% Version: $Id: graebe-04-5.tex,v 1.1 2008/09/05 09:52:58 graebe Exp $
\documentclass[11pt]{article}  
\usepackage{kosemnet,ko-math,ngerman}  

\author{Hans-Gert Gr�be, Leipzig}
\title{Ortslinien in der Geometrie\kosemnetlicensemark\\
Arbeitsmaterial f�r Klasse 7}
\date{}

\begin{document} 
\maketitle         

Ortslinien haben eine gro"se Bedeutung in der Geometrie, da sie es
erlauben, logische Bedingungen in geometrische Aussagen zu verwandeln
und umgekehrt. So kann man etwa die Aussage
\begin{center}
$X$ ist von $A$ und $B$ gleichweit entfernt
\end{center}
umformulieren zu 
\begin{center}
$X$ liegt auf der Mittelsenkrechten der Strecke $\ksegment{AB}$,
\end{center}
denn die Mittelsenkrechte ist gerade der entsprechende geometrische
Ort. 

Soll ein Punkt $X$ mehrere Bedingungen erf"ullen (wie z.B. in einer
Konstruktionsaufgabe), so ergibt sich aus der Umformulierung in die
Sprache der geometrischen Orte, dass $X$ der Schnittpunkt der
entsprechenden Linien ist, was man oft zu dessen Konstruktion
verwenden kann.
\begin{aufgabe}
  Konstruiere ein Dreieck $\ktriangle{ABC}$, von dem die L"ange $c$ der Seite
  $AB$, die Gr"o"se $\alpha$ des Winkels $\kangle{CAB}$ und die L"ange $R$ des
  Umkreisradius gegeben sind.
\end{aufgabe}

\ul{L"osung:} Wir beginnen mit der Seite $\ksegment{AB}$, die wir entsprechend
der vorgegebenen L"ange zeichnen. Der fehlende Punkt $C$ wird durch zwei
Bedingungen bestimmt: $\mangle{CAB}=\alpha$ und die Gr"o"se $R$ des
Umkreisradius. Die zugeh"origen geometrischen Orte sind der Schenkel eines
Winkels und eine Kreislinie. Den Schenkel des Winkels kann man sofort
konstruieren, indem man $\alpha$ an der Seite $\ksegment{AB}$ in $A$ abtr"agt.
Etwas schwieriger wird es mit dem Umkreis, von dem wir wissen, dass er durch
$A,B$ geht und den Radius $R$ hat. Wo liegt sein Mittelpunkt $M$~?  Antwort,
wieder mit Ortslinien: $M$ ist erstens von $A,B$ gleichweit entfernt (liegt
also auf der Mittelsenkrechten der Strecke $\ksegment{AB}$) und zweitens gilt
$\msegment{AM}=R$, d.h. $M$ liegt auf dem Kreis um $A$ mit dem Radius
$R$. Damit bekommen wir $M$ als Schnitt der beiden genannten Linien. Nun
k"onnen wir den Umkreis des gesuchten Dreiecks zeichnen und finden
schlie"slich $C$ als Schnittpunkt des Schenkels des in $A$ angetragenen
Winkels und des Umkreises.

\begin{aufgabe}
  Beachte, dass die L"osung nicht eindeutig ist. Wieviel zueinander nicht
  kongruente L"osungen hat die Aufgabe? An welcher Stelle muss man genauer
  argumentieren, um diese weiteren L"osungen nicht zu "ubersehen.
\end{aufgabe}

Umgekehrt kann man in geometrischen Beweisen, in denen Linien
vorkomemn, die geometrische Orte sind, oft die logischen Eigenschaften
verwenden, die sich mit ihnen verbinden. Erinnern wir uns an den
\begin{satz}[Satz vom Schnittpunkt der Mittelsenkrechten] In jedem Dreieck
gehen die drei Mittelsenkrechten durch einen gemeinsamen Punkt, den
Umkreismittelpunkt des Dreiecks.
\end{satz}
und dessen \ul{Beweis:} Sei $M$ der Schnittpunkt der Mittelsenkrechten auf
$\ksegment{AB}$ und $\ksegment{AC}$. Weil $M$ auf der Mittelsenkrechten von
$\ksegment{AB}$ liegt, ist er von $A$ und $B$ gleichweit entfernt,
d.h. $\msegment{AM}=\msegment{BM}$. Weil $M$ auf der Mittelsenkrechten von
$\ksegment{AC}$ liegt, ist er von $A$ und $C$ gleichweit entfernt, d.h.\
$\msegment{AM}=\msegment{CM}$. Damit ist auch $\msegment{BM}=\msegment{CM}$
und folglich liegt $M$ auf der Mittelsenkrechten der Strecke
$\ksegment{BC}$. Daraus folgen alle getroffenen Aussagen unmittelbar.
\medskip

Du siehst also, dass der Wechsel zwischen logischer und geometrischer
Formulierung desselben Sachverhalts in vielen Aufgabenstellungen
hilfreich ist. Eine {\bf Aufstellung aus der Schule bekannter
geometrischer Ortslinien} w�re hier noch zu erg�nzen.

\subsubsection*{Wie findet man eine Ortslinie~?}

Die folgenden Regeln helfen oft, eine passende Ortslinie zu finden:
\begin{enumerate}
\item Suche zuerst nach einzelnen Punkten, die auf der Ortslinie liegen
m"ussen.
\item Stelle eine Vermutung "uber die Art der Ortslinie auf (meist sind es
Geraden oder Kreise; sonst kann man damit wenig anfangen).
\item Und zuletzt: Beweise Deine Vermutung.
\end{enumerate}


\begin{attribution}
graebe (2004-09-02):\\ Dieses Material wurde vor einiger Zeit als
Begleitmaterial f�r den LSGM-Korrespondenzzirkel in der Klasse 7 erstellt und
nun nach den Regeln der KoSemNet-Literatursammlung aufbereitet.
\end{attribution}

\end{document}
