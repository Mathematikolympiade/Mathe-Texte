\documentclass[11pt]{article}
\usepackage{kosemnet,ko-math,ngerman}

\title{Die Jensensche Ungleichung\kosemnetlicensemark}

\author{Hans-Gert Gr"abe, Univ. Leipzig} 

\date{3.~Februar 1998}

\begin{document}

\maketitle

\section{Konvexe und konkave Funktionen}

Wir betrachten eine stetige Funktion $y=f(x)$, die auf einem offenen
Intervall $]a,b[$ definiert sein m"oge. Eine solche Funktion k"onnen
wir in einem x-y-Koordinatensystem zeichnen, wobei man die Menge
\[\Gamma_f:=\{(x,f(x))\ :\ x\in ]a,b[\}\]
als den {\em Graphen} von $f$ bezeichnet. 
\medskip

\begin{definition}
Wir nennen eine solche Funktion $f$ {\em konvex} oder {\em nach unten
gekr"ummt}, wenn jede Sekante, die zwei Funktionswerte auf dem Graphen
von $f$ verbindet, vollst"andig oberhalb $\Gamma_f$ verl"auft.
\medskip

Wir nennen eine solche Funktion $f$ {\em konkav} oder {\em nach oben
gekr"ummt}, wenn jede Sekante, die zwei Funktionswerte auf dem Graphen
von $f$ verbindet, vollst"andig unterhalb $\Gamma_f$ verl"auft.
\end{definition}

Beispiele f"ur derartige Funktionen, deren Graph aus dem
Schulunterricht gut bekannt ist.
\begin{enumerate}
\item $f(x)=\log(x)$. Die Logarithmusfunktion (zu einer Basis $b>1$)
ist in ihrem gesamten Definitionsbereich $x>0$ nach oben gekr"ummt,
also {\em konkav}.
\item $f(x)=x^k$. Die Potenzfunktion (mit $k>1$) ist f"ur $x\geq 0$
nach unten gekr"ummt, also {\em konvex}.
\item $f(x)=\frac{1}{x}$. Diese Funktion ist f"ur $x> 0$ nach unten
gekr"ummt, also {\em konvex}.
\item $f(x)=\sin(x)$. Die Sinusfunktion ist im Bereich $[0,\pi]$ nach
oben gekr"ummt, also {\em konkav}, im Bereich $[\pi,2\,\pi]$ dagegen
nach unten gekr"ummt, also {\em konvex}.
\item $f(x)=\tan(x)$. Die Tangensfunktion ist im Bereich
$]-\frac{\pi}{2},0]$ konkav und im Bereich $[0,\frac{\pi}{2}[$
konvex. 
\end{enumerate}

F"ur gen"ugend gute Funktionen, deren Graph uns nicht so gel"aufig ist
wie die Graphen der bisher aufgef"uhrten Funktionen, kann man das
Kr"ummungsverhalten auch mit Mitteln der Differentialrechnung
bestimmen. Offensichtlich trennen die Wendepunkte der Funktion $f$ die
Bereiche unterschiedlichen Kr"ummungsverhaltens voneinander. Ist eine
Funktion $f$ (glatt und) nach oben gekr"ummt und legen wir die
Tangente an verschiedene Punkte des Graphen von $f$, so sehen wir,
dass der Anstieg dieser Tangenten, also der Wert von $f'(x)$, f"ur
wachsendes $x$ immer kleiner wird. $f'$ muss also eine monoton
fallende Funktion sein. Eine monoton fallende Funktion erkennen wir
aber daran, dass deren Ableitung, also hier $f''(x)$, kleiner als 0
ist. "Ahnlich verh"alt es sich mit konvexen Funktionen.

\begin{satz}
Eine (gen"ugend oft stetig differenzierbare) Funktion $f$ ist genau
dann in einem Intervall $]a,b[$ konvex (konkav), wenn dort ihre erste
Ableitung $f'$ monoton w"achst (f"allt). Das gilt genau dann, wenn
ihre zweite Ableitung $f''(x)$ im gesamten Intervall $\geq 0$ ($\leq
0$) ist.
\end{satz}

Beispiel: Betrachten wir die Funktion $f(x)=\sin(x)+\tan(x)$ im
Intervall $[0,\frac{\pi}{2}[$. Als Summe einer nach oben gekr"ummten
und einer nach unten gekr"ummten Funktion ist das Kr"ummungsverhalten
nicht auf den ersten Blick zu erkennen. Allerdings gilt 
\[f''(x)=-{\rm sin}(x) +2\,{{\rm tan}(x)}^{3} +2\,{\rm tan}(x)\geq
0,\]
da bekanntlich $\sin(x)\leq \tan(x)$ ist. Die Funktion $f$ ist also
konvex. 

\section{Die Jensensche Ungleichung}

Sei $y=f(x)$ eine im Intervall $]a,b[$ (streng) konvexe Funktion,
$x_1,x_2\in ]a,b[$ und $y_1=f(x_1),\,y_2=f(x_2)$ die zugeh"origen
Funktionswerte. Dann verl"auft die Sekante von $P_1=(x_1,y_1)$ nach
$P_2=(x_2,y_2)$ vollkommen oberhalb des Graphen von $f$. Insbesondere
liegt der Mittelpunkt $S=(\frac{x_1+x_2}{2}, \frac{y_1+y_2}{2})$ der
Sekante $P_1P_2$ {\em oberhalb} des Funktionswerts
$f(\frac{x_1+x_2}{2})$ an der Stelle $\frac{x_1+x_2}{2}$. Mithin gilt
{\em f"ur eine konvexe Funktion} stets
\[f(\frac{x_1+x_2}{2})\leq \frac{f(x_1)+f(x_2)}{2}\]
und analog {\em f"ur eine konkave Funktion}
\[f(\frac{x_1+x_2}{2})\geq \frac{f(x_1)+f(x_2)}{2}.\]
Mehr noch, Gleichheit gilt in beiden F"allen nur dann, wenn die
Sekante entartet, also f"ur $x_1=x_2$.

Diese recht einfachen "Uberlegungen f"uhrten zu zwei Ungleichungen,
die in vielen Anwendungen ein wichtiges Hilfsmittel sein
k"onnen. Bevor wir im n"achsten Punkt derartige Anwendungen
betrachten, wollen wir diese "Uberlegungen noch verallgemeinern.
\medskip

Betrachten wir dazu $n$ verschiedene Argumente $x_1,x_2,\ldots,x_n$
aus dem gegebenen Intervall, deren Funktionswerte $y_i=f(x_i),\,
i=1,\ldots,n$ und die zugeh"origen Punkte $P_i=(x_i,y_i)$ auf dem
Graphen der Funktion $f$. Diese Punkte spannen ein $n$-Eck auf, dessen
Schwerpunkt $S$ bekanntlich die Koordinaten
\[S=\left(\frac{x_1+x_2+\ldots+x_n}{n},\,\frac{y_1+y_2+\ldots+y_n}{n}
\right) \]
hat. Ist $f$ wiederum eine konvexe Funktion, dann liegt zusammen mit
jeder der Sekanten, die zwei der Punkte verbinden, auch das gesamte
$n$-Eck {\em oberhalb} des Graphen von $f$. Insbesondere liegt der
Schwerpunkt $S$ {\em oberhalb} des Punktes auf dem Graphen mit
derselben Abszisse.

Es gilt also 
\begin{satz}[Jensensche Ungleichung]  Ist $f$ eine im Intervall
$]a,b[$ konvexe Funktion und $x_1,\ldots,x_n$ aus diesem Intervall, so
gilt die Ungleichung
\[f\left(\frac{x_1+x_2+\ldots+x_n}{n}\right)\leq \frac{f(x_1)+f(x_2)
+\ldots +f(x_n)}{n}.\]

Analog gilt f"ur eine konkave Funktion $f$ die Ungleichung
\[f\left(\frac{x_1+x_2+\ldots+x_n}{n}\right)\geq \frac{f(x_1)+f(x_2)
+\ldots +f(x_n)}{n}.\] 

Gleichheit gilt (f"ur streng konvexes bzw. konkaves $f$) wiederum nur
dann, wenn alle Argumente $x_i$ "ubereinstimmen.
\end{satz}

\section{Anwendungen}

Als erste Anwendung wollen wir zeigen, dass in jedem Dreieck f"ur die
Innenwinkel 
\[\sin(\alpha)+\sin(\beta)+\sin(\gamma)\leq \frac{3\,\sqrt{3}}{2}\]
gilt.

Diese Ungleichung ergibt sich als eine einfache Folgerung aus der
Jensenschen Ungleichung, denn $\sin(x)$ ist im Intervall $[0,\pi]$
konkav. Also gilt
\[\frac{\sin(\alpha)+\sin(\beta)+\sin(\gamma)}{3}\leq
\sin\left(\frac{\alpha +\beta +\gamma}{3} \right) =\sin(60^0) =
\frac{\sqrt{3}}{2}.\] Gleichheit gilt bei dieser Ungleichung nur f"ur
das gleichseitige Dreieck.

\begin{aufgabe}
 Zeigen Sie, dass in einem Dreieck f"ur die Innenwinkel stets
\[\tan^2(\frac{\alpha}{2})+ \tan^2(\frac{\beta}{2})+
\tan^2(\frac{\gamma}{2}) \geq 1\]
gilt.
\end{aufgabe}

In einem Dreieck gilt f"ur die Innenwinkel stets auch 
\[\sin(\frac{\alpha}{2}) \cdot\sin(\frac{\beta}{2})
\cdot\sin(\frac{\gamma}{2}) \leq \frac{1}{8}\]

Hier betrachten wir die Funktion $f(x)=\log(\sin(x))$ im Intervall
$]0, \frac{\pi}{2}]$. Aus der zweiten Ableitung $f''(x)= -\,
\frac{1}{{{\rm sin}(x)}^{2}}$ (oder durch genauere Analyse des
Graphen) erkennen wir, dass diese Funktion im ganzen
Definitionsbereich konkav ist, d.h. dass
\[\log(\sin(\frac{\alpha}{2}))+\log(\sin(\frac{\beta}{2})) +
\log(\sin(\frac{\gamma}{2})) \leq 3\, \log(\sin(30^0)) =
\log(2^{-3})
\] 
gilt, woraus die Behauptung unmittelbar folgt (denn $\log(x)$ ist ja
eine monotone wachsende Funktion).
\medskip

Aus der Jensenschen Ungleichung lassen sich auch viele andere
wohlbekannte Ungleichungen herleiten. Betrachen wir z.B. die Funktion
$f(x)=\log(x)$, die ja im ganzen Definitionsbereich $x>0$ konkav ist,
so ergibt sich nach den Logarithmengesetzen
\[\log\left(\frac{x_1+x_2+\ldots+x_n}{n}\right) \geq \frac{\log(x_1)
+\log(x_2) +\ldots +\log(x_n)}{n} =
\log(\sqrt[n]{x_1\,x_2\,\ldots\,x_n}), \]
also die {\bf Ungleichung vom arithmetisch-geometrischen Mittel}.


\begin{aufgabe}
Leiten Sie unter Verwendung der Potenzfunktion als $f$ die Ungleichung
vom quadratischen Mittel und vom harmonischen Mittel her.
\end{aufgabe}

Auch kompliziertere Ungleichungen lassen sich auf die Jensensche
zur"uck\-f"uhren. Die Aufgabe 061243 der 6.~Mathematikolympiade lautete etwa
\begin{quote}
$a_1,\ldots,a_n$ seien positive reelle Zahlen und
$s=\sum_{i=1}^n{a_i}$ deren Summe. Beweisen Sie die Ungleichung
\[\Sigma:=\sum_{i=1}^n{\frac{a_i}{s-a_i}}\geq \frac{n}{n-1}\]
\end{quote}

\begin{aufgabe}
Zeigen Sie die Ungleichung f"ur $n=3$, d.h.
\[\frac{a_1}{a_2+a_3}+\frac{a_2}{a_2+a_3}+\frac{a_3}{a_1+a_2} \geq
\frac{3}{2}, \] 
ohne die Jensensche Ungleichung zu verwenden.
\end{aufgabe}
 
F"ur den Beweis der Ungleichung f"ur eine beliebige nat"urliche Zahl
$n$ betrachten wir die Funktion $f(x)=\frac{x}{s-x}$ im Intervall
$0<x<s$. Da deren zweite Ableitung $f''(x)=\frac{2\,s}{{(s -x)}^{3}}$
"uberall positiv, $f$ also konvex ist, erhalten wir
\[f\left(\frac{a_1+a_2+\ldots+a_n}{n}\right)=
f\left(\frac{s}{n}\right) \leq \frac{f(x_1)+f(x_2) +\ldots
+f(x_n)}{n}={\frac{\Sigma}{n}}.\]
Daraus ergibt sich die Behauptung unmittelbar.

\section{Eine weitere Verallgemeinerung}

Betrachten wir noch einmal das von den Punkten $P_i=(x_i,y_i)$
aufgespannte $n$-Eck. Zum Beweis der Jensenschen Ungleichung haben wir
einzig die Tatsache verwendet, dass der Schwerpunkt $S$ f"ur eine
konvexe Funktion $f$ {\em oberhalb} des Graphen von $f$ liegt. Das ist
aber f"ur {\em jeden} Punkt aus dem Inneren des $n$-Ecks richtig. Die
x-y-Koordinaten eines solchen Punktes kann man aus seinen {\em
baryzentrischen Koordinaten} bzgl. $P_1,\ldots,P_n$ bestimmen. Dazu
versehen wir die Eckpunkte mit nichtnegativen Gewichten
$w_1,\ldots,w_n\in {\R}$, die in der Summe gerade 1 ergeben. Der
Punkt $P$ mit den Koordinaten
\[P=w_1P_1+\ldots+w_nP_n=\left(w_1x_1+\ldots+w_nx_n,\,
w_1y_1+\ldots+w_ny_n \right)\]
ist dann ein Punkt im Inneren des $n$-Ecks und liegt somit oberhalb
des Punkts auf dem Graphen von $f$ mit derselben Abszisse. Wir
erhalten als Verallgemeinerung der oben bewiesenen Ungleichung den
folgenden 
\begin{satz}[Gewichtete Jensensche Ungleichung]  Ist $f$ eine im
Intervall $]a,b[$ konvexe \linebreak Funktion, $x_1,\ldots,x_n$
aus diesem Intervall und $w_1,\ldots,w_n$ nichtnegative Gewichte
mit $w_1+\ldots+w_n=1$, so gilt die Ungleichung
\[f\left(w_1x_1+w_2x_2+\ldots+w_nx_n\right)\leq w_1f(x_1)+w_2f(x_2)
+\ldots +w_nf(x_n).\]

Analog gilt f"ur eine konkave Funktion $f$ die Ungleichung
\[f\left(w_1x_1+w_2x_2+\ldots+w_nx_n\right)\geq w_1f(x_1)+w_2f(x_2)
+\ldots +w_nf(x_n).\]

Gleichheit gilt (f"ur streng konvexes bzw. konkaves $f$) wiederum nur
dann, wenn alle Argumente $x_i$, deren Gewicht $w_i$ positiv ist,
"ubereinstimmen.
\end{satz}

\end{document}

