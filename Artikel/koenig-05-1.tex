\documentclass[11pt]{article}
\usepackage{ko-math,ko-graf,kosemnet,ngerman}
\usepackage[latin1]{inputenc}

\newcommand{\Bild}[3]{
\begin{center}
\includegraphics[#1]{#2}\nopagebreak\\[12pt] #3
\end{center}
}

\title{Wo steckt der Fehler?\kosemnetlicensemark} 
\author{Helmut K�nig, Chemnitz} 
\date{4. Februar 2005}

\begin{document}
\maketitle

Wenn man im au�erunterrichtlichen Bereich bei den Sch�lern die F�higkeit zum
probleml�senden Denken entwickeln will, dann sollte man nicht vergessen,
bisweilen auch nach Fehlern in vorgelegten L�sungen suchen zu lassen. Ein
hierf�r gut geeignetes Beispiel soll hier vorgestellt werden.

\subsection*{Die Aufgabenstellung}


  Wir betrachten ein Rechteck $ABCD$, dessen Seite $\ksegment{AB}$ halb so
  lang ist wie die Seite $\ksegment{BC}$ und bei dem $M$ der Mittelpunkt von
  $\ksegment{BC}$ ist. Die Strecke $\ksegment{MA}$ schneide die Diagonale
  $\ksegment{BD}$ im Punkt $P$ und $Q$ sei der Fu�punkt des Lots von $P$ auf
  $\ksegment{BC}$. Die Strecke $\ksegment{MD}$ schneide die Diagonale
  $\ksegment{AC}$ im Punkt $S$ und $R$ sei der Fu�punkt des Lots von $S$ auf
  $\ksegment{BC}$.

  Weise nach, dass unter diesen Voraussetzungen $PQRS$ ein Rechteck ist,
  dessen Seite $\ksegment{PQ}$ halb so lang ist wie die Seite $\ksegment{QR}$
  und dessen Fl�cheninhalt gleich dem neunten Teil des Fl�cheninhalts von
  $ABCD$ ist!


\subsection*{Auftrag an Sch�ler}

Beweise den ersten Teil der Behauptung! 

Pr�fe nach, ob die folgenden drei Beweise des zweiten Teils der Behauptung
korrekt sind!  Wenn dies nicht der Fall ist, dann gib an, wo der Fehler
steckt!

\begin{minipage}{0.75\textwidth}
  \emph{1. Beweis}: Man teile die Seiten von $ABCD$ jeweils in drei gleiche
  Teile. Die in der Abbildung eingezeichneten Verbindungsstrecken sind dann zu
  einer der beiden Seiten parallel und zerlegen $ABCD$ in neun Rechtecke, die
  offensichtlich kongruent und daher auch inhaltsgleich sind und deren
  Seitenl�ngen sich wie $1:2$ verhalten. Das rechts neben dem zentralen
  Teilrechteck liegende Rechteck ist unser Rechteck $PQRS$. Hieraus folgt
  unmittelbar die Behauptung.\vspace*{20pt}
\end{minipage}\hfill
\begin{minipage}{0.2\textwidth}
\psset{unit=3cm}
%Bildbeginn: Breite=3cm
\begin{flushright}
\psset{unit=0.1000}
\pspicture*(0,1)(10,15)
\Hauptlinien
\psline{-}(2,2)(8,2)(8,14)(2,14)(2,2)
\psline{-}(8,6)(6,6)(6,10)(8,10)
\psline{-}(2,2)(8,14)
\psline{-}(2,14)(8,2)
\Hilfslinien
\psline{-}(4,2)(4,14)
\psline{-}(6,2)(6,6)(2,6)
\psline{-}(6,14)(6,10)(2,10)
\ZPunkt{2,2}{210}{A}
\ZPunkt{4,2}{0}{}
\ZPunkt{6,2}{0}{}
\ZPunkt{8,2}{330}{B}
\ZPunkt{2,6}{0}{}
\ZPunkt{4,6}{0}{}
\ZPunkt{6,6}{60}{P}
\ZPunkt{8,6}{0}{Q}
\ZPunkt{2,10}{0}{}
\ZPunkt{4,10}{0}{}
\ZPunkt{6,10}{300}{S}
\ZPunkt{8,10}{0}{R}
\ZPunkt{2,14}{150}{D}
\ZPunkt{4,14}{0}{}
\ZPunkt{6,14}{0}{}
\ZPunkt{8,14}{30}{C}
\endpspicture
\end{flushright}

\end{minipage}
\bigskip

\begin{minipage}{0.7\textwidth}
  \emph{2. Beweis}: Die Verl�ngerung von $\ksegment{QP}$ �ber $P$ hinaus
  schneide $\ksegment{AC}$ in $T$ und $\ksegment{AD}$ in $Q'$; die
  Verl�ngerung von $\ksegment{RS}$ �ber $S$ hinaus schneide $\ksegment{BD}$ in
  $U$ und $\ksegment{AD}$ in $R'$.

  Da $PQRS$ ein Rechteck ist, folgt hieraus $\kline{Q'Q}\parallel\kline{R'R}$
  und $\kline{PS}\parallel\kline{BC}$.

  Da $ABCD$ ein Rechteck ist, sind seine beiden Mittellinien die beiden
  Symmetrieachsen $m$ und $n$ dieses Vierecks.

  Aus Eigenschaften der Geradenspiegelung folgt, dass bei der Spiegelung an
  $m$ die Diagonale $\ksegment{BD}$ das Bild der Diagonalen $\ksegment{AC}$
  ist und dass die Gerade $\kline{QQ'}$ in sich �bergeht. Folglich geht $P$ in
  $T$ und analog auch $S$ in $U$ �ber.

  Hieraus folgt, dass $TQ'R'U$ das Bild unseres Rechtecks $PQRS$ bei
  Spiegelung an $m$ ist und dass daher diese beiden Rechtecke kongruent und
  folglich auch inhaltsgleich sind. 
\end{minipage}\hfill
\begin{minipage}{0.27\textwidth}
\psset{unit=4cm}
%Bildbeginn: Breite=4cm
\begin{flushright}
\psset{unit=0.1000}
\pspicture*(0,0)(10,15)
\Hauptlinien
\psline{-}(2,2)(8,2)(8,14)(2,14)(2,2)
\psline{-}(8,6)(6,6)(6,10)(8,10)
\psline{-}(8,14)(2,2)
\psline{-}(2,14)(8,2)
\psline{-}(1,8)(9,8)
\psline{-}(5,1)(5,15)
\psline{<->}(4.3,1.3)(5.7,1.3)
\psline{<->}(8.7,7.3)(8.7,8.7)
\Hilfslinien
\psline{-}(4,2)(4,14)
\psline{-}(6,2)(6,6)(2,6)
\psline{-}(6,14)(6,10)(2,10)
\ZPunkt{2,2}{210}{A}
\ZPunkt{4,2}{0}{}
\ZPunkt{6,2}{0}{}
\ZPunkt{8,2}{330}{B}
\ZPunkt{2,6}{210}{Q'}
\ZPunkt{4,6}{210}{T}
\ZPunkt{6,6}{330}{P}
\ZPunkt{8,6}{330}{Q}
\ZPunkt{2,10}{150}{R'}
\ZPunkt{4,10}{150}{U}
\ZPunkt{6,10}{30}{S}
\ZPunkt{8,10}{30}{R}
\ZPunkt{2,14}{150}{D}
\ZPunkt{4,14}{0}{}
\ZPunkt{6,14}{0}{}
\ZPunkt{8,14}{30}{C}
\uput[0]{*0}(8.7,8){\ensuremath{n}}
\uput[270]{*0}(5,1.3){\ensuremath{m}}
\endpspicture
\end{flushright}
\end{minipage}
\bigskip

  F�hrt man anschlie�end noch die Spiegelung an der Symmetrieachse $n$ durch,
  dann erkennt man, dass dies f�r alle neun Teilrechtecke zutrifft. 

  Hieraus folgt unmittelbar, dass alle neun Teilrechtecke und daher auch unser
  Rechteck $PQRS$ einen Fl�cheninhalt besitzen, der gleich dem neunten Teil
  des Fl�cheninhalts von $ABCD$ ist.
\bigskip

\begin{minipage}{0.7\textwidth}
  \emph{3. Beweis}: Seien $T$ und $U$ die im 2.~Beweis eingef�hrten Hilfspunkte.
  Da $PQRS$ ein Rechteck ist, folgt mithilfe des Nebenwinkelsatzes
\[\mangle{RQP} = \mangle{SPT} = \mangle{CRS} (=90\grad) . \tag{1} \]
Ferner folgt $\kline{PQ}\parallel\kline{RS}$ und mithilfe des
Stufenwinkelsatzes $\mangle{PTS}=\mangle{RSC}$. Wegen
$\kline{AC}\parallel\kline{PR}$ folgt analog $\mangle{PTS}=\mangle{QPR}$ und
daher
\[\mangle{QPR}=\mangle{PTS}=\mangle{RSC}. \tag{2}\] 
Da $PQRS$ ein Rechteck ist, gilt auch 
\[\msegment{QR}= \msegment{PS}\ \text{und}\  \msegment{PQ}=
\msegment{SR}. \tag{3}\] 
\end{minipage}\hfill
\begin{minipage}{0.27\textwidth}
\psset{unit=4cm}
%Bildbeginn: Breite=4cm
\begin{flushright}
\psset{unit=0.1000}
\pspicture*(0,1)(10,15)
\MarkFlaeche{(4,6)(6,6)(6,10)}
\MarkFlaeche{(6,6)(8,6)(8,10)}
\MarkFlaeche{(6,10)(8,10)(8,14)}
\Hauptlinien
\psline{-}(2,2)(8,2)(8,14)(2,14)(2,2)
\psline{-}(8,6)(6,6)(6,10)(8,10)
\psline{-}(8,14)(2,2)
\psline{-}(2,14)(8,2)
\Hilfslinien
\psline{-}(4,2)(4,14)
\psline{-}(6,2)(6,6)(2,6)
\psline{-}(6,14)(6,10)(2,10)
\Hilfswinkel{8,10}{90}{180}
\Hilfswinkel{8,6}{90}{180}
\Hilfswinkel{6,10}{0}{63.434949}
\Hilfswinkel{6,6}{0}{63.434949}
\Hilfswinkel{6,6}{116.565051}{180}
\Hilfswinkel{4,6}{0}{63.434949}
\ZPunkt{2,2}{210}{A}
\ZPunkt{4,2}{0}{}
\ZPunkt{6,2}{0}{}
\ZPunkt{8,2}{330}{B}
\ZPunkt{2,6}{0}{}
\ZPunkt{4,6}{210}{T}
\ZPunkt{6,6}{210}{P}
\ZPunkt{8,6}{330}{Q}
\ZPunkt{2,10}{0}{}
\ZPunkt{4,10}{150}{U}
\ZPunkt{6,10}{150}{S}
\ZPunkt{8,10}{30}{R}
\ZPunkt{2,14}{150}{D}
\ZPunkt{4,14}{0}{}
\ZPunkt{6,14}{0}{}
\ZPunkt{8,14}{30}{C}
\endpspicture
\end{flushright}
\end{minipage}
\bigskip

Aus (1), (2), (3) folgt mithilfe des Kongruenzsatzes (wsw) die Kongruenz der
Dreiecke $PQR$, $TPS$ und $SRC$, woraus dann leicht die Kongruenz aller neun
Teilrechtecke abgeleitet werden kann. Hieraus folgt dann unmittelbar unsere
Behauptung. 

\subsection*{Auswertung mit den Sch�lern}

{\bf Der 1. Beweis ist unkorrekt.} \\
Hier liegt der grobe Beweisfehler vor, dass bei einem (direkten) Beweis {\em
von der Behauptung ausgegangen} wird. 

Dass $ABCD$ in neun kongruente Teilrechtecke zerlegt wird ist ein
hinreichendes Teilziel, aus dem die Behauptung unmittelbar folgt und das aus
den Voraussetzungen abgeleitet werden muss.

Man k�nnte auch sagen, dass hier nicht unser Satz sondern eine Umkehrung
dieses Satzes bewiesen wurde. 

{\bf Der 2. Beweis ist unkorrekt.} \\ 
Hier wird nur bewiesen, dass die vier {\glqq}Eckrechtecke{\grqq} und die vier
{\glqq}Seitenrechtecke{\grqq} untereinander kongruent sind. Dass ein
{\glqq}Eckrechteck{\grqq}, ein {\glqq}Seitenrechteck{\grqq} und das
{\glqq}zentrale Rechteck{\grqq} nicht kongruent sein m�ssen erkennt man, wenn
man das {\glqq}zentrale Rechteck{\grqq} �hnlich vergr��ert oder verkleinert.

In diesem Fall liegt eine {\em Beweisl�cke} vor. 

{\bf Der 3. Beweis ist unkorrekt.}\\
Die f�r den Beweis unverzichtbare {\em Feststellung}
$\kline{AC}\parallel\kline{PR}$ wurde {\em der Anschauung entnommen}, aber
{\em nicht aus den Voraussetzungen hergeleitet}. (Dies w�re mit dem
Hilfsmittel �hnlichkeit/Strahlens�tze m�glich.)

\subsection*{Ein korrekter Beweis unserer Aufgabe}

  Sei $\ksegment{MN}$ eine Mittellinie des Rechtecks $ABCD$ und $O$ der auf
  $\ksegment{MN}$ liegende Diagonalenschnittpunkt. Sei $S'$ bzw. $S''$ der
  Schnittpunkt der Geraden $\kline{PS}$ mit $\ksegment{MN}$ bzw.
  $\ksegment{CD}$ und $U''$ der Schnittpunkt von $\kline{TU}$ mit
  $\ksegment{CD}$. 

\begin{minipage}{0.7\textwidth}

  Nach Voraussetzung ist $\ksegment{AB}$ halb so lang wie $\ksegment{BC}$.
  Hieraus folgt, dass $NMCD$ ein Quadrat ist, auf dessen Diagonale
  $\ksegment{MD}$ nach Voraussetzung der Punkt $S$ liegt.

  Hieraus folgt, dass auch $S'MRS$ ein Quadrat ist und dass $PQRS$ ein
  Rechteck ist, deren Seite $\ksegment{PQ}$ halb so lang ist wie die Seite
  $\ksegment{QR}$.

  Aus den Voraussetzungen folgt $\kline{OM}\parallel\kline{CD}$ und
  $\msegment{OM}:\msegment{CD} = 1:2$. Mithilfe des {\em Strahlensatzes} folgt
  hieraus $\msegment{S'S}:\msegment{SS''} = 1:2$, woraus dann $\msegment{MR} =
  \msegment{RC} = \msegment{QR}$ und hieraus $\msegment{BQ} = \msegment{QR} =
  \msegment{RC}$ folgt. 

  Da man zeigen kann (siehe 2. Beweis, Spiegelung an $m$) dass
  $\kline{PS}\parallel\kline{TU}$ gilt, folgt analog $\msegment{CS''} =
  \msegment{S''U''} = \msegment{U''D}$. Der Rest ist einfach.
\end{minipage}\hfill
\begin{minipage}{0.3\textwidth}
\psset{unit=4.5cm}
%Bildbeginn: Breite=4.5cm
\begin{flushright}
\psset{unit=0.1000}
\pspicture*(0,1)(10,15)
\Hauptlinien
\psline{-}(2,2)(8,2)(8,14)(2,14)(2,2)
\psline{-}(8,6)(6,6)(6,10)(8,10)
\psline{-}(8,14)(2,2)(8,8)(2,14)(8,2)
\psline{-}(2,8)(8,8)
\Hilfslinien
\psline{-}(4,2)(4,14)
\psline{-}(6,2)(6,6)(2,6)
\psline{-}(6,14)(6,10)(2,10)
\ZPunkt{2,2}{210}{A}
\ZPunkt{4,2}{0}{}
\ZPunkt{6,2}{0}{}
\ZPunkt{8,2}{330}{B}
\ZPunkt{2,6}{0}{}
\ZPunkt{4,6}{210}{T}
\ZPunkt{6,6}{330}{P}
\ZPunkt{8,6}{330}{Q}
\ZPunkt{2,10}{0}{}
\ZPunkt{4,10}{150}{U}
\ZPunkt{6,10}{30}{S}
\ZPunkt{8,10}{30}{R}
\ZPunkt{2,14}{150}{D}
\ZPunkt{4,14}{90}{U''}
\ZPunkt{6,14}{90}{S''}
\ZPunkt{8,14}{30}{C}
\ZPunkt{2,8}{180}{N}
\ZPunkt{5,8}{150}{O}
\ZPunkt{6,8}{30}{S'}
\ZPunkt{8,8}{0}{M}
\endpspicture
\end{flushright}
\end{minipage}



\begin{attribution}
koenig (2005-01-25): Contributed to KoSemNet
\end{attribution}


\end{document}

