% Version: $Id: graebe-05-1.tex,v 1.1 2008/09/05 09:52:58 graebe Exp $
\documentclass[11pt]{article}  
\usepackage{kosemnet,ko-math,ngerman}  

\author{Hans-Gert Gr�be, Leipzig}
\title{Regul�re Polyeder\kosemnetlicensemark}
\date{}

\begin{document}

\maketitle

\section*{Der Eulersche Polyedersatz}

Als {\em konvexes Polyeder} bezeichnet man einen durch endlich viele
ebene Fl�\-chen begrenzten K�rper, wo zusammen mit je zwei Punkten
auch alle Punkte der Verbindungsstrecke zu diesem K�rper geh�ren.

Jede Seitenfl�che definiert eine Ebene, die {\em St�tzebene} des
Polyeders zu dieser Seitenfl�che, so dass das Polyeder vollst�ndig in
einem Halbraum bzgl.\ dieser Ebene liegt. Das Polyeder ist als
Punktmenge genau der Durchschnitt all dieser Halbr�ume.

\begin{satz}[Polyedersatz, Euler 1758]
Hat ein konvexes Polyeder $e$ Ecken, $k$ Kanten und $f$ Seitenfl�chen,
so gilt stets \[e-k+f=2\,.\]
\end{satz}

\ul{Beweisskizze}: Es gibt verschiedene Beweise dieses Satzes. Eine
besonders elementare �ber\-legung verwendet das schrittweise
``Aufschneiden'' in ein ebenes K�rpernetz und untersucht, wie sich
dabei die Zahlen $e,k,f$ und $Z=e-k+f$ ver�ndern. Wir bezeichnen die
neuen Werte jeweils mit $e',k',f',Z'$.

Zun�chst wird eine Seitenfl�che herausgeschnitten: $e'=e, k'=k,
f'=f-1$, also $Z'=Z-1$.

Danach werden weitere Kanten aufgeschnitten, bis das K�rpernetz
schlie�lich ausgebreitet werden kann. Bei jedem Schnitt wird es eine
Ecke und eine Kante mehr: $e'=e+1, k'=k+1, f'=f, Z'=Z$.

Das K�rpernetz ist ein ebener (�berschneidungsfreier) Graph mit Ecken
und Kanten. Die Seitenfl�chen nennen wir jetzt ``Gebiete''.  Nun
werden in diesem Netz nacheinander die inneren Kanten entfernt:
\begin{itemize}
\item[(1)] Trennt die innere Kante zwei Gebiete, so werden diese beiden
  Gebiete vereinigt: $e'=e, k'=k-1, f'=f-1, Z'=Z$.
\item [(2)] ``H�ngt'' die innere Kante nur noch an einer Ecke und ragt
  in ein Gebiet hinein, so wird beim Entfernen der Kante die Anzahl
  der Gebiete nicht ge�ndert. Die Kante verschwindet zusammen mit dem
  an ihr ``h�ngenden'' Eckpunkt: $e'=e-1,k'=k-1,f'=f, Z'=Z$.
\end{itemize}
Schlie�lich bleibt nur noch ein einziges Gebiet �brig, das von $n$
Ecken und dann logischerweise auch $n$ Kanten begrenzt wird, die einen
Ring bilden $e=k=n, f=Z=1$. $\Box$\medskip

\section*{Klassifikation der regul�ren Polyeder im Raum}

Als {\em regul�res Polyeder} bezeichnet man ein Polyeder, dessen
Seitenfl�chen alle zueinander kongruente regelm��ige Vielecke
($q$-Ecke) sind \ul{und} in jeder Ecke die gleich Anzahl $p$ von
Kanten zusammenst��t.

$p\cdot e$ ist die Summe der von den Ecken ausgehenden Kanten. Dabei
haben wir jede Kante doppelt gez�hlt, denn jede Kante hat zwei
Endpunkte: $2\,k = p\cdot e$. Genauso bekommen wir $2\,k = q\cdot f$,
wenn wir die Anzahlen der Kanten aller Fl�chen aufsummieren, da jede
Kante an zwei Fl�chen angrenzt. Zusammen mit der Eulerformel $e-k+f=2$
begrenzt das die M�glichkeiten, denn es muss gelten (warum?):
\[\frac{1}{p} + \frac{1}{q} =  \frac12 +\frac{1}{k}\]

Diese Gleichung hat genau die folgenden f�nf L�sungen (warum?):
\begin{center}
  \begin{tabular}{|cc|ccc|l|}\hline
    $p$ & $q$ & $e$ & $k$ & $f$ & Name des Polyeders \\\hline
    3 & 3 &  4 &  6 &  4 & Tetraeder\\
    3 & 4 &  8 & 12 &  6 & W�rfel (Hexaeder)\\
    4 & 3 &  6 & 12 &  8 & Oktaeder\\
    3 & 5 & 20 & 30 & 12 & Dodekaeder\\
    5 & 3 & 12 & 30 & 20 & Ikosaeder\\\hline
  \end{tabular}
\end{center}

\begin{attribution}
graebe (2005-07-24):\\ Dieses Material wurde f�r die Projektarbeit in Klasse
11/12 f�rs Mathelager 2005 in Ilmenau erstellt und dort auch eingesetzt. 
\end{attribution}

\end{document}