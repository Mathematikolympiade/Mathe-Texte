\documentclass[11pt,a4paper]{article}
\usepackage{kosemnet,ko-math,ngerman,url}
\usepackage[utf8]{inputenc}

\title{Zahlentheorie\kosemnetlicensemark}
\author{Lisa Sauermann}
\date{März 2013}
\begin{document}
\maketitle

Hier sollen einige grundlegende Lösungsmethoden für Zahlentheorieaufgaben bei
Olympiaden und anderen Wettbewerben vermittelt werden.

\subsection*{Der Chinesische Restsatz}
\begin{satz}
Seien $m_{1}$, $m_{2}$, \dots , $m_{k}$ teilerfremde positive ganze Zahlen und
$M=m_{1}\cdot m_{2}\cdots m_{k}$ deren Produkt.  

Für ganze Zahlen $n_{1}$, $n_{2}$, \dots , $n_{k}$ gibt es genau eine ganze
Zahl $z$ mit $0\le z<M$ mit
\begin{align*}
z&\equiv n_{1}\pmod{m_{1}}\\ z&\equiv n_{2}\pmod{m_{2}}\\ 
z&\equiv n_{k}\pmod{m_{k}}
\end{align*}
\end{satz}
\begin{beweis}
Wir zeigen zuerst, dass es mindestens ein solches $z$ gibt. Es ist
$M_1=m_{2}\cdot m_{3}\cdots m_{k}$ zu $m_{1}$ teilerfremd. Wir wissen, dass
$m_{2}\cdot m_{3}\cdots m_{k}$ damit ein Inverses modulo $m_{1}$ hat,
d.\,h.\ es gibt eine ganze Zahl $a_{1}$ mit $a_{1}\cdot M_1\equiv
1\pmod{m_{1}}$.

Analog ist $M_i=\frac{M}{m_{i}}$ eine positive ganze Zahl und es gibt eine
ganze Zahl $a_{i}$ mit $a_{i}\cdot M_i\equiv 1\pmod{m_{i}}$, $i=2,\dots,k$.

Nun wähle $z$ mit $0\le z<M$ so, dass
\begin{gather*}
  z\equiv n_{1}\cdot a_{1}\cdot M_1 +n_{2}\cdot a_{2}\cdot M_2 +\dots
  +n_{k}\cdot a_{k}\cdot M_k \pmod{M}
\end{gather*}
gilt.  Da $M_j$ für $j\neq i$ durch $m_i$ teilbar ist, ergibt sich $M_j\equiv
0\pmod{m_i}$ für $j\neq i$ und damit 
\begin{gather*}
  z\equiv n_{i}\cdot a_{i}\cdot M_i \equiv n_{i}\cdot 1\pmod{m_{i}}\,.
\end{gather*}
Damit erfüllt dieses $z$ die Bedingungen.

Angenommen, es gäbe $z>z'$, die beide die Bedingungen erfüllen. Dann gilt
$z-z'\equiv n_{i}-n_{i}\equiv 0\pmod{m_{i}}$ für alle $1\leq i\leq k$. Weil
die $m_{i}$ paarweise teilerfremd sind, folgt daraus $z-z'\equiv 0
\pmod{M}$. Das ist aber ein Widerspruch zu $1\leq z-z'<M$.

Damit gibt es genau ein solches $z$.
\end{beweis}

Der Chinesische Restsatz ist sehr wichtig und nützlich. Mit ihm allein kann
man die folgende IMO-Aufgabe lösen.

\paragraph{Aufgabe 1 (5. Aufgabe der IMO 1989)} 
Für welche positiven ganzen Zahlen $n$ gibt es eine positive ganze Zahl $N$,
sodass keine der Zahlen $1+N$, $2+N$, \dots , $n+N$ eine Potenz einer Primzahl
ist?

\subsection*{Die $\varphi$- und die $\tau$-Funktion}

Für eine positive ganze Zahl $n$ sei mit $\varphi(n)$ die Anzahl der zu $n$
teilerfremden positiven ganzen Zahlen $\leq n$ bezeichnet. Beispielsweise gilt
$\varphi(1)=1$, $\varphi(3)=2$ und $\varphi(6)=2$.

Für eine Primzahlpotenz $p^{m}$ (mit $p$ prim, $m\geq 1$ ganz) gilt
$\varphi(p^{m})=p^{m}-p^{m-1}$. Denn es sind genau die Zahlen zu $p^{m}$
teilerfremd, die zu $p$ teilerfremd sind und es gibt genau
$p^{m}\cdot\frac{p-1}{p}=p^{m}-p^{m-1}$ nicht durch $p$ teilbare positive
ganze Zahlen $\leq p^{m}$.

Für zueinander teilerfremde positive ganze Zahlen $m$, $n$ gilt
$\varphi(m\cdot n)=\varphi(m)\cdot\varphi(n)$. Denn nach dem Chinesischen
Restsatz gibt es für jedes Paar von Resten modulo $m$ und $n$ genau einen
entsprechenden Rest modulo $mn$. Damit gibt es für jedes der
$\varphi(m)\cdot\varphi(n)$ Paare eines jeweils teilerfremden Restes modulo
$m$ und $n$ genau einen entsprechenden Rest modulo $mn$. Dies sind aber genau
die $\varphi(m\cdot n)$ zu $m\cdot n$ teilerfremden Reste.

Sei nun allgemein $n=p_{1}^{a_{1}}p_{2}^{a_{2}}\cdots p_{k}^{a_{k}}$
($a_{1},\dots ,a_{k}\geq 1$) eine beliebige natürliche Zahl in ihrer
Primfaktorenzerlegung. Dann gilt 
\begin{align*}
  \varphi(n)&= \varphi(p_{1}^{a_{1}})\cdot \varphi(p_{2}^{a_{2}})
  \cdot\ldots\cdot \varphi(p_{k}^{a_{k}})\\
  &=\br{p_{1}^{a_{1}}-p_{1}^{a_{1}-1}}\cdot
  \br{p_{2}^{a_{2}}-p_{2}^{a_{2}-1}}\cdot\ldots\cdot
  \br{p_{k}^{a_{k}}-p_{k}^{a_{k}-1}}\\
  &=n\,\br{1-\frac{1}{p_{1}}}\cdot \br{1-\frac{1}{p_{2}}} \cdot\ldots\cdot
  \br{1-\frac{1}{p_{k}}}\,. 
\end{align*}
Im Satz von Euler-Fermat wird die $\varphi$-Funktion noch eine wichtige Rolle
spielen. Nach ihrem Entdecker wird sie oft auch Eulersche $\varphi$-Funktion
genannt.

Für eine positive ganze Zahl $n$ sei mit $\tau(n)$ die Anzahl der positiven
Teiler von $n$ bezeichnet. Beispielsweise gilt $\tau(1)=1$, $\tau(3)=2$ und
$\tau(6)=4$.

Sei nun allgemein $n=p_{1}^{a_{1}}p_{2}^{a_{2}}\cdots p_{k}^{a_{k}}$ eine
beliebige natürliche Zahl in ihrer Primfaktorenzerlegung. Dann hat jeder
Teiler von $n$ die Form $p_{1}^{b_{1}}p_{2}^{b_{2}}\cdots p_{k}^{b_{k}}$ mit
$b_{i}\leq a_{i}$ für $1\leq i\leq k$. Damit gibt es für $b_{i}$ jeweils
$a_{i}+1$ Möglichkeiten (nämlich $0,1,2,\dots a_{i}$). Insgesamt beträgt die
Anzahl der Teiler von $n$ also 
\[\tau(n)=(a_{1}+1)(a_{2}+1)\cdots(a_{k}+1)\,.\]

Diese sehr nützliche Formel erschlägt die folgende Aufgabe sofort. Nicht ganz
so leicht ist dagegen Aufgabe 3.

\paragraph{Aufgabe 2} 
Gibt es eine dreistellige Zahl mit genau 11 Teilern (im Dezimalsystem)? 

\paragraph{Aufgabe 3 (431046)} 
Beweise: Bei jeder natürlichen Zahl $n$, bei der $n+1$ durch 24 teilbar ist,
ist auch die Summe aller Teiler von $n$ durch 24 teilbar.

\subsection*{Der Satz von Euler-Fermat}

\begin{satz}
Sei $m$ eine positive ganze Zahl und $a$ eine zu ihr teilerfremde ganze Zahl.
Dann gilt \[a^{\varphi(m)}\equiv 1\pmod{m}\,.\]
\end{satz}
\begin{beweis}
Es seien $x_{1}=1, x_{2}, \dots ,x_{\varphi(m)}$ die zu $m$ teilerfremden
ganzen Zahlen von 1 bis $m$. Nun sind auch $a\,x_{1}, a\,x_{2}, \dots ,
a\,x_{\varphi(m)}$ zu $m$ teilerfremd. Wegen $x_{j}\not\equiv x_{i}\pmod{m}$
gilt $a\,x_{j}\not\equiv a\,x_{i}\pmod{m}$ für $1\leq i<j\leq \varphi(m)$.
Damit haben die zu $m$ teilerfremden Zahlen $a\,x_{1}, a\,x_{2}, \dots ,
a\,x_{\varphi(m)}$ paarweise verschiedene Reste. Es gibt aber nur die
$\varphi(m)$ zu $m$ teilerfremden Reste $x_{1},x_{2}, \dots ,
x_{\varphi(m)}$. Damit sind die Reste von $a\,x_{1}, a\,x_{2}, \dots ,
a\,x_{\varphi(m)}$ und $x_{1},x_{2}, \dots ,x_{\varphi(m)}$ bei Teilung duch
$m$ bis auf die Reihenfolge gleich. Also gilt
\begin{gather*}
  a^{\varphi(m)}\cdot x_{1}\cdot x_{2}\cdot\ldots\cdot x_{\varphi(m)}\equiv
  a\,x_{1}\cdot a\,x_{2}\cdots a\,x_{\varphi(m)}\equiv x_{1}\cdot x_{2}
  \cdot\ldots\cdot x_{\varphi(m)}\pmod{m}\,.
\end{gather*}
Daraus folgt $a^{\varphi(m)}\equiv 1\pmod{m}$, da $x_{1}\cdot x_{2}\cdots
x_{\varphi(m)}$ teilerfremd zu $m$ ist und deshalb auf beiden Seiten der
Kongruenz gekürzt werden kann. 
\end{beweis}

Ein bekannter Spezialfall des Satzes von Euler-Fermat ist der kleine Satz von
Fermat: Für eine Primzahl $p$ und eine nicht durch diese teilbare Zahl $a$
gilt $a^{p-1}\equiv 1\pmod{p}$.

\subsection*{Die Identität von Sophie Germain}

Es gilt für beliebige ganze Zahlen $a$, $b$
\begin{gather*}
  a^{4}+4\,b^{4} =\br{a^{2}+2b^{2}}^{2}-4\,a^{2}\,b^{2}
  =\br{a^{2}-2\,a\,b+2\,b^{2}}\br{a^{2}+2\,a\,b+2\,b^{2}}\,.
\end{gather*}

\paragraph{Aufgabe 4 (380942)} 
In einem Zahlensystem mit der Basis $a$ ($a\in\N$, $a>1$) sei die Zahl
$z=100\,000\,004$ gegeben. Beweise, dass es keine natürliche Zahl $a$ gibt,
für die $z$ eine Primzahl ist.

\subsection*{Potenzreste}

\paragraph{Aufgabe 5 (400941)} 
\begin{itemize}
\item [(a)] Beweise: Für jede natürliche Zahl ist ihr Quadrat entweder von der
  Form $4k$ oder von der Form $8\,k+1$, wobei jeweils $k$ eine natürliche Zahl
  ist.
\item [(b)] Gibt es eine $n$-stellige Quadratzahl mit $n>1$, die aus lauter
  gleichen Ziffern besteht? Beweise deine Antwort.
\end{itemize}

Diese recht einfache Aufgabe regt dazu an, sich mit den möglichen Resten von
Quadratzahlen oder höheren Potenzen bei der Teilung durch bestimmte Zahlen zu
befassen.

Durch Ausprobieren für alle Restklassen können wir uns leicht überlegen,
welche Reste Quadratzahlen modulo 4 und 8 lassen können. Das Ergebnis ist
durchaus interessant und wurde ja schon in Teil (a) der obigen Aufgabe
verraten.

Auch die möglichen Reste von Quadratzahlen modulo 3, 9, 5 und 7 sollte man als
Olympaideteilnehmer immer im Kopf haben. Bei vielen Aufgaben kann man aus den
möglichen Resten modulo einer geschickt gewählten Primzahl nämlich bedeutende
Schlüsse ziehen. Die folgenden Aufgaben sollen dies verdeutlichen.

\paragraph{Aufgabe 6 (420944)} 
Ermittle alle diejenigen Tripel $(a,b,c)$ ganzer Zahlen, für die die Gleichung
$2\,a^{2}+b^{2}=5\,c^{2}$ gilt.

\paragraph{Aufgabe 7} 
Ermittle alle diejenigen Paare $(a,b)$ ganzer Zahlen, für die die Gleichung
$a^{5}=b^{2}+4$ gilt.

\paragraph{Aufgabe 8  (N1 2002)} 
Was ist die kleinste positive ganze Zahl $t$, sodass es ganze Zahlen $x_{1},
x_{2},\dots,x_{t}$ mit $x_{1}^{3}+x_{2}^{3}+\dots+x_{t}^{3}=2002^{2002}$ gibt?

\paragraph{Aufgabe 9  (1. Aufgabe der IMO 1986)} 
Die Menge $S=\cbr{2,5,13}$ hat die Eigenschaft, dass für alle $a,b\in S, a\neq
b$ die Zahl $a\,b-1$ keine Quadratzahl ist. Zeige, dass für jede nicht in $S$
enthaltene positve ganze Zahl $d$, die Menge $S\cup \cbr{d}$ diese Eigenschaft
nicht hat.

\paragraph{Aufgabe 10  (N1 2000)} 
Finde alle positiven ganzen Zahlen $n\geq 2$, die folgende Bedingung erfüllen:
Für alle $a$, $b$ teilerfremd zu $n$ gilt
\[a\equiv b\pmod{n}\quad \text{genau dann wenn}\quad a\,b\equiv 1\pmod{n}\,.\] 

\subsection*{Einschachtelung von Quadratzahlen}

Im vorhergehenden Abschnitt haben wir gesehen, wie man durch Betrachtung von
Restklassen ausschließen kann, dass eine bestimmte Zahl eine Quadratzahl ist.
Eine andere Methode ist die Betrachtung der Vielfachheiten ihrer Primfaktoren
in der Primfaktorenzerlegung.

Eine dritte sehr wichtige und oft benötigte Strategie zum Beweis, dass eine
Zahl keine Quadratzahl ist, ist die folgende: Wir zeigen, dass unsere Zahl
zwischen zwei aufeinanderfolgenden Quadratzahlen liegt. Damit kann sie selbst
keine Quadratzahl sein. Oft muss man etwas rechnen, damit die Einschachtelung
gelingt, aber der Aufwand lohnt sich.

Hier eine zwei Beispielaufgaben

\paragraph{Aufgabe 11 (480846)} 
Ermittle alle geordneten Paare $(a,b)$ positiver ganzer Zahlen, für die
$a^{2}+3\,b$ und $b^2+3\,a$ Quadratzahlen sind.

\paragraph{Aufgabe 12 (471346)} 
Man bestimme alle reellen Zahlen $x$, für welche die beiden Zahlen
$4\,x^{5}-7$ und $4\,x^{13}-7$ gleichzeitig Quadrate ganzer Zahlen sind.

\subsection*{Dezimalsystembasteleien}

Bei einigen Aufgaben wird besonderes Augenmerk auf das Dezimalsystem gelegt.
Bei den folgenden drei Aufgaben muss man geschickt Vielfache, deren
Dezimaldarstellungen bestimmte Eigenschaften haben, konstruieren. Ein
wichtiger Trick ist dabei der folgende: Für jede zu 10 teilerfremde natürliche
Zahl $n$ gibt es eine Zehnerpotenz $10^{m}$, die Rest 1 bei Teilung durch $n$
lässt (nach Euler-Fermat). Nun ist die Zahl
$1+10^{m}+10^{2m}+\dots+10^{(n-1)m}$ durch $n$ teilbar und hat eine sehr
praktische Dezimaldarstellung. Man kann sie nämlich mit irgendeiner
$m$-stelligen Zahl multiplizieren und erhält deren Dezimaldarstellung $n$ mal
hintereinander. Oft kann man dadurch die gewünschten Eigenschaften
erfüllen. Ist die Ausgangszahl $n$ durch 2 oder 5 teilbar, muss man dies durch
geschickte Wahl von $m$ berücksichtigen.

\paragraph{Aufgabe 13  (BWM 2009 1. Runde)} 
Eine positive ganze Zahl heiße Dezimal-Palindrom, wenn ihre Dezimaldarstellung
$z_{n}\dots z_{0}$ mit $z_{0}\neq 0$ spiegelsymmetrisch ist, d.\,h.\ wenn
$z_{k}=z_{n-k}$ für alle $k=0,\dots,n$ gilt. Zeige, dass jede nicht durch 10
teilbare ganze Zahl ein positives Vielfaches besitzt, das ein
Dezimal-Palindrom ist.

\paragraph{Aufgabe 14} 
Man finde alle positiven ganzen Zahlen $k$ mit folgender Eigenschaft: Jede
Zahl, die durch Umkehren der Dezimaldarstellung eines Vielfachen von $k$
entsteht, ist ebenfalls ein Vielfaches von $k$.

\paragraph{Aufgabe 15  (6. Aufgabe der IMO 2004)} 
Wir nennen eine positive ganze Zahl alternierend, wenn ihre Dezimaldarstellung
abwechselnd gerade und ungerade Ziffern hat. Finde alle positiven ganzen
Zahlen, die ein alternierendes Vielfaches besitzen.

\subsection*{Weitere Aufgaben}

\paragraph{Aufgabe 16 (410943)} 
Bestimme den größten gemeinsamen Teiler aller Zahlen der Form $n^{4}-4\,n^{2}$
mit $n\in\cbr{1,2,3,\dots}$, welche auf Null enden.

\paragraph{Aufgabe 17 (441044)} 
Man bestimme alle natürlichen Zahlen $n>1$ mit folgender Eigenschaft: Für
jeden Teiler $d>1$ von $n$ ist die Zahl $d-1$ ein Teiler von $n-1$.

\begin{attribution}
sauermann (März 2013): Für KoSemNet freigegeben.

graebe (2014-01-01): Nach den KoSemNet Regeln aufbereitet.
\end{attribution}
\end{document}
