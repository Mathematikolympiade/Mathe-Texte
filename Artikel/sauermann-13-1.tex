\documentclass[11pt,a4paper]{article}
\usepackage{kosemnet,ko-math,ngerman,url}
\usepackage[utf8]{inputenc}

\title{Hilfsmittel bei Geometrieaufgaben.\\ Ein Kompendium für Klasse
  8\kosemnetlicensemark} 
\author{Lisa Sauermann}
\date{März 2013}

\begin{document}
\maketitle

Geometrie ist ein wichtiges Gebiet bei der Olympiade, das neben viel
Kreativität und einem geübtem Auge auch einige theoretische Kenntnisse
erfordert. Im Folgenden sollen deshalb die wichtigsten grundlegenden
geometrischen Sachverhalte sowie einige Beweisstrategien zusammengefasst
werden. Es sind dabei auch schwierigere Resultate erwähnt, vielleicht weckt
dies das Interesse an der selbstständigen Beschäftigung mit diesen. Dazu sei
beispielsweise das Geometriekapitel des Buches „Ein-Blick in die Mathematik“
(Richard Bamler, Christian Reiher et al., Aulis Verlag Deubner) empfohlen.

\section{Beweismethoden und Aufgabentypen}

\subsection{Aufgabentypen}

\begin{itemize}
\item Bei allen Aufgabentypen: Man darf Lagebeziehungen nicht einfach
  voraussetzen, sondern muss sie gegebenenfalls in einer Lagebetrachtung
  begründen! Manchmal sind auch Fallunterscheidungen notwendig.
\item Bestimmungsaufgaben einer Länge oder eines Winkels: Aus den
  Voraussetzungen ist die gesuchte Größe abzuleiten.
\item Bestimmungsaufgaben aller möglichen Größen einer Länge oder eines
  Winkels: Es müssen alle möglichen Werte angegeben werden, gezeigt werden,
  dass es keine weiteren gibt, und nachgewiesen werden, dass diese Werte
  tatsächlich möglich sind.
\item Aussagen vom Typ „Wenn, dann“: Aus den Voraussetzungen muss die
  Behauptung geschlussfolgert werden.
\item Aussagen vom Typ „Genau dann, wenn“: Es sind zwei Richtungen zu zeigen.
\item Aufgaben vom Typ „Bestimme den geometrischen Ort“: Es ist eine Aussage
  vom Typ „Genau dann, wenn“ (2 Richtungen!) zu zeigen. Die Behauptung muss
  man aber selbst noch finden. Dazu helfen mehrere Skizzen, um eine Vermutung
  abzuleiten.
\item Konstruktionsaufgaben (im Allgemeinen mit Zirkel und Lineal, wenn nicht
  anders vermerkt):
\begin{itemize}
\item Konstruktionsbeschreibung;
\item Angabe, wann sich bei der Konstruktion wie viele gesuchte Objekte
  ergeben;
\item Nachweis, dass es neben den konstruierten keine weiteren Objekte geben
  kann, die die Bedingungen erfüllen;
\item Nachweis, dass die konstruierten Objekte tatsächlich die Bedingungen
  erfüllen;
\item eventuell Konstruktionszeichnung (Durchführung der Konstruktion an einem
  Beispiel).
\end{itemize}
Absolute Grundkonstruktionen dürfen vorausgesetzt werden (Konstruktion von
Mittelsenkrechten, Winkelhalbierenden, Senkrechte auf Gerade durch Punkt,
Parallele zu Gerade durch Punkt)
\end{itemize}

\subsection{Beweismethoden}

\begin{itemize}
\item Geometrische Örter: Einige geometrische Objekte lassen sich als
  geometrische Örter interpretieren. Beispielsweise ist die Mittelsenkrechte
  einer Strecke $\ksegment{AB}$ der geometrische Ort aller Punkte der Ebene,
  die von den Punkten $A$ und $B$ den gleichen Abstand haben. Die
  Innenwinkelhalbierende und die Außenwinkelhalbierende eines Winkels
  $\kangle{BAC}$ bilden den geometrischen Ort aller Punkte der Ebene, die von
  den Geraden $\kline{AB}$ und $\kline{AC}$ den gleichen Abstand haben.
\item „Winkeljagd“: So lange alle Winkel ausrechnen (d.\,h.\ durch andere
  Winkel, wie beispielsweise die Innenwinkel eines Dreiecks, ausdrücken), bis
  die Behauptung gezeigt ist.
\item Herumrechnen mit Streckenlängen: beispielsweise im Zusammenhang mit den
  Sätzen von Ceva und Menelaos.
\item Suchen von Sehnenvierecken: Sehnenvierecke spielen in vielen Aufagben
  den entscheidenden Schritt bei der Lösungsfindung. Auch bei der „Winkeljagd“
  sind sie hilfreiche Geschütze.
\item Suchen von ähnlichen oder kongruenten Dreiecken: Aus Aussagen über
  Strecken ergeben sich so Aussagen über Winkel und umgekehrt.
\item Betrachten von Drehungen, zentrischen Streckungen oder Drehstreckungen:
  geht ein Teil der Figur durch eine Drehung, zentrischen Streckung oder
  Drehstreckung (oder seltener auch Spiegelung) in einen anderen Teil über,
  lassen sich meist wertvolle Aussagen ableiten.
\end{itemize}

\subsection{Herangehensweisen beim Finden von Beweisen}

\begin{itemize}
\item korrekte Skizze!
\item Wenn man nicht weiter kommt: Neue Skizze machen!
\item Wie kann man Voraussetzungen benutzen? Was würde reichen, um die
  Behauptung zu zeigen?
\item Aufstellen von Zwischenbehauptungen, diesbezügliche Vermutungen anhand
  von Skizze(n) finden und überprüfen.
\item Alle Erkenntnisse in Skizze deutlich machen (z.\,B.\ gleich große Winkel
  mit gleicher Farbe markieren), dafür Bunststifte zu Klausur mitbringen!
\item Neu eingeführte Punkte sinnvoll benennen.
\end{itemize}

\section{Sätze und Sachverhalte}

\subsection{Allgemeines}

\begin{itemize}
\item Nebenwinkelsatz, an parallelen Geraden Stufenwinkelsatz und
  Wechselwinkelsatz
\item Kongruenzsätze sss, sws, wsw und SsW
\item Ähnlichkeitssätze www, sss, sws und SsW (s steht hier für gleiche
  Seitenverhältnisse)
\item Strahlensätze: Sind $\kline{AB}$ und $\kline{A'B'}$ parallele Geraden
  und $P$ der Schnittpunkt der Geraden $\kline{AA'}$ und $\kline{BB'}$, so
  gilt:
\begin{itemize}
\item erster Strahlensatz:
  \begin{gather*}
    \frac{\msegment{PA}}{\msegment{PA'}}
    =\frac{\msegment{PB}}{\msegment{PB'}}\,.
  \end{gather*}
\item zweiter Strahlensatz:
  \begin{gather*}
    \frac{\msegment{PA}}{\msegment{PA'}}
    =\frac{\msegment{AB}}{\msegment{A'B'}}\,.
  \end{gather*}
\item dritter Strahlensatz: Liegt außerdem noch $C$ auf der Geraden
  $\kline{AB}$, $C'$ auf der Geraden $\kline{A'B'}$ und $P$ ebenfalls auf der
  Geraden $\kline{CC'}$, so gilt
  \begin{gather*}    
    \frac{\msegment{AB}}{\msegment{A'B'}}
    =\frac{\msegment{BC}}{\msegment{B'C'}}\,. 
  \end{gather*}
\end{itemize}
Umkehrbar sind nur der erste und der zweite Strahlensatz.
\end{itemize}

\subsection{Am Kreis}

\begin{itemize}
\item Ein konvexes Viereck $ABCD$ ist genau dann ein Sehnenviereck
  (d.\,h.\ alle vier Ecken liegen auf einem gemeinsamen Kreis), wenn
  $\mangle{CBA} + \mangle{ADC}=180\grad$ gilt.
\item Peripheriewinkelsatz: Liegen die vier Punkte $A$, $B$, $C$ und $D$ in
  dieser Reihenfolge auf einem Kreis, so gilt $\mangle{ACB}=\mangle{ADB}$.
\item Zentri-Peripheriewinkelsatz: Liegen die drei Punkte $A$, $B$ und $C$ auf
  einem Kreis um den Punkt $O$, so gilt $2\cdot\mangle{ACB}=\mangle{AOB}$.
\item Sehnentangentenwinkelsatz: Liegen die drei Punkte $A$, $B$ und $C$ auf
  einem Kreis und der Punkt $P$ auf der Tangenten an diesen Kreis in $A$,
  wobei $P$ und $C$ auf verschiedenen Seiten der Geraden $AB$ liegen, so gilt
  $\mangle{ACB}=\mangle{PAB}$.
\item Tangentenabschnittssatz: Berühren die beiden Tangenten von einem Punkt
  $P$ an einen Kreis diesen in den Punkten $A$ und $B$, so gilt
  $\msegment{PA}=\msegment{PB}$.
\item Sehnen-Satz: Liegen die vier Punkte $A$, $B$, $C$ und $D$ auf einem
  Kreis und der Schnittpunkt $P$ der Geraden $\kline{AB}$ und $\kline{CD}$
  innerhalb dieses Kreises, so gilt $\msegment{PA}\cdot \msegment{PB}
  =\msegment{PC}\cdot \msegment{PD}$.
\item Sekanten-Satz: Liegen die vier Punkte $A$, $B$, $C$ und $D$ auf einem
  Kreis und der Schnittpunkt $P$ der Geraden $\kline{AB}$ und $\kline{CD}$
  außerhalb dieses Kreises, so gilt $\msegment{PA}\cdot \msegment{PC}
  =\msegment{PB}\cdot \msegment{PD}$.
\item Sehnen-Sekanten-Satz: Liegen die vier Punkte $A$, $B$, $C$ und $D$ auf
  einem Kreis und ist $P$ der Schnittpunkt der Geraden $\kline{AB}$ und
  $\kline{CD}$, so gilt $\msegment{PA}\cdot \msegment{PC} =\msegment{PB}\cdot
  \msegment{PD}$.
\end{itemize}

\subsection{Dreiecksgeometrie}

Über Höhenschnittpunkt und Umkreismittelpunkt:
\begin{itemize}
\item Die drei Höhen, Seitenhalbierenden, Mittelsenkrechten und
  Winkelhalbierenden schneiden sich jeweils in einem Punkt (Höhenschnittpunkt,
  Schwerpunkt, Umkreismittelpunkt bzw.\ Inkreismittelpunkt).
\item Die Spiegelbilder vom Höhenschnittpunkt an den Dreiecksseiten liegen auf
  dem Umkreis.
\item Das Spiegelbild vom Höhenschnittpunkt eines Dreiecks $\ktriangle*{ABC}$
  am Mittelpunkt der Strecke $\ksegment{BC}$ liegt auf dem Umkreis und bildet
  verbunden mit dem Punkt $A$ einen Durchmesser des Umkreises.
\item Das Spiegelbild der Höhe an der Winkelhalbierenden der gleichen Ecke
  verläuft durch den Umkreismittelpunkt.
\item Bei zentrischer Streckung am Schwerpunkt mit Streckfaktor $-\frac{1}{2}$
  geht das Dreieck in sein Seitenmittendreieck über. Der Höhenschnittpunkt
  geht dabei in den Umkreismittelpunkt über.
\item Eulergerade: Höhenschnittpunkt, Schwerpunkt und Umkreismittelpunkt
  liegen auf einer Geraden.
\item Feuerbachkreis: Die drei Seitenmittelpunkte, die drei Höhenfußpunkte und
  die drei Mittelpunkte der Verbindungsstrecken des Höhenschnittpunktes mit
  den Ecken liegen auf einem Kreis. Der Mittelpunkt dieses Kreises liegt
  ebenfalls auf der Eulergeraden.
\end{itemize}

Über die Winkelhalbierenden:
\begin{itemize}
\item Winkelhalbierende und gegenüberliegende Mittelsenkrechte schneiden sich
  auf dem Umkreis.
\item Es seien $a$, $b$, $c$ die Seitenlängen und $s=\frac{a+b+c}{2}$ der
  Halbumfang eines Dreiecks. Dann betragen die Längen der Tangentenabschnitte
  von den Ecken an den Inkreis $s-a$, $s-b$ bzw.\ $s-c$. Die
  Tangentenabschnitte jeder Ecke an den gegenüberliegenden Ankreis haben die
  Länge $s$.
\item Ist $r$ der Inkreisradius und $s$ der Halbumfang eines Dreiecks, dann
  beträgt dessen Fläche $r\cdot s$.
\item Inkreis- und Ankreisberührpunkt an eine Dreiecksseite liegen bezüglich
  des Mittelpunkts dieser Dreiecksseite gespiegelt.
\item Innenwinkelhalbierende und Außenwinkelhalbierende stehen aufeinander
  senkrecht.
\item Der Inkreismittelpunkt ist der Höhenschnittpunkt des
  Ankreisberührpunktdreiecks.
\item Der Höhenschnittpunkt ist der Inkreismittelpunkt der
  Höhenfußpunktdreiecks.
\item Satz von der Winkelhalbierenden: Wenn die Winkelhalbierende durch $A$ in
  einem Dreieck $\ktriangle*{ABC}$ die Seite $\ksegment{BC}$ im Punkt $D$
  schneidet, so gilt
  \begin{gather*}
    \frac{\msegment{BD}}{\msegment{DC}}=\frac{\msegment{AB}}{\msegment{AC}}\,.
  \end{gather*}
\item Die drei Verbindungsstrecken der Ecken mit den jeweils
  gegenüberliegenden Inkreisberührpunkten schneiden sich in einem Punkt, dem
  Gergonne-Punkt.
\item Die drei Verbindungsstrecken der Ecken mit den jeweils
  gegenüberliegenden Ankreisberührpunkten schneiden sich in einem Punkt, dem
  Nagelschen Punkt.
\item Heron-Formel: Die Dreiecksfläche beträgt
  $\sqrt{s\,(s-a)\,(s-b)\,(s-c)}$.
\end{itemize}

\subsection{Die Sätze von Ceva und Menelaos}

Ist $P$ ein Punkt auf einer Gerade $\kline{AB}$, so können wir das
Teilungsverhältnis $t(A,B;P)$ von $P$ bezüglich der Strecke $\ksegment{AB}$
definieren. Ist $P$ ein innerer Punkt der Strecke $\ksegment{AB}$, so beträgt
dieses wie gewohnt einfach
\begin{gather*}
  t(A,B;P)=\frac{\msegment{AP}}{\msegment{PB}}\,.
\end{gather*}
Liegt der Punkt $P$ außerhalb der Strecke $\ksegment{AB}$, so definieren wir
das Teilungsverhältnis als 
\begin{gather*}
  t(A,B;P)=-\frac{\msegment{AP}}{\msegment{PB}}\,.
\end{gather*}

\begin{satz}[Satz von Ceva] 
  Es seien $\ktriangle*{ABC}$ ein Dreieck und $D$, $E$ und $F$ von dessen
  Eckpunkten verschiedene Punkte auf den Geraden $\kline{BC}$, $\kline{CA}$
  bzw.\ $\kline{AB}$. 

  Die Geraden $\kline{AD}$, $\kline{BE}$ und $\kline{CF}$ schneiden sich genau
  dann in einem Punkt, wenn 
  \begin{gather*}
    t(B,C;D)\cdot t(C,A;E)\cdot t(A,B;F) =1
  \end{gather*}
  gilt.
\end{satz}
\begin{satz}[Satz von Menelaos] 
  Es seien $\ktriangle*{ABC}$ ein Dreieck und $D$, $E$ und $F$ von dessen
  Eckpunkten verschiedene Punkte auf den Geraden $\kline{BC}$, $\kline{CA}$
  bzw.\ $\kline{AB}$.  

  Die Punkte $D$, $E$ und $F$ liegen nun genau dann auf einer Geraden, wenn
  \begin{gather*}
    t(B,C;D)\cdot t(C,A;E)\cdot t(A,B;F) =-1
  \end{gather*}
  gilt.
\end{satz}

\begin{attribution}
sauermann (März 2013): Für KoSemNet freigegeben.

graebe (2014-01-01): Nach den KoSemNet Regeln aufbereitet.
\end{attribution}
\end{document}
