\documentclass[10pt]{article}  
\usepackage{kosemnet,ko-math,ngerman} 
\usepackage[utf8]{inputenc}    

\author{Dr. Wolfgang Moldenhauer (Bad Berka), Carsten Moldenhauer (Dresden)}
\title{Die {\glqq}bestmöglichen{\grqq} Dreiecke\kosemnetlicensemark}
\date{}

\begin{document} 
\maketitle 

Zur Lösung einer Geometrieaufgabe {\glqq}Gegeben sei ein Dreieck $ABC$
\ldots{\grqq} fertigt man zumeist eine Skizze an. Das Dreieck wird gezeichnet.
Doch es wird gleichseitig. Die nächste Skizze zeigt ein gleichschenkliges. Ein
weiterer Versuch ergibt ein rechtwinkliges -- wieder ein Spezialfall. Es wirkt
Murphys Gesetz: Wenn etwas schief gehen kann, dann wird es auch schief gehen.
(If anything can go wrong, it will.)

Aber: Welches ist denn nun das {\glqq}beste{\grqq} Dreieck?  Die Suche nach
diesem bestmöglichen nicht-speziellen Dreieck basiert auf dem Grundsatz von
G.~Polya: {\glqq}Die Figur darf nicht eine unangebrachte Spezialisierung nahe
legen{\grqq} ([1, S.~108]).

Es seien $\alpha,\, \beta,\, \gamma$ die Größen der Innenwinkel eines
spitzwinkligen Dreiecks mit o.\,B.\,d.\,A. $90\grad>\alpha>\beta>\gamma$.
Dann misst $90\grad-\alpha$ die Differenz zu einem rechtwinkligen und
$\alpha-\beta$ bzw.\ $\beta-\gamma$ die Differenz zu einem gleichschenkligen
bzw.\ gleichseitigen Dreieck. Es sei $\delta = \min(90\grad-\alpha,\,
\alpha-\beta,\, \beta-\gamma)$. Da $\delta$ den kleinsten Abstand zu den
Spezialfällen (rechtwinklig, gleichschenklig, gleichseitig) misst,
unterscheidet sich das Dreieck mit dem größten $\delta$ dann am meisten von
den Spezialfällen.

Nun gilt für das gewichtete arithmetische Mittel der Differenzen
$90\grad-\alpha,\, \alpha-\beta,\, \beta-\gamma$ die Beziehung
\begin{gather*}
  \frac{3\,\br{90\grad-\alpha} + 2\,\br{\alpha-\beta} + \br{\beta-\gamma}}{6}
  = \frac{270\grad-\br{\alpha+\beta+\gamma}}{6} = 15\grad\,.
\end{gather*}
Ist $\alpha=75\grad,\, \beta=60\grad,\, \gamma=45\grad$, so gilt
$\delta=15\grad$.  Gilt aber nicht $\alpha=75\grad,\, \beta=60\grad,\,
\gamma=45\grad$, so ist eine der drei genannten Differenzen nach dem
Schubfachschluss kleiner als $15\grad$.

Mithin:
\begin{satz}
  Das bestmögliche nicht-spezielle spitzwinklige Dreieck hat die Innenwinkel
  $\alpha=75\grad,\, \beta=60\grad,\, \gamma=45\grad$ und es ist
  $\delta=15\grad$.
\end{satz}

Jetzt seien $\alpha,\, \beta,\, \gamma$ die Größen der Innenwinkel eines
stumpfwinkligen Dreiecks mit o.\,B.\,d.\,A. $\alpha>90\grad>\beta>\gamma$.
Das Minimum der Differenzen $\alpha-90\grad,\, 90\grad-\beta,\,
\beta-\gamma,\, \gamma-0\grad$ (sie messen wieder die Abweichungen von den
Spezialfällen.)  muss wieder möglichst groß sein. Mit
$\alpha=180\grad-\beta-\gamma$ ist $\alpha-90\grad = 90\grad-\beta-\gamma <
90\grad-\beta$, so dass die Differenz $90\grad-\beta$ nicht weiter
einzubeziehen ist.

Für das gewichtete arithmetische Mitte der Differenzen $\alpha-90\grad,\,
\beta-\gamma,\, \gamma-0\grad$ gilt
\begin{gather*}
  \frac{\br{\alpha-90\grad}+\br{\beta-\gamma}+2\,\br{\gamma-0\grad}}{4} =
  \frac{\alpha+\beta+\gamma-90\grad}{4}=22,5\grad\,. 
\end{gather*}
Für $\alpha=112,5\grad,\, \beta=45\grad,\, \gamma=22,5\grad$ ist
$\delta=22,5\grad$.  Gilt aber nicht $\alpha=112,5\grad,\, \beta=45\grad,\,
\gamma=22,5\grad$, so ist eine der drei genannten Differenzen nach dem
Schubfachschluss kleiner als $22,5\grad$.

Also gilt:
\begin{satz}
  Das bestmögliche nicht-spezielle stumpfwinklige Dreieck hat die Innenwinkel
  $\alpha=112,5\grad$, $\beta=45\grad,\, \gamma=22,5\grad$ und es ist
  $\delta=22,5\grad$.
\end{satz}

In [2] wird $\alpha=80\grad,\, \beta=60\grad,\, \gamma=40\grad$ mit dem
Abstand $\delta=10\grad$ zu den Spezialfällen (gleichseitig, rechtwinklig und
gleichschenklig) und für stumpfwinklige Dreiecke $\alpha=108\grad,\,
\beta=54\grad,\, \gamma=18\grad$ mit $\delta=18\grad$ angegeben. Diese
angegebenen Werte führen nicht auf das beste $\delta$.

\subsection*{Literatur:}
\begin{itemize}
\item[{[1]}] Polya, George: Schule des Denkens. A.~Franke Verlag, Tübingen und
  Basel 1995.
\item[{[2]}] Hendriks, Björn, Schöbel, Konrad: Immer Ärger mit den Dreiecken\
  \ldots. Wurzel 9+10/02, S.~226 -- 229. 
\end{itemize}


\begin{attribution}
moldenhauer (2006-07-20): Text für KoSemNet freigegeben. 

graebe (2006-08-10): Umsetzung in \LaTeX\ für das KoSemNet-Projekt.
\end{attribution}

\end{document}
