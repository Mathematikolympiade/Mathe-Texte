\documentclass[11pt,a4paper]{article}
\usepackage{a4wide,ngerman,schueler,url,enumitem}
\usepackage{kosemnet,ko-math}

\setlist{noitemsep}
\newcommand{\cas}[1]{\textsc{#1}}

\title{Rekursive Folgen (Supplement)\kosemnetlicensemark} 
\author{Hans-Gert Gräbe, Leipzig}
\date{14. Februar 2024}

\parindent0pt
\parskip3pt

\begin{document}
\maketitle

Ergänzungen zu \cite{b-schueler}.

\section{Rekursive Folgen}

In \cite{b-schueler} werden Aufgaben vorgestellt, die auf die Arbeit mit
rekursiven Folgen zurückgeführt werden können und insbesondere die Theorie der
\emph{homogenen linearen Rekursionen mit konstanten Koeffizienten} (HLR)
genauer besprochen.  Dies sind Rekursionen der Form
\begin{align*}\label{e-linrek}
a_{n+k}=c_1 a_{n+k-1} +c_2 a_{n+k-2} +\cdots + c_k a_n,\quad n\in \N.\tag{H}
\end{align*}
mit konstanten Koeffizienten $c_i$. 

\subsection{Rekursive Folgen mit CAS berechnen}

Viele Abzählaufgaben lassen sich über Rekursionsbeziehungen lösen.

\begin{aufgabe}
Es sei $a_n$ die Anzahl der Möglichkeiten, ein Rechteck vom Format $2\times n$
in Dominosteine vom Format $2\times 1$ zu zerlegen.

Man bestimme $a_n$.
\end{aufgabe}

\begin{loesung}
  Die Anzahl $a_n$ erfüllt $a_1=1, a_2=2$ und die Rekursion
  $a_n=a_{n-1}+a_{n-2}$ für $n>2$, denn am rechten Ende kann entweder ein
  Dominostein hochkant oder zwei Dominosteine horizontal liegen. 
\end{loesung}

Diese Folge ist im Wesentlichen die \emph{Fibonacci-Folge}. Genauer: Der Index
ist gegenüber der Fibonacci-Folge um 1 verschoben.

Bei der Analyse von Folgen ist es oft wichtig, sich einen Überblick über ein
Anfangsstück der Folgenglieder zu verschaffen, um angemessene Hypothesen zu
formulieren oder andere zu verwerfen.  Für derartige Rechnungen kann man
Computer-Algebra-Systeme (CAS) oder andere Pakete zum symbolischen
Rechnen\footnote{Etwa das Python-Paket \emph{SymPy},
\url{https://www.sympy.org/en/index.html} } verwenden, die hinreichend gut
programmierbar sind.  Rechnungen werden im Weiteren mit dem freien CAS
\cas{Maxima}\footnote{\url{https://maxima.sourceforge.io/}} ausgeführt.

Bei der Analyse ist eine Besonderheit rekursiver Funktionen zu beachten, die
am gerade betrachteten Beispiel demonstriert werden soll.  Anfangsstücke
dieser Folge können etwa wie folgt mit \cas{Maxima} berechnet werden:
\begin{code}
  a(n):=if n=1 then 1\+\\
  else if n=2 then 2\\
  else a(n-1)+a(n-2);\-\\[4pt]
  makelist(a(n),n,1,10);
\end{code}
\begin{gather*}
  [1, 2, 3, 5, 8, 13, 21, 34, 55, 89]
\end{gather*}
Allerdings gibt es bereits Probleme, Werte wie etwa $a(26)$ zu berechnen, da
die Berechnung sehr lange dauert.  Wir haben hier mehr noch das typische
Verhalten exponentiellen Wachstums -- in einem Anfangsbereich (hier bis etwa
$n=23$) laufen die Rechnungen sehr schnell ab, in einem sehr kleinen
Zwischenbereich (hier etwa für $n=24..28$) wächst die Rechenzeit rasch an, im
Bereich darüber hinaus sind die Rechnungen nicht mehr in sinnvoller Zeit
ausführbar.

Dies liegt hier darin begründet, dass bei der Berechnung von $a(n)$ die
Berechnung von $a(n-i)$ mehrfach aufgerufen wird und diese Anzahlen mit
wachsendem $i$ schnell steigen.  Ist $z_i$ die Zahl der Aufrufe von $a(n-i)$
bei der Berechnung von $a(n)$, so ist $z_i=z_{i-1}+z_{i-2}$ mit $z_0=z_1=1$,
denn $a(n-i)$ wird genau bei der Berechnung von $a(n-i+1)$ und $a(n-i+2)$
aufgerufen.  Die Folge $(z_i)$ (im Wesentlichen wieder die Fibonacci-Zahlen)
wächst exponentiell, wie wir später noch sehen werden.

Dieses Problem kann man umgehen, indem die einmal berechneten Funktionswerte
zwischengespeichert werden.  Im folgenden Code wird in der vierten Zeile der
berechnete Wert $a(n)$ in einem Array $A$ an der Stelle $A[n]$ gespeichert
(beachte hier, dass die Zuweisung $A[n]:...$ den berechneten Wert auch noch
zurückgibt) und in der dritten Zeile vorab geprüft, ob der Wert bereits
berechnet wurde. In anderen CAS (etwa \cas{Mathematica}) kann explizit
untersucht werden, ob die Variable $A[n]$ belegt ist, in \cas{Maxima} wird
hier mit \texttt{integerp} geprüft, ob $A[n]$ mit einer ganzen Zahl belegt
ist. 
\begin{code}
  a(n):=if n=1 then 1\+\\
  else if n=2 then 2\\
  else if integerp(A[n]) then A[n]\\
  else A[n]:a(n-1)+a(n-2);\-\\[4pt]
  a(200);
\end{code}
\begin{gather*}
  453973694165307953197296969697410619233826
\end{gather*}
Die bessere Rechenzeit wird also mit mehr Speicherplatz erkauft. Alternativ
kann man $a(n)$ auch iterativ über eine Schleife berechnen, in der die im
Weiteren benötigten Werte aus den vorherigen Berechnungen in
Schleifenvariablen zwischengespeichert werden.
\begin{code}
  a(n):=block([a1:1,a2:2,a3],\+\\
  for i:1 thru n-1 do (a3:a1+a2, a1:a2, a2:a3),\\
  a1);
\end{code}

\subsection{Zwei weitere Beispiele}

Oft schränken rekursive Regeln die Möglichkeiten so weit ein, dass sich eine
überschaubare Menge von Folgen oder auch nur eine einzige ergibt, die allen
Bedingungen genügen. 

\begin{aufgabe} (MO 591036)
  Für die Folge $(a_n)_{n\ge 1}$ positiver ganzer Zahlen gilt 
  \begin{itemize}
  \item[(1)] $a_m\le a_n$ für alle $m,n$ mit $0<m<n$.
  \item[(2)] $a_m\m a_n=a_{m\m n}$ für alle $m,n$ mit $0<m\le n$ sowie
  \item[(3)] $a_{59}=59$.
  \end{itemize}
  Zeigen Sie, dass dann stets $a_n=n$ erfüllt ist.
\end{aufgabe}

\begin{loesung}
  Wäre $a_k=a_n$ für $k<n$, dann wäre wegen (1) $a_k=a_{k+1}=\dots=a_n$.

  Wäre $a_k=a_{k+1}$ für ein $k$, so wäre $a_k=a_n$ für alle $n\ge k$.
  Anderenfalls gäbe es ein kleinstes $t>k$ mit $a_t>a_k$. Das führt aber zu
  folgendem Widerspruch:
  \begin{gather*}
    a_k^2=a_{t-1}^2\stackrel{(2)}{=}a_{t^2-2t+1}\ge a_{t^2-2t}=a_t\m
    a_{t-2}=a_t\m a_k
  \end{gather*}
  und damit $a_t\le a_k$.  Die Folge $(a_n)$ ist also streng monoton
  wachsend und damit $a_n\ge n$ für alle $n$. 

  Weiter ist $a_{1\m 59}=a_1\m a_{59}$ und damit $a_1=1$. Mit Induktion zeigt
  man, dass $a_{59^k}=59^k$ ist. Also muss auch dazwischen $a_n=n$ sein.
\end{loesung}

\paragraph{Die Thue-Morse-Folge}
\begin{aufgabe}
  Die Folge $t_0,t_1,t_2,\dots$ sei wie folgt definiert: $t_0=1$,
  $t_{2^k+j}=-t_j$ für $0\le j\le 2^k-1$ und $k=0,1,\dots$.

  Man zeige, dass die Folge nicht periodisch ist.
\end{aufgabe}
\begin{loesung}
  Für ein $k$ ergibt sich $t_j$ für alle $j\in\cbr{2^k,\dots,2^{k+1}-1}$ durch
  Negation der bisher berechneten Folgenglieder $t_j$ mit $j\in\cbr{0,2^k-1}$:
  \begin{center}
    \begin{tabular}{c|l} 
      $k=0$ & 1, -1\\
      $k=1$ & 1, -1, -1, 1\\
      $k=2$ & 1, -1, -1, 1, -1, 1, 1, -1\\
      $k=3$ & 1, -1, -1, 1, -1, 1, 1, -1, -1, 1, 1, -1, 1, -1, -1, 1\\
    \end{tabular}  
  \end{center}

  Wir sehen, dass insbesondere das Vorzeichen des letzten Elements in jeder
  der obigen Teilfolgen alterniert, d.h., dass $t_{2^m-1}=(-1)^m$ für $m\ge 0$
  gilt.  Dies ergibt sich in der Tat daraus, dass $2^m-1=2^{m-1}+j$ mit
  $j=2^{m-1}-1$ ist und folglich nach der Bildungsvorschrift
  $t_{2^m-1}=-t_{2^{m-1}-1}$ gilt.

  Wäre nun die Folge periodisch, dann gäbe es Zahlen $l,p>0$, so dass
  $t_{k}=t_{k+p}$ für alle $k\ge l$ gilt.  Für alle $m$ mit $2^m>l,p$ gilt
  dann $t_{2^m-1}=t_{2^m+(p-1)}=-t_{p-1}$ und damit
  $t_{2^m-1}=t_{2^{m+1}-1}=-t_{p-1}$.  Dies widerspricht der gerade gezeigten
  Eigenschaft.
\end{loesung}

Es handelt sich um eine Variante der \emph{Thue-Morse-Folge}, eine Folge mit
vielen spannenden Eigenschaften. So gilt zum Beispiel
\begin{itemize}
\item[(1)] $t_n=(-1)^{v(n)}$, wobei $v(n)$ die Anzahl der Einsen in der
  Binärdarstellung von $n$ ist.
\item[(2)] $t_{2n}=t_n, t_{2n+1}=-t_n$.
\item[(3)] Die Folge ist selbstähnlich: Streicht man alle Folgenglieder mit
  ungeradem Index, so erhält man die Ausgangsfolge zurück.
\item[(4)] Ist $X_k=\cbr{0\le n<2^{k+1}\ :\ t_n=1}$ und $Y_k= \cbr{0\le
  n<2^{k+1}\ :\ t_n=-1}$, so gilt 
  \begin{gather*}
    \sum_{n\in X_k}{f(n)}=\sum_{n\in Y_k}{f(n)}
  \end{gather*}
  für jedes Polynom vom Grad $\le k$.

  Für $k=2$ gilt insbesondere $0+3+5+6=1+2+4+7(=14)$ und
  $0^2+3^2+5^2+6^2=1^2+2^2+4^2+7^2(=70)$.  
\end{itemize}
(1) folgt unmittelbar daraus, dass dies für ein Anfangsstück $0\le j<2^k$ gilt
und $t_{2^k+j}=-t_j$ ist für $j<2^k$.  Die Binärdarstellungen der beiden Teile
unterscheiden sich nur dadurch, dass im Stück ab $2^k$ das $k$-te Bit gleich 1
ist.

(2) ergibt sich unmittelbar daraus, denn $2n$ entsteht durch Anhängen einer
Null an die Binärdarstellung von $n$, $2n+1$ durch das Anhängen einer Eins.

(3) folgt unmittelbar aus $t_{2n}=t_n$.

(4) kann mit Induktion nach $k$ gezeigt werden. Es ist nach der
Konstruktionsvorschrift  
\begin{gather*}
  X_{k+1}=X_k\cup\cbr{2^{k+1}+n\ :\ n\in Y_k}\ \text{und}\ 
  Y_{k+1}=Y_k\cup\cbr{2^{k+1}+n\ :\ n\in X_k}.
\end{gather*}
Ist $f(x)$ ein Polynom vom Grad $\le k+1$, so ist $h(x)=f\br{x+2^{k+1}}-f(x)$
als Differenz von $f(x)$ und einem Shift dieses Polynoms ein Polynom vom Grad
$\le k$. Es gilt deshalb nach Induktionsvoraussetzung
\begin{align*}
  \sum_{n\in X_k}{h(n)}&=\sum_{n\in Y_k}{h(n)}
  \intertext{und damit auch}
  \sum_{n\in X_k}{f\br{2^{k+1}+n}}-\sum_{n\in X_k}{f(n)}
  &=\sum_{n\in Y_k}{f\br{2^{k+1}+n}}-\sum_{n\in Y_k}{f(n)}\\
  \sum_{n\in Y_k}{f(n)}+\sum_{n\in X_k}{f\br{2^{k+1}+n}}
  &=\sum_{n\in X_k}{f(n)}+\sum_{n\in Y_k}{f\br{2^{k+1}+n}}\\
  \sum_{n\in Y_{k+1}}{f(n)}&=\sum_{n\in X_{k+1}}{f(n)}.\\
\end{align*}

Die \emph{Thue-Morse-Folge} ist als Folge
A010060\footnote{\url{https://oeis.org/A010060}} in der \emph{Online
Encyclopedia of Integer Sequences} (OEIS) zu finden. Mehr zu dieser Folge auch
unter
\begin{itemize}
\item \url{https://de.wikipedia.org/wiki/Morse-Folge}
\item \url{https://de.wikibrief.org/wiki/Thue-Morse_sequence}
\item \url{https://mathworld.wolfram.com/Thue-MorseSequence.html}
\item 
  \url{https://sites.math.washington.edu/~morrow/336_12/papers/christopher.pdf}.
\end{itemize}

%%%%%%%%%%%%%%%%%%%%%%%%%%%%%%%%%%%%%%%%%%%%%%%%%%%%%%%%
\section{Lösung der allgemeinen homogenen linearen Rekursion \rf[e-linrek]}

Es wird nun die in \cite{b-schueler} genauer ausgeführte Theorie des Lösens
von homogenen linearen Rekursionen \rf[e-linrek] kurz rekapituliert.  Beim
Lösen spielt das \emph{charakteristische Polynom}
\begin{gather*}
  p(x)=x^k-c_1 x^{k-1}-c_2 x^{k-2} - \dots -c_k
\end{gather*}
eine wichtige Rolle. Es ist ein Polynom vom Grad $k$, das nach dem
Fundamentalsatz der Algebra genau $k$ komplexe Nullstellen (gezählt mit ihrer
Vielfachheit) $q_1,q_2,\cdots,q_k$ besitzt.

Im einfachsten Fall sind diese $k$ Nullstellen alle voneinander verschieden.
Dann erfüllen die $k$ voneinander verschiedenen geometrischen Folgen
\begin{gather*}
(q_1^n),\, (q_2^n),\dots,(q_k^n)
\end{gather*}
alle die Rekursion \rf[e-linrek]. Damit ist dann auch die Folge
\begin{align}\label{e-allg}
x_n=A_1 q_1^n +A_2 q_2^n +\cdots  +A_k q_k^n, n\ge 0,
\end{align}
mit beliebigen Koeffizienten $A_1,\dots ,A_k$ eine Lösung der Rekursion.

Umgekehrt zeigt sich, dass für jede rekursive Folge $(x_n)$, die \rf[e-linrek]
genügt, eine solche Darstellung \rf[e-allg] existiert.  Das Bestimmen der
Koeffizienten $A_1,\dots,A_k$ führt dabei auf ein lineares Gleichungssystem
mit den Folgenwerten $x_0,\dots,x_{k-1}$ als rechten Seiten, das in diesem
Fall stets eindeutig lösbar ist.

Hat $p(x)$ Mehrfachnullstellen und ist
\begin{gather*}
  p(x)=\br{x-q_1}^{m_1}\m \dots \m \br{x-q_s}^{m_s}
\end{gather*}
die Zerlegung von $p(x)$ in Linearfaktoren, so bilden die $k$ Folgen
\begin{gather*}
  \br{n^iq_j^n}, i=0,\dots, m_j-1, j=1,\dots,s
\end{gather*}
eine Basis des Lösungsraums von \rf[e-linrek], d.h. für jede Folge $(x_n)$,
die \rf[e-linrek] genügt, gibt es (eindeutig bestimmte) Polynome $A_1(n)$
vom Grad $m_1-1$, $A_2(n)$ vom Grad $m_2-1$, \dots, $A_s(n)$ vom Grad $m_s-1$
mit
\begin{gather*}
  a_n=A_1(n)q_1^n+A_2(n)q_2^n+\dots+A_s(n)q_s^n.
\end{gather*}
Ein genauer Beweis dieser Zusammenhänge erfordert fortgeschrittenere Methoden
der linearen Algebra sowie der Theorie der Polynome und ist deshalb auch in
\cite{b-schueler} nur skizziert. 

Um Beweise in Olympiadeaufgaben vollständig zu führen, kann man aber eine nach
diesen Regeln gefundene explizite Darstellung durch (verallgemeinerte)
vollständige Induktion beweisen. Dazu zeigt man als Induktionsanfang, dass die
Beziehung für die ersten $k$ Werte gilt und schließt weiter im
Induktionsschritt von der Gültigkeit der Formel für alle $k<n$ auf die
Gültigkeit der Formel für $n$.

\subsection*{Beispiele:}

\paragraph{(1)}
Die Fibonacci-Folge $a_{n+2}=a_{n+1} +a_n$, $a_1=a_2=1$ hat das
charakteristische Polynom $p(x)=x^2-x-1$ mit den beiden rellen Nullstellen
$q_{1,2} =\half \pm \half \sqrt{5}$. Die allgemeine Lösung lautet also
\begin{gather*}
a_n=A \br{\frac{1+\sqrt{5}}{2}}^n +B \br{\frac{1-\sqrt{5}}{2}}^n.
\end{gather*}
Beachtet man $a_0=0$, $a_1=1$, so hat man
\begin{gather*}
0=A+B,\quad 1=\frac12(A+B) +\frac{\sqrt{5}}{2}(A-B).
\end{gather*}
Dies liefert $A=-B=\frac{1}{\sqrt{5}}$, also
\begin{gather*}
  a_n=\frac{1}{\sqrt{5}}\br{\br{\frac{1+\sqrt{5}}{2}}^n
    -\br{\frac{1-\sqrt{5}}{2}}^n}.
\end{gather*}
Diese Formel wird auch als \emph{Formel von Binet} bezeichnet. 

\paragraph{(2)}
$a_n=4a_{n-1}-3a_{n-2}$ für $n\ge 2$ und $a_0=1, a_1=2$.

Das charakteristische Polynom ist $p(x)=x^2-4x+3=(x-1)(x-3)$, was auf den
Ansatz $a_n=A\m 1^n+B\m 3^n$ führt. Aus den Gleichungen $a_0=1=A+B$ und
$a_1=2=A+3B$ bestimmt man $A=B=\frac12$, was zur expliziten Formel
\begin{gather*}
  a_n=\frac12\m\br{3^n+1}
\end{gather*}
führt.

\paragraph{(3)}
$a_n=a_{n-1}+6a_{n-2}$ für $n\ge 2$ und $a_0=4, a_1=3$.

Das charakteristische Polynom ist $p(x)=x^2-x-6=(x-3)(x+2)$, was auf den Ansatz
$a_n=A\m 3^n+B\m (-2)^n$ führt. Aus den Gleichungen $a_0=4=A+B$ und
$a_1=3=3A-2B$ bestimmt man $A=\frac{11}{5}$ und $B=\frac95$, was zur expliziten
Formel
\begin{gather*}
  a_n=\frac{11}{5}\m 3^n+\frac95\m (-2)^n
\end{gather*}
führt.

\paragraph{(4)}
$a_n=6a_{n-1}-9a_{n-2}$ für $n\ge 2$ und $a_0=2, a_1=5$.

Das charakteristische Polynom ist $p(x)=x^2-6x+9=(x-3)^2$, was auf den Ansatz
$a_n=(A+B\m n)\m 3^n$ führt. Aus den Gleichungen $a_0=2=A$ und $a_1=5=3(A+B)$
bestimmt man $A=2, B=-\frac13$, was zur expliziten Formel
\begin{gather*}
  a_n=\br{2-\frac{n}{3}}\m 3^n
\end{gather*}
führt.

\paragraph{(5)}
$a_n=2a_{n-1}-2a_{n-2}$ für $n\ge 2$ und $a_0=3, a_1=1$.

Das charakteristische Polynom ist $p(x)=x^2-2x+2$, das die (komplexen)
Nullstellen $q_1=1+\ii$ und $q_2=1-\ii$ hat. Dies führt auf den Ansatz
$a_n=A\m(1+\ii)^n+B\m(1-\ii)^n$. Aus den Gleichungen $a_0=3=A+B$ und
$a_1=1=(A+B)+(A-B)\ii$ bestimmt man nacheinander $(A-B)\ii=-2$, $A-B=2\ii$ und
schließlich $A=\frac32+\ii, B=\frac32-\ii$, was zur expliziten Formel
\begin{gather*}
  a_n=\br{\frac32+\ii}(1+\ii)^n+\br{\frac32-\ii}(1-\ii)^n
\end{gather*}
führt.  Da die Folge $(a_n)$ ganzzahlig ist, kürzen sich für jedes konkrete
$n$ die entsprechenden Imaginärteile der beiden Summanden gegenseitig weg.

Wir sehen am letzten Beispiel, dass die explizite Formel kompliziert
werden kann und für Fragen einer weitergehenden Analyse nicht in jedem Fall
brauchbar ist. 

\begin{aufgabe} (MO 281224)
  Es ist $x_1=y_1=2023$ sowie $x_n=2x_{n-1}-1$, $y_n=2y_{n-1}-2^n$ für $n>1$.
  Untersuchen Sie für beide Folgen, ob alle Folgenglieder positiv sind.
\end{aufgabe}

\begin{loesung}
  Die erste Folge kann mit $z_n=x_n-1$ auf die Potenzfolge $z_{n+1}=2z_n$
  zurückgeführt werden. Es ist $z_n=2^{n-1}z_1$ und damit
  $x_n=2^{n-1}\br{x_1-1}+1$ immer positiv.  Wir hätten auch die ersten Glieder
  mit \cas{Maxima} mit einem variablen Startwert $a$ berechnen können
  \begin{code}
    x(n):=if n=1 then a\+\\
    else expand(2*x(n-1)-1);\-\\[4pt]
    makelist(x(n),n,1,10);
  \end{code}
  \begin{gather*}
    [a, 2 a - 1, 4 a - 3, 8 a - 7, 16 a - 15, 32 a - 31, 64 a - 63, 128 a -
      127, 256 a - 255, 512 a - 511],
  \end{gather*}
dann die Formel $x_n=a\m 2^{n-1}-2^{n-1}+1=(a-1)\m 2^{n-1}+1$ vermuten und
diese mit vollständiger Induktion beweisen können.

Die zweite Folge scheint ebenfalls schnell zu wachsen, wenn wir die ersten
Folgenglieder berechnen.  \cas{Maxima} liefert für die ersten Gleider mit
einem variablen Startwert~$a$
\begin{code}
  x(n):=if n=1 then a\+\\
  else expand(2*x(n-1)-2\pw n);\-\\[4pt]
  makelist(x(n),n,1,8);
\end{code}
\begin{gather*}
  [a, 2 a - 4, 4 a - 16, 8 a - 48, 16 a - 128, 32 a - 320, 64 a - 768, 128 a -
    1792]
\end{gather*}
Wir vermuten $y_n=a\m 2^{n-1}-(n-1)\m 2^n=(a-2(n-1))\m 2^{n-1}$, denn es ist
\begin{code}
  makelist((n-1)*2\pw n,n,1,10);
\end{code}
\begin{gather*}
  [0, 4, 16, 48, 128, 320, 768, 1792, 4096, 9216]
\end{gather*}
Wir können wieder mit Induktion prüfen, dass die Formel korrekt ist:
\begin{align*}
  y_{n+1}&=2y_n-2^{n+1}=2(a-2(n-1))\m2^{n-1}-2^{n+1}\\
  &=(a-2(n-1)-2)\m2^n =(a-2n)\m 2^n.
\end{align*}
Für den Startwert $a=2023$ ist also $y_n$ für $n>1013$ negativ. 
\end{loesung}

Die gegebenen Rekursionsbeziehungen sind beide nicht homogen, so dass die
allgemeine Theorie nicht unmittelbar angewendet werden kann.  Beide Folgen
lassen sich auch als HLR anschreiben.

Für $\br{x_n}$ ergibt sich aus
\begin{gather*}
  x_n-2x_{n-1}=x_{n-1}-2x_{n-2}(=-1)
\end{gather*}
für $n>2$ die HLR $x_n=3x_{n-1}-2x_{n-2}$ der Ordnung 2.  Für die eindeutige
Auflösung dieser Rekursion muss noch $x_2$ aus der ursprünglichen Rekursion
bestimmt werden.  Wir erhalten $x_2=4045$. Das charakteristische Polynom ist
$p(x)=x^2-3x+2=(x-1)(x-2)$ mit den Nullstellen $q_1=1$ und $q_2=2$. Dies führt
auf den Ansatz $x_n=A\m1^n+B\m2^n$. $A$ und $B$ lassen sich aus den
Gleichungen $x_1=A+2B=2023$, $x_2=A+4B=4045$ bestimmen.  Es ergibt sich $A=1,
B=1011$ und $x_n=1011\m2^n+1$ wie bereits oben berechnet.

Für $\br{y_n}$ ergibt sich aus
\begin{gather*}
  y_n-2y_{n-1}=2\br{y_{n-1}-2y_{n-2}}(=-2^n)
\end{gather*}
die HLR $y_n=4y_{n-1}-4y_{n-2}$ mit dem charakteristischen Polynom
$p(x)=x^2-4x+4=(x-2)^2$.  Dies führt auf den Lösungsansatz $y_n=(A+B\m n)2^n$,
wobei zur Bestimmung der Koeffizienten $A$ und $B$ noch $y_2=2y_1-4=4042$
benötigt wird.  Es ergibt sich 
\begin{gather*}
  y_n=\br{\frac{2025}{2}-n}\m2^n =(2025-2n)\m2^{n-1}
\end{gather*}
in Übereinstimmung mit dem früher berechneten Ergebnis.

\section{Aufgaben aus \cite{b-schueler}}
\begin{aufgabe}
(Landesseminar Sachsen 2005, Klausur Klasse 9/10)\\ Bestimme die Einerziffer
  und die erste Ziffer nach dem Komma von $b_{2005}$, wenn
\begin{quote}
(a) $ b_n=(2+\sqrt{3})^{n}$.\\
(b) $ b_n=(3+\sqrt{7})^{n}$.
\end{quote}
\end{aufgabe}

\begin{loesung}
  Beide Ausdrücke lassen sich jeweils zu einer HLR der Ordnung 2 ergänzen,
  deren Folgenglieder ganzzahlig sind.

  (a) Betrachte dazu die Folge $a_n=(2+\sqrt{3})^{n}+(2-\sqrt{3})^{n}$, die
  eine Kombination aus den Nullstellen $q_{1,2}=2\pm\sqrt3$ ist und so auf das
  charakteristsche Polynom $p(x)=(x-2)^2-3=x^2-4x+1$ und die Rekursion
  $a_n=4a_{n-1}-a_{n-2}$ mit $a_0=2, a_1=4$ führt.  Die letzte Ziffer von
  $a_n$ ergibt sich als Rest modulo 10, also $a_n=(2,4,4,2,4,4,\dots
  \mod{10})$.  Die Reste wiederholen sich periodisch mit der Periodenlänge 3.
  Wegen $2023\equiv 1\mod{3}$ ist also $a_{2023}\equiv a_1=4\mod{10}$. Wegen
  $0<(2-\sqrt{3})<1$ ist weiter $a_{2023}=\dots4$ und
  $b_{2023}=\dots3{,}9999\dots$.

Die letzten $k$ Ziffern von $a_n$ lassen sich mit \cas{Maxima} auch direkt
bestimmen, wenn alle Zwischenergebnisse bereits $\mod{10^k}$ reduziert werden.
Wie früher erläutert, sind die Rechnungen mit einer rekursiven Implementierung
nicht erfolgreich, deshalb wird hier eine iterative Implementierung der
Berechnung von $a_n\mod{10^k}$ angegeben.
\begin{code}
a(n,k):=block([u:2,v:4,w],\+\\
  thru n do (w:mod(4*v-u,10\pw k), u:v, v:w),\\
  u);
\end{code}
Die letzten 60 Stellen von $a_{2023}$ können so leicht berechnet werden.
\begin{code}
  a(2023,60);
\end{code}
\begin{gather*}
  a_{2023}=\dots544477607903430234502941226114843894762534706639567206625764
\end{gather*}

(b) Betrachte die Folge $a_n=(2+\sqrt{5})^{n}+(2-\sqrt{5})^{n}$.  Dies führt
wie oben auf die Nullstellen $q_{1,2}=2\pm\sqrt5$, weiter auf das
charakteristische Polynom $p(x)=(x-2)^2-5=x^2-4x-1$ und die Rekursion
$a_n=4a_{n-1}+a_{n-2}$, $a_0=2, a_1=4$.  Für die letzte Ziffer erhalten wir
$a_n=(2,4,8,6,2,4,8,\dots \mod{10})$, die Periodenlänge ist 4. Dann ist
$a_{2023}\equiv a_3=6\mod{10}$, also $a_{2023}=\dots6$ und
$a_{2023}<b_{2023}=\dots6{,}000\dots$.  Beachte dabei $0>(2-\sqrt{5})>-1$.
Wie oben lassen sich die letzten 60 Stellen von $a_{2023}$ bestimmen:
\begin{gather*}
  a_{2023}=\dots974562703306449804829226825960632014186192343935889477077276
\end{gather*}

\emph{Bemerkung:} Da als Reste modulo $10 $ nur die fünf geraden Zahlen
$0,2,4,6,8$ in Frage kommen, gibt es maximal $5^2=25 $ Paare $(r_1,r_2)$ von
Resten modulo $10$, die hier auftreten. Erstaunlicherweise treten alle Paare
tatsächlich auf bis auf $(0,0)$, welches ein Orbit für sich ist.
\end{loesung}

\begin{aufgabe} (Engel, S. 209)
Man finde eine explizite Formel für
\begin{gather*}
a_{n+1}= \frac{1}{16} \left( 1+4 a_n + \sqrt{1 +24 a_n}\right), \quad a_1=1.
\end{gather*}
\end{aufgabe}

\begin{loesung} 
Wir bestimmen zunächst Zahlenwerte, etwa mit folgendem \cas{Maxima}-Code:
\begin{code}
  a(n):=if n=1 then 1\\
  else (1/16*(1+4*a(n-1)+sqrt(1+24*a(n-1))));\\[6pt]

  makelist(a(n),n,1,10);
\end{code}
\begin{gather*}
  \left[1, \frac{5}{8}, \frac{15}{32}, \frac{51}{128}, \frac{187}{512},
    \frac{715}{2048}, \frac{2795}{8192}, \frac{11051}{32768},
    \frac{43947}{131072}, \frac{175275}{524288}\right]
\end{gather*}
Wir erhalten nur rationale Zahlen, also steht unter der Wurzel aus
irgendwelchen Gründen immer ein vollständiges Quadrat einer rationalen Zahl.
Um diesem Phänomen auf den Grund zu gehen, betrachten wird die abgeleitete
Folge $b_n=\sqrt{1+24a_n}$ mit $b_1=5, b_2=4$ und $a_n=\frac{b_n^2-1}{24}$.
Setzt man dies in die Rekursion für $a_n$ ein, so hat man nacheinander
\begin{align*}
  a_{n+1}=\frac{b_{n+1}^2-1}{24}&=\frac{1}{16}\br{1+\frac{b_n^2-1}{6}+b_n}\\
  b_{n+1}^2-1&=\frac{3}{2}\m\frac{6+b_n^2-1+6b_n}{6}\\
  b_{n+1}^2&=\frac{1}{4}\br{b_n^2+6b_n+9}=\br{\frac{b_n+3}{2}}^2.
\end{align*}
Da alle Glieder positiv sind, folgt daraus $b_{n+1} = \frac12 b_n+\frac32.$

Die Rekursion ist von der Ordnung 1, aber keine HLR, die Theorie also nicht
unmittelbar anwendbar.  Wir versuchen den (nicht durch die Theorie gestützten)
Ansatz $b_n=A\m2^{-n}+B$. Aus den Startwerten $A=4, B=3$ ergibt sich die
Formel
\begin{gather*}
  b_n=4\br{\frac12}^n+3, n\ge 1,
\end{gather*}
die durch vollständige Induktion dann auch bewiesen werden kann.  Für $n=1$
prüft man die Gültigkeit unmittelbar, der Induktionsschritt ergibt sich aus
\begin{gather*}
  b_{n+1} = \frac12 b_n+\frac32=\frac12\m\br{4\br{\frac12}^n+3}+\frac32
  =4\m\br{\frac12}^{n+1}+2\m\frac32=4\m\br{\frac12}^{n+1}+3.
\end{gather*}
Man kann die Rekursion aber auch wieder in eine HLR verwandeln, indem die
Rekursionsbeziehungen für $n$ und für $n+1$ verwendet werden, um den
inhomogenen Teil zu eliminieren:
\begin{gather*}
  b_{n+2}-\frac12 b_{n+1}=b_{n+1}-\frac12 b_n,\\
  b_{n+2}=\frac32 b_{n+1}-\frac12 b_n,
  \intertext{was auf das charakteristische Polynom}
  p(x)=x^2-\frac32 x-\frac12=(x-1)\br{x-\frac12}
\end{gather*}
und so auf denselben Ansatz führt. 
\end{loesung}

%%% Bulg Aufg. 77, S. 16 %%%%%%%%%%%%%%%
\begin{aufgabe}\label{r-1} (BB, Aufgabe 77, S. 16)
  Gegeben sei die rekursive Folge $(a_n)$ mit
  \begin{gather*}
    a_1=a_2=1,\quad \quad a_{n}=\frac{a_{n-1}^2 +2}{a_{n-2}}, \quad n\ge 3.
  \end{gather*}
  Beweisen Sie, dass alle Folgenglieder ganzzahlig sind.
\end{aufgabe}

\begin{loesung}
Die Berechnung von einigen Folgengliedern ergibt
\begin{gather*}
  (a_n)=(1, 1, 3, 11, 41, 153, 571, 2131, 7953, 29681, \dots)
\end{gather*}
und legt die Vermutung nahe, dass dies in der Tat der Fall ist.

Wir formen die Rekursionsbeziehung zunächst so um, dass in der Formel kein
Quotient mehr enthalten ist:
\begin{gather*}
    a_na_{n-2}=a_{n-1}^2+1.
\end{gather*}
Diese Beziehung gilt für alle $n>3$. Wir betrachten neben dieser Beziehung für
den Index $n$ wieder dieselbe Beziehung für den Index $n+1$:
\begin{gather*}
  a_na_{n-2}=a_{n-1}^2+1,\\ a_{n+1}a_{n-1}=a_n^2+1.
\end{gather*}
Die Differenz ergibt $a_n\br{a_n+a_{n-2}}=a_{n-1}\br{a_{n+1}+a_{n-1}}$ und
weiter 
\begin{gather*}
  c_n=\frac{a_n+a_{n-2}}{a_{n-1}}=\frac{a_{n+1}+a_{n-1}}{a_n}=c_{n+1}.
\end{gather*}
Die Folge $(c_n)$ ist also konstant mit $c_n=4$ und $(a_n)$ genügt der
Rekursionsbeziehung 
\begin{gather*}
  a_{n+2}=4a_{n+1}-a_n.
\end{gather*}
Damit ist die Ganzzahligkeit der Folgenglieder bereits gezeigt.
\end{loesung}

Wir berechnen noch die explizite Darstellung.  Das charakteristische Polynom
$p(x)=x^2-4x+1$ hat die beiden Nullstellen $q_1=2+\sqrt{3}$ und
$q_2=2-\sqrt{3}=q_1^{-1}$.  Es gilt also $a_n=Aq_1^n+Bq_2^n$ für geeignete
$A,B$, die sich aus $a_1=a_2=1$ als $A=\frac{9-5\sqrt3}{6},
B=\frac{9+5\sqrt3}{6}$ bestimmen lassen.  Daraus ergibt sich die explizite
Formel
\begin{gather*}
  a_n=\frac{9-5\sqrt3}{6}\br{2+\sqrt3}^n +\dfrac{9+5\sqrt3}{6}\br{2-\sqrt3}^n.
\end{gather*}

\begin{aufgabe}\label{r-2}
  Gegeben sei eine Folge $(a_n)$ mit $a_1=A$, $a_2=B$ und
\begin{gather*}
  a_n=\frac{a_{n-1}^2 +C}{a_{n-2}},\quad n\ge 3.
\end{gather*}
Beweisen Sie, dass aus der Ganzzahligkeit von
\begin{gather*}
  A, \, B,\,\,\frac{A^2 +B^2 +C}{AB}
\end{gather*}
folgt, dass alle Folgenglieder $a_n$ ganzzahlig sind.
\end{aufgabe}

\begin{loesung}
  Diese Aufgabe verallgemeinert Aufgabe \ref{r-1}, die sich für $A=B=C=1$
  ergibt.

Wir setzen $D=\dfrac{A^2+B^2+C}{AB}$. Damit ist 
\begin{gather*}
  C=ABD-A^2-B^2\ \text{und}\ a_3=\frac{B^2+C}{A}=\frac{ABD-A^2}{A}=BD-A.
\end{gather*}
Aus der Rekursionsbeziehung für $n$ und $n+1$ erhalten wir
\begin{gather*}
  a_na_{n-2}=a_{n-1}^2+C,\ a_{n+1}a_{n-1}=a_n^2+C.
\end{gather*}
Die Differenz ergibt $a_n\br{a_n+a_{n-2}}=a_{n-1}\br{a_{n+1}+a_{n-1}}$ und
weiter 
\begin{gather*}
  c_n=\frac{a_n+a_{n-2}}{a_{n-1}}=\frac{a_{n+1}+a_{n-1}}{a_n}=c_{n+1}.
\end{gather*}
Die Folge $\br{c_n}$ ist also konstant und wir erhalten
$c_n=c_3=\dfrac{a_3+a_1}{a_2}=\dfrac{BD-A+A}{B}=D$.  $\br{a_n}$ genügt also
der Rekursion $a_{n+2}=Da_{n+1}-a_n$.
\end{loesung}

\begin{aufgabe} (MO 261242)
  Ermittle alle Zahlenfolgen mit $a_1=1$, $a_2=\frac52$, für die
  \begin{gather*}
    a_{m+n}\m a_{m-n}=a_m^2-a_n^2\ \text{für alle}\ m>n>0
  \end{gather*}
  gilt.
\end{aufgabe}
\begin{loesung}
  Für $m=1$ ergibt sich $a_{n+1}=\dfrac{a_n^2-1}{a_{n-1}}$, womit die
  Zahlenfolge zumindest eindeutig bestimmt ist, wenn sie existiert.

  Die Rekursionsbeziehung ist von ähnlicher Bauart wie die in Aufgabe
  \ref{r-2}, wobei die dortige Ableitung der linearen Rekursion nicht daran
  gebunden ist, dass $A,B,C$ ganzzahlig sind.  Hier ist $A=1, B=\frac52, C=-1$
  und damit $D=\dfrac{A^2+B^2+C}{AB}=\frac52$.  Die Folge $a_n$ erfüllt also
  die Rekursion $a_{n+2}=\frac52a_{n+1}-a_n$.  Das charakteristische Polynom
  ist $p(x)=x^2-\frac52x+1$ mit den Nullstellen $q_1=2, q_2=\frac12$.  Die
  explizite Bildungsvorschrift ergibt sich aus dem Ansatz $a_n=Aq_1^n+Bq_2^n$,
  für den man aus den Startwerten $A=\frac23, B=-\frac23$ ermittelt. Einziger
  Lösungskandidat ist also die Folge
  \begin{gather*}
    a_n=\frac23\br{2^n-\br{\frac12}^n}.
  \end{gather*}
  Die Gültigkeit der Beziehung
  \begin{gather*}
    a_{m+n}\m a_{m-n}=a_m^2-a_n^2\ \text{für alle}\ m>n>0
  \end{gather*}
  rechnet man leicht nach.
\end{loesung}


\begin{aufgabe} Gegeben sei die rekursive Folge $(a_n)_{n\in\N } $ mit
\begin{gather*}
a_1=a_2=a_3=1,\quad a_{n}=\frac{a_{n-1}a_{n-2} +1}{a_{n-3}}, \quad n\ge 4.
\end{gather*}
Beweisen Sie, dass alle Folgenglieder ganzzahlig sind.
\end{aufgabe}

Diese Aufgabe ist ähnlich zur Aufgabe 2.2 im BWM 2003 \cite{BWM2003.2}. 
\begin{quote}
  Zeigen Sie, dass alle Glieder der durch 
  \begin{gather*}
    a_1=1, a_2=1, a_3=2\ \text{sowie}\ a_{n+3}=\frac{a_{n+2}a_{n+1}+7}{a_n}\
    \text{für}\ n>0 
  \end{gather*}
  gegebenen Folge ganzzahlig sind.  
\end{quote}
Berechnet man numerisch die Quotienten aufeinanderfolgender Glieder eines
Anfangsstücks der Folge, so ergeben sich näherungsweise alternierend zwei
Werte, was die Vermutung nahe legt, dass -- wie in der dritten Lösung der
BWM-Aufgabe -- die Folge über zwei verschränkte Rekursionsbeziehungen für
gerade und ungerade Indizes berechnet werden kann.  Dies ist in der Tat so.

\begin{loesung}
Wir bilden wieder die Differenz der Rekursionsbeziehung für $n$ und $n+1$:
\begin{gather*}
  a_na_{n-3}=a_{n-1}a_{n-2}+1,\ a_{n+1}a_{n-2}=a_na_{n-1}+1,\\
  a_n\br{a_{n-1}+a_{n-3}}=a_{n-2}\br{a_{n+1}+a_{n-1}},\\
  c_{n+1}=\frac{a_{n+1}+a_{n-1}}{a_n}=\frac{a_{n-1}+a_{n-3}}{a_{n-2}}=c_{n-1}.
\end{gather*}
Im Gegensatz zur Aufgabe \ref{r-1} und \ref{r-2} ist hier also
\begin{gather*}
  c_3=c_5=\dots=\frac{a_3+a_1}{a_2}=2\ \text{und}\
  c_4=c_6=\dots=\frac{a_4+a_2}{a_3}=3, 
\end{gather*}
was die Rekursionen
\begin{gather*}
  a_{2n}=3a_{2n-1}-a_{2n-2}\ \text{und}\ a_{2n+1}=2a_{2n}-a_{2n-1}
\end{gather*}
liefert.  Daraus folgt zunächst die Ganzzahligkeit der Folgenglieder.

In der ersten Lösung der BWM-Aufgabe wird eine einfache Rekursion der Ordnung
4 angegeben, der die Folge ebenfalls genügt.  Das soll nun für die hier
gegebene Folge ebenfalls hergeleitet werden.  Es gilt
\begin{align*}
  a_{2n+2}&=3a_{2n+1}-a_{2n}, a_{2n}=3a_{2n-1}-a_{2n-2}\\
  a_{2n+2}+a_{2n}&=3\br{a_{2n+1}+a_{2n-1}}-a_{2n}-a_{2n-2}\\
  a_{2n+2}&=6a_{2n}-2a_{2n}-a_{2n-2}=4a_{2n}-a_{2n-2} 
\end{align*}
und ähnlich
\begin{align*}
  a_{2n+1}&=2a_{2n}-a_{2n-1}, a_{2n-1}=3a_{2n-2}-a_{2n-3}\\
  a_{2n+1}&=2a_{2n}-a_{2n-1}, a_{2n-1}=3a_{2n-2}-a_{2n-3}\\
  a_{2n+1}&=2\br{a_{2n}+a_{2n-2}}-2a_{2n-1}-a_{2n-3}\\
  &=6a_{2n-1}-2a_{2n-1}-a_{2n-3}=4a_{2n-1}-a_{2n-3}. 
\end{align*}
Die Folge genügt also der gemeinsamen Rekursion $a_{n+4}=4a_{n+2}-a_n$.
Daraus ergibt sich als explizite Formel der Ansatz 
\begin{gather*}
  a_n=Aq_1^n+B\br{-q_1}^n+Cq_2^n+D\br{-q_2}^n
\end{gather*}
mit $q_1=\sqrt{2+\sqrt{3}}$ und $q_2=\sqrt{2-\sqrt{3}}=q_1^{-1}$, da das
charakteristische Polynom $p(x)=x^4-4x^2+1$ die vier Nullstellen $\pm q_1, \pm
q_2$ hat.
\end{loesung}

%%% Bulgarisches Buch, Seite 16, Aufg. 76 %%%%
\begin{aufgabe} (BB, S. 16, Aufgabe 76)
Gegeben sei eine Folge $(a_n)_{n\ge 0}$ von ganzen Zahlen mit
\begin{gather*}
3 a_{n}= a_{n-1}  +a_{n+1}, \quad n\ge 1.
\end{gather*}
Beweisen Sie, dass $5a_n^2 +4( a_0^2 +a_1^2 -3a_0a_1) $ stets eine Quadratzahl
ist.
\end{aufgabe}

\begin{loesung}
  Symbolische Berechnung der ersten Werte der Folge
\begin{gather*}
  b_n=5a_n^2+4\br{a_0^2+a_1^2-3a_0a_1}
\end{gather*}
ergibt $b_n=c_n^2$ mit einer Folge $c_0=3a_0-2a_1$, $c_1=2a_0-3a_1=3a_1-2a_2$
und $c_{n+2}=3c_{n+1}-c_n$.  Das legt die Vermutung $c_n=3a_n-2a_{n+1}$ nahe.
Es bleibt zu zeigen, dass dies allgemein gilt.  Beweis mit Induktion und
$a_n=3a_{n+1}-a_{n+2}$, wobei der Induktionsanfang schon gezeigt wurde.
\begin{align*}
  b_{n+1}&=b_n+5a_{n+1}^2-5a_n^2=\br{3a_n-2a_{n+1}}^2+5a_{n+1}^2-5a_n^2\\
  &=\br{7a_{n+1}-3a_{n+2}}^2+5a_{n+1}^2-5\br{3a_{n+1}-a_{n+2}}^2\\
  &=9a_{n+1}^2-12a_{n+1}a_{n+2}+4a_{n+2}^2=\br{3a_{n+1}-2a_{n+2}}^2=c_{n+1}^2.
\end{align*}
\end{loesung}

\begin{aufgabe} 
%%% Bulg Aufgabe 79, S. 17 %%%%
Es seien $(x_n)$ und $(y_n)$ rekursive Folgen, gegeben durch
\begin{alignat*}{3}
x_1&=1,\,\,& x_2& = 1,& \,\, x_{n+2}& =x_{n+1} +2 x_{n} ,\quad n\in\N,\\
y_1&= 1,&\,\, y_2&= 7,&\,\, y_{n+2}& = 2y_{n+1} +3 y_{n},\quad n\in \N.
\end{alignat*}
Beweisen Sie, dass außer $x_1=y_1=1$ die beiden Folgen keine gemeinsamen
Folgenglieder besitzen.
\end{aufgabe}

\begin{loesung} Wir betrachten beide Folgen modulo $8$ und erhalten
\begin{align*}
x_n&\equiv 1,1,3,5,3,5,\cdots \mod{8},\\
y_n&\equiv  1,-1,1,-1,\cdots \mod{8}.
\end{align*}
Folglich können nur die ersten Folgenglieder übereinstimmen.
\end{loesung}

Man kann hier auch die expliziten Bildungsvorschriften bestimmen, was
allerdings in der gestellten Frage nicht weiterhilft. 

Die Folge $x_n$ hat $p(x)=x^2-x-2=(x-2)(x+1)$ als charakteristisches Polynom
und damit gilt $x_n=A\m2^n+B\m(-1)^n$.  Einsetzen der Startwerte liefert
$A=\frac13, B=-\frac13$ und somit $x_n=\frac13\br{2^n-(-1)^n}$.

Für die Folge $y_n$ ist $p(x)=x^2-2x-3=(x-3)(x+1)$ und damit
$y_n=A\m3^n+B\m(-1)^n$.  Einsetzen der Startwerte liefert $A=\frac23, B=1$ und
somit $y_n=3^{n-1}+(-1)^n$.

%%% Bulgar Aufg. 80 S. 17 %%%%
\begin{aufgabe} 
Es sei $f(x) $ ein Polynom mit ganzzahligen Koeffizienten und $f(0)=f(1)=1$ und
$a_1\in \Z $ beliebig gegeben.  Ferner sei $a_{n+1} =f(a_n)$ für alle $n\ge 1$.

Beweisen Sie, dass die Folgenglieder paarweise teilerfremd sind.
\end{aufgabe}

\begin{loesung}
  Für $f(x)=c_nx^n+\dots+c_1x+c_0$ ist $f(0)=c_0=1$ und
  $f(1)-c_0=c_n+c_{n-1}+\dots+c_1=0$.  Damit ist
\begin{align*}
  a_{k+1}&=f(a_k)=a_k\br{c_na_k^{n-1}+\dots+c_1}+1\\
  &=a_k\br{c_n\br{a_k^{n-1}-1}+\dots+c_1(1-1)}+1\\
  &=a_k\br{a_k-1}\m p_k+1
\end{align*}
und zunächst $\gcd\br{a_{k+1},a_k}=1$.  Mit Induktion zeigt man
$a_k-1=a_{k-1}\m\dots\m a_1\m q_k$.  In der Tat, setzt man das oben ein, so
ergibt sich 
\begin{gather*}
  a_{k+1}-1=a_k\br{a_k-1}\m p_k=a_k\m a_{k-1}\m\dots\m a_1\m q_k\m p_k
\end{gather*}
und so $q_{k+1}=q_k\m p_k$.  Damit sind die $a_k$ aber paarweise
teilerfremd. 
\end{loesung}

\subsection{Periodizität der Fibonacci-Zahlen modulo $m$}

\begin{aufgabe} Es sei $m\in \N $ eine fixierte natürliche Zahl.
  \begin{itemize}
  \item[(a)] Beweise, dass es eine \textsc{Fibonacci}-Zahl $a_n$ gibt, die
    durch $m$ teilbar ist.

  \item[(b)] Zeige, dass es eine \textsc{Fibonacci}-Zahl gibt, die auf $m$
    Neunen endet.
  \end{itemize}
\end{aufgabe}

\begin{beweis} 
(a) Wir betrachten sämtliche Paare von Resten $(a_n, a_{n+1})$ modulo $m$. Da
  es höchstens $m\times m=m^2 $ solcher Paare gibt, wiederholt sich nach
  spätestens $p\le m^2$ Schritten ein Paar:
  \begin{gather*}
    (a_n, a_{n+1})=(a_{n+p},a_{n+p+1}) \pmod{m}.
  \end{gather*}
  Da sich aus $(a_n, a_{n+1})$ und der Beziehung $a_{n+2}=a_{n+1}+a_n$ bzw.
  $a_{n-1}=a_{n+1}-a_n$ das vorherige und das nächste Folgenglied eindeutig
  ergeben, kann es keine Vorperiode geben und die Paare $(a_0,a_1)$
  bzw. $(a_{-2},a_{-1})$ wiederholen sich nach $p$ Schritten. Nun ist aber
\begin{gather*}
a_0=0,\quad a_{-1} =1,\quad a_{-2}=-1.
\end{gather*}
Daher treten die Reste $0$, $1$ und $-1$ modulo $m$ stets auf.

(b) Man betrachte als Modul $10^m $ und die Folgenglieder, welche kongruent
$-1$ modulo $10^m$ sind.
\end{beweis}

Die Periodizität der modularen Fibonaccifolge kann mit folgender Darstellung
genauer untersucht werden: Es gilt
\begin{gather*}
  \begin{pmatrix} a_{n+1} \\ a_n \end{pmatrix}=
  M \m \begin{pmatrix} a_n \\ a_{n-1} \end{pmatrix}\text{ mit }
  M=\begin{pmatrix} 1 & 1\\ 1 & 0 \end{pmatrix}
  \intertext{und damit}
  \begin{pmatrix} a_{n+1} \\ a_n \end{pmatrix}=M^n
  \begin{pmatrix} a_1 \\ a_0 \end{pmatrix}. 
\end{gather*}
Die Fibonaccifolge hat damit eine Periode $p$ modulo $m$ genau dann, wenn
$M^p\equiv E \pmod{m}$ gilt, wobei $E$ die Einheitsmatrix ist.     

Für $m=10^k$ (also die Frage nach den Neunen) ergibt sich $M^{15\m m} \equiv E
\pmod{m}$.  Hat man dies für kleine $k$ überprüft, so ergibt sich wegen
\begin{gather*}
  M^{15\m 10^{k+1}} =\br{M^{15\m 10^k}}^{10} \equiv \br{E+a\m 10^k} \equiv
  E+\binom{10}{1}\m a\m 10^k \equiv E \pmod{10^{k+1}},
\end{gather*}
dass dies auch für alle $k>2$ gilt.  

\begin{thebibliography}{Wor77}

\bibitem{b-schueler} A.~Schüler.  \newblock \emph{Rekursive Folgen}.
  \newblock KoSemNet Textsammlung.\\
  \url{https://lsgm.uni-leipzig.de/KoSemNet/pdf/schueler-05-1.pdf}
\bibitem{BWM2003.2} Bundeswettbewerb Mathematik 2003. 2. Runde.\\
  \url{https://www.mathe-wettbewerbe.de/aufgaben}

\end{thebibliography}

\begin{attribution}
graebe (2024-02-14): Contributed to KoSemNet
\end{attribution}

 
\end{document}




