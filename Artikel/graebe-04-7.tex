% Version: $Id: graebe-04-7.tex,v 1.1 2008/09/05 09:52:58 graebe Exp $
\documentclass[11pt]{article}  
\usepackage{kosemnet,ko-math,ngerman}  

\author{Hans-Gert Gr�be, Leipzig}
\title{Das Rechnen mit Kongruenzen. Teil 2\kosemnetlicensemark\\ 
Arbeitsmaterial f�r Klasse 8}
\date{}

\begin{document} 
\maketitle 

Dieser Text baut auf dem Text {\glqq}Rechnen mit Kongruenzen{\grqq}
(graebe-04-4) auf.   

\subsection*{Die K�rzungsregeln}

Wir wollen das Arbeitsblatt {\glqq}Rechnen mit Kongruenzen{\grqq} noch
um zwei Rechenregeln erweitern, die das {\glqq}Dividieren{\grqq} von Resten
betreffen. Da Restklassen aus dem Bereich der ganzen Zahlen entstanden sind,
in dem man bekanntlich Divisionen nicht uneingeschr"ankt ausf"uhren kann (dazu
muss man sie zu den rationalen Zahlen erweitern), ist dabei jedoch Vorsicht am
Platze! Insbesondere, da Reste ja ein Ausdruck daf"ur sind, wie sehr die
Division nicht aufgeht, wollen wir den Begriff {\glqq}Division{\grqq} im
Zusammenhang mit Resten g"anzlich vermeiden und von {\em K"urzungsregeln}
sprechen.
\medskip

\begin{quote} {\bf 1. K"urzungsregel}: Enthalten in der Kongruenz
\[a\equiv b\pmod{m}\]
die Zahlen $a,b,m$ alle einen gemeinsamen Faktor $d$, d.h. lassen sie
sich als $a=d\cdot a',\  b=d\cdot b',\  m=d\cdot m'$ darstellen, so
gilt auch
\[a'\equiv b'\pmod{m'}.\]
\medskip

 {\bf 2. K"urzungsregel}: Enthalten in der Kongruenz
\[a\equiv b\pmod{m}\]
die Zahlen $a,b$ einen gemeinsamen Faktor $d$, d.h. lassen sie sich
als $a=d\cdot a',\ b=d\cdot b'$ darstellen, und sind weiterhin $d$ und
$m$ teilerfremd, so gilt auch
\[a'\equiv b'\pmod{m}.\]
\end{quote}

Zum {\bf Beweis} beider Aussagen erinnern wir uns daran, dass wir
$a\equiv b\pmod{m}$ auf drei verschiedene Arten aufschreiben
k"onnen:
\[a\equiv b\pmod{m}\ \Leftrightarrow\ m\teilt (a-b)\ \Leftrightarrow\
\exists\,t\in {\Z}\ :\ a=b+m\cdot t.\] 
In beiden K"urzungsregeln gilt $a=d\cdot a',\ b=d\cdot b'$, also
$(a-b)=d\cdot(a'-b')$. F"ur die K"urzungsregel 1 schlie"sen wir weiter
\[m=d\cdot m'\teilt d\cdot(a'-b')\ \Rightarrow\ m'\teilt (a'-b')\ \Rightarrow\ 
a'\equiv b'\ (mod\ m') \]
und f"ur die K"urzungsregel 2 folgt aus $m\teilt d\cdot(a'-b')$, dass
$m\teilt (a'-b')$, denn $m$ und $d$ waren als teilerfremd vorausgesetzt
worden. 
\medskip

{\bf Beispiel :} Aus $30\equiv 105\pmod{25}$ k"onnen wir also den Faktor 5
nach der 1.~K"urzungsregel herausk"urzen, womit sich $6\equiv 21\pmod{5}$
ergibt.  Weiter kann man den gemeinsamen Faktor 3 nach der zweiten
K"urzungsregel herausk"urzen. Man erh"alt schlie"slich $2\equiv 7\pmod{5}$.

\subsection*{\centering Lineare Kongruenzen}

Lineare Kongruenzen sind, sehr "ahnlich zu gew"ohnlichen Gleichungen,
Bestimmungsaufgaben, bei denen alle ganzen Zahlen zu finden sind, deren Rest
eine bestimmte Bedingung erf"ullt. Eine typische Aufgabe hat die Gestalt
\[71\,x\equiv 12\pmod{93}.\]
Gesucht sind dabei alle diejenigen ganzen Zahlen $x$, f"ur die $71\,x$ bei
Division durch $93$ den Rest $12$ l"asst.
\medskip

Eine L"osung dieser Aufgabe ist $x=84$. Wie man darauf kommt m"oge an dieser
Stelle offenbleiben. Dass es wirklich eine L"osung ist, kannst Du aber einfach
durch eine Probe sehen.

Da es bei einer linearen Kongruenz nicht auf die Zahl $x$ selbst, sondern nur
auf deren Rest (hier $\pmod{93}$) ankommt, ist auch jede andere Zahl mit
demselben Rest eine L"osung, also etwa $x=177, x=270, x=363$ usw., aber auch
$x=-102, x=-9$ usw.\ und insgesamt jede Zahl der Form $x=84+k\cdot 93, k\in
{\Z}$. Gibt es eine ganzzahlige L"osung, so gibt es also gleich unendlich
viele. Deshalb fragen wir nicht nach den ganzzahligen L"osungen einer linearen
Kongruenz, sondern nach entsprechenden Restklassen. Wir schreiben also
stattdessen:
\begin{quote}
Die Restklasse $x\equiv 84\pmod{93}$ oder kurz $x\equiv 84\pmod{93}$ ist eine
L"osung obiger linearer Kongruenz.
\end{quote}
\medskip

Wie findet man nun alle L"osungen einer linearen Kongruenz. Die sicherste,
aber auch aufw�ndigste Methode ist {\bf das (vollst"andige\,!)  Probieren}. Da
wir wissen, dass es (ganz im Gegensatz zu den ganzen Zahlen) nur {\bf endlich
viele} Restklassen gibt, brauchen wir nur alle durchzuprobieren und die
herausfiltern, f"ur welche die Kongruenz erf"ullt ist.

Dieses Verfahren ist nat"urlich recht aufw�ndig und nur dann sinnvoll, wenn es
nur wenige Restklassen gibt, wenn also der Modul klein ist.  Daf"ur kann man
es nicht nur f"ur lineare, sondern f"ur beliebige Kongruenzen anwenden. Daf"ur
drei Beispiele:

\begin{aufgabe}
  Bestimme die L"osungen der linearen Kongruenz $2x\equiv 1\pmod{3}$.
\end{aufgabe}

\ul{L"osung}: Es gibt $\pmod{3}$ die Restklassen $0,1,2$ als m"ogliche Werte
f"ur $x$. Einsetzen zeigt, dass genau f"ur $x\equiv 2\pmod{3}$ die obige
Kongruenz erf"ullt ist.

\begin{aufgabe}
  Bestimme die L"osungen der linearen Kongruenz $3x\equiv 5\pmod{13}$. 
\end{aufgabe}

\ul{L"osung}: F"ur die verschiedenen Restklassen $x$ stellen wir die Werte
in einer Tabelle zusammen:
\begin{center}
\begin{tabular}{c*{13}{|c}}
$x$ & 0& 1& 2& 3& 4& 5& 6& 7& 8& 9& 10& 11& 12  \\\hline
$3x\pmod{13}$ & 0& 3& 6& 9& 12& 2& 5& 8& 11& 1& 4& 7& 10
\end{tabular}
\end{center}
Wir sehen, dass $x\equiv 6\pmod{13}$ die einzige L"osung ist.

\begin{aufgabe}
  Untersuche, f"ur welche Zahlen $n$ die Zahl $n^3+2n^2+4$ durch 7 teilbar
  ist.
\end{aufgabe}

\ul{L"osung}: Die Aufgabe kann man umformulieren: Es sind alle Restklassen
$n\pmod{7}$ mit $n^3+2n^2+4\equiv 0\pmod{7}$ gesucht. Stellen wir f"ur die 7
m"oglichen Reste wieder eine Tabelle auf:
\begin{center}
\begin{tabular}{c*{7}{|c}}
$n$ & 0& 1& 2& 3& 4& 5& 6 \\\hline
$n^3+2n^2+4\pmod{7}$ & 4& 0& 6& 0& 2& 4& 5
\end{tabular}
\end{center}
Damit haben also genau diejenigen ganzen Zahlen $n$, die bei Division
durch 7 einen der Reste 1 oder 3 lassen, die Eigenschaft, dass
$n^3+2n^2+4$ durch 7 teilbar ist.
\medskip

Manche lineare Kongruenz hat "uberhaupt keine L"osung. Besitzen
n"amlich in
\[a\cdot x\equiv b\pmod{m}\]
$a$ und $m$ einen gemeinsamen Teiler $d$, so muss f"ur die Existenz
einer L"osung auch $d\teilt b$ gelten. Das sieht man am besten, wenn man
die Kongruenz in der alternativen Form
\[a\cdot x\equiv b\pmod{m}\ \Leftrightarrow\ \exists\,t\in {\Z}\
:\ a\cdot x=b+m\cdot t   \] 
darstellt. Wegen $b=ax-mt$ ist also $gcd(a,m)\teilt b$ eine {\em
notwendige} Bedingung f"ur die Existenz einer L"osung.
\medskip

\ul{Beispiel}:  
\[12x\equiv 7\ (10)\]
besitzt keine L"osung, denn $gcd(12,10)=2$ ist kein Teiler von 7. Und
tats"achlich ist $12x$ immer eine gerade Zahl, kann also niemals auf 7
enden (genau das bedeutet ja {\glqq}$\equiv 7\ (10)${\grqq}).
\medskip

Ist die notwendige Bedingung erf"ullt und der Modul so gro"s, dass
vollst"andiges Probieren zu aufw�ndig wird, so empfiehlt es sich, die
lineare Kongruenz zun"achst zu vereinfachen.
\medskip

\ul{Beispiel}: 
\[12x\equiv 8\ (10)\]
Wir reduzieren auf kleinstm"ogliche Reste
\[2x\equiv 8\ (10)\]
und wenden die 1.~K"urzungsregel an, um den gemeinsamen Faktor 2
herauszuk"urzen. Wir erhalten
\[x\equiv 4\ (5)\]
als L"osung.
 \medskip

\ul{Beispiel}: 
\[27x\equiv 9\ (21)\]
Wir reduzieren auf kleinstm"ogliche Reste
\[6x\equiv 9\ (21)\]
und wenden die 1. K"urzungsregel an, um den gemeinsamen Faktor 3
herauszuk"urzen. Wir erhalten
\[2x\equiv 3\ (7).\]
Die L"osung dieser linearen Kongruenz k"onnte man durch vollst"andiges
Probieren ermitteln. Stattdessen k"onnen wir aber auch die rechte
Seite gezielt durch einen gleichwertigen Rest ersetzen, der gerade
ist, um dann die 2.~K"urzungsregel anwenden zu k"onnen:
\[2x\equiv 3\equiv 10\ (7)\ \Rightarrow\ x\equiv 5\ (7).\]
Damit haben wir die L"osung gefunden.

\begin{attribution}
graebe (2004-09-03):\\ Dieses Material wurde vor einiger Zeit als
Begleitmaterial f�r den LSGM-Korrespondenzzirkel in der Klasse 8 erstellt und
nun nach den Regeln der KoSemNet-Literatursammlung aufbereitet.
\end{attribution}

\end{document}
