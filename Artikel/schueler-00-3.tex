\documentclass[11pt]{article}
\usepackage{kosemnet,ko-math,ngerman,url}

\title{Das Invarianzprinzip\kosemnetlicensemark}  
\author{Axel Sch�ler, Mathematisches Institut, Univ. Leipzig\\[8pt]
\url{mailto:schueler@mathematik.uni-leipzig.de}}
\date{Juli 2000}


\begin{document}
\maketitle

\section*{Das Invarianzprinzip}
In vielen Bereichen der Mathematik spielen Invarianten eine Rolle. Man denke
nur an Fixpunkts"atze und Eigenwertaufgaben. Das Invarianzprinzip wird im
Olympiadebereich meist bei Aufgaben verwendet, in denen Spiele oder
Algorithmen vorkommen.

Gegeben sind eine {\em Zustandsmenge} und ein {\em Algorithmus}, der die
Zustandsmenge in sich "uber\-f"uhrt. Gesucht ist eine Funktion, die jedem
Zustand eine Zahl, eine Restklasse oder einen Wahrheitswert zuordnet. Diese
Funktion $f$  soll {\em invariant} sein, das hei"st unver"anderlich, unter dem
Algorithmus. Ferner sind meist noch ein {\em Anfangszustand} $Z_a$ und ein
angestrebter {\em Endzustand} $Z_e$ gegeben. Gilt f"ur die Invariante $f$
\[
f(Z_a)\ne f(Z_e),
\]
dann ist es nicht m"oglich, mit Hilfe des Algorithmus', den Anfangszustand
in einer endlichen Zahl von Schritten in den Endzustand zu "uberf"uhren.

\subsection*{Aufgaben zum Invarianzprinzip}

\begin{aufgabe}
  In $2n$ kreisf"ormig angeordneten Sch"usseln befinden sich je eine Kugel. In
  jedem Zug wird in eine Sch"ussel je eine Kugel aus den beiden benachbarten
  Sch"usseln gelegt.
  \\
  Man zeige, da"s zu keinem Zeitpunkt alle Kugeln in einer Sch"ussel liegen!
\end{aufgabe}

\begin{loesung}
  Wir setzen $m:=2n$. Den Sch"usseln werden die Restklassen
  $[0],[1],\dots,[m-1]$ modulo $m$ zugeordnet. Als Zustandssumme definieren
  wir
\[
s:=\sum_{k=0}^{m-1} a_k [k],
\]
wobei $a_k$ die Anzahl der Kugeln in der Schale mit der Nummer $k$ ist.  In
jedem Schritt "andert sich die Zustandssumme wiefolgt. Angenommen, wir
f"ullten die Schale mit der Nummer $k$ mit je einer Kugel aus den
Nachbarschalen, dann gilt f"ur die neue Zustandssumme $s'$
\[
s':= s-[k-1]-[k+1]+2 [k] =s,
\]
das hei"st, $s$ ist invariant. Nun ist aber die Zustandssumme am Anfang gleich
\[
s_0=\sum_{k=0}^{m-1}[k]{\cdot}1=m(m-1)/2\not\equiv 0 \mod m,
\]
da $(m-1)/2$ keine nat"urliche Zahl ist. Andererseits hat der angestrebte
Endzustand die Zustandssumme $s_1:=[k]m\equiv 0 \mod m$, was offensichtlich
von $s_0$ verschieden ist.
\end{loesung}

\begin{aufgabe}
  Auf einem Kreisbogen sind $100$ ganze Zahlen angeordnet, deren Summe gleich
  $1$ ist. Als Kette bezeichnen wir solche Zahlen, die nebeneinander stehen.
  \\
  Wieviele Ketten mit positiver Summe gibt es?
\end{aufgabe}


\begin{aufgabe}
  (Das MU-R"atsel) Gegeben sei das \glqq Alphabet\grqq\ $\{M,U,I\}$ und
  folgende \glqq Regeln\grqq:
\begin{enumerate}
\item $X$ sei ein beliebiges Wort, gebildet aus den Buchstaben des Alphabets.
  Wenn $MX$ gilt, so gilt auch $MXX$.
\item Endet ein Wort auf $I$, so kann man ein $U$ anh"angen.
\item Drei aufeinanderfolgende Buchstaben $I$ oder zwei aufeinanderfolgende
  Buchstaben $U$ k"onnen gestrichen werden.
\end{enumerate}

L"asst sich das Wort $MI$ durch Anwenden einer endlichen Zahl von obigen
Regeln in das Wort $MU$ umwandeln? Wenn ja, so gebe man eine derartige Folge
an!
\end{aufgabe}


\begin{loesung}
  Die Invariante ist: Teilbarkeit durch $3$ der Anzahl der Buchstaben $I$ des
  Wortes.
\end{loesung}

Die folgenden Aufgaben sind aus Arthur Engels Buch {\glqq}Problem-solving
Strategies{\grqq} entnommen.

\begin{aufgabe}
  (a) Es sei $n$ eine ungerade nat"urliche Zahl. Jemand schreibt zun"achst die
  Zahlen $1,2,\dots, 2n$ an die Tafel. Dann w"ahlt er zwei beliebige Zahlen
  $a$ und $b$ von ihnen aus, streicht sie und ersetzt sie durch ihre absolute
  Differenz $|a-b|$. So f"ahrt er fort bis zum Schluss nur noch eine Zahl an
  der Tafel steht.
  \\
  Beweise, dass diese Zahl ungerade ist!
  
  (b) Jemand schreibt die Zahlen $1,\dots, n$ auf und streicht immer zwei und
  ersetzt sie durch ihre absolute Differenz. Bleibt am Schluss eine gerade
  oder ungerade Zahl stehen?
\end{aufgabe}

\begin{loesung}
  (a) Es sei $S$ die Summe der Zahlen an der Tafel, also $S=n(2n+1)$ und damit
  ungerade, da $n$ ungerade war. In jedem Schritt reduziert sich $S$ um
  $a+b-|a-b|=2\min(a,b)$, also um eine gerade Zahl. Somit bleibt die
  Ungeradzahligkeit von $S$ erhalten. Somit ist die letzte Zahl ungerade.
  
  (b) Es sei $s$ die Summe der an der Tafel stehenden Zahlen modulo $2$.
  \\
  Wegen ${a+b\equiv |a-b|}{ \mod 2}$ ist $s$ invariant und folglich ist
\[
s=s_0=\sum_{k=1}^n k=n(n+1)/2\equiv
\begin{cases}
  0,&\quad\text{falls }\quad n\equiv 0,3 \mod 4
  \\
  1,&\quad \text{falls }\quad n\equiv 1,2 \mod 4
\end{cases}
\]
\end{loesung}

\begin{aufgabe}
  Ein Kreis ist in $6$ Sektoren eingeteilt. Die Zahlen $1,0,1,0,0,0$ sind
  entgegen dem Uhrzeiger in dieser Reihenfolge auf die Sektoren geschrieben.
  In jedem Schritt darf man zwei benachbarte Sektoreninhalte um $1$ erh"ohen.
  \\
  Kann man erreichen, dass nach einer gewissen Zahl von Schritten alle Zahlen
  gleich sind?
\end{aufgabe}

\begin{loesung}
  Es seien $a_1,\dots ,a_6$ die aktuellen Zahlen der Sektoren. Dann ist
  $I:=a_1-a_2+a_3-a_4+a_5-a_6$ eine Invariante des Algorithmus'. Wegen $I_0=2$
  kann daher das Ziel $I=0$ nie erreicht werden.
\end{loesung}

\begin{aufgabe}
  Im Landtag von Sachringen hat jeder Abgeordnete h"ochstens drei Feinde. Man
  zeige, dass man den Landtag in zwei H"auser einteilen kann, so dass jeder
  Abgeordnete h"ochstens einen Feind im eigenen Hause hat!
\end{aufgabe}

\begin{loesung}
  Anfangs teilen wir die Abgeordneten beliebig ein. Es sei $H$ die Gesamtsumme
  der Anzahl der Feinde, die jeder in seinem eigenen Haus hat. Angenommen $A$
  hat zumindest zwei Feinde in seinem eigenen Haus. Wenn $A$ das Haus
  wechselt, so verringert sich $H$. Dieses Verringern kann nicht beliebig
  lange fortgesetzt werden. Wenn $H$ sein Minimum angenommen hat, dann ist die
  gew"unschte Verteilung erreicht.
\end{loesung}

\begin{aufgabe}
  Angenommen, die vier ganzen Zahlen $a,b,c,d$ sind nicht alle gleich. Wir
  starten mit dem Quadrupel $(a,b,c,d)$ und ersetzen es durch
  $(a-b,b-c,c-d,d-a)$. Dann wird mindestens eine Zahl des Quadrupels beliebig
  gro"s.
\end{aufgabe}

\begin{aufgabe}
  Jede der Zahlen $a_1,\dots, a_n$ ist $1$ oder $-1$ und es gilt
\[
S=a_1a_2a_3a_4+a_2a_3a_4a_5+\dots +a_na_1a_2a_3=0.
\]
Man beweise, dass $4\teilt n$.
\end{aufgabe}

\begin{aufgabe}
  $2\,n$ Botschafter sind zu einem Bankett eingeladen. Jeder Botschafter hat
  h"ochstens $n-1$ Feinde.
  \\
  Man beweise, dass die Botschafter so an einem runden Tisch Platz nehmen
  k"onnen, dass keiner von ihnen neben einem seiner Feinde sitzt!
\end{aufgabe}

\begin{aufgabe}
  Wir starten mit einem Quadrupel $(a,b,c,d)$ von positiven ganzen Zahlen und
  gehen "uber zu $(|a-b|, |b-c|, |c-d|, |d-a|)$. Endet dieser Prozess stets
  bei $(0,0,0,0)$?
\end{aufgabe}

\begin{aufgabe}
  Auf einem $n\times n$-Feld sind die Zahlen $a_{ij}=\pm 1$ f"ur $i,j=1,\dots
  ,n$ eingetragen. Wir bezeichnen die Produkte der Zahlen der $i$-ten Zeile
  mit $\alpha_i$ und die der $j$-ten Spalte mit $\beta_j$. F"ur ungerades $n$
  beweise man, dass
\[
s=\sum_i\alpha_i+\sum_j\beta_j
\]
stets von Null verschieden ist!
\end{aufgabe}

\begin{loesung}
  F"ur alle $i,j$ sei zun"achst $a_{ij}=+1$. Dann ist $s_0=n+n=2n\equiv 2 \mod
  4$. Sei nun $(a_{ij})$ beliebig. "Andert man bei genau einem $a_{ij}$ das
  Vorzeichen, so ver"andern sich $\alpha_i$ und $\beta_j$ jeweils um $2$.
  Daher gilt $s'-s\in\{-4,0,4\}$. Also ist stets $s'\equiv s \mod 4$ und daher
  niemals $s_1\equiv 0 \mod 4$.
\end{loesung}

\begin{aufgabe}
  Auf einem Kreis seien zyklisch $4$ Einsen und $5$ Nullen angeordnet.
  Zwischen zwei Zahlen schreibt man nun ihre Summe modulo $2$ und l"oscht die
  alten Zahlen.
  \\
  Man zeige, dass zu keinem Zeitpunkt $9$ Nullen auftauchen!
\end{aufgabe}

\begin{loesung}
  Die Invariante ist $s:=\sum_{k=1}^9 a_k \mod 2$, wobei $a_k$ die $k$-te Zahl
  modulo $2$ auf dem Kreis ist. Nach dem ersten Schritt ist
\[
s'=\sum_{k=1}^9 (a_k +a_{k+1})=2\sum_{k=1}^9 a_k\equiv 0 \mod 2.
\]
Angenommen, zu einem Zeitpunkt tritt zum ersten Male $(0,0,\dots, 0)$ auf.
Dann muss der Vorg"anger dieser Konfiguration gerade gleich $(1,\dots, 1)$
gewesen sein. Die Zusatandssumme dieses Vorg"angers ist aber gleich $s=9\equiv
1\mod 2$. Hingegen ist $s\equiv 0\mod 2 $ vom zweiten Schritt an.
\end{loesung}

\begin{aufgabe}
  Ein Computer kann die folgenden beiden Operationen mit nat"urlichen Zahlen
  durchf"uhren:
\begin{enumerate}
\item Quadrieren,
\item Eine $n$-stellige Zahl $x$ mit $n>3$ geht "uber in $a+b$, wobei $a$ die
  aus den ersten $n-3$ und $b$ die aus den letzten $3$ Ziffern von $x$
  gebildete Zahl ist.
\end{enumerate}

Kann man mit Hilfe dieses Computers aus der Zahl $604$ die Zahl $703$
erhalten?
\end{aufgabe}

\begin{loesung}
  Wir betrachten die zweite Operation $1000a+b\mapsto a+b$. Die Differenz
  dieser beiden Zahlen ist $999a$, also durch $27$ und $37$ teilbar. Der Rest
  modulo $m$, $m\in\{3,9,27,37\}$ bleibt also bei der zweiten Operation
  erhalten. Bei der ersten Operation bleibt aber nur die Teilbarkeit bzw.
  Nicht-Teilbarkeit durch $m$ erhalten. Nun sind aber $604$ und $703$ beide
  durch $9$ und beide nicht durch $27$ teilbar. Aber es ist $703=37*19$ aber
  $604=37*16+12$. Die gesuchte Invariante ist also die Teilbarkeit durch $37$,
  die f"ur die eine Zahl erf"ullt ist, f"ur die andere aber nicht. Der
  Computer kann also nicht die $604$ in die $703$ "uberf"uhren.
\end{loesung}

\begin{aufgabe}
  Zwei Personen spielen das folgende Spiel. An der Tafel steht die Zahl $2$.
  Wer am Zug ist, ersetzt die Zahl $n$ an der Tafel durch die Zahl $n+d$,
  wobei $d$ ein beliebiger Teiler von $n$ ist mit $1\le d<n$. Verloren hat,
  wer eine Zahl gr"o"ser als 1991 an die Tafel schreibt.
  \\
  a) Wer gewinnt bei genauem Spiel, der Anziehende oder sein Gegner?
  \\
  b) Ersetze $1991$ durch $1992$ und beantworte dieselbe Frage!
\end{aufgabe}

\begin{aufgabe}
  Gegeben sei eine beliebige Stellung des $15$-er Schiebespiels --- das ist
  ein $4\times 4$-Quadrat mit $15$ waagerecht oder senkrecht verschiebbaren
  Plastikquadraten, auf denen die Zahlen von $1$ bis $15$ geschrieben stehen.
  Das Spiel hat ein (quadratisches) Loch, das das Verschieben erst
  erm"oglicht.
  \\
  Man zeige, dass es nicht m"oglich ist, zwei Zahlen zu vertauschen unter
  Beibehaltung der Lage aller anderen Zahlen!
\end{aufgabe}

\begin{loesung}
  Insbesondere bleibt die Lage des Loches erhalten. Jeder m"oglichen
  Konfiguration entspricht dann eine Permutation der Zahlen $\{1,\dots ,15\}$.
  Jede Kette von Operationen, die das Loch in sich selbst "uberf"uhrt, besteht
  aber aus einer {\em geraden} Anzahl von waagerechten und einer {\em geraden}
  Anzahl von senkrechten Z"ugen (Vertauschung von Loch und Scheibchen), da das
  Loch ja wieder seine urspr"ungliche horizontale und vertikale Lage einnehmen
  soll. Dies liefert somit eine {\em gerade} Permutation der Zahlen
  $\{1,\dots,15\}$. Die Vertauschung zweier Zahlen allein ist aber eine {\em
    ungerade} Permutation.
\end{loesung}

\begin{aufgabe}
  (Vorschlag $1$ zur Lagerolympiade.) Die Zahlen $1,\dots, 2n$ seien in
  beliebiger Reihenfolge aufgeschrieben. Nun addiert man zu jeder Zahl seine
  Nummer in dieser Reihenfolge.
  \\
  Man beweise, dass dann mindestens zwei gleiche Summen modulo $2n$ auftreten!
\end{aufgabe}

\begin{loesung}
  Die Summe der Summen h"angt nicht von der gew"ahlten Permutation $\pi\in
  S_{2n}$ ab und betr"agt modulo $2n$
\[
s:=\sum_{k=1}^{2n} (k +\pi(k))\equiv 2n(2n+1)\equiv 0\mod{2n}.
\]
Angenommen, kein Rest unter diesen Summen $k +\pi(k)$ tritt doppelt auf, dann
muss jeder Rest genau einmal auftreten, also ist andererseits
\[
s=\sum_{k=1}^{2n} k \equiv 2n(2n+1)/2\equiv n\mod{2n}.
\]
Dies widerspricht der obigen Rechnung.
\end{loesung}

\subsection*{Parkettierungen}

\begin{aufgabe}
  Entfernt man von einem $8\times 8$ Quadrat die gegen"uberliegenden
  Eckfelder, so l"asst sich die verbleibende Fl"ache nicht mit $31$
  Dominosteinen "uberdecken.
\end{aufgabe}

\begin{loesung}
  $31$ Dominosteine "uberdecken stets $31$ schwarze und $31$ wei"se Felder
  einer schachbrettartig gef"arbten Fl"ache. Die oben beschriebene Restfl"ache
  hat jedoch $30$ schwarze und $32$ wei"se Felder.
\end{loesung}

\begin{aufgabe}
  (a) Ein $8\times 8$ Brett l"asst sich nicht durch $15$ Rechtecke der Form
  $4\times 1$ und ein $2\times 2$ Quadrat "uberdecken.
  \\
  (b) Ist ein Rechteck mit $2\times 2$ und $4\times 1$ Platten "uberdeckt, so
  kann man es nicht mehr "uberdecken, wenn man eine Platte der einen Sorte
  durch eine Platte der anderen Sorte ersetzt.
\end{aufgabe}

\begin{loesung}
  Wir "uberdecken das Rechteck regelm"a"sig mit den Resten $0,1,2$ und $3$
  modulo $4$. Jedes $4\times 1$ Rechteck "uberdeckt dann in jeder Lage genau
  einen dieser Reste. Jedes $2\times 2$ Quadrat "uberdeckt genau einen Rest
  zweimal und einen gar nicht. Daher ist ein Austausch nicht m"oglich.
\end{loesung}

\begin{aufgabe}
  Kann man ein $8\times 8$ Brett so mit $21$ Rechtecken der Form $3\times 1$
  so auslegen, dass genau ein $1\times 1$ Quadrat nicht "uberdeckt wird? Wenn
  ja, wo kann dieses Quadrat liegen?
\end{aufgabe}

\begin{loesung}
  Man "uberdecke das Quadrat mit den Resten modulo $3$ in regelm"a"siger
  Weise. Dabei treten diagonal immer dieselben Reste auf: Beginnt man mit Rest
  $0$ in der linken oberen Ecke und Rest $1$ darunter, so hat man den Rest $0$
  genau $1+4+7+6+3=21$ Mal, den Rest $1$ genau $2+5+8+5+2=22$ Mal und den Rest
  $2$ schlie"slich $3+6+7+4+3=21$ Mal. Bei der "Uberdeckung bleibt also der
  Rest $1$ einmal "ubrig. Das freie Feld muss also auf der Hauptdiagonalen
  $a1$ -- $h8$ oder seinen dritten bzw. sechsten Nebendiagonalen (jeweils $2$)
  liegen. Aus Symmetriegr"unden gilt dasselbe f"ur die Hauptdiagonale $a8$ --
  $h1$ (was man etwa auch durch eine andere Parkettierung mit den Resten
  modulo $3$ erreichen kann). Einzig die Felder $c3$, $f3$, $f6$ und $c6$
  erf"ullen beide Bedingungen. Eine m"ogliche Parkettierung mit $1\times 3$
  Rechtecken "uberlegt man sich schnell.
\end{loesung}

\begin{aufgabe}
  Auf einem $8\times 8$ Brett befindet sich in der rechten oberen Ecke ein
  schwarzer Spielstein. Mit jedem Zug wird ein wei"ser Spielstein auf ein
  freies Feld gesetzt. Dabei werden Spielsteine auf den benachbarten Feldern
  umgef"arbt (schwarz zu wei"s und umgekehrt). Kann man erreichen, dass das
  ganze Brett mit wei"sen Steinen besetzt ist?
\end{aufgabe}

(Bem.: Zwei Felder seien benachbart, wenn sie eine gemeinsame Ecke haben).

\begin{loesung}
  Bei jedem Setzen eines Steins auf ein Feld werde der Mittelpunkt dieses
  Feldes mit denen der freien Nachbarfeldern verbunden, falls es noch welche
  gibt.  Offenbar werden dabei beliebige zwei benachbarte Felder (mit Ausnahme
  des besetzten Eckfeldes) mit genau einer Strecke verbunden.  Die
  vollst"andige Anzahl der Nachbarschaftsstrecken ist in jeden $n\times n$
  Quadrat gerade, denn zu den $4$ Eckfeldern geh"oren jeweils $3$ Strecken, zu
  den $4(n-2)$ Randfeldern jeweils $5$ und zu den $(n-1)^2$ inneren Feldern
  jeweils $8$ Nachbarn. Da man doppelt gez"ahlt hat, muss man noch durch $2$
  teilen. Es bleibt dennoch eine gerade Anzahl "ubrig. Durch das schwarze
  Eckfeld fallen $3$ Nachbarschaften heraus.  Die Gesamtzahl der Strecken ist
  also ungerade.  Daher muss es einen in einem bestimmten Moment gesetzten
  Stein geben, der eine ungerade Anzahl von freien Nachbarfelder hat. Nach
  allen Umf"arbungen (Besetzen der ungeraden Zahl von Nachbarfeldern) wird
  dieser Stein schwarz sein.
\end{loesung} 

\begin{aufgabe}
  Ein Quadrat der Seitenl"ange $2n-1$ wird durch Figuren der folgenden Bauart
  "uberdeckt

\begin{center}
%\input steine2.pic
{% Picture saved by xtexcad 2.4
  \unitlength=1.000000pt
\begin{picture}(332.00,33.00)(0.00,0.00)
\put(319.00,1.00){\line(0,1){31.00}}
\put(288.00,1.00){\line(1,0){31.00}}
\put(288.00,16.00){\line(0,-1){15.00}}
\put(332.00,16.00){\line(-1,0){44.00}}
\put(332.00,32.00){\line(0,-1){16.00}}
\put(303.00,32.00){\line(1,0){29.00}}
\put(303.00,0.00){\line(0,1){32.00}}
\put(258.00,21.00){\makebox(0.00,0.00){$(3)$}}
\put(128.00,21.00){\makebox(0.00,0.00){$(2)$}}
\put(0.00,21.00){\makebox(0.00,0.00){$(1)$}}
\put(173.00,32.00){\line(0,-1){28.00}}
\put(159.00,18.00){\line(1,0){30.00}}
\put(158.00,33.00){\line(0,-1){9.00}}
\put(189.00,33.00){\line(-1,0){31.00}}
\put(189.00,4.00){\line(0,1){29.00}}
\put(158.00,4.00){\line(1,0){31.00}}
\put(158.00,30.00){\line(0,-1){26.00}}
\put(63.00,31.00){\line(-1,0){14.00}}
\put(63.00,17.00){\line(0,1){14.00}}
\put(34.00,17.00){\line(1,0){29.00}}
\put(34.00,3.00){\line(0,1){28.00}}
\put(48.00,3.00){\line(-1,0){14.00}}
\put(48.00,31.00){\line(0,-1){28.00}}
\put(34.00,31.00){\line(1,0){14.00}}
\end{picture}}
\end{center}

Man beweise, dass dazu mindestens $4n-1$ Figuren vom Typ $(1)$ ben"otigt
werden!
\end{aufgabe}

\begin{loesung}
  Die folgenden $n^2$ Felder des gegebenen Quadrates seien markiert:
  \\
  $a1$, $a3,\dots, a(2n-1)$, $c1$, $c3,\dots, c(2n-1)$, usw. Es seien $x$
  bzw.{} $y$ die Anzahl der Steine vom Typ $(1)$ bzw. vom Typ $(2)$ und $(3)$.
  Dann gilt
\[
3x+4y=4n^2-4n+1.
\]
Die drei Steintypen k"onnen aber jeweils h"ochstens ein markiertes Feld
"uberdecken. Somit gilt
\[
x+y\ge n^2
\]
Multipliziert man diese Ungleichung mit $4$ und setzt die obige Gleichung ein,
so hat man:
\[
4x+ (4n^2-4n+1)-3x\ge 4n^2.
\]
Hieraus folgt sofort $x\ge 4n-1$.
\end{loesung}

\begin{aufgabe}
  (a) Gegeben sei ein $6\times 6$ Quadrat oder ein $k\times n$ Rechteck mit
  $k\le 4$. Diese seien mit Dominos "uberdeckt. Man zeige, dass es stets eine
  Gittergerade gibt, die kein Domino in der Mitte zerschneidet!
  
  (b) Man zeige, dass es f"ur alle anderen Rechtecke "Uberdeckungen mit
  Dominos derart gibt, so dass alle Gittergeraden zumindest jeweils ein Domino
  in der Mitte zerteilen!
\end{aufgabe}

\begin{loesung}
  (a1) Der Fall des $6\times 6$ Quadrats. Wir zeigen: Schneidet eine
  Gittergerade ein Domino, so schneidet sie auch ein weiteres Domino. Durch
  jede Gittergerade wird das Domino in zwei Rechtecke geteilt, die eine {\em
    gerade} Zahl als Fl"ache besitzen. Zerschnittene Dominos auf einer
  Gittergeraden m"ussen also immer paarweise auftreten.
  \\
  Nun gibt es aber $10$ schneidende Gittergeraden. Angenommen, jede
  Gittergerade zerschneidet mindestens ein Domino, so schneidet sie auch ein
  zweites Domino. Da jedes Domino nur von h"ochstens einer Gittergeraden
  zerschnitten werden kann, existieren also mindestens $20$ Dominos mit einer
  Fl"ache von $40$. Dies ist aber ein Widerspruch, da das Quadrat nur die
  Fl"ache $36$ hat.
  \\
  (a2) Die F"alle der Rechtecke $1\times n$ und $2\times n$ "uberlegt man sich
  schnell anhand einer Skizze. Wir betrachten den Fall $3\times 2n$ und nehmen
  wieder an, dass alle $2n-1$ vertikalen Gittergeraden, mindestens ein Domino
  zerschneiden. Wie in (a1) argumentiert man, dass die $n-1$ vertikalen
  Gittergeraden, die das Rechteck in zwei Rechtecke mit gerader Fl"ache
  zerschneiden wiederum genau $2$ Dominos zerschneiden m"ussen und die anderen
  $n$ mindestens ein Domino. Die Gesamtzahl der Dominos ist also
\[
2{\cdot}(n-1)+1{\cdot} n=3n-2.
\]
Andererseits m"ussen aber mindestens $4$ senkrechte Dominos von den beiden
waagerechten Gittergeraden zerschnitten werden. Somit hat man mindestens
$3n+2$ Dominos zu zerschneiden, es gibt aber nur $3n$.
\\
(a3) Wir betrachten den Fall des $4\times n$ Rechtecks. Die $n-1$ kurzen
Gittergeraden m"ussen dann immer mindestens $2$ Dominos schneiden. Somit hat
man mindestens $2n-2$ Dominos, die von den kurzen Gittergeraden geschnitten
werden. Die drei langen Gittergeraden schneiden jeweils mindestens ein Domino,
so dass man mindestens $2n+1$ Dominos hat; es gibt aber nur $2n$ Dominos.

(b) Es existieren Zerlegungen der $5\times 6$ und $8\times 6$ Rechtecke, wo
jede Gittergerade mindestens ein Domino zerschneidet:
\begin{center}
\begin{tabular}{|l|l|l|l|l|}
\hline
1&2&2&3&3
\\ \hline
1&4&5&5&6
\\ \hline
7&4&8&8&6
\\ \hline
7&9&9&10&11
\\ \hline
12&12&13&10&11
\\ \hline 
14&14&13&15&15
\\
\hline
\end{tabular}
\qquad
\begin{tabular}{|l|l|l|l|l|l|l|l|}
\hline
1&2&2&3&4&4&5&5
\\ \hline 
1&6&6&3&7&7&8&9
\\ \hline
10&10&11&12&12&13&8&9 
\\ \hline
14&15&11&16&16&13&17&17
\\ \hline
14&15&18&18&19&20&20&21
\\ \hline
22&22&23&23&19&24&24&21
\\ \hline
\end{tabular}
\end{center}
Jedes andere Rechteck l"asst sich durch schrittweises Hinzuf"ugen von zwei
Zeilen bzw.{} Spalten aus den obigen beiden Rechtecken $5\times 6$ und
$6\times 8$ erzeugen. Wir geben nun eine "Uberdeckung des Rechtecks $n\times
(k+2)$ an, die aus einer "Uberdeckung des Rechtecks $n\times k$ ($n$ Zeilen
und $k$ Spalten) konstruiert wird und ebenfalls die Eigenschaft hat, dass jede
Gittergerade mindestens ein Domino schneidet. Wir f"ugen rechts $2$ Spalten
hinzu.  In der Spalte, rechtsau"sen, des Ausgangsrechtecks findet man ein
vertikales Domino; (angenommen, es g"abe nur horizontale Dominos in den
letzten beiden Spalten, dann w"urde die vorletzte vertikale Gittergerade kein
Domino schneiden, entgegen der Annahme). Dieses vertikale Domino wird um zwei
Spalten nach rechts verschoben. Alle anderen freien Felder werden durch
horizontale Dominos aufgef"ullt. Man sieht, dass dadurch die beiden neu
hinzugekommenen vertikalen Gittergeraden $2$ bzw.{} $n-2$ Dominos
durchschneiden, wogegen die horizontalen Gittergeraden nach wie vor dieselben
Dominos zerschneiden.
\end{loesung}

\begin{aufgabe}
  (Vorschlag $2$ zur Lagerolympiade.) Ist es m"oglich, ein Rechteck mit den 5
  verschiedenen Tetraminos zu "uberdecken?
\end{aufgabe}

\begin{loesung}
  Nein, es ist nicht m"oglich. Das Rechteck hat $20$ Einheitsquadrate und
  l"asst sich daher schachbrettartig f"arben: $10$ wei"se und $10$ schwarze
  Felder hat man. Nun "uberdecken aber das gerade, das quadratische, das L-
  und das S-Tetramino jeweils zwei schwarze und zwei wei"se Felder. Nur das
  "ubriggebliebene T-Tetramino "uberdeckt von einer Farbe $3$ Felder und von
  der anderen nur ein Feld. Daher ist die Bilanz der schwarzen und wei"sen
  Felder in jeder "uberdecketen Fl"ache unausgeglichen. Die "Uberdeckung ist
  somit nicht m"oglich.
\end{loesung}

\begin{aufgabe}
  Man beweise, dass sich das $10\times 10$ Quadrat nicht mit $25$ T-Tetraminos
  "uberdecken l"asst!
\end{aufgabe}

\begin{loesung}
  (Nach Susanne K"ursten.) Bei der schachbrettartigen F"arbung des Quadrates
  treten gleich viele schwarze wie wei"se Felder auf. Somit muss die Anzahl
  der \glqq schwarzen\grqq\ T-Tetraminos gleich der Anzahl der \glqq
  wei"sen\grqq\ T-Tetraminos in der "Uberdeckung sein (Ein T-Tetramino in der
  "Uberdeckung hei"se schwarz bzw.{} wei"s, wenn es mehr schwarze bzw.{} mehr
  wei"se Felder "uberdeckt). Da aber eine ungerade Zahl von T-Tetraminos zur
  "Uberdeckung verwendet wird, k"onnen nicht gleich viele schwarze wie wei"se
  T-Tetraminos auftreten.
\end{loesung}

\begin{aufgabe}
  Man zeige, dass sich der $10\times 10\times 10$ W"urfel nicht mit $250$
  Quadern vom Format $4\times 1\times 1$ auslegen l"asst.
\end{aufgabe}

\begin{loesung}
  Der $10\times 10\times 10$ W"urfel wird regelm"a"sig mit den Farben $0,1,2$
  und $3$ so gef"arbt, dass jedes gerade (r"aumliche) Tetramino jeden Rest
  genau einmal "uberdeckt. Beginnt man in der hinteren, oberen, linken Ecke
  mit der Farbe $0$ und geht in die $3$ m"oglichen Richtungen jeweils mit $1$,
  $2$ und $3$ weiter, so erh"alt man eine ausgeglichene Bilanz von gef"arbten
  Elementarw"urfeln bis auf die vordere, rechte, untere $2\times 2\times
  2$-Ecke, die in den beiden Schichten, von hinten beginnend, die F"arbung
\[
\begin{pmatrix}0&1\\ 1&2\end{pmatrix}
\quad \text{und}\quad
\begin{pmatrix}1 &2\\ 2&3\end{pmatrix}
\]
hat. Die Farben $1$ und $2$ "uberwiegen also, was bei einer Parkettierung mit
geraden Tetraminos nicht m"oglich ist.
\end{loesung}

\begin{attribution}
schueler (2004-09-09): Contributed to KoSemNet

graebe (2004-09-09): Prepared along the KoSemNet rules

schueler (2016-01-30): Vermeintliche L\"osung zu Aufgabe 7 wurde zu Aufgabe 8 korrigiert.
\end{attribution}

\end{document}


\nein{
\aufg{
{\bf 3.} (Doppeltes Abz"ahlen) Es sei $n$ eine nat"urliche Zahl und
$A_1,\dots,A_{2n+1}$ Teilmengen einer Menge $B$. Es gelte
\begin{enumerate}
\item Jede Menge $A_i$ enth"alt genau $2n$ Elemente.
\item F"ur alle $i$ und $j$, $i\ne j$, enth"alt die Menge $A_i\cap A_j$ genau
  ein Element.
\item Jedes Element von $B$ geh"ort zu mindestens zwei der Mengen $A_i$.
\end{enumerate}
(a) Man zeige, dass $B$ genau $n(2n+1)$ Elemente besitzt, dass keine drei
verschiedenen Mengen $A_i$ gemeinsame Elemente besitzen, und dass
jedes Element von $B$ in genau 2 der Mengen $A_i$ liegt.
\\
(b) F"ur welche $n$ gibt es eine Zuordnung von $B$ nach $\{0,1\}$, so dass
genau $n$ Elementen jeder Menge $A_i$ die Zahl $0$ zugeordnet wird?
}

















