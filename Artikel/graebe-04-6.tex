% Version: $Id: graebe-04-6.tex,v 1.1 2008/09/05 09:52:58 graebe Exp $
\documentclass[11pt]{article}  
\usepackage{kosemnet,ko-math,ngerman}  

\author{Hans-Gert Gr�be, Leipzig}
\title{Das Dirichletsche Schubfachprinzip\kosemnetlicensemark\\ 
Arbeitsmaterial f�r Klasse 8}
\date{}

\begin{document} 
\maketitle 

\begin{quote}
Felix behauptet am ersten Tag im Ferienlager: {\glqq}In unserer Gruppe gibt es
unter den 15 Kindern zwei, die im selben Monat Geburtstag haben.{\grqq} Hat
Felix recht\,?
\end{quote}
Wie kann Felix so sicher sein, wo er doch die Kinder noch gar nicht
kennt und demzufolge bestimmt auch nicht wei"s, in welchem Monat jeder
einzelne Geburtstag hat? 

Felix' "Uberlegung ist denkbar einfach: Denke ich mir 12 Schubk"asten,
auf denen die verschiedenen Monatsnamen stehen, lie"se die 15 Kinder
ihre Namen auf kleine Zettel schreiben und jedes Kind dann seinen
Zettel in das Fach legen, auf dem sein Geburtsmonat steht, dann liegen
in einem der K"asten mehr als ein Zettel, denn es gibt mehr Kinder in
der Gruppe als Monate im Jahr.

Auf eine solche "Uberlegung, die man das {\em Dirichletsche Schubfachprinzip}
nennt, kann man manche Knobelaufgaben zur"uckf"uhren. Dabei ist es egal,
welcher Natur die K"asten und die verteilten Objekte sind und nat"urlich auch,
ob sie wirklich verteilt worden sind oder (wie hier) man sich die Verteilung
nur vorgestellt hat.

Zur Verdeutlichung spricht man deshalb oft von Kugeln und K"asten. In
obiger Aufgabe haben wir die Aussage
\begin{quote}\it
Hat man mehr Kugeln zum Verteilen als K"asten, wo sie hineinkommen, so
liegen nach dem Verteilen in wenigstens einem Kasten mehrere Kugeln.
\end{quote}
verwendet.

Man kann genauer folgende Aussage treffen:
\begin{quote} \it
Hat man (wenigstens) $k\cdot n+1$ Kugeln zum Verteilen auf $n$
K"asten, wo sie hineinkommen, so liegen nach dem Verteilen in
einem Kasten mindestens $k+1$ Kugeln.
\end{quote}

\begin{aufgabe}
  Gibt es an Eurer Schule zwei Sch"uler, die am gleichen Tag
  Geburtstag haben\,?
\end{aufgabe}

\ul{Antwort}: Da das Jahr h"ochstens 366 Tage hat, m"ussten an Deiner
Schule also mindestens 367 Sch"uler sein, um die Frage mit
{\glqq}ja{\grqq} zu beantworten. Gibt es wenigstens $2\cdot
366+1=733$ Sch"uler, so gibt es sogar einen {\glqq}Kasten, in dem
drei Sch"uler liegen{\grqq}, wie man manchmal etwas salopp sagt.

\begin{aufgabe}
  Was kann man "uber ein f"unfz"ugiges Gynmasium mit
  durchschnittlich 28 Sch"ulern pro Klasse aussagen\,?
\end{aufgabe}

\begin{aufgabe}
  Gibt es auf der Erde 100 Menschen, die alle gleich viele Haare
  auf dem Kopf haben\,?
\end{aufgabe}

Versuche, diese Fragen selbst zu beantworten\,!

\begin{attribution}
graebe (2004-09-03):\\ Dieses Material wurde vor einiger Zeit als
Begleitmaterial f�r den LSGM-Korrespondenzzirkel in der Klasse 8 erstellt und
nun nach den Regeln der KoSemNet-Literatursammlung aufbereitet.
\end{attribution}

\end{document}
