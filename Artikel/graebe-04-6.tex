\documentclass[11pt]{article}  
\usepackage{kosemnet,ko-math,ngerman}
\usepackage[utf8]{inputenc}

\author{Hans-Gert Gräbe, Leipzig}
\title{Das Dirichletsche Schubfachprinzip\kosemnetlicensemark\\ 
Arbeitsmaterial für Klasse 8}
\date{Version vom 29. September 2025}

\begin{document} 
\maketitle 

\section{Das Schubfachprinzip}
\begin{quote}
Felix behauptet am ersten Tag im Ferienlager: „In unserer Gruppe gibt es unter
den 15 Kindern zwei, die im selben Monat Geburtstag haben.“ Hat Felix recht?
\end{quote}
Wie kann Felix so sicher sein, wo er doch die Kinder noch gar nicht kennt und
demzufolge bestimmt auch nicht wei"s, in welchem Monat jeder einzelne
Geburtstag hat?

Felix' "Uberlegung ist denkbar einfach: Denke ich mir 12 Schubfächer, auf
denen die verschiedenen Monatsnamen stehen, lie"se die 15 Kinder ihre Namen
auf kleine Zettel schreiben und jedes Kind dann seinen Zettel in das Fach
legen, auf dem sein Geburtsmonat steht, dann liegen in einem der K"asten mehr
als ein Zettel, denn es gibt mehr Kinder in der Gruppe als Monate im Jahr.

Auf eine solche "Uberlegung, die man das \emph{Dirichletsche Schubfachprinzip}
nennt, kann man manche Aufgaben zur"uckf"uhren. Dabei ist es egal, welcher
Natur die Fächer und die verteilten Objekte sind und nat"urlich auch, ob sie
wirklich verteilt worden sind oder (wie hier) man sich die Verteilung nur
vorgestellt hat.

Zur Verdeutlichung spricht man deshalb oft von Kugeln und Fächern. In obiger
Aufgabe haben wir die Aussage
\begin{quote}\it
Hat man mehr Kugeln zum Verteilen als Fächer, wo sie hineinkommen, so liegen
nach dem Verteilen in wenigstens einem Fach mehrere Kugeln.
\end{quote}
verwendet.

Man kann genauer folgende Aussage treffen:
\begin{quote} \it
Hat man (wenigstens) $k\m n+1$ Kugeln zum Verteilen auf $n$ Fächer, so liegen
nach dem Verteilen in einem Fach mindestens $k+1$ Kugeln.
\end{quote}

Im Englischen spricht man auch vom \emph{pigeonhole principle} (dem „Prinzip
des Taubenschlags“); Wenn 10 Tauben in drei Taubenschlägen sitzen, dann müssen
in einem der Schläge (mindestens) vier Tauben sitzen.

\section{Aufgaben}

\begin{aufgabe}
  Gibt es an Eurer Schule zwei Sch"uler, die am gleichen Tag Geburtstag
  haben?
\end{aufgabe}
\begin{loesung}
Da das Jahr h"ochstens 366 Tage hat, m"ussen an Deiner Schule also mindestens
367 Sch"uler sein, um die Frage mit „ja“ zu beantworten. Gibt es wenigstens
$2\cdot 366+1=733$ Sch"uler, so gibt es sogar ein „Fach, in dem drei Sch"uler
liegen“, wie man manchmal etwas salopp sagt, dass also 3 Schüler am gleichen
Tag Geburtstag haben.
\end{loesung}

\begin{aufgabe}
  Was kann man "uber ein f"unfz"ugiges Gymnasium mit durchschnittlich 28
  Sch"ulern pro Klasse aussagen?
\end{aufgabe}

\begin{aufgabe}
  Wie viele Schüler muss eine Schule mindestens haben, damit mit Sicherheit
  mindestens fünf von ihnen am gleichen Tag Geburtstag haben?
\end{aufgabe}
\begin{loesung}
  Es gilt hier $n = 365$ (Anzahl der Tage eines Jahres) sowie $k+1=5$, also
  $k=4$ und somit $nk+1 = 1461$. Folglich muss die Schule mindestens 1461
  Schüler haben, damit die genannte Bedingung erfüllt ist.
\end{loesung}

\begin{aufgabe}
  Gibt es in Deutschland 100 Menschen, die (keine Glatze haben und dennoch
  alle gleich viele Haare auf dem Kopf haben?
\end{aufgabe}

\begin{aufgabe}
  Gibt es eine (hinreichend große) Schüleranzahl, für die man behaupten kann,
  dass mit Sicherheit an \emph{zwei} verschiedenen Tagen jeweils mehr als ein
  Schüler Geburtstag hat? 
\end{aufgabe}
\begin{loesung}
  Nein, es könnten alle am selben Tag Geburtstag haben.
\end{loesung}

\begin{aufgabe}
  Beweise: Unter $n+1$ ganzen Zahlen kann man stets mindestens zwei finden,
  deren Differenz durch $n$ teilbar ist ($n\ge 2, n\in\N$).
\end{aufgabe}
(Hinweis: Betrachte die Reste der $n+1$ Zahlen bei Division durch die Zahl
$n$). 

\begin{aufgabe}
  Eine Schießscheibe habe die Form eines gleichseitigen Dreiecks mit der
  Seitenlänge 2. Sie werde fünfmal getroffen. Zeige, dass es dann stets
  mindestens zwei Einschusslöcher gibt, deren Abstand kleiner oder gleich 1
  ist.
\end{aufgabe}
(Hinweis: Zerlege das Dreieck in vier kongruente Dreiecke mit der Seitenlänge
1.)

Wir betrachten nun zwei etwas schwerere Aufgaben:

\begin{aufgabe}
  In einem Saal seien $n\ge 2$ Personen anwesend. Zeige, dass es unter ihnen
  stets mindestens zwei Personen gibt, die im Saal dieselbe Anzahl von
  Bekannten haben.
\end{aufgabe}

\begin{loesung}
  Wir ordnen einer Person, die im Saal genau $i$ Bekannte hat, das
  Schubfach mit der Nummer $i$ zu.

  Die $n$ Personen verteilen sich also auf die $n$ Schubfächer $0,
  1,\dots,n-1$.  Eines der beiden Schubfächer mit der Nummer $0$ oder der
  Nummer $n-1$ muss aber frei bleiben, weil es nicht möglich ist, dass eine
  Person niemanden im Saal kennt, eine andere dagegen alle.  Folglich werden
  die $n$ Personen auf nur $n-1$ Schubfächer verteilt und es gibt ein
  Schubfach mit (mindestens) zwei Personen.
\end{loesung}

\begin{aufgabe}
  Gegeben seien $n$ nicht notwendig verschiedene ganze Zahlen $a_1, a_2,\dots,
  a_n$.  Zeige, dass es stets eine Teilmenge dieser Zahlen gibt, deren Summe
  durch $n$ teilbar ist ($n\ge 1,n\in\N$).
\end{aufgabe}
\begin{loesung}
  Wir betrachten die $n$ Zahlen $a_1, (a_1+a_2), (a_1+a_2+a_3),\dots,
  (a_1+a_2+\dots+a_n)$.  Falls eine dieser Zahlen durch $n$ teilbar ist, gilt
  die Behauptung. Anderenfalls lassen mindestens zwei dieser Zahlen bei
  Division durch $n$ den gleichen Rest. Man wähle dazu die $n-1$ Reste
  $1,2,\dots,n-1$ als Schubfächer. Dann ist aber die Differenz dieser beiden
  Zahlen durch $n$ teilbar. Daraus folgt leicht die Behauptung.
\end{loesung}

\section{Für welche Aufgaben eignet sich das Schubfachprinzip?}

Wenn man weiß, dass eine Aufgabe mit dem Schubfachprinzip lösbar ist, dann
kommt es nur darauf an, die „Elemente“ und die „Schubfächer“ geschickt zu
bestimmen. Hierin kann die Hauptschwierigkeit einer solchen Aufgabe liegen.

Wie sieht man einer Aufgabe an, dass sie mit dem Schubfachprinzip lösbar ist?
Hier kann man folgende Faustregel benutzen: Eine Aufgabe lässt sich mit dem
Schubfachprinzip lösen, wenn in der Aufgabenstellung eine Menge beschrieben
wird, in der zwei oder mehr Elemente gleich sein oder eine gewisse Bedingung
erfüllen sollen. Dabei kommt in der Aufgabenstellung häufig die Formulierung
„mindestens“ vor. Im allgemeinen sind Existenzaussagen über endliche Mengen
mit dem Schubfachprinzip beweisbar.  Häufig kann der Aufgabentext so
umformuliert werden, dass diese äußeren Merkmale sichtbar werden.

\begin{attribution}
graebe (2004-09-03): Begleitmaterial für den LSGM-Korrespondenzzirkel in der
Klasse 8.

graebe (2025-09-29): Um Material des BK Chemnitz ergänzt.
\end{attribution}

\end{document}
