\documentclass[11pt,a4paper]{article}
\usepackage{ngerman,schueler,url}
\usepackage{kosemnet,ko-math}

\title{Rekursive Folgen\kosemnetlicensemark} 
\author{Axel Schüler, Mathematisches Institut, Univ. Leipzig\\[8pt]
\url{mailto:Axel.Schueler@math.uni-leipzig.de}}
\date{01. März 2010}
%  Änderung in der Lösung von Aufgabe 2
\usepackage{times}

%\usepackage{prelim2e}
%\usepackage{showkeys}

%%%%%%%%%%%%% Seitenformat %%%%%%%%%%%%%%%%%%%%%
\textwidth 16cm
\textheight 24cm
\oddsidemargin 2.3cm
\evensidemargin 1.5cm
%\leftmargin 2cm
\hoffset -2cm
\voffset -1.6cm
\sloppy
\parindent0pt
%%%%%%%%%%%%%%%%%%%%%%%%%%%%%%%%%%%%%%%%%5
%\DeclareGraphicsExtensions{.eps}


\begin{document}
\maketitle
%%%%%%%%%\tableofcontents

\section{Rekursive Folgen}
\subsection{Einleitung}
\nocite{b-Markuschewitsch}

Rekursive Folgen umfassen viele aus dem Unterricht bekannte Folgen:
geometrische Folgen $(cq^n)$, $c,q\in \R$, arithmetische Folgen
$(c+dn)_{n\in\N}$, $c,d\in\R $ gegeben, periodische Folgen, wie z.\,B.{}
$(-1,1,-1,1,\cdots)$ oder arithmetische Folgen höherer Ordnung, wie
quadratische Folgen $(n^2)$ oder kubische Folgen, wie $(an^3 +bn^2 +cn+d)$,
$a,b,c,d\in \R $.

Die vorliegende kleine mathematische Theorie ist abgeschlossen, einfach und
klar.  Ihre Grundzüge wurden vom französischen Mathematiker {\sc Moivre}
(1720), von {\sc Daniel Bernoulli} und {\sc Leonhard Euler} entwickelt bzw.{}
weiter entwickelt.

Das lateinische \emph{recurro} bedeutet „umkehren“ oder „zurückgehen“. Grob
gesprochen erhält man das Glied $a_n$ einer rekursiven Folge, indem man $a_n$
aus einer festen Anzahl vorhergehender Glieder berechnet, etwa $a_{n+2}=a_{n+1}
+a_n$. Ist hingegen $a_n$ als Funktion von $n$ allein (und nicht in
Abhängigkeit von $a_{n-1}$, $a_{n-2}$ usw.{}) gegeben, so sagt man, dass die
Folge $(a_n)$ \emph{explizit} gegeben ist, z.\,B.{} ist $a_n=\sqrt{n^3 -1} $
eine explizite Bildungsvorschrift.

Die folgenden Aufgabenstellungen treten häufig bei rekursiven Folgen auf:
\begin{itemize}
\item Man ermittle aus der rekursiven Vorschrift die explizite
  Bildungsvorschrift. Dies ist in dieser Allgemeinheit eine sehr schwierige, in
  vielen Fällen unlösbare Aufgabenstellung.  Wir werden aber sehen, dass sie
  für die große Klasse der \emph{linearen rekursiven Folgen mit konstanten
    Koeffizienten} vollständig lösbar ist.
\begin{align}\label{e-linrek}
a_{n+k}=c_1 a_{n+k-1} +c_2 a_{n+k-2} +\cdots + c_k a_n,\quad n\in \N.
\end{align}

\item Man beweise, dass die Glieder einer rekursiv oder explizit gegebenen
  Folge ganzzahlig, durch $3$ teilbar, paarweise teilerfremd, usw.{} sind.

\item Man bestimme Einerziffer, die Zehnerziffer, die ersten 5 Nachkommastellen
  eines Folgenelements.

\item Man beweise die Periodizität einer Folge.

\item Mitunter sind mehrere gekoppelte rekursive Folgen gegeben und man soll
  eine der oben genannten Aufgaben lösen (siehe MO 441134).
\end{itemize}

\subsection{Beispiele}

\begin{beispiel}\label{b-1}
(a) $a_n=a_{n-1}$, $a_1:=a$, $a\in \R $ gegeben. Hier erhält man die
  \emph{konstante Folge} $(a,a,a,\cdots)$.

(b) $a_{n+2}=a_{n}$, $a_1=a$, $a_2=b$, $a,b\in\R $ gegeben. Man erhält hier
  eine \emph{periodische Folge} der Periodenlänge $1$, falls $a=b$ nämlich die
  konstante Folge oder eine der Periode $2$, nämlich $(a,b,a,b,\cdots)$.

(c) $a_n=(a_{n-1})^ 2$, $a_1=2$. Man erhäl die Folge $(2,4,16,256,\cdots)$ mit
  der expliziten Vorschrift $\ds a_n=2^{2^{n-1}}$.  Dies ist ein Beispiel für
  eine nicht-lineare Rekursionsformel. Der Beweis, dass dies tatsächlich die
  explizite Bildungsvorschrift ist, erfolgt durch \emph{vollständige
    Induktion.}  Offenbar ist $a_1=2^{2^0}=2^1=2$ --- der Induktionsanfang ist
  erfüllt. Nun gelte die explizite Formel für ein festes $n$. Wir zeigen, dass
  sie auch für $n+1$ gilt, also $a_{n+1}=2^{2^{n}}$.

\begin{beweis} Nach der Rekursionsformel und der Induktionsvoraussetzung ist
\begin{gather*}
  a_{n+1}=(a_n)^2 \underset{\mathrm{Ind.Vor}}{=}\left(2^{2^{n-1}}\right)^2
  =2^{2^{n-1} \cdot 2}= 2^{2^n}\,,
\end{gather*}
was zu zeigen war.
\end{beweis}

(d) $a_{n+1}=q \,a_n$, $a_1=c$.  Wir lösen die Rekursion auf, indem wir in der
Rekursionsformel nacheinander $n-1$, $n-2,\dots, 1$ anstelle von $n$
einsetzen. Wir erhalten dadurch $n-1$ Gleichungen: $a_n=q a_{n-1}$, $a_{n-1}=
qa_{n-2}$,\dots, $a_2 =qa_1$.  Setzt man diese Gleichungen nacheinander in sich
ein, so erhält man
\begin{gather*}
  a_{n+1}=qa_{n}=q^2 a_{n-1} =q^3 a_{n-2}=\cdots =q^{n} a_1= c\, q^n\,.
\end{gather*}
Wir erhalten somit die geometrische Folge mit dem Faktor $q$ und dem 
Anfangsglied $c$. Der genaue Beweis erfolgt wie oben durch vollständige
Induktion.

(e) $a_{n+2}=2\,a_{n+1} -a_n$. Diese Rekursionsformel wird durch eine beliebige
arithmetische Folge (erster Ordnung) $a_n=c +d\,(n-1)$, $c,d\in\R $, erfüllt.
In der Tat ist
\begin{align*}
2\,a_{n+1} -a_n& = 2\,(c+ d\,n) -(c + d\,(n-1))= c+2\,d\,n -d\,n +d = c +
d\,(n+1)= a_{n+2}\,.
\end{align*}
Man erkennt, dass es zu einer expliziten Vorschrift durchaus mehrere
Rekursionen geben kann. So erfüllt $a_n=1$ neben der Rekursion (e) auch die in
(a).

(f) $a_{n+2}=a_{n+1} +a_n$, $a_1=a_2=1$. Dies ist die {\sc Fibonacci}-Folge.
Ihre ersten Glieder lauten
\begin{center}
\begin{tabular}{l|llllllllllll}
$n$ & $1$ & $2$ & $3$ & $4$ & $5$ & $6$ & $7$ & $8$ & $9$ & $ 10$ & $11$ & $12 $
\\ \hline 
$a_n$ & $1$ & $1$ & $2$ & $3$ & $5$ & $8$ & $13$ & $21$ & $34$ & $55$ & $89$ & $144$
\end{tabular}
\end{center}

Wir werden später die explizite Vorschrift ermitteln und zeigen, dass es zu
jedem $m\in\N $ ein $n\in \N $ gibt mit $m\mid a_n$. Viele weitere interessante
Eigenschaften über die Fibonacci-Zahlen findet man in \cite{b-Worobjow}.

\end{beispiel}

\begin{beispiel} \label{b-2}
Wir betrachten die Folge $a_n=n^2$, $n\in \N $,  der Quadratqzahlen und wollen
umgekehrt zu dieser expliziten Bildungsvorschrift eine rekursive Vorschrift 
finden. Zunächst ist
\begin{gather*}
  a_{n+1}=(n+1)^2 =n^2 +2\,n +1= a_n +2\,n +1\,.
\end{gather*}
Vergrößert man hier $n$ um $1$, so erhält man
\begin{gather*}
  a_{n+2} =a_{n+1} +2\,n +3\,.
\end{gather*}
Bildet man nun die Differenz aus diesen beiden Gleichungen, so hat man
\begin{gather*}
  a_{n+2} -a_{n+1} = a_{n+1}- a_{n} +2 \quad\text{bzw.}\quad a_{n+2}=2a_{n+1} -
  a_n +2.
\end{gather*}
Erhöht man erneut $n$ um $1$, so hat man
\begin{gather*}
  a_{n+3}= 2\,a_{n+2} -a_{+1} +2
\end{gather*}
und Differenzbildung liefert
\begin{gather*}
  a_{n+3} - a_{n+2} = 2\,a_{n+2} -3\, a_{n+1} +a_n \quad\text{bzw.}\quad
  a_{n+3} = 3\, a_{n+2} -3\, a_{n+1} + a_n\,.
\end{gather*}
Somit genügt die Folge der Quadratzahlen einer linearen rekursiven Gleichung
\emph{dritter Ordnung}. Analog kann man sich überlegen, dass die Folgen $(n^3)$
der Kuben einer linearen Rekursion vierter Ordnung genügt:
\begin{gather*}
  a_{n+4} =4\,a_{n+3} -6\, a_{n+2} +4\, a_{n+1} - a_n\,.
\end{gather*}
\end{beispiel}

\begin{beispiel}\label{b-dezimal}
Wir wollen die Ziffernfolge $0,a_1a_2a_3\cdots $ bei der
Dezimalbruchentwicklung einer rationalen Zahl
\begin{gather*}
  \frac{761}{1332}= 0{,}57\ov{132}\cdots 
\end{gather*}
betrachten.  In diesem Beispiel ist $a_1=5$, $a_2=7$, $a_3=a_6=a_9=\cdots =
1$, $a_4=a_7=a_{10}=\cdots = 3$ und $a_5=a_8=a_{11} =\cdots =2$. Offensichtlich
ist für alle $n\ge 3$
\begin{gather*}
  a_{n+3} = a_n\,.
\end{gather*}
Dies ist wieder eine lineare rekursive  Folge dritter Ordnung
mit konstanten Koeffizienten.
\end{beispiel}

\begin{beispiel}\label{b-polynom}
  Dividiert man ein Polynom $p(x)=p_0 +p_1 x +\cdots +p_r x^r $ durch ein
  Polynom $q(x)=q_0 +q_1 x+ \cdots + q_s x^s $, $q_0\ne 0$, so erhält man
  i.\,a. als Quotient eine sogenannte \emph{Potenzreihe}
  \begin{gather*}
    r(x)=a_0 +a_1 x +a_2 x^2 +\cdots +a_n x^n +\cdots\,,
  \end{gather*}
  die nicht notwendig abbricht. Die Koeffizienten $(a_n)$ des Quotienten $r(x)$
  genügen der linearen Rekursion
  \begin{gather*}
    a_{n+s} q_0 +a_{n+s-1} q_1 +\cdots + a_n q_{s} =0,\quad n+s\ge r+1\,.
  \end{gather*}
  Dies folgt durch Koeffizientenvergleich vor $x^{n+s}$ nach Ausmultiplizieren
  der Gleichung $p(x)=q(x)r(x)$.  

Zum Beispiel ist
\begin{gather*}
  \frac{1}{1-x-x^2}=1+x+2x^2 +3 x^3 +5x^4 +8 x^5 +\cdots +a_n x^n+ \cdots\,,
\end{gather*}
wobei $(a_n)$ die {\sc Fibonacci}-Folge ist. Die Funktion auf der linken Seite
bezeichnet man dann auch als \emph{erzeugende Funktion} für die Folge $(a_n)$.
\end{beispiel}

\begin{beispiel}\label{b-Reihe} 
Genügt $(a_n)$ der linearen Rekursion \rf[e-linrek] der Ordnung $k$ mit
konstanten Koeffizienten $c_1,\dots, c_k$, so genügt die Folge der
Partialsummen
\begin{gather*}
  s_n= a_1 +a_2 +\cdots +a_n
\end{gather*}
einer linearen Rekursion \rf[e-linrek] der Ordnung $k+1$. Es gilt nämlich
\begin{gather*}
  s_{n+k+1} =(1+c_1)\,s_{n+k} +(c_2 -c_1)\, s_{n+k-1} +\cdots + (c_k
  -c_{k-1})\,s_{n+1} -c_k\, s_n\,.
\end{gather*}
\end{beispiel}
%%%%%%%%%%%%%%%%%%%%%%%%%%%%%%%%%%%%%%%%%%%%%%%%%%%%%%%%
\section{Lösung der allgemeinen linearen Rekursion \rf[e-linrek]}
Die erste wichtige Tatsache ist, dass alle Folgen, die \rf[e-linrek] erfüllen
einen \emph{linearen Raum} bilden.

\begin{satz} \label{s-1}
Wenn $(x_n)$ und $(y_n)$ die Rekursion \rf[e-linrek] erfüllen, so auch die
Folgen
\begin{gather*}
  (x_n + y_n) \quad\text{und}\quad (\lam x_n),\quad \lam\in\R.
\end{gather*}
\end{satz}

Mit anderen Worten, die Summe $z_n=x_n +y_n$ von Lösungen $(x_n) $ und $(y_n)$
ist wieder eine Lösung und skalare Vielfache von Lösungen sind wieder Lösungen.

\begin{beweis} Bildet man die Summe von
\begin{align*}
x_{n+k}& =c_1\, x_{n+k-1} +c_2\,x_{n+k-2} +\cdots + c_k\,x_n\,,\\
y_{n+k}&=c_1\, y_{n+k-1} +c_2\, y_{n+k-2} +\cdots + c_k\, y_n\,,
\end{align*}
so hat man
\begin{gather*}
  x_{n+k} +y_{n+k} = c_1\,(x_{n+k-1} +y_{n+k-1}) +c_2\, (x_{n+k-2} +y_{n+k-2})
  +\cdots c_k\,( x_n +y_n), \quad n\in \N\,.
\end{gather*}
Analog gilt
\begin{gather*}
  \lam x_{n+k} =c_1\,(\lam x_{n+k-1}) +c_2\,(\lam x_{n+k-2}) +\cdots + c_k
  \,(\lam x_n)\,,
\end{gather*}
so dass $z_n=\lam x_n$ auch \rf[e-linrek] erfüllt.  

Die Angabe des Lösungsraumes ist daher äquivalent zur Angabe einer \emph{Basis}
von linear unabhängigen \emph{Fundamentallösungen}. Wir werden sehen, dass die
Dimension des Lösungsraumes stets mit der Ordnung $k$ der Rekursion überein
stimmt.
\end{beweis}

\subsection{Der Ansatz --- geometrische Folgen} Als Ansatz versuchen wir 
Lösungen der Form $a_n=q^{n}$ (also geometrische Folgen) zu finden.
Setzt man dies in \rf[e-linrek] ein, so hat man
\begin{align}
q^{n+k} &=c_1 q^{n+k-1} +\cdots + c_k q^n\quad \mid :q^n\label{e-char1} \\
q^k&= c_1q^{k-1} +\cdots +c_{k-1} q + c_k\nn\\
q^k&-c_1q^{k-1} -c_2 q^{k-2}- \cdots -c_k=0\label{e-char}.
\end{align}
Das Polynom $\chi(q) $ auf der linken Seite bezeichnet man als
\emph{charakteristisches Polynom} der Rekursion \rf[e-linrek]. Es ist ein
Polynom vom Grade $k$, das nach dem Fundamentalsatz der Algebra genau $k$
komplexe Nullstellen (gezählt in ihrer Vielfachheit) $q_1,q_2,\cdots ,q_k$
besitzt.

\subsubsection{Alles einfache Nullstellen}
Im einfachsten Fall sind diese $k$ Nullstellen alle voneinander verschieden.
Dann erfüllen die $k$ voneinander verschiedenen geometrischen Folgen
\begin{gather*}
(q_1^n),\, (q_2^n),\dots,(q_k)^n
\end{gather*}
alle die Rekursion \rf[e-linrek]. Nach Satz\,\ref{s-1}
ist dann auch
\begin{align}\label{e-allg}
x_n=A_1 q_1^n +A_2 q_2^n +\cdots  +A_k q_k^n
\end{align}
mit beliebigen Koeffizienten $A_1,\dots ,A_k$ eine Lösung der Rekursion.

Wir wollen zeigen, dass diese $k$ Lösungen tatsächlich eine Basis bilden,
d.\,h., dass sich \emph{jede} rekursive Folge \rf[e-linrek] in der obigen Form
\rf[e-allg] mit gewissen Koeffizienten $A_i$, $i=1,\dots , k$ schreiben lässt.
Dies liefert gleichzeitig eine Vorschrift, wie man die explizite
Bildungsvorschrift erhält, wenn $(x_n)$ die Rekursion erfüllt und zusätzlich
die $k$ Anfangsglieder $x_0,x_1,\dots , x_{k-1}$ gegeben sind.  

Zunächst werden in \rf[e-allg] nacheinander $n=0,1,\dots, k-1$ eingesetzt; man
erhält aus den ersten $k$ Folgenglieder ein $k\times k$ lineares
Gleichungssystem für die Koeffizienten $A_i$:
\begin{align*}
\begin{matrix}
A_1   &  +& A_2 +\cdots &  +  & A_k & = & x_0,\\
q_1A_1 & + & q_2 A_2 +\cdots &  +&  q_k A_k &= & x_1,\\
\vdots &&&&&& \vdots\\
q_1^{k-1} A_1&  + & q_2^{k-1} A_2 +\cdots & + & q_k^{k-1} A_k & = & x_k.
\end{matrix}
\end{align*}
Man kann zeigen, dass dieses GS für beliebige Wahl der Anfangsglieder
$x_0,x_1,\dots,x_{k-1}$ eine eindeutige Lösung $(A_1,A_2,\dots A_k)$ besitzt.
Somit haben wir eine Folge $(x_n')$ gefunden, die die Rekursion \rf[e-linrek]
erfüllt und außerdem in den ersten $k$ Gliedern von $(x_n)$ überein stimmt.  Da
durch diese beiden Vorgaben die Folge jedoch schon eindeutig bestimmt ist, gilt
$(x_n')=(x_n)$.

\begin{beispiel} Die Fibonacci-Folge $a_{n+2}=a_{n+1} +a_n$, $a_1=a_2=1$ hat die 
charakteristische Gleichung $q^2 =q+1 $ mit den beiden rellen Lösungen $q_{1,2}
=\half \pm \half \sqrt{5}$. Die allgemeine Lösung lautet also
\begin{gather*}
a_n=A \left(\frac{1+\sqrt{5}}{2}\right)^n +B \left(\frac{1-\sqrt{5}}{2}\right)^n
\end{gather*}
Beachtet man $a_0=0$, $a_1=1$, so hat man
\begin{gather*}
0=A+B,\quad 1=\half\left(A+B\right) +\frac{\sqrt{5}}{2}\left(A-B\right).
\end{gather*}
Dies liefert $A=-B=\frac{1}{\sqrt{5}}$, also
\begin{gather*}
a_n=\frac{1}{\sqrt{5}} \left(q_1^n -q_2^n\right).
\end{gather*}
\end{beispiel}

\subsubsection{Mehrfache Nullstellen}
Angenommen, das charakteristische Polynom lautet $\chi(x)=(x-q)^m\chi_1(x)$,
$m\le k$, hat also eine $m$-fache Nullstelle $q$. Wir zeigen, dass dann die $m$
Folgen
\begin{gather*}
(q^n), (nq^n),\cdots, (n^{m-1} \, q^n)
\end{gather*}
\rf[e-linrek] erfüllen.

Wir zeigen dies für $m=2$. Die wesentliche Eigenschaft, die benutzt wird: Ist
$q$ eine $m$-fache Nullstelle von $\chi(x) $, so ist $q$ eine $(m-1)$-fache
Nullstelle der Ableitung $\chi'(x)$.  Insbesondere gilt für $m=2$
\begin{gather*}
kq^{k-1}= c_1 (k-1)q^{k-2} +\cdots + c_{k-2} \cdot 2q +c_{k-1}.
\end{gather*}
Multipliziert man diese Gleichung mit $q^{n+1}$ und addiert man hierzu
das $n$-fache von  \rf[e-char1]
\begin{gather*}
n q^{n+k} = nc_1 q^{n+k-1} +nc_2 q^{n+k-2} +\cdots +n c_k q^n,
\end{gather*}
so erhält man
\begin{gather*}
(n+k)q^{n+k} = c_1 (n+k-1)q^{n+k-1} +c_2 (n+k-2)q^{n+k-2} +\cdots + c_k nq^n,
\end{gather*}
mit anderen Worten, $b_n= n\, q^n $ erfüllt die Rekursion \rf[e-linrek].

\begin{beispiel} 
$a_{n+2}=2a_{n+1} -a_n$. Es ist $\chi(x)=x^2 -2x +1=(x-1)^2 $ mit doppelter
  Nullstelle $q=1$. Somit bilden $(1) $ und $(n)$ ein Fundamentalsystem von
  Lösungen.
\end{beispiel}

Wenn man beide Methoden (einfache Nullstellen und mehrfaache Nullstellen)
koppelt, erhält man zu jeder charakteristischen Gleichung ein Fundamentalsystem
von $k$ unabhängigen Folgen, die den Lösungsraum aufspannen.

\emph{Bemerkung:} Die Partialsummen $s_n=a_1 +a_2 +\cdots +a_n$ einer linearen
rekursiven Folge $k$-ter Ordnung bilden eine lineare rekursive Folge $(k+1)$ter
Ordnung.

\section{Aufgaben}

\begin{aufgabe} %%% MONOID Heft 83, September 2005
Es sei $(a_n)$ die Fibonacci-Folge und 
\begin{gather*} 
  s=\sum_{n=1}^\infty \frac{a_{n}}{10^{n+1}} =\frac{1}{100} +\frac{1}{1000}
  +\frac{2}{10000} +\cdots .
\end{gather*}
Zeige, dass die Reihe konvergiert und dass $s=\frac{1}{89}$ ist.
\end{aufgabe}

\begin{beweis}
Da die Fibonacci-Folge im wesentlichen wie $q^n$ wächst mit $q=(\sqrt{5} +1)/2$
und $q/10 <1$, kann die Reihe durch eine konvergente geometrische Reihe
majorisiert werden, also konvergiert sie.

Sei $b_n=\frac{a_n}{10^{n+1}}$. Dann  haben wir eine Rekursion für $b_n$,
$b_{n+2} =\frac{1}{10} b_{n+1} +\frac{1}{100} b_n$.
Summiert man diese Gleichung auf für $n=1,\dots , k$, so hat man
\begin{gather*}
s_{k+2} - b_1 -b_2 = \frac{1}{10}( s_{k+1} -b_1) + \frac{1}{100} s_k.
\end{gather*}
Bildet man nun den Limes $k\to \infty $, so hat man
\begin{gather*}
s-0.011 =0.11 s -0.001 \RRA 0.89s=0.1\RRA s=\frac{1}{89}.
\end{gather*}
\end{beweis}

\begin{aufgabe} %%% Engel, S. 209 %%%
Man finde eine explizite Formel für
\begin{gather*}
a_{n+1}= \frac{1}{16} \left( 1+4 a_n + \sqrt{1 +24 a_n}\right), \quad a_1=1.
\end{gather*}
\end{aufgabe}

\begin{loesung} 
\begin{gather*}
a_n=\third +\frac{1}{2^n} +\frac{2}{3\cdot 4^n}.
\end{gather*}
Wenn man die ersten Glieder berechnet, könnte man vermuten, dass $a_n$ stets
\emph{rational} ist, dass
\begin{gather*}
b_n= \sqrt{1 +24 a_n}
\end{gather*}
stets eine rationale Zahl ist. Mit dieser Substitution wäre $a_n = (b_n^2
-1)/24$. Setzt man dies in die Rekursion ein, so hat man nach einigen
Umformungen eine binomische Formel:
\begin{gather*}
b_{n+1}^2  = 9/4 + b_n^2/4  +3/2 b_n = (3/2 +b_n/2)^2. 
\end{gather*} 
Da alle Glieder positiv sind, kann man die Wurzel ziehen und hat $b_{n+1} =
b_n/2 + 3/2.$

Setzt man dies immer nacheinander in sich selbst ein und beachtet $b_1 =5$, so
hat man $b_n=3 + 1/2^{n-2}.$ Dies liefert die Lösung für $a_n$.
\end{loesung}

\begin{aufgabe}
Es sei $a_n$ die Anzahl der Möglichkeiten, ein Rechteck vom Format $2\times n$
in Dominosteine vom Format $2\times 1$ zu zerlegen.

Man bestimme $a_n$.
\end{aufgabe}

\begin{aufgabe} Gegeben sei eine Folge $(a_n)$ mit $a_1=A$, $a_2=B$ und
\begin{gather*}
  a_n=\frac{a_{n-1}^2 +C}{a_{n-2}},\quad n\ge 3.
\end{gather*}
Beweisen Sie, dass aus der Ganzzahligkeit von
\begin{gather*}
  A, \, B,\,\,\frac{A^2 +B^2 +C}{AB}
\end{gather*}
folgt, dass alle Folgenglieder $a_n$ ganzzahlig sind.
\end{aufgabe}

\subsubsection*{Teleskopsummen}

\begin{aufgabe} 
Es sei $a_1= \half $ und $\ds a_{n+1} =\frac{n+1}{n^2 +2n} \,a_{n}$, $n\ge 1$.

Ermitteln Sie eine explizite Bildungsvorschrift für $(a_n)$ und berechnen Sie
$\sum_{n=1}^\infty a_n$.
\end{aufgabe}

\begin{loesung}
(a) Zunächst bestimmt man die ersten Folgenglieder; man erkennt schnell, dass
  es alles Brüche sind mit Zähler $1$, deren Nenner schnell wachsen.
\begin{center}
\begin{tabular}{l||lllll}
$n$ & $1$ & $2$ & $3$ & $4 $ & $5$\\  \hline
$a_n$ & $\half $ & $\third $ & $\frac{1}{8} $ & $\frac{1}{30} $ & 
$\frac{1}{144}$ 
\end{tabular}
\end{center}

Man kommt somit leicht auf die Vermutung $\ds a_n=\frac{1}{(n-1)! (n+1)}$, die
man mit vollständiger Induktion beweist: Für $n=1,2,3,4,5$ stimmt die
Vermutung. Angenommen, die Aussage gilt für ein festes $n$. Wir habe zu zeigen,
dass sie dann auch für $n+1$ gilt: $\ds a_{n+1}= \frac{1}{n! (n+2)}.$

In der Tat gilt wegen der gegebenen Rekursionsformel und nach
Induktionsvoraussetzung
\begin{gather*}
a_{n+1}= \frac{n+1}{n^2 +2n} \,a_{n} = \frac{n+1}{n(n+2)} \, \frac{1}{(n-1)!
  (n+1)}=\frac{1}{n! (n+2)},
\end{gather*}
was zu zeigen war.

(b) Durch Erweitern von $a_n $ mit $n$ hat man $\ds a_n=\frac{n}{(n+1)!}$. Die
gesuchte unendliche Reihe ließe sich berechnen, wenn man sie als
„Teleskopsumme“ darstellen könnte. Das ist aber der Fall, denn
\begin{gather*}
a_n=\frac{n}{(n+1)!} = \frac{(n+1) -1 }{(n+1)!} =\frac{1}{n!}
-\frac{1}{(n+1)!}.
\end{gather*} 
Somit ist
\begin{gather*}
\sum_{k=1}^n a_k=\sum_{k=1}^n\left(\frac{1}{k!} -\frac{1}{(k+1)!}\right)
=1-\frac{1}{(n+1)!}.
\end{gather*}
Da der Subtrahend für $n\to \infty $ gegen $0$ geht, ist $\sum_n a_n=1$.
\end{loesung}

\begin{aufgabe}
  Man ermittle die Summe der unendlichen Reihe
\begin{gather*}
s=\frac{1}{1\cdot 2\cdot 3\cdot 4} +\frac{1}{2\cdot 3\cdot 4\cdot 5} +\cdots.
\end{gather*}
\end{aufgabe}

\begin{loesung} Mit Hilfe der \emph{Partialbruchzerlegung} 
\begin{gather*}
\frac{1}{(x-2)(x-1)x(x+1)} =\frac{A}{x-2} +\frac{B}{x-1} +\frac{C}{x}
+\frac{D}{x+1}
\end{gather*}
erhält man $A=\frac{1}{6}$, $B=-\half$, $C=\half$, $D=-\frac{1}{6}$, so dass
\begin{gather*}
\frac{1}{n}{n+1}{n+2}{n+3}=\frac{1}{6} \frac{1}{n} -\half \frac{1}{n+1} +\half
\frac{1}{n+2} -\frac{1}{6} \frac{1}{n+3}.
\end{gather*}
Summiert man von $n=1$ bis $\infty$, so bleiben die ersten 6 Summanden stehen
(von $n=1,2,3 $ kommend)
\begin{gather*}
s=\left(\frac{1}{6} 1 -\half\,\half +\half \third\right) +\left( \frac{1}{6}
\,\frac{1}{2} -\half\,\third\right) +\frac{1}{6} \,\third = \frac{1}{18}.
\end{gather*}
\end{loesung}

\begin{aufgabe}
Es sei $w_1=1$ und $\ds w_{n+1}= 1 +\frac{n}{w_n}$, für $n\ge 1$.  Zeigen Sie,
dass stets gilt
\begin{gather*}
\sqrt{n} \le w_n \le \sqrt{n}+1.
\end{gather*}
\end{aufgabe}

\begin{beweis} Wir beweisen dies durch vollständige Induktion über $n$.
Offenbar gelten beide Ungleichungen für $n=1$, denn $1 \le w_1 =1 \le 2$.
Angenommen, beide Ungleichungen gelten für ein festes $n$.  Das heißt,
\begin{align}\label{u-1}
\sqrt{n} \le w_{n} \le \sqrt{n} +1
\end{align}
Wir müssen dann zeigen, dass sie auch für $n+1$ gelten, dass also
\begin{align}\label{u-2}
\sqrt{n+1} \le w_{n+1} \le \sqrt{n+1} +1
\end{align}
gilt.  Dazu benutzen wir im wesentlichen die Monotonie und die Positivität der
Funktion $f(x)=\sqrt{x}$. Durch Reziprokenbildung von \rf[u-1] drehen sich sie
Relationszeichen um und wir haben:
\begin{align}
\frac{1}{\sqrt{n}}& \ge \frac{1}{w_n} \ge \frac{1}{\sqrt{n} +1} 
\quad \left. \right| \cdot n \quad \quad  \mid +1  \nn\\
\sqrt{n} +1 &\ge \frac{n}{w_n}+ 1 \ge 1+\frac{n}{\sqrt{n} +1} \nn \\ 
\sqrt{n} +1 &\ge w_{n+1} \ge 1+\frac{n}{\sqrt{n} +1}.
\label{u-3}
\end{align}
Wegen $\sqrt{n+1}> \sqrt{n} $ können wir auf der linken Seite fortsetzen zu
$\sqrt{n+1} +1 \ge w_{n+1}$, was den ersten Teil der Induktionsbehauptung
\rf[u-2] beweist.

Offenbar gilt für alle reellen Zahlen $A$ und $B$ mit $A>B\ge 1$,
$(A-B)(A-1)\ge 0$. Dies gilt insbesondere für $A=\sqrt{n+1}$ und $B=\sqrt{n}$,
$n\in\N$, also
\begin{gather*}
(\sqrt{n+1}-\sqrt{n})(\sqrt{n+1} -1)=n+1 -\sqrt{n(n+1)} -\sqrt{n+1} +\sqrt{n}
  \ge 0. 
\end{gather*}
Bringt man alle Summanden bis auf den ersten auf die rechte Seite, so hat man
\begin{align*}
n&\ge \sqrt{n}\, \sqrt{n+1} -\sqrt{n} +\sqrt{n+1} -1 =
(\sqrt{n+1} -1)(\sqrt{n} +1)\\
\RRA \frac{n}{\sqrt{n} +1} &\ge \sqrt{n+1} -1\\
\RRA 1+  \frac{n}{\sqrt{n} +1} &\ge \sqrt{n+1} .
\end{align*}
Zusammen mit \rf[u-3] folgt der zweite Teil $w_{n+1} \ge \sqrt{n+1} $ der
Induktionsbehauptung \rf[u-2].
\end{beweis}

\begin{aufgabe} (A 156)  Gegeben sei die Folge $(x_n)$ mit $x_0=5$ und 
\begin{gather*}
x_{n+1} = x_n +\frac{1}{x_n}, \quad n\ge 0.
\end{gather*}
Beweise, dass
\begin{gather*}
45 < x_{1000} < 45{,}1.
\end{gather*}
\end{aufgabe}

\begin{beweis} Wir zeigen, dass für alle natürlichen Zahlen $n\in \N $ gilt
\begin{align}\label{in-1}
\sqrt{25 +2n} < x_n < 0{,}1 +\sqrt{25 +2n}.
\end{align}
Aus $ x_n=x_{n-1} + \frac{1}{x_{n-1}}$ folgt durch Quadrieren $x_n^ 2=x_{n-1}^
2 +2 +\frac{1}{x_{n-1}^2}> x_{n-1}^ 2 +2$.  Durch Auflösen dieser Rekursion hat
man
\begin{gather*}
x_n^2 > x_0^2 +2n = 25 +2n
\end{gather*}
und die linke Seite der Ungleichung folgt.  Da $(x_n)$ streng monoton wachsend
ist, gilt ${x_n\ge x_0=5}$ und somit $-\frac{1}{x_{n-1}}\ge -0,2$. Also ist
\begin{gather*}
x_{n-1}=x_n -\frac{1}{x_{n-1}} \ge x_n -0,2.
\end{gather*} 
Hieraus und mit $2x_{n-1}(x_n -x_{n-1})=(x_{n-1} +x_{n-1})(x_n -x_{n-1})=2 $
folgt
\begin{gather*}
2\ge (x_n -0,2 +x_{n-1})(x_n- x_{n-1})=x_n^ 2 -x_{n-1}^2 -0,2 (x_n -x_{n-1}).
\end{gather*}
Durch Auflösen dieser Rekursion hat man
\begin{gather*}
x_n^2 -0,2 x_n -2n -24 \le 0.
\end{gather*} 
Da $0,1 +\sqrt{2n +24,01} $ eine Wurzel der entsprechenden quadratischen
Gleichung ist, ist
\begin{gather*}
x_n\le 0,1 +\sqrt{2n +24,01} < 0,1 +\sqrt{2n +25}.
\end{gather*} 
Für $n=1000$ erhält man genau die Behauptung.
\end{beweis}

%%% Bulgarisches Buch, Seite 16, Aufg. 76 %%%%
\begin{aufgabe}
Gegeben sei eine Folge $(a_n)_{n\ge 0}$ von ganzen Zahlen mit
\begin{gather*}
3 a_{n}= a_{n-1}  +a_{n+1}, \quad n\ge 1.
\end{gather*}
Beweisen Sie, dass $5a_n^2 +4( a_0^2 +a_1^2 -3a_0a_1) $ stets eine Quadratzahl
ist.
\end{aufgabe}

%%% Bulg Aufg. 77, S. 16 %%%%%%%%%%%%%%%
\begin{aufgabe} Gegeben sei die rekursive Folge $(a_n)_{n\in\N } $ mit
\begin{gather*}
a_1=a_2=1,\quad \quad a_{n}=\frac{a_{n-1}^2 +2}{a_{n-2}}, \quad n\ge 3.
\end{gather*}
Beweisen Sie, dass alle Folgenglieder ganzzahlig sind.
\end{aufgabe}

\begin{aufgabe} Gegeben sei die rekursive Folge $(a_n)_{n\in\N } $ mit
\begin{gather*}
a_1=a_2=a_3=1,\quad a_{n}=\frac{a_{n-1}a_{n-2} +1}{a_{n-3}}, \quad n\ge 4.
\end{gather*}
Beweisen Sie, dass alle Folgenglieder ganzzahlig sind.
\end{aufgabe}

\begin{aufgabe} 
%%% Bulg Aufgabe 79, S. 17 %%%%
Es seien $(x_n)$ und $(y_n)$ rekursive Folgen, gegeben durch
\begin{alignat*}{3}
x_1&=1,\,\,& x_2& = 1,& \,\, x_{n+2}& =x_{n+1} +2 x_{n} ,\quad n\in\N,\\
y_1&= 1,&\,\, y_2&= 7,&\,\, y_{n+2}& = 2y_{n+1} +3 y_{n},\quad n\in \N.
\end{alignat*}
Beweisen Sie, dass außer $x_1=y_1=1$ die beiden Folgen keine gemeinsamen
Folgenglieder besitzen.
\end{aufgabe}

\begin{beweis} Wir betrachten beide Folgen modulo $8$ und erhalten
\begin{align*}
x_n&\equiv 1,1,3,5,3,5,\cdots \mod{8},\\
y_n&\equiv  1,-1,1,-1,\cdots \mod{8}.
\end{align*}
Folglich können nur die ersten Folgenglieder übereinstimmen.
\end{beweis}

%%% Bulgar Aufg. 80 S. 17 %%%%
\begin{aufgabe} 
Es sei $f(x) $ ein Polynom mit ganzzahligen Koeffizienten und $f(0)=f(1)=1$ und
$a_1\in \Z $ beliebig gegeben.  Ferner sei $a_{n+1} =f(a_n)$ für alle $n\ge 1$.

Beweisen Sie, dass die Folgenglieder paarweise teilerfremd sind.
\end{aufgabe}

%% bulg. Aufg. 81 S.17 %%%%
\begin{aufgabe} (einfach)
Gegeben sei die ganzzahlige Folge $a_n=n^2+1$.

Beweisen Sie, dass es unter den Folgengliedern unendlich viele zusammengesetzte
Glieder der Form $a_n=a_k \,a_l$ gibt.
\end{aufgabe}

\begin{beweis} Etwa $a_{l^2 +l+1}= a_{l} \,a_{l+1}$ oder 
$a_{2l^3 + l}=a_l \, a_{2l^2}$.
\end{beweis}

%% bulg Aufg. 82, S.17 %%%
\begin{aufgabe}
Zeigen Sie, dass es unter den Zahlen $2^n-3$, $n\in \N $, unendlich viele gibt,
die paarweise teilerfremd sind.
\end{aufgabe}


\begin{aufgabe} (einfach)
%%%% Korrespondenzseminar %
Die Folge $a_n$ ist bestimmt durch $a_1=1337$ und $a_{2n+1}=a_{2n}=n-a_n$ für
ganze Zahlen $n>0$.

Bestimme den Wert von $a_{2004}$.
\end{aufgabe}

\begin{loesung}
Durch Rückwärtsarbeiten findet man Schritt für Schritt
\begin{alignat*}{3}
a_{2004}&= 1002 -a_{1002},&\quad a_{1002}&=501 -a_{501},&\quad
a_{501}&=a_{500}=250 -a_{250},\\ 
a_{250}&= 125 -a_{125},&\quad a_{125}&=a_{124}=62 -a_{62},&\quad a_{62}&= 31
-a_{31},\\
a_{31}&=a_{30}=15 -a_{15},&\quad a_{15}&=a_{14} = 7- a_7,& \quad
a_7&=a_6=3-a_3,\\ 
a_3&=a_2=2-a_1.
\end{alignat*}
Setzt man dies nacheinander ineinander ein, so hat man
\begin{align*}
a_{2004}&= 1002-501 + 250 -125 +62 -31 +15 -7 +3 -1+ a_1\\
&= 501 +125 +31 +8 +2 +a_1= 667 +1337 =2004.
\end{align*}
\end{loesung}

\begin{aufgabe} 
%%%  MO 431336 %%%%%
Die reellen Zahlen $x_1,x_2,\cdots $ seien durch die Bildungsvorschrift
\begin{gather*}
x_1=1,\quad x_{k+1} = \frac{1}{1+x_k}, \quad k\in\N
\end{gather*}
gegeben. Man untersuche, ob  $x_{2004}^2 +x_{2004} -1 $ positiv, negativ oder
gleich $0$ ist.
\end{aufgabe}

\begin{loesung} 
Es sei $f(x)=x^2 +x -1$ mit den Nullstellen $q_{1,2}=\frac{-1\pm \sqrt{5}}{2}$,
$q=q_1$. Man zeigt:
\begin{gather*}
x_{2n} < q,\quad x_{2n+1} > q.
\end{gather*}
Somit gilt für alle geraden Folgenglieder $f(x_{2n}) < 0 $ und für alle
ungeraden Folgenglieder $f(x_{2n-1}) >0$.
\end{loesung}

\begin{aufgabe}
%%%% Vorbereitungsseminar 15.1.2005, Klausur, Klasse 9/10 %%%%%%%%%%
Bestimme die Einerziffer und die erste Ziffer nach dem Komma von $b_{2005}$,
wenn
\begin{quote}
(a) $ b_n=(2+\sqrt{3})^{n}$.\\
(b) $ b_{n}= (3+\sqrt{7})^{n}  $.
\end{quote}
\end{aufgabe}

\begin{loesung}
(b) Wir betrachten die rekursive Folge zweiter Ordnung
\begin{gather*}
a_n= (3+\sqrt{7})^n + (3-\sqrt{7})^n
\end{gather*}
mit $q_{1,2}=3\pm \sqrt{7}$. Wegen $q_1 +q_2 =6 $ und $q_1q_2=2$ lautet die 
charakteristische Gleichung für $(a_n)$, $0=x^2 -6x +2$ und somit gilt
\begin{gather*}
a_{n+2} = 6 a_{n+1} -2 a_n,\quad a_0= 2,\quad a_1=6,\quad a_2=32.
\end{gather*}
Wegen der Rekursion gilt $a_n\in \N$ und wegen $c_n=(3-\sqrt{7})^n < 0,1$, hat
$b_n=a_n -c_n$ als erste Stelle nach dem Komma stets eine $9$.  Um die
Einerstelle von $b_{2005}$ zu ermitteln betrachten wir $a_n\mod 10$.  Diese
Folge ist periodisch mit der Periodenlänge $24$:
\begin{gather*}
(a_n\mod{10})=(2,6,2,0,6,6,4,2,4,0,2,2,8,8,0,4,4,6,8,6,0,8,8,2,6,\cdots).
\end{gather*}
Somit gilt $a_n \equiv a_{n+24} \mod{10}$, also wegen $2005\equiv 13\mod{24} $
haben wir  $a_{2005}\equiv a_{13} \equiv 4\mod{10}$.
Somit ist die Einerstelle von $b_{2005} $ gleich $4-1=3$.

\emph{Bemerkung:} Da als Reste modulo $10 $ nur die fünf geraden Zahlen
$0,2,4,6,8$ in Frage kommen, gibt es maximal $5^2=25 $ Paare $(r_1,r_2)$ von
Resten modulo $10$, die hier auftreten. Erstaunlicher Weise treten alle Paare
tatsächlich auf bis auf $(0,0)$, welches ein Orbit für sich ist.

(a) Analog zu (b). Die Rekursion lautet hier $a_{n+2}= 4 a_{n+1} -a_n$.  Die
Periodenlänge modulo $10$ ist hier gleich $3$.
\end{loesung}

\begin{aufgabe} Es sei $m\in \N $ eine fixierte natürliche Zahl.

(a) Beweise, dass es eine {\sc Fibonacci}-Zahl $a_n$ gibt, die durch $m$
  teilbar ist.

(b) Zeige, dass es eine {\sc Fibonacci}-Zahl gibt, die auf $m $ Neunen endet.
\end{aufgabe}

\begin{beweis} 
(a) Wir betrachten wieder sämtliche Paare von Resten $(a_n, a_{n+1}) $ modulo
  $m$. Da es höchstens $m\times m =m^2 $ solcher Paare gibt, ist die
  Periodenlänge von $(a_n\mod{m})$ höchstens $m^2$. Nun ist aber
\begin{gather*}
a_0=0,\quad a_{-1} =1,\quad a_{-2}=-1.
\end{gather*}
Daher treten die Reste $0$, $1$ und $-1$ modulo $m$ stets auf.

(b) Man betrachte als Modul $10^m $ und die Folgenglieder, welche kongruent
$-1$ modulo $10^m$ sind.
\end{beweis}

\begin{aufgabe} Wie lautet die Einerziffer von $a_{2005}$ bei der 
Fibonacci-Folge $(a_n)$?
\end{aufgabe}
 
\begin{aufgabe}
Die Zahlenfolge $\{a_n\}$ ist gegeben durch die Anfangsbedingungen
$a_0=1,\ a_1=\frac13$ und die Rekursionsbeziehung
\[a_n=\frac23\,a_{n-1} - a_{n-2}\quad\text{für } n\ge 2.\]

Zeigen Sie, dass es ein Folgenglied $a_n>0{,}99999$ gibt.
\end{aufgabe}

\begin{loesung} 
Die charakteristische Gleichung der linearen Rekursion zweiter Ordnung $ q^2
=\frac{2}{3}\,q -1 $ hat die komplexen Wurzeln
\begin{gather*}
q_{1,2}=\third \pm\sqrt{\frac{1}{9} -\frac{9}{9}} =\third\left( 1\pm
2\sqrt{2}\ii\right) .
\end{gather*}
Man beachte, dass $q_1 q_2=1$ und $q_2=\ov{q_1}$, $q=q_1$, beide auf dem
Einheitskreis liegen.  Der Ansatz $a_n=Aq_1^n +Bq_2^n $ führt auf $1=A+B$ und
$\third= \third(A+B) +\third \,2\sqrt{2}\ii (A-B)$ und somit auf
$A=B=\half$. Somit gilt
\begin{gather*}
a_n=\half\left( q^n +\ov{q^n}\right) =\Re (q^n).
\end{gather*}
Ist $q=\ee^{\ii \vp}$ eine $N$-te Einheitswurzel, so ist $a_N=1$ und die
Behauptung ist gezeigt. Ist hingegen $\vp$ kein rationales Vielfaches vom
$2\pi$, etwa $\vp = 2\pi\al$, $\al\not\in \Q$, dann gibt es zu $\ve>0$ stets
$N,M\in\N $, so dass $0< N\al -M< \ve$ (rationale Zahlen sind dicht in $\R $
und approximieren die irrationalen).  Somit gilt $2\pi M < N\vp < 2\pi M +2\pi
\ve$. Das liefert
\begin{gather*}
\Re\left(\ee^{2\pi \ii \ve} \right) < \Re q^N < 1.
\end{gather*}
und beweist die Behauptung, da die linke Seite beliebig dicht an $1$ heran
kommt.
\end{loesung}

\subsection*{Aufgaben aus dem bulgarischen Aufgabenbuch}
 \S 3 Folgen, Nr. 2 Rekursive Folgen.

\begin{aufgabe}(A 150, schwer)
Es sei $x_n$ eine rekursive Folge von Polynomen in $x$, gegeben durch
\begin{gather*}
x_{n+1} = x\cdot x_n -x_{n-1}, \quad x_0=2,\quad x_1= x.
\end{gather*}
Beweisen Sie, dass das Polynom
\begin{gather*}
(x^2-4) (x_n^2 -4)
\end{gather*}
ein vollständiges Quadrat ist.
\end{aufgabe}

\begin{beweis} 
Das Auflösen der Rekursion liefert $x_n=q_1^n +q_2^n$, wobei
\begin{gather*}
q_{1,2}=\half\left(x \pm \sqrt{x^2 -4}\right)
\end{gather*}
die Wurzeln der charakteristischen Gleichung $q^2 -xq +1 =0$ sind. Somit gilt
\begin{gather*}
x_n^2-4 =(q_1^n +q_2^n)^2 =q_1^{2n} +2 +q_2^{2n } -4 =(q_1^n -q_2^n)^2
=(q_1 -q_2)^2(q_1^{n-1} +q_1^{n-2}q_2 +\cdots +q_2^{n-1})^2.
\end{gather*}
Weil $(q_1 -q_2)^2 = x^2 -4$  und $x_n^2 -4$ beides Polynome sind, ist auch
\begin{gather*}
A(x)= q_1^{n-1} +q_1^{n-2}q_2 +\cdots +q_2^{n-1}
\end{gather*} 
ein Polynom mit ganzen Koeffizienten (Welche Rekursion?). Also gilt
\begin{gather*}
(x^2-4)(x_n^2 -4)=(x^2-4)^2 A(x)^2=((x^2-4) A(x))^2.
\end{gather*}
\end{beweis}

\begin{aufgabe} (A 149) Die Folge $a_n$ sei gegeben durch $a_{n+2}=4a_{n+1}
  -a_n$, $ a_1= 4$, $a_2=14$.

Beweisen Sie, dass das Dreieckk mit den Seitenlängen $a_n-1$, $a_n$, $a_{n} +1$
einen ganzzahligen Flächeninhalt besitzt.
\end{aufgabe}
\begin{beweis} 
Das folgt mit der {\sc Heron}schen Dreieicksformel und der obigen A 150 mit
$x=4$.
\end{beweis}

\begin{aufgabe} (A 153) Beweisen Sie, dass es genau eine Folge 
$(a_0, a_1,\dots )$ von positiven Zahlen gibt mit $a_0=1$ und 
$a_{n+2}=-a_{n+1} +a_n$.
\end{aufgabe}

\begin{beweis} In der expliziten Darstellung $a_n=Aq_1^n  +B q_2 ^n$ mit 
$q_{1,2}=\half\left(-1 \pm \sqrt{5}\right)$ ist $0< q_1 < 1$ und $q_2 <-1$.
  Daher muss der Koeffizient $B$ vor $q_2^n $ verschwinden und es bleibt $A=1$
  wegen $a_0=1=Aq_1^0$.
\end{beweis}

\begin{aufgabe} (A 168) Es  sei $(a_n)$ eine Folge mit $0\le a_n \le 1 $ und
$a_{n+2} -2a_{n+1} +a_n \ge 0$ für alle $n\in \N $. Man beweise, dass $0\le
  a_{n}-a_{n+1} \le \frac{2}{n+1}$.
\end{aufgabe}

\begin{aufgabe} (MO 331336) Man ermittle für jede natürliche Zahl $n$ die
größte Zweierpotenz, die ein Teiler  der Zahl $\left[(4+\sqrt{18})^n\right]$
ist.

\emph{Hinweis:} Für eine reelle Zahl $r$ bezeichnet $[r] $ die größte ganze
Zahl, kleiner oder gleich $r$, also $[r]\le r < [r] +1$ und $[r]\in \Z $.
\end{aufgabe}

\begin{loesung} Es sei $a_n=(4+\sqrt{18})^n + (4-\sqrt{18})^n$. Dann gilt
$a_0=2$, $a_1=8$ und $a_{n+2}= 8 a_{n+1} + 2 a_n$, $n\ge 2$. Wegen $ -1 <
  4-\sqrt{18} <0$ ist $b_n= (4+\sqrt{18})^n$ für gerades $n$ immer etwas
  kleiner und für ungerades $n$ immer etwas größer als die nächstgelegene ganze
  Zahl $a_n$. Da $a_n$ immer gerade ist, ist $[b_n]$ für gerades $n$ immer
  ungerade. Für ungerades $n =2k -1$ vermutet man nach erstem Probieren, dass
  die maximale Zweierpotenz $2^{k+2}$ ist und in $a_{2k} $ ist sie $2^{k+1}$.
  Dies beweist man mit vollständiger Induktion mittels Rekursionsformel.
\end{loesung}

\begin{aufgabe} (MO 381323)
Eine Zahlenfolge $(x_n)$ sei durch das rekursive Bildungsgesetz
\begin{gather*}
x_{n+1} = \left( \frac{n}{3} +\frac{1}{n}\right) x_n ^2 -\frac{n^3}{3} +1,
\quad n=1,2,\dots
\end{gather*}
und ihr Anfangsglied $x_1=1 $ gegeben. Bestimmen Sie $x_{1999} $.
\end{aufgabe}

\begin{aufgabe} (60th William Putnam Mathematical Competition 1999)
A-3. Betrachten Sie die Reihenentwicklung
\begin{gather*}
\frac{1}{1-2x -x^2} =\sum_{n=0}^\infty a_n x^n.
\end{gather*}
Zeigen Sie, dass für alle natürlichen Zahlen $n\in\N_0$ gilt
\begin{gather*}
a_n^2 +a_{n+1} ^2 = a_m
\end{gather*}
für ein gewissens $m\in \N $.
\end{aufgabe}

\begin{loesung}
Man hat für $(a_n)$ die Rekursion 2.Ordnung:
\begin{gather*}
a_{n+1} = 2 a_{n} + a_{n-1},\quad a_0=1, \, a_1 = 2.
\end{gather*}
Definiert man $ b_{2n}= a_{n}^2 +a_{n-1}^2 $ und $ b_{2n+1}= a_n\,(a_{n-1}
+a_{n+1})$, so hat man $2\,b_{2n+1} +b_{2n} = b_{2n+2}$ und $2\,b_{2n}+b_{2n-1}
= b_{2n+1} $, so dass $(b_n)$ dieselbe Rekusion wie $(a_n)$ erfüllt. Ferner ist
$b_0=1$ und $b_1=2$ und damit $a_n=b_n$.
\end{loesung}

%\nein{

\begin{thebibliography}{Wor77}

\bibitem[Eng98]{b-Engel} A.~Engel.  \newblock \emph{Problem-solving
  strategies}.  \newblock Springer, New York, 1998.

\bibitem[Mar77]{b-Markuschewitsch} A.~I. Markuschewitsch.  \newblock
  \emph{{R}ekursive {F}olgen}.  \newblock Number~xi in Kleine
  Erg{\"a}nzungsreihe zu den Hochschulb{\"u}chern f{\"u}r Mathematik. VEB
  Deutscher Verlag der Wissenschaften, Berlin, 4 edition, 1977.

\bibitem[Wor77]{b-Worobjow} N.~N. Worobjow.  \newblock \emph{Die Fibonaccischen
  Zahlen}.  \newblock Number~1 in Kleine Erg{\"a}nzungsreihe zu den
  Hochschulb{\"u}chern f{\"u}r Mathematik. VEB Deutscher Verlag der
  Wissenschaften, Berlin, 1977.

\end{thebibliography}
%} %%% ende nein


\nocite{b-Engel}

%\bibliographystyle{alpha}
%\bibliographystyle{/home/schueler/latex/axel}
%\bibliography{bibdat}

\begin{attribution}
schueler (2005-04-14): Contributed to KoSemNet
\end{attribution}

 
\end{document}




