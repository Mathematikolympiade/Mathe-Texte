\documentclass{article}
\usepackage{kosemnet,ko-math,ngerman,url}

\title{Primzahl-Testverfahren\kosemnetlicensemark} 
\author{Hans-Gert Gr"abe, Leipzig\\[8pt]
\url{http://www.informatik.uni-leipzig.de/~graebe}}

\date{2.~Februar 2005} 

\begin{document} 

\maketitle

\section{Einfache Primzahl-Testverfahren}

Ist eine gro"se ganze Zahl gegeben, so ist es in vielen F"allen einfach zu
erkennen, dass es sich um eine zusammengesetzte Zahl handelt. So sind
z.B. 50\,\% aller Zahlen gerade. Eine Probedivision durch die 4 Primzahlen
$<10$ l"asst uns bereits 77\,\% aller Zahlen als zusammengesetzt
erkennen. "Ubrig bleibt eine Grauzone von m"oglichen Kandidaten von
Primzahlen, f"ur die ein aufw�ndigeres Verfahren herangezogen werden muss, um
die Primzahleigenschaft auch wirklich zu beweisen.

Ein erstes solches Verfahren ist die {\bf Methode der Probedivision}.  Die
folgenden Prozeduren sind in {\bf MuPAD} \cite{MuPAD} geschrieben, sehen aber
in jedem anderen Computeralgebra-System (CAS) "ahnlich aus.
\begin{verbatim}
primeTestByTrialDivision:=proc(m:Dom::Integer) local z;
begin  
  if (m<3) then return(bool(m=2)) end_if;
  z:=2;
  while z*z<=m do 
      if m mod z = 0 then return(FALSE) end_if; z:=z+1;
  end_while;
  TRUE; 
end;
\end{verbatim}
oder in etwas effizienterer Form (Aufwand auf $\frac13$ reduziert, indem nur
auf Faktoren der Form $6k\pm 1$ getestet wird)
\begin{verbatim}
moreEfficientPrimeTestByTrialDivision:=proc(m:Dom::Integer) local z;
begin 
  if (m<2) then return(FALSE)
    elif (m mod 2 = 0) then return(bool(m=2));
    elif (m mod 3 = 0) then return(bool(m=3))
  end_if;
  z:=5;
  while z*z<=m do
      if m mod z = 0 then return(FALSE) end_if; z:=z+2;
      if m mod z = 0 then return(FALSE) end_if; z:=z+4;
  end_while;
  TRUE;
end;
\end{verbatim}
{\tt bool} erzwingt dabei die boolesche Auswertung des zur"uckzugebenden
Ausdrucks, da sonst standardm"a"sig angenommen wird, dass es sich um eine
Gleichung handelt, der man in einem symbolischen Kontext nur dann einen
Wahrheitswert zuordnen kann, wenn in ihr keine (ungebundenen) Symbole
auftreten.

Der Aufwand f"ur dieses Verfahren ist am gr"o"sten, wenn $m$ eine Primzahl
ist, d.h. wirklich alle Tests bis zum Ende durchgef"uhrt werden m"ussen. Die
Laufzeit ist dabei von der Gr"o"senordnung $O(\sqrt{m})$, also exponentiell in
der Bitl"ange $l(m)=\log_2(m)$ der zu untersuchenden Zahl
($\sqrt{m}=2^{l(m)/2}$).

Dieses Verfahren liefert uns allerdings f"ur eine zusammengesetzte Zahl neben
der Antwort auch einen Faktor, so dass eine rekursive Anwendung nicht nur die
Primzahleigenschaft pr"ufen kann, sondern f"ur Faktorisierungen geeignet
ist. Experimente mit CAS zeigen dagegen, dass Faktorisieren offensichtlich um
Gr"o"senordnungen schwieriger ist als der reine Primzahltest.

Gleichwohl wird in allen CAS der Test mit Probedivision f"ur eine Liste von
kleinen Primzahlen als Vortest angewendet und die aufw�ndigeren Verfahren nur
f"ur solche Zahlen eingesetzt, die {\glqq}nicht offensichtlich{\grqq} keine
Primzahlen sind. In der folgenden Prozedur ist {\tt smallPrimes} eine Liste
aller {\glqq}kleinen{\grqq} Primzahlen:
\begin{verbatim}
smallPrimesTest:=proc(m:Dom::Integer) local i,smallPrimes;
begin
  smallPrimes:=[2, 3, 5, 7, 11, 13, 17, 19, 23, 29];
  if (m<2) then return(FALSE) end_if;
  for i in smallPrimes do
       if (m mod i = 0) then return(bool(m=i)) end_if;
  end_for;
  FAIL;
end;
\end{verbatim}
In dieser Prozedurdefinition wird eine dreiwertige Logik ausgenutzt, die neben
den (sicheren) Wahrheitswerten {\tt TRUE} und {\tt FALSE} auch noch die
M"oglichkeit {\tt FAIL} ({\glqq}nicht entschieden{\grqq}) erlaubt.

\section{Der Restklassenring ${\Z}_m$}

Die weiteren Primtestverfahren beruhen alle auf zahlentheoretischen
Zusammenh"angen, die sich beim Rechnen mit Resten ergeben.

Bekanntlich nennt man zwei Zahlen $a,b\in {\Z}$ {\em kongruent modulo m} (und
schreibt $a\equiv b\pmod{m}$), wenn ihre Differenz durch $m$ teilbar ist, also
bei Division durch $m$ der Rest $0$ bleibt. So gilt $127\equiv 1\pmod{7}$,
aber ebenso $127\equiv 8\pmod{7}$, denn in beiden F"allen ist die Differenz
durch $7$ teilbar.

Die eingef"uhrte Relation ist eine "Aquivalenzrelation, so dass wir die
zugeh"origen "Aquivalenzklassen betrachten k"onnen, die als {\em Restklassen}
bezeichnet werden. Die Restklasse $\pmod{7}$, in der sich die Zahl $1$
befindet, besteht etwa aus den Zahlen
\[[1]_7=\{\ldots,-20,-13,-6,1,8,15,\ldots,127,\ldots\} = \{7k+1\teilt
k\in {\Z}\}.
\] 
Die Darstellungen $z\equiv 1\pmod{7},\ 7\teilt (z-1),\ z=7k+1,\ z\in [1]_7$
und $[z]_7 = [1]_7$ sind also "aquivalent zueinander. Wir werden diese
unterschiedlichen Schreibweisen im Weiteren frei wechselnd verwenden.  Die
Menge der Restklassen modulo $m$ bezeichnen wir mit ${\Z}_m$.

Addition und Multiplikation sind mit der Restklassenbildung vertr"aglich, so
dass die Menge ${\Z}_m$ sogar einen Ring bildet.  Im Gegensatz zu den ganzen
Zahlen kann dieser Ring aber Nullteiler besitzen. So ist etwa $2,3\not\equiv
0\pmod{6}$, dagegen $2\cdot 3 = 6 \equiv 0\pmod{6}$.

\subsection{Ein wichtiger Satz "uber endliche Mengen}

In den weiteren Betrachtungen wird mehrfach der folgende wichtige Satz
"uber endliche Mengen angewendet.
\begin{satz}
Sei $\phi:\ M_1\longrightarrow M_2$ eine Abbildung zwischen zwei
gleichm"achtigen endlichen Mengen
in sich selbst. Dann gilt
\begin{align*}
&\phi\ \text{ist injektiv, d.h.}\ \phi(x_1)=\phi(x_2)\ \Rightarrow
    x_1=x_2\tag{1} 
\intertext{genau dann, wenn}
&\phi\ \text{ist surjektiv, d.h.}\ \forall\ y\in M_2\ \exists\ x\in M_1\,:\, 
    y=\phi(x)\tag{2}
\end{align*}
\end{satz}

\begin{beweis}
Offensichtlich, denn
\begin{quote}
(1) hei"st: jedes $y\in M_2$ hat {\em h"ochstens} ein Urbild,

(2) hei"st: jedes $y\in M_2$ hat {\em mindestens} ein Urbild
\end{quote}
In Wirklichkeit hat wegen der Gleichm"achtigkeit in beiden F"allen
jedes $y\in M_2$ {\em genau} ein Urbild $x\in M_1$.
\end{beweis}

Dieser Satz ist f"ur unendliche Mengen falsch. So ist z.B. die Abbildung
$\phi_1: {\N}\rightarrow {\N}$ via $\phi_1(n)=2\,n$ zwar injektiv, aber nicht
surjektiv, die Abbildung $\phi_2: {\N}\rightarrow {\N}$ via $\phi_2(n)=n\ div\ 
10$ surjektiv, aber nicht injektiv.

\subsection{Die Gruppe der prime Restklassen}

Eine Restklasse $[a]_m$ hei"st {\em prim}, wenn ein (und damit jeder)
Vertreter dieser Restklasse zu $m$ teilerfremd ist, wenn also $\gcd(a,m)=1$
gilt. So sind etwa $\pmod{7}$ alle Restklassen verschieden von $[0]_7$ prim,
$\pmod{8}$ dagegen nur die Restklassen $[1]_8, [3]_8, [5]_8$ und $[7]_8$ und
$\pmod{6}$ gar nur die beiden Restklassen $[1]_6$ und $[5]_6$. Wir bezeichnen
die Menge der primen Restklassen mit ${\Z}_m^\ast$.

Prime Restklassen haben bzgl. der Multiplikation eine besondere
Eigenschaft. Es gilt f"ur eine prime Restklasse $a\pmod{m}$ die {\em
K"urzungsregel} 
\[a\cdot x\equiv a\cdot y\pmod{m}\ \Rightarrow\ x\equiv y\pmod{m}.\] 
Dies l"asst sich sofort aus $m\teilt (a\,x-a\,y)=a\,(x-y)$ und $\gcd(a,m)=1$
herleiten. 

Anders formuliert: Die Multiplikationsabbildung
\[\phi_a:\ {\Z}_m\longrightarrow {\Z}_m\quad \text{via}\quad
x\pmod{m}\mapsto a\,x\pmod{m}
\] 
ist injektiv und somit nach obigem Satz als Abbildung zwischen
gleichm"achtigen endlichen Mengen auch surjektiv. Zu einer primen Restklasse
$a\in {\Z}_m$ gibt es also stets ein (eindeutig bestimmtes) $a'\in {\Z}_m$, so
dass $\phi_a(a')=a\cdot a'\equiv 1\pmod{m}$ gilt. $a'$ (auch $a^{-1}$)
bezeichnet man als die zu $a$ {\em inverse Restklasse}.

Da die Menge aller Restklassen ${\Z}_m$ endlich ist, ist es auch
die Menge der primen Restklassen ${\Z}_m^\ast$. Ihre Anzahl
bezeichnet man mit dem Symbol $\phi(m)$. Die zugeh"orige Funktion in
Abh"angigkeit von $m$ bezeichnet man als die {\em Eulersche
$\phi$-Funktion}. 

Produkt- und Inversenbildung f"uhren nicht aus der Menge der primen
Restklassen ${\Z}_m^\ast$ heraus. Eine solche unter Multiplikation
und Inversenbildung abgeschlossene Struktur bezeichnet man auch als
{\bf Gruppe}. Die Gruppe ${\Z}_m^\ast$ enth"alt genau $\phi(m)$
Elemente.  Allgemeine Gruppeneigenschaften spezifizieren zu
interessanten zahlentheoretischen S"atzen.

So bezeichnet man etwa f"ur ein Element $a$ einer Gruppe $G$ die
M"achtigkeit der von $a$ erzeugten Untergruppe $\langle a \rangle =
\{1,a,a^2,\ldots\} \subset G$ als die {\em Ordnung} $ord(a)$ von
$a$. Im Falle endlicher Gruppen ist diese Ordnung immer endlich und es
gilt
\[ord(a)=n_0=\min\{n>0\ :\ a^n=1\}.\]
Weiter gilt
\[a^n=1\ \Rightarrow\ n_0\teilt n,\]
denn aus $n=q\,n_0+r$ mit $0\leq r<n_0$ (Division mit Rest) und
$1=a^n=(a^{n_0})^q\cdot a^r = 1\cdot a^r$ folgt $r=0$.

\begin{satz}[Satz von Lagrange]
Ist $H$ eine Untergruppe von $G$, so ist $|H|$ ein Teiler von $|G|$.
\end{satz}

\begin{beweis}
F"ur $a\in G$ betrachten wir die Mengen $aH=\{ah\teilt h\in H\}$, die
man auch als {\em Nebenklassen} von $a$ bzgl. $H$ bezeichnet. Zwei
solche Nebenklassen $a_1H$ und $a_2H$ sind
\begin{quote}
(1) gleichm"achtig (die Multiplikationsabbildung $\phi_{a}$ bildet $H$
    umkehrbar eindeutig auf $aH$ ab, Umkehrabbildung ist
    $\phi_{a^{-1}}$) 

(2) disjunkt, wenn $a_1^{-1}a_2\not\in H$ (w"are $a\in a_1H\cap a_2H$,
    also $a=a_1h_1=a_2h_2$, so w"are $a_1^{-1}a_2= h_1h_2^{-1}\in H$)

(3) und fallen f"ur $a_1^{-1}a_2\in H$ zusammen ($a_2\cdot h= a_1\cdot
    (a_1^{-1}a_2h$).
\end{quote}
$G$ zerf"allt also in paarweise disjunkte gleichgro"se Teile, von
denen ein Teil $H=1\cdot H$ ist.
\end{beweis}

Insbesondere ist die Gruppenordnung $|G|$ durch die Ordnung $ord(a)$
jedes Elements teilbar und es gilt stets $a^{|G|}=1$. F"ur die Gruppe
der primen Restklassen bedeutet das:

\begin{folgerung}[Satz von Euler]
Ist $(a,m)=1$, so gilt $a^{\phi(m)}\equiv 1\pmod{m}$.
\end{folgerung}

Ein Spezialfall dieses Satzes:
\begin{folgerung}[Kleiner Satz von Fermat]
  Ist $m$ eine Primzahl und $1<a<m$, so gilt\\ $a^{m-1}\equiv 1\pmod{m}$.
\end{folgerung}

\section{Der Fermat-Test}

Den kleinen Satz von Fermat kann man zu einem Verfahren ausbauen, mit
dem zusammengesetzte Zahlen sicher erkannt werden k"onnen, ohne deren
Faktorzerlegung zu berechnen. Dieser Test beruht auf der folgenden
Umkehrung des Satzes:
\begin{quote}\it 
Gilt f"ur {\bf eine} Zahl $a$ mit $1<a<m$ {\bf nicht} $a^{m-1}\equiv
1\pmod{m}$, so kann $m$ keine Primzahl sein.
\end{quote}
Eine Realisierung in MuPAD h"atte etwa folgende Gestalt:
\begin{verbatim}
singleFermatTest:=proc(m:Dom::Integer, a:Dom::Integer) local D;
begin 
D:=Dom::IntegerMod(m); 
iszero(D(a)^(m-1)-1);
end;
\end{verbatim}
Hierbei wird das Potenzieren in $\Z_m$ ausgef�hrt, d.h. in jedem
Zwischenschritt sofort $\pmod{m}$ reduziert und nicht zuerst $a^{m-1}$ als
ganze Zahl berechnet (was zu einer Riesenzahl f�hren w�rde) und dann erst
$\pmod{m}$ reduziert. Wir haben dabei die M�glichkeiten des Domain-Konzepts
von MuPAD genutzt, das so in anderen CAS nicht zur Verf�gung steht.

Mit bin"arem Potenzieren (wiederholtes Quadrieren) sind die Kosten dieses
Verfahrens von der Gr"o"senordnung $O(l(m)^3)$ ($\phi(m)\sim m$, $\log(m)$
Multiplikationen von Zahlen der L"ange $l(m)$ mit anschlie"sender
Restreduktion), also polynomial in der Bitl"ange der zu untersuchenden Zahl.

Der Fermat-Test ist allerdings nur ein notwendiges Kriterium. Genauer
gesagt k"onnen wir aus $a^{m-1}\not\equiv 1\pmod{m}$ mit Sicherheit
schlie"sen, dass $m$ eine zusammengesetzte Zahl ist. Ansonsten k"onnen
wir den Test mit einer anderen Basis $a$ wiederholen, weil vielleicht
zuf"allig $ord(a)\teilt m-1$ galt.

Auf dieser Basis kann man folgenden {\bf Las-Vegas-Test} ansetzen:
\begin{quote}\it
Mache den Fermat-Test f"ur $c$ verschiedene (zuf"allig gew"ahlte)
Basen $a_1,\ldots,a_c$.

Ist einmal $a_i^{m-1}\not\equiv 1\pmod{m}$, so ist $m$ garantiert
eine zusammengesetzte Zahl.

Ist stets $a_i^{m-1}\equiv 1\pmod{m}$, so ist $m$ wahrscheinlich
(hoffentlich mit gro"ser Wahrscheinlichkeit) eine Primzahl.
\end{quote}
Eine MuPAD-Implementierung h"atte etwa folgende Gestalt:
\begin{verbatim}
LasVegasFermatTest:=proc(m:Dom::Integer, c:Dom::Integer)
  local D,i,a,r;
begin
  D:=Dom::IntegerMod(m);
  r:=random(m); /* r ist Zufallszahlengenerator */
  for i from 1 to c do
      a:=r();
      if igcd(a,m) <> 1 then return(FALSE) end_if;
         /* a ist keine prime Restklasse, m also zusammengesetzt */
      if not iszero(D(a)^(m-1)-1) then return(FALSE) end_if;
         /* der FermatTest schl�gt zu */
  end_for;
  TRUE;
end;
\end{verbatim}
{\tt r:=random(m)} gibt dabei gem"a"s der Dokumentation keine Zufallszahl,
sondern eine (nullstellige) Funktionsdefinition zur"uck, {\tt r()} ruft diese
auf.  Ist $a$ zuf"allig keine prime Restklasse, dann ist $m$ zusammengesetzt
und der berechnete gcd sogar ein echter Teiler von $m$. Dieser (sehr selten
auftretende) Fall wird gesondert abgefangen.
\medskip

Was k"onnen wir "uber die Wahrscheinlichkeit im unsicheren Zweig dieses Tests
aussagen?

\begin{satz}
Die Menge 
\[P_m:=\{a\in {\Z}_m^\ast\ :\ a^{m-1}\equiv 1\pmod{m}\}\]
der Restklassen $\pmod{m}$, f"ur welche der Fermat-Test kein Ergebnis liefert,
ist eine Untergruppe der Gruppe der primen Restklassen ${\Z}_m^\ast$.
\end{satz}

Nach dem Satz von Lagrange ist somit $|P_m|$ ein Teiler von $\phi(m) =
|{\Z}_m^\ast|$.  Ist $m$ also zusammengesetzt und $P_m\neq {\Z}_m^\ast$, dann
erkennt der Fermat-Test f"ur eine zuf"allig gew"ahlte Basis in wenigstens der
H"alfte der F"alle, dass $m$ zusammengesetzt ist.

In diesem Fall ist die Wahrscheinlichkeit, dass im unsicheren Zweig
des LasVegas-FermatTests $m$ keine Primzahl ist, h"ochstens $2^{-c}$, also
bei gen"ugend vielen Tests verschwindend klein.
\medskip

Ist andererseits $m$ eine zusammengesetzte Zahl und $P_m = {\Z}_m^\ast$, so
kann der Fermat-Test $m$ prinzipiell nicht von Primzahlen unterscheiden. Gibt
es solche Zahlen ?  \medskip

Antwort: Ja, z.B. die Zahl $561=3\cdot 11\cdot 17$, denn f"ur
$a\in{\Z}_{561}^\ast$ gilt 
\[a^{560} \equiv 1\pmod{3},\ a^{560} \equiv 1\pmod{11}\ \text{und}\ a^{560}
\equiv 1\pmod{17},\ \text{also}\ a^{560} \equiv 1\pmod{561}.\]

\section{Der Satz von Carmichael}

Dieses Beispiel, in welchem der Fermat-Test fehlschl"agt, ist nicht zuf"allig
gefunden worden, sondern ergibt sich aus folgenden "Uberlegungen:

Ist $m=p_1^{a_1}\ldots p_k^{a_k}$ die Primfaktorzerlegung von $m$ und
$m_i=p_i^{a_i}$, so gilt mit $x^{\phi(m_i)}\equiv 1\pmod{m_i}$ auch f"ur jedes
Vielfache $c$ von $m_i$ die Beziehung $x^c\equiv 1\pmod{m_i}$. Nehmen wir
insbesondere ein {\bf gemeinsames} Vielfaches der $m_i$, etwa $c = {\rm
lcm}(\phi(m_1), \ldots, \phi(m_n))$, so gilt $x^c\equiv 1\pmod{m_i}$ f"ur {\bf
alle} $i=1,\ldots,n$, also $x^c-1\teilt {\rm lcm}(m_1,\ldots,m_n)=m$, denn die
Faktoren $m_i$ sind paarweise teilerfremd. 
\begin{satz}[Satz von Carmichael]
Ist $m=p_1^{a_1}\ldots p_k^{a_k}$ die Primfaktorzerlegung von $m$, so gilt
\begin{align*}
&a^{\psi(m)} \equiv 1\pmod{m}
\intertext{mit}
&\psi(m)={\rm lcm}(p_1^{a_1-1}(p_1-1),\ldots,p_k^{a_k-1}(p_k-1))
\end{align*}
Die Zahl $\psi(m)$ bezeichnet man auch als die {\em Carmichael-Zahl} von $m$.
\end{satz}
\begin{beweis}
F"ur eine Primzahlpotenz $p^a$ gilt offensichtlich $\phi(p^a)=p^a-p^{a-1}$.
\end{beweis}

\begin{beispiel}
  $\psi(12)=2,\ \psi(18)=6,\ \psi(24)=4,\ \psi(36)=6$. 
\end{beispiel}

$c=\psi(m)$ ist f�r zusammengesetzte Zahlen $m$ also ein deutlich kleinerer
Exponent als $\phi(m)$, f�r welchen immer noch 
\[a\in\Z_m^\ast\ \yields a^c\equiv 1\pmod{m}\]
gilt und welcher damit ein gemeinsames Vielfaches der Gruppenordnungen
\ul{aller} Elemente $a\in\Z_m^\ast$ ist.  Es stellt sich heraus, dass $c$ in
vielen F�llen der kleinste solche Exponent ist und in anderen F�llen der
kleinste solche Exponent gerade gleich $\frac{c}{2}$ ist.  Damit k"onnen wir
auch genauer sagen, f"ur welche Zahlen der Fermat-Test fehlschl"agt.
\begin{satz}
F"ur die ungerade zusammengesetzte Zahl $m$ schl"agt der Fermat-Test genau
dann systematisch fehl, wenn $\psi(m)$ ein Teiler von $m-1$ ist.

Solche Zahlen kann der Fermat-Test also prinzipiell nicht von Primzahlen
unterscheiden, wenn der Test mit einem $a\in\Z_m^\ast$ ausgef�hrt wird.
\end{satz}
Zusammengesetzte Zahlen dieser Art nennt man {\em Carmichael-Zahlen}.  Weitere
Carmichael-Zahlen sind $1105=5\cdot 13\cdot 17$ und $1729=7\cdot 13\cdot 19$.

\begin{aufgabe}\mbox{}
\begin{itemize}
\item[a)] Zeigen Sie, dass Carmichael-Zahlen $m$ stets quadratfrei
sind und immer wenigstens 3 Primfaktoren haben. F"uhren Sie dazu die
Annahmen $m=p^a\cdot q$ und $m=p\cdot q$ jeweils zum Widerspruch.
\item[b)] Zeigen Sie, dass $N=(6t+1)(12t+1)(18t+1)$ eine
Carmichaelzahl ist, wenn $6t+1, 12t+1$ und $18t+1$ Primzahlen sind.
\end{itemize}
\end{aufgabe}

Carmichaelzahlen sind unter den Primzahlen also recht selten. So gibt es unter
den Zahlen $<10^{15}$ nur etwa $10^5$ solcher Zahlen.  Andererseits gibt es
unendlich viele solche Zahlen und Alford, Granville und Pomerance (1994) haben
sogar gezeigt, dass sich die Anzahl $C(x)$ der Carmichaelzahlen kleiner $x$
durch $C(x)\geq x^{2/7}$ nach unten hin absch"atzen l"asst, d.h. ihre Zahl
exponentiell mit der Bitl"ange von $x$ w"achst.

\section{Der Rabin-Miller-Test} 

Ein Primzahltest ohne solche systematischen Ausnahmen beruht auf der folgenden
Verfeinerung des Fermat-Tests: F"ur eine Primzahl $m$ und eine zuf"allig
gew"ahlte Zahl $1<a<m$ muss $a^{m-1}\equiv 1\pmod{m}$ gelten. Ist
$m-1=2^t\cdot q$ die Zerlegung des Exponenten in eine Zweierpotenz und einen
ungeraden Anteil $q$, so ergibt die Restklasse $b=a^q\pmod{m}$ nach
$t$-maligem Quadrieren den Rest $1$. Bezeichnet $u$ das Element in der Folge
$\{b, b^2, b^4, b^8, \ldots, b^{2^t}\}$, wo erstmals $u^2\equiv 1\pmod{m}$
gilt, so muss, wenn $m$ eine Primzahl ist, $u\equiv -1\pmod{m}$ gelten. Ist
dagegen $m$ keine Primzahl, so hat die Kongruenz $u^2\equiv 1\pmod{m}$ noch
andere L"osungen.

In der Tat, ist $m=p\cdot q$ f"ur zwei teilerfremde Faktoren $p,q$, so gibt es
nach dem chinesischen Restklassensatz eine Restklasse $a_1\pmod{m}$ mit
$a_1\equiv 1\pmod{p}$ und $a_1\equiv -1\pmod{q}$ und eine weitere Restklasse
$a_2\pmod{m}$ mit $a_2\equiv -1\pmod{p}$ und $a_2\equiv 1\pmod{q}$. F"ur beide
gilt $a_i^2\equiv 1\pmod{p}$ und $a_i^2\equiv 1\pmod{q}$, also auch
$a_i^2\equiv 1\pmod{m}$, aber $a_i\not\equiv \pm\,1\pmod{m}$.

W"ahlt man $a$ zuf"allig aus, so ist die Wahrscheinlichkeit, dass $u$
bei zusammengesetztem $m$ auf eine solche Restklasse trifft, d.h.
$u\not\equiv -1\pmod{m}$, aber $u^2\equiv 1\pmod{m}$ gilt, gro"s.
Eine genauere Analyse zeigt, dass diese Wahrscheinlichkeit sogar
wenigstens $\frac34$ betr"agt, d.h. ein Las-Vegas-Test auf dieser
Basis mit $c$ Durchl"aufen nur mit der Wahrscheinlichkeit $4^{-c}$
fehlerhaft antwortet.

Auf der beschriebenen Verfeinerung beruht der folgende {\bf
Rabin-Miller-Test} 
\begin{verbatim}
primeTestRabinMiller:=proc(m:Dom::Integer,c:Dom::Integer) 
  local a,q,b,t,i,j,r,D;
begin
  D:=Dom::IntegerMod(m);
  q:=m-1; t:=0; r:=random(m);
  while q mod 2 = 0 do q:=q/2; t:=t+1 end_while;
        /* nun ist  m-1 = 2^t * q */
  for i from 1 to c do
    a:=r(); b:=D(a)^q;
    if iszero(b-1) or iszero(b+1) then next end_if;
    /* keine Information, wenn b = 1 oder b = -1 (mod m) */
    for j from 1 to (t-1) do
      /* nun ist  b <> 1 oder -1 (m) */
      b:=b*b;
        if iszero(b-1) then return(FALSE) end_if;
        if iszero(b+1) then break end_if;   /* keine Information */
    end_for;
    if iszero(b+1) then next;         /* keine Information */
    else return(FALSE) end_if;
  end_for;
  TRUE;
end_proc;
\end{verbatim}
Die Hauptschleife (Iterationsvariable $i$) f"uhrt den eigentlichen Test f"ur
$c$ verschiedene zuf"allig gew"ahlte Basen $a$ aus. In jedem dieser Tests wird
zun"achst $b\equiv a^q\pmod{m}$ berechnet. Ist bereits $b\equiv \pm
1\pmod{m}$, so kann f"ur diese Basis keine Aussage getroffen werden. Ansonsten
quadrieren wir $b$:
\begin{quote}
Erhalten wir den Rest $1$, so war $b\not\equiv \pm 1\pmod{m}$, aber $b^2\equiv
1\pmod{m}$, also ist $m$ nicht prim.

Erhalten wir den Rest $-1$, so wird im n"achsten Schritt $b^2\equiv
1\pmod{m}$, woraus wir jedoch kein Kapital schlagen k"onnen. Aus der
gew"ahlten Basis $a$ kann keine Aussage getroffen werden.

Anderenfalls quadrieren wir noch einmal.
\end{quote}
Das Quadrieren ist nach $(t-1)$ Schritten abzubrechen, denn $b^{2^t}
\equiv a^{m-1}\pmod{m}$. Da nach Verlassen der inneren Schleife
stets $b\not\equiv 1\pmod{m}$ gilt, kommen folgende F"alle in Frage:
\begin{quote}
$b^2\not\equiv 1\pmod{m}$, also $m$ nicht prim nach dem Fermat-Test.

$b\not\equiv - 1\pmod{m}$, aber $b^2\equiv 1\pmod{m}$, also ist
$m$ nicht prim.

$b\equiv - 1\pmod{m}$. In diesem Fall kann aus der gew"ahlten Basis
$a$ keine Aussage getroffen werden.
\end{quote}

\section{Primzahl-Zertifikate}

Neben dem Rabin-Miller-Test gibt es noch eine Reihe weiterer Primzahltests,
die andere zahlentheoretische Zusammenh"ange (etwa "uber quadratische Reste --
der Solovay-Strassen-Test) verwenden. All diesen Tests haftet der Makel an,
dass Primzahlen zwar mit hoher Wahrscheinlichkeit korrekt erkannt werden, aber
nicht mit Sicherheit bekannt ist, ob es sich wirklich um Primzahlen handelt.
Die genannten Algorithmen sind f"ur praktische Belange, d.h. in Bereichen, in
denen sie noch mit vertretbarem Zeitaufwand angewendet werden k"onnen,
ausreichend und wurden auch in der Form in den verschiedenen CAS
implementiert.
\bigskip

Aus dem Axiom-Handbuch:
\begin{quote}
{\tt prime?(n)} returns true if n is prime and false if not.  The algorithm
used is Rabin's probabilistic primality test.  If {\tt prime? n} returns
false, $n$ is proven composite.  If {\tt prime? n} returns true, {\tt prime?}
may be in error however, the probability of error is very low and is zero
below $25*10^9$ (due to a result of Pomerance et al), below $10^{12}$ and
$10^{13}$ due to results of Pinch, and below 341550071728321 due to a result
of Jaeschke.  Specifically, this implementation does at least 10 pseudo prime
tests and so the probability of error is $< 4^{-10}$\ldots
\end{quote}

Aus dem Maple-Handbuch:
\begin{itemize}
\item The function {\tt isprime} is a probabilistic primality testing routine.
\item It returns false if $n$ is shown to be composite within one strong
pseudo-primality test and one Lucas test and returns true otherwise. If {\tt
isprime} returns true, $n$ is {\glqq}very probably prime - see
\cite{Knuth_91a}, Band 2, Section 4.5.4, Algorithm P for a reference and
\cite{Riesel:94}. No counter example is known and it has been conjectured that
such a counter example must be hundreds of digits long.
\end{itemize}

Aus dem Mathematica-Handbuch:
\begin{itemize}
\item {\tt PrimeQ} first tests for divisibility using small primes, then uses
the Miller-Rabin strong pseudoprime test base 2 and base 3, and then uses a
Lucas test.
\item As of 1997, this procedure is known to be correct only for $n<10^{16}$,
and it is conceivable that for larger $n$ it could claim a composite number to
be prime.
\end{itemize}
M"ochte man f"ur gewisse Anwendungen sicher gehen, dass es sich bei der
untersuchten Zahl garantiert um eine Primzahl handelt, ben"otigen wir ein {\em
Zertifikat} f"ur die Primzahleigenschaft.
\medskip

Ein solches Zertifikat kann erstellt werden, wenn man zu der vermuteten
Primzahl $m$ ein Erzeugendes $a$ der zyklischen Gruppe ${\Z}_m^\ast$ angibt
(zusammen mit einem Beweis, dass dieses Element die behauptete Eigenschaft
besitzt).

\begin{satz}
  Die ungerade Zahl $m$ ist genau dann eine Primzahl, wenn ${\Z}_m^\ast$ eine
  zyklische Gruppe der Ordnung $m-1$ ist, d.h. wenn es ein Element $a\in
  {\Z}_m^\ast$ gibt, so dass $ord(a)=m-1$ gilt.

Ein Element $a\in {\Z}_m^\ast$ hat genau dann die Ordnung $ord(a)=m-1$, wenn
$a^{m-1}\equiv 1\pmod{m}$, aber f"ur alle Primteiler $p$ der Zahl $m-1$ die
Beziehung $a^\frac{m-1}{p}\not\equiv 1\pmod{m}$ gilt.
\end{satz}

Die letzte Bedingung sichert, dass $ord(a)$ durch $m-1$, aber keinen Teiler
von $m-1$ teilbar ist.
\medskip

Betrachten wir als Beispiel die Primzahl $m=55499821019$. F"ur einen {\em
Beweis} der Primzahleigenschaft pr"ufen wir die Voraussetzungen des Satzes
f"ur $a=2$. Mit MuPAD erhalten wir
\begin{verbatim}
m:=55499821019;
Z:=Dom::IntegerMod(m);
u:=Factored::factors(factor(m-1));
\end{verbatim}
\[\sbr{2,17,1447,1128091}\]
\begin{verbatim}
map(u,p->Z(2)^((m-1)/p));
\end{verbatim}
\[\sbr{55499821018, 44871471456, 31660914932, 44713683701}\]
\begin{verbatim}
Z(2)^(m-1);
\end{verbatim}
\[1\]
Die Voraussetzungen des Satzes sind also erf"ullt, so dass $2$ die Gruppe
${\Z}_m^\ast$ erzeugt.
\medskip

Oftmals ist allerdings f"ur eine konkrete Restklasse $a$ die Beziehung
$a^\frac{m-1}{p}\not\equiv 1\pmod{m}$ nur f"ur einige der Primfaktoren von
$m-1$ erf"ullt und es ist schwierig, eine Restklasse zu finden, die f"ur {\em
alle} Primteiler passt.
\begin{verbatim}
m:=nextprime(2*10^10); // = 20000000089
Z:=Dom::IntegerMod(m);
u:=Factored::factors(factor(m-1));
\end{verbatim}
\[[2, 3, 67, 1381979]\]
\begin{verbatim}
for c in [2,3,5,7,11,13,17,23,29] do
print(map(u,p->mods(expr(Z(c)^((m-1)/p)),m)))
end_for;
\end{verbatim}
\begin{gather*} 
[1, 1, 3108157330, -2457953819]\\
[1, 1, -4865156001, -7521929036]\\
[1, 5697767833, 6300564838, -3927279952]\\
[-1, 1, 1, -855778974]\\
[1, 1, 8042780317, -7269258111]\\
[1, -5697767834, 8735956541, -5244188757]\\
[1, 1, -2293750453, 9516320400]\\
[1, 5697767833, -8450150690, -4573270973]\\
[-1, -5697767834, 8111724708, -1296001895]
\end{gather*}
Erst $a=29$ hat die geforderte Eigenschaft, dass $a^\frac{m-1}{p}\neq 1$ f"ur
{\em alle} Primteiler $p$ von $m$ gilt.  Es stellt sich, mit Blick auf den
Chinesischen Restklassensatz nicht verwunderlich, heraus, dass es gen"ugt,
f"ur jeden Primteiler {\em seine} Restklasse $a$ zu finden, womit die
Rechnungen in diesem Beispiel bereits f"ur $a=7$ beendet werden k"onnen.

\begin{satz} 
Seien $\{p_1,\ldots,p_k\}$ die (verschiedenen) Primfaktoren von $m-1$. Dann
gilt
\begin{align*}
&\exists\, a\in {\Z}_m^\ast\ \forall i\ a^\frac{m-1}{p_i}\not\equiv 1\pmod{m}
\intertext{genau dann, wenn}
&\forall i\ \exists\, a_i\in {\Z}_m^\ast\ a_i^\frac{m-1}{p_i} \not\equiv
  1\pmod{m},
\end{align*}
d.h. es gibt eine gemeinsame Basis f"ur alle Primteiler von $m-1$, wenn es
f"ur jeden Primteiler einzeln eine passende Basis gibt.
\end{satz}
Zur Bestimmung eines Primzahlzertifikats f"ur die primzahlverd"achtige Zahl
$m$ reicht es also aus, f"ur jeden Teiler $p$ der Zahl $m-1$ in einer Liste
von kleinen Zahlen eine solche Zahl $a_p$ zu finden, dass
$a_p^\frac{m-1}{p}\not\equiv 1\pmod{m}$ gilt:
\begin{verbatim}
certifyPrime:=proc(m) local u,v,p,a,D;
begin
  if not isprime(m) then error(expr2text(m)." ist nicht prim") end_if;
  D:=Dom::IntegerMod(m);
  u:=null(); v:=Factored::factors(factor(m-1));
  for p in v do
    for a in [2,3,5,7,11,13,17,19,23] do
       if not iszero(D(a)^((m-1)/p)-1) then u:=u,[p,a]; break end_if;
    end_for;
  end_for;
  if nops([u])<>nops(v) then 
    error("Kein Zertifikat f�r ".expr2text(m)." gefunden");
  end_if;
  [u];
end_proc;
\end{verbatim}
Auf obiges Beispiel angewendet erhalten wir damit folgendes
Zertifikat:
\begin{verbatim}
certifyPrime(m);
\end{verbatim}
\[[[2, 7], [3, 5], [67, 2], [1381979, 2]]\]
Der erste Eintrag gibt dabei jeweils den Primfaktor von $m-1$, der zweite die
f"ur diesen Primfaktor $p$ geeignete Basis $a_p$ an.
\medskip

Dieses Kriterium geht also davon aus, dass eine Faktorzerlegung der Zahl $m-1$
berechnet werden kann. F"ur Zahlen in der Nachbarschaft einer Primzahl $m$ ist
das erstaunlicherweise oft m"oglich.

Besonders einfach ist ein solcher Nachweis f"ur Primzahlen der Form $m=2^a+1$,
da hier die Zahl $m-1$ nur durch 2 teilbar ist. Allerdings stellt es sich
heraus, dass dies h"ochstens dann der Fall ist, wenn $a$ selbst Zweierpotenz
ist. Zahlen der Form $F_n:=2^{2^n}+1$ bezeichnet man als {\em Fermatzahlen},
da sie erstmals von Fermat untersucht wurden, der in einem Brief an Mersenne
die Behauptung aufstellte, dass alle Zahlen $F_n$ prim sind. Er konnte dies
f"ur die ersten 5 Fermatzahlen $3, 5, 17, 257$ und $65537$ nachweisen,
vermerkte allerdings, dass er die Frage f"ur die n"achste Fermatzahl
$F_5=4294967297$ nicht entscheiden k"onne. Dies w"are allerdings mit dem
Fermat-Test f"ur die Basis $a=3$ (vom Rechenaufwand abgesehen) gar nicht so
schwierig gewesen:
\begin{verbatim}
m:=2^(2^5)+1;
Z:=Dom::IntegerMod(m);
Z(3)^(m-1);
\end{verbatim}
\[3029026160\]
Die Basis 3 kann man generell f"ur den Fermattest von $F_n$ verwenden und
zeigen, dass auch die n"achsten Fermatzahlen (deren Stellenzahl sich
allerdings jeweils verdoppelt) zusammengesetzt sind.

Primzahlen dieser Gestalt spielen eine gro"se Rolle in der Frage der
Konstruierbarkeit regelm"a"siger $n$-Ecke mit Zirkel und Lineal. So konnte
Gauss die Konstruierbarkeit des regelm"a"sigen 17-Ecks nachweisen, weil die
Primzahl 17 in dieser Reihe auftritt, vgl. etwa \cite{Kunz:91}.

Die Faktorisierung $F_5=641\cdot 6700417$ fand erstmals Euler, allerdings
scheiterte er bereits an der n"achsten Zahl $F_6=18446744073709551617$, deren
Faktorisierung
\[F_6=67280421310721\cdot 274177\] 
erst im Jahre 1880 entdeckt wurde.  Ein modernes CAS berechnet diese Zerlegung
heute im Bruchteil einer Sekunde, kommt aber bei der n"achsten Fermatzahl
\[F_7=340282366920938463463374607431768211457\] 
bereits in Schwierigkeiten. Man geht heute davon aus, dass es au"ser den
bereits gefundenen keine weiteren primen Fermatzahlen gibt. F"ur einen Beweis
dieser Vermutung gibt es jedoch nicht einmal ansatzweise Ideen,
vgl. \cite{Riesel:94}.
\medskip

Neben den Fermatzahlen spielen auch Primzahlen der Form $m=2^a-1$ eine
wichtige Rolle. Die gr"o"sten heute bekannten Primzahlen haben diese Gestalt.
Da $m-1$ keine so einfache Faktorisierung wie f"ur Fermatzahlen besitzt, ist
hier die Erstellung eines Primzahlzertifikats schwieriger, jedoch hat
$m-1=2\,(2^{a-1}-1)$ wieder einen Faktor derselben Gestalt, was f"ur eine
teilweise Faktorisierung oft sehr hilfreich ist.  Primzahlen der Form
$M_p=2^p-1$ bezeichnet man als {\em Mersennesche Primzahlen}.
\medskip

Moderne Primtestverfahren verwenden auch andere Gruppen als ${\Z}_m^\ast$ zum
Test, insbesondere solche, die mit elliptischen Kurven verbunden sind.
Allerdings kennt man bis heute noch keinen sicheren Primzahltest mit
garantiert polynomialem (in $l(m)$) Laufzeitverhalten.

\begin{thebibliography}{1}

\bibitem{Knuth_91a} D.E. Knuth.  \newblock {\em The art of computer
programming}.  \newblock Addison Wesley, 1991.

\bibitem{Kunz:91} E.~Kunz.  \newblock {\em Algebra}.  \newblock Vieweg, 1991.

\bibitem{MuPAD} Das Computeralgebra-System MuPAD, \url{http://www.mupad.de}.

\bibitem{Riesel:94} H.~Riesel.  \newblock {\em Prime numbers and computer
methods for factorization}.  \newblock Birkh\"auser, 1994.

\end{thebibliography}

\begin{attribution}
graebe (2004-09-03): Contributed to KoSemNet

graebe (2005-02-02): Revision and translation to MuPAD
\end{attribution}

\end{document}




