% Beitrag f"ur das Heft zum Landesseminar M"arz 2000

\documentclass[11pt]{article}
\usepackage{graphicx,kosemnet,ko-math,ngerman}

\title{Bewegungsgeometrie\kosemnetlicensemark}
\author{Hans-Gert Gr"abe, Leipzig}
\date{23.~Januar 2000}

\newcommand{\Bild}[3]{
\begin{center}
\includegraphics[#1]{graebe-00-1/#2}\nopagebreak\\[12pt] #3
\end{center}
}

\begin{document}
\maketitle

Geometrische Argumentationen verwenden an vielen Stellen Kongruenzaussagen,
die ge\-w"ohn\-lich mit verschiedenen Kongruenzs"atzen begr"undet werden.
Bekanntlich hei"sen zwei geometrische Figuren {\em kongruent}, wenn es eine
Bewegung gibt, die die eine Figur mit der anderen zur Deckung bringt.

In diesem Beitrag soll an einigen Beispielen aufgezeigt werden, wie f"ur
Kongruenzaussagen direkt mit solchen Deckbewegungen argumentiert werden kann.

\begin{aufgabe}
  Konstruieren Sie ein gleichseitiges Dreieck ${ABC}$, dessen Eckpunkte auf
  drei vorgegebenen parallelen Geraden $g_A, g_B, g_C$ liegen.
\end{aufgabe}

\Bild{width=12cm}{Bild1}{Bild 1: Konstruktion des Dreiecks $ABC$}

Wegen der Translationsinvarianz kann man einen der Eckpunkte, etwa $B$, auf
seiner Geraden beliebig w"ahlen. Sei nun $ABC$ ein Dreieck mit den geforderten
Eigenschaften. Dreht man die Ebene in $B$ um $60\grad$, so geht $A$ in $C$
"uber.  $C$ muss also auf $g_C$ und der Bildgeraden $g_A'$ liegen, die man aus
$g_A$ durch die Drehung erh"alt. Da $g_A'$ nicht parallel zu $g_C$ verl"auft,
ist $C$ bei fixiertem $B$ als Schnittpunkt dieser beiden Geraden eindeutig
bestimmt.

Ein Dreieck mit den geforderten Eigenschaften kann man also wie folgt
konstruieren: 
\begin{itemize}
\item[(1)] W"ahle $B\in g_B$ beliebig aus.
\item[(2)] Konstruiere das Bild $g_A'$ von $g_A$ bei Drehung um $60\grad$ mit
  Zentrum $B$. (W"ahle dazu zwei Punkte auf der Originalgeraden, konstruiere
  deren Bilder und verbinde diese zur Bildgeraden).
  
  Der Schnittpunkt von $g_A'$ und $g_C$ ist $C$.
\item[(3)] Der dritte Eckpunkt $A$ ist der Schnittpunkt von $g_A$ mit der
  Mittelsenkrechten auf $\ksegment{BC}$.
\end{itemize}

Aus unseren bisherigen Betrachtungen ergibt sich, dass das so konstruierte
Dreieck $ABC$ die Bedingungen der Aufgabenstellung erf"ullt und umgekehrt
jedes solche Dreieck auch auf diese Weise konstruiert werden kann. Bei
fixiertem $B$ erhalten wir zwei L"osungen, da man die Drehung um $60\grad$ mit
Zentrum $B$ in {\em zwei} Richtungen ausf"uhren kann. Die beiden L"osungen
liegen symmetrisch bzgl.\ einer Achse durch $B$, die senkrecht zu den drei
parallelen Geraden verl"auft.

\Bild{width=12cm}{Bild2}{Bild 2: Die beiden L"osungen bei gegebenem $B$}

\begin{aufgabe}
  In einem regelm"a"sigen Tetraeder $ABCD$ mit der Kantenl"ange $a$ seien
  $E,F,G$ und $H$ die Mittelpunkte der Kanten $\ksegment{AB}, \ksegment{BC},
  \ksegment{CD}$ und $\ksegment{DA}$.

Zeigen Sie, dass $EFGH$ ein (ebenes) Quadrat ist.
\end{aufgabe}

\Bild{width=12cm}{Bild3}{Bild 3: Tetraeder $ABCD$ mit Quadrat $EFGH$}

Da $\kline{EF}$ die Mittellinie im Dreieck $ABC$ ist, gilt
$\kline{EF}\parallel \kline{AC}$ und $\msegment{EF}=\frac{a}{2}$. Analog
erhalten wir $\kline{GH}\parallel \kline{AC},\ \kline{EH}\parallel
\kline{BD}\parallel \kline{FG}$ und
$\msegment{FG}=\msegment{GH}=\msegment{EH}=\frac{a}{2}$. Die vier Punkte
liegen also in einer Ebene (die z.B.\ von den parallelen Geraden $\kline{EF}$
und $\kline{GH}$ bestimmt wird) und spannen ein Rhombus auf. Um zu zeigen,
dass das Rhombus ein Quadrat ist, zeigen wir, dass es gleichlange Diagonalen
$\msegment{EG}=\msegment{FH}$ besitzt.

Dazu betrachten wir zun"achst die Deckbewegungen, die ein regul"ares Tetraeder
in sich selbst "uberf"uhren. Es gibt genau 12 solche Drehbewegungen:
\begin{itemize}
\item[(1)] 8 Drehungen, deren Achse durch je einen Eckpunkt und den
  Mittelpunkt der gegen"uberliegenden Seitenfl"ache verl"auft (es gibt 4
  solche Achsen; um jede kann man um $\pm 120\grad$ drehen. Bezeichnung:
  $d_A^+, d_A^-$ f"ur die Drehungen um die Achse durch $A$),
\item[(2)] 3 Drehungen, deren Achse jeweils durch durch die Mitten
  gegen"uberliegender Kanten verl"auft (es gibt 3 solche Achsen; um jede muss
  man um $180\grad$ drehen. Bezeichnung: $d_{AB}^{CD}$ f"ur die Drehung um die
  Achse durch die Kantenmitten von $AB$ und $CD$) und
\item[(3)] die identische Bewegung.
\end{itemize}
Die Drehung $d_D^+$ vertauscht die Eckpunkte $A,B,C$ zyklisch, bildet damit
$E$ auf $F$ und $G$ auf $H$ und so auch $\ksegment{EG}$ auf $\ksegment{FH}$
ab. Die beiden Strecken sind also gleichlang. (NB: Der Schnittpunkt beider
Strecken ist unter der Drehung invariant, liegt also auf der Drehachse.)
\medskip

Werden zwei Bewegungen aus obiger Liste nacheinander ausgef"uhrt, so ergibt
wieder eine Bewegung aus der Liste. F"uhren wir z.B.\ zuerst die Drehung
$d_D^+$ aus, die $A,B,C$ zyklisch vertauscht, und danach die Drehung $d_B^-$
um die Achse durch den (neuen\,!) Eckpunkt $B$, so entspricht das genau der
Drehung $d_{AD}^{BC}$.  In der Tat, die Punkte werden dabei wie folgt
abgebildet: $A\mapsto B\mapsto D,\ B\mapsto C\mapsto C,\ C\mapsto A\mapsto B,\ 
D\mapsto D\mapsto A$.
\begin{quote}
  {\bf Weiterf"uhrende Aufgabe:} Erstellen Sie eine "Ubersicht, zu welcher
  Bewegung sich je zwei Bewegungen aus obiger Liste zusammen setzen.
\end{quote}
\medskip

Wir wollen nun die Zusammensetzung von Drehungen in der Ebene n"aher
betrachten. Sei $ABC$ ein {\em gleichseitiges} Dreieck. Bildpunkte nach den
ersten $k$-ten Drehungen werden wir im Folgenden mit dem Index $k$
kennzeichnen.

\Bild{width=12cm}{Bild4}{Bild 4: Gleichseitiges Dreieck $ABC$ in
den verschiedenen Positionen}

Drehen wir die Ebene in $A$ um $120\grad$, so geht $ABC$ in das Dreieck
$A_1B_1C_1$ mit $A_1=A$ "uber.  Drehen wir weiter in $B_1$ um $120\grad$, so
geht $A_1B_1C_1$ in das Dreieck $A_2B_2C_2$ mit $B_2=B_1$ und $C_2=C$ "uber.
Und drehen wir schlie"slich in $C_2$ um $120\grad$, so geht $A_2B_2C_2$ in das
Ausgangsdreieck $ABC$ "uber. Da eine Bewegung durch die Angabe der Bildpunkte
von drei nicht kollinearen Punkten eindeutig bestimmt ist, ergibt die
Nacheinanderausf"uhrung der drei angegebenen Drehungen also die identische
Abbildung.

\begin{aufgabe}
  Zeigen Sie, dass eine analoge Aussage f"ur {\em jedes} beliebige Dreieck
  $ABC$ gilt. Genauer: Zeigen Sie, dass das Nacheinanderausf"uhren der Drehung
  mit Zentrum $A$ um den Winkel $2\alpha$, der Drehung mit Zentrum $B_1$ um
  den Winkel $2\beta$ und der Drehung mit Zentrum $C_2$ um den Winkel
  $2\gamma$ die identische Abbildung ist.
\end{aufgabe}

\Bild{width=12cm}{Bild5}{Bild 5: Beliebiges Dreieck $ABC$ in den
verschiedenen Positionen}

Die erste Drehung "uberf"uhrt das Dreieck $ABC$ in das Dreieck $A_1B_1C_1$ mit
$A_1=A$, wobei nach Konstruktion $\mangle{BAC}= \mangle{CAB_1}=
\mangle{B_1AC_1} =\alpha$ gilt. Das Dreieck $CAB_1$ ist also nach (sws) zum
Dreieck $ABC$ kongruent und die zweite Drehung "uberf"uhrt auch im allgemeinen
Fall das Dreieck $A_1B_1C_1$ in ein Dreieck $B_2C_2A_2$ mit $B_2=B_1$ und
$C_2=C$ und $\mangle{A_2CB_1}= \mangle{B_1CA}= \mangle{ACB} =\gamma$.  Die
dritte Drehung "uberf"uhrt das Dreieck also wieder in seine Ausgangslage.
\medskip

Ein gleichseitiges Dreieck $ABC$ hat eine zweite interessante Eigenschaft.
F"uhren wir Drehungen um jeweils $60\grad$ um $A,\ B_1,\ C_2,\ A_3,\ B_4,\ 
C_5$ aus (Bezeichungen wie oben), so erhalten wir ebenfalls die identische
Abbildung.

\Bild{width=12cm}{Bild6}{Bild 6: Gleichseitiges Dreieck $ABC$ in den
  verschiedenen Positionen}

Das Dreieck wechselt dabei st"andig zwischen zwei verschiedenen Lagen hin und
her und ``dreht'' sich um sich selbst.  Die Lagen der drei Eckpunkte nach
jeder Drehung sind in der folgenden Tabelle aufgelistet.
\[\begin{array}{lccc}
\mbox{Ausgangslage : } & A_0=A & B_0=B & C_0=C\\
\mbox{nach 1. Drehung : } & A_1=A & B_1=C & C_1=D\\
\mbox{nach 2. Drehung : } & A_2=B & B_2=C & C_2=A\\
\mbox{nach 3. Drehung : } & A_3=C & B_3=D & C_3=A\\
\mbox{nach 4. Drehung : } & A_4=C & B_4=A & C_4=B\\
\mbox{nach 5. Drehung : } & A_5=D & B_5=A & C_5=C\\
\mbox{nach 6. Drehung : } & A_6=A & B_6=B & C_6=C\\
\end{array}\]

Auch hier k"onnen wir dasselbe Problem f"ur ein allgemeines Dreieck
$ABC$ stellen:
\begin{aufgabe}
  Welche Bewegung ergibt sich, wenn man die Drehung mit Zentrum $A$ um den
  Winkel $\alpha$, die Drehung mit Zentrum $B_1$ um den Winkel $\beta$, die
  Drehung mit Zentrum $C_3$ um den Winkel $\gamma$, die Drehung mit Zentrum
  $A_3$ um den Winkel $\alpha$, die Drehung mit Zentrum $B_4$ um den Winkel
  $\beta$ und die Drehung mit Zentrum $C_5$ um den Winkel $\gamma$
  nacheinander ausf"uhrt\,?
\end{aufgabe}

Offensichtlich wird insgesamt um den Winkel $2(\alpha+\beta+\gamma) =360\grad$
gedreht. Die Zusammensetzung all dieser Drehungen ist also eine Verschiebung
um einen Vektor $\kvector{XX_6}$, wobei $X$ ein beliebiger Punkt und $X_6$
dessen Bildpunkt unter der zusammen gesetzten Bewegung ist.
\begin{quote}
  {\bf Weiterf"uhrende Aufgabe:} Zeigen Sie mit elementargeometrischen
  Mitteln, dass die Verschiebung in Wirklichkeit die identische Abbildung ist.
\end{quote}

Wir wollen hier einen anderen Weg beschreiten und eine analytische L"osung
unter Verwendung der komplexen Zahlen ${\C}$ angeben. Ich setze dabei die
Kenntnis entsprechender Zusammenh"ange voraus und verweise den unkundigen
Leser bzw.\ Leserin auf das sch"one Buch \cite{Pieper} von H. Pieper.

Fixieren wir ein Koordinatensystem und identifizieren $x$- bzw.\ $y$-Achse mit
der reellen bzw.\ imagin"aren Achse der Gau"sschen Zahlenebene, so k"onnen wir
den Punkt $X=(x,y)$ unserer Ebene als komplexe Zahlen $z=x+i\,y$
interpretieren (und umgekehrt). Eine Drehung mit Zentrum $z_0\in {\C}$ um den
Winkel $\alpha$ kann man dann wie folgt beschreiben: Der Bildpunkt $z'\in
{\C}$ eines Originalpunkts $z\in {\C}$ berechnet sich aus der Formel
\[z'-z_0=e^{i\,\alpha}(z-z_0)\quad\text{bzw.}\quad z'=e^{i\,\alpha}\,z
+ (1-e^{i\,\alpha})\,z_0,\]
wobei $e^{i\,\alpha}=\cos(\alpha)+i\,\sin(\alpha)$ eine komplexe Zahl
vom Betrag 1 ist, die die gegebene Drehung charakterisiert. 

Sind $a_0,b_0,c_0\in {\C}$ die Eckpunkte unseres Dreiecks und $z_0\in {\C}$
ein beliebiger Punkt, so gilt demnach
\[\begin{array}{c}
z_1=e^{i\,\alpha}\,z_0 + (1-e^{i\,\alpha})\,a_0\\
z_2=e^{i\,\beta}\,z_1 + (1-e^{i\,\beta})\,b_1\\
z_3=e^{i\,\gamma}\,z_2 + (1-e^{i\,\gamma})\,c_2\\
z_4=e^{i\,\alpha}\,z_3 + (1-e^{i\,\alpha})\,a_3\\
z_5=e^{i\,\beta}\,z_4 + (1-e^{i\,\beta})\,b_4\\
z_6=e^{i\,\gamma}\,z_5 + (1-e^{i\,\gamma})\,c_5\\
\end{array}\]
wobei $b_1, c_2,\ldots$ das Bild von $b_0$ unter der ersten Drehung, das Bild
von $c_0$ unter den ersten beiden Drehungen usw.\ ist. Wir k"onnen diese
Gleichungen verwenden, um $z_6$ durch die Eckpunkte $a_0,b_0,c_0$ des Dreiecks
in Ausgangslage und dessen Innenwinkel auszudr"ucken.  Die entsprechenden
Umformungen sind nicht schwierig, aber langweilig, so dass ich das
Computeralgebrasystem {\sc Maple} damit beauftragt habe. Zur Abk"urzung ist
$u=e^{i\,\alpha}, v=e^{i\,\beta}, w=e^{i\,\gamma}$ gesetzt:
\begin{verbatim}
z1:=u*z0+(1-u)*a0;
z2:=expand(v*z1+(1-v)*subs(z0=b0,z1));
z3:=expand(w*z2+(1-w)*subs(z0=c0,z2));
z4:=expand(u*z3+(1-u)*subs(z0=a0,z3));
z5:=expand(v*z4+(1-v)*subs(z0=b0,z4));
z6:=expand(w*z5+(1-w)*subs(z0=c0,z5));
\end{verbatim}
\begin{align*}
z_1=&u{z_0}+(1-u){a_0}\\ 
z_2=&(1-u){a_0}+u(1-v){b_0}+vu{z_0}\\
z_3=&wvu{z_0}+(1-u){a_0}+u(1-v){b_0}+ vu(1-w){c_0}\\
z_4=&wvu^2{z_0}+(u-1)(1+wvu){a_0}+u(1-v){b_0}+vu(1-w){c_0}\\
z_5=&wv^2u^2{z_0}+(u-1)(1+wvu){a_0}+u(1-v)(1+wvu){b_0}+vu(1-w){c_0}\\
z_6=&(u-1)(1+wvu){a_0}+u(1-v)(1+wvu){b_0}+uv(1-w)(1+wvu){c_0}+w^2v^2u^2{z_0}
\end{align*}
Beachten wir, dass $uvw=e^{i(\alpha+\beta+\gamma)}=e^{i\,\pi}=-1$ gilt, so
folgt $z_6=z_0$, d.h.\ die Zusammensetzung der sechs Drehungen ist in der Tat
die identische Bewegung.


\begin{thebibliography}{xxx}
\bibitem{Pieper} H. Pieper: {\em Komplexe Zahlen}. Verlag der Wissenschaften,
  Berlin und Harry Deutsch Verlag, Frankfurt/M., 1988.
\end{thebibliography}

\begin{attribution}
graebe (2004-09-09): Contributed to KoSemNet
\end{attribution}

\end{document}

