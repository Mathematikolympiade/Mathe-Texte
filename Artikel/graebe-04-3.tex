% Version: $Id: graebe-04-3.tex,v 1.1 2008/09/05 09:52:58 graebe Exp $
\documentclass[11pt]{article}  
\usepackage{kosemnet,ko-math,ngerman}  

\author{Hans-Gert Gr�be, Leipzig}
\title{Aussagenlogik\kosemnetlicensemark\\ 
Arbeitsmaterial f�r Klasse 7}
\date{}

\begin{document} 
\maketitle         

In vielen Aufgaben spielen Aussagekombinationen eine Rolle, von
denen dar"uberhinaus gesagt wird, dass einige falsch und andere
richtig sein sollen. Hier ist es oftmals sinnvoll, zuerst die Aussagen
in solche zu transformieren, die alle wahr sind. Dazu ist meist eine
Fallunterscheidung sinnvoll, wie etwa in der folgenden 
\begin{quote}
\ul{Aufgabe 240631}: Drei Sch"uler Wolfgang, Ralph und Udo belegten bei
einem Sportfest die ersten drei Pl"atze im Weitsprung. Marcus, der in
einer anderen Disziplin starten musste, erkundigte sich hinterher bei
Elke nach dem Ausgang beim Weitsprung. Elke konnte sich nicht mehr
genau erinnern und sagte: {\glqq}Ich glaube\\
(a) Wolfgang wurde nicht Erster,\\
(b) Ralph wurde nicht Zweiter, sondern\\
(c) Udo wurde Zweiter.{\grqq}

Es stellte sich heraus, dass Elke einmal etwas richtiges gesagt hatte,
sich aber in den beiden anderen F"allen geirrt hatte. Was kann man
"uber die Reihenfolge sagen?
\end{quote}
Um diese Aufgabe zu l�sen, m�ssen wir die Aussagen (a), (b) und (c)
entsprechend ihres (uns nicht genau bekannten) Wahrheitsgehalts kombinieren.
Jede der einzelnen Aussagen kann wahr oder falsch sein (im letzteren Fall ist
das Gegenteil wahr), und Aussagen k�nnen gemeinsam ({\bf und}-verkn�pft) oder
alternativ ({\bf oder}-verkn�pft). 

Um in solchen Aussagekombinationen nicht die �bersicht zu verlieren, wollen
auch wir die folgende in der Mathematik �bliche Notation verwenden. F�r zwei
Aussagen (A) und (B) gibt es die folgenden wichtigen Aussagekombinationen:
\begin{center}
  \begin{tabular}{c|c|c}
    Zeichen & Bedeutung & wie das die Mathematiker nennen\\\hline
    $\neg\,(A)$ & nicht (A) & Negation der Aussage (A)\\
    $(A) \wedge (B)$ & (A) und (B) & konjunktive Verkn�pfung \\
    $(A) \vee (B)$ & (A) oder (B) & disjunktive Verkn�pfung \\
  \end{tabular}
\end{center}
Beachte, dass {\glqq}(A) oder (B){\grqq} den Fall {\glqq}(A) und (B){\grqq}
einschlie�en (inklusives oder) oder ausschlie�en kann (exklusives oder).  Die
Mathematiker haben sich darauf geeinigt, dass immer das inklusive
{\glqq}oder{\grqq} gemeint ist, wenn nichts anderes gesagt wird. An diese
vereinbarung werden wir uns nat�rlich auch halten. 

Kehren wir zur Analyse von Elkes Aussage zur�ck. Wir unterscheiden drei
F"alle, je nachdem, ob (a), (b) oder (c) wahr sind. Unter Verwendung der eben
eingef�hrten Symbole und der Buchstaben $W, R, U$ f"ur den Platz des jeweilgen
Sportlers lauten die Aussagen f"ur die drei F"alle
\begin{align*}
\text{(a) wahr:\quad}& (W\neq 1) \wedge (R = 2) \wedge (U \neq 2)\tag{1}\\
\text{(b) wahr:\quad}& (W = 1) \wedge (R \neq 2) \wedge (U \neq 2)\tag{2}\\
\text{(c) wahr:\quad}& (W = 1) \wedge (R = 2) \wedge (U = 2)\tag{3}
\end{align*}
Wir sehen, dass Fall (2) (kein zweiter Platz) und Fall (3) (zwei
zweite Pl"atze) ausscheiden. Im Fall (1) schlie"sen wir nacheinander
$R = 2$, wegen $W\neq 1$ dann $W=3$ und $U=1$. Das w"are die einzig
m"ogliche Platzverteilung im Fall (1). Nun muss noch gepr"uft werden,
ob diese M"oglichkeit auch keine der Bedingungen verletzt, also die
Probe gemacht werden! Das ist hier sehr einfach, deshalb wollen wir es
so formulieren:
\begin{center}
Die Probe zeigt, dass es sich wirklich um eine L"osung handelt.
\end{center}
Also nicht zu fr"uh gefreut: {\glqq}Hurra, das ist die L"osung{\grqq}, denn es
k"onnte ja sein, dass das gefundene Ergebnis an einer Stelle, die noch nicht
beachtet wurde, ebenfalls nicht passt und die Aufgabe gar keine L"osung
besitzt.

Eine andere L"osung k"onnte 6 F"alle unterscheiden, n"amlich die 6
m"oglichen Reihenfolgen, und untersuchen, welche der Aussagen wahr
bzw. falsch sind:
\begin{center}
\begin{tabular}{ccc|ccc}
W& R& U& (1)&  (2)&  (3)\\\hline
1& 2& 3& f& f& f\\
1& 3& 2& f& w& w\\
2& 1& 3& w& w& f\\
2& 3& 1& w& w& f\\
3& 1& 2& w& w& w\\
3& 2& 1& w& f& f
\end{tabular}\end{center}
Nur im letzten Fall sind genau zwei der Aussagen falsch. Eine Probe
ist in diesem Fall {\bf nicht} notwendig, denn wir haben ja eine
vollst"andige Fallunterscheidung durchgef"uhrt, d.h. in {\em jedem}
Fall {\em alle} Bedingungen gepr"uft.
\medskip

Aussagekombinationen lassen sich nach einfachen Regeln umformen.  Die
wichtigsten sind die folgenden {\bf de'Morgan-schen Regeln}:
\begin{center}
  \begin{tabular}{c|c}
Regel & Interpretation\\\hline
$\neg(A\wedge B) \Leftrightarrow (\neg A) \vee (\neg B)$ & Das Gegenteil von
{\glqq}A {\bf und} B{\grqq} ist {\glqq}nicht A {\bf oder} nicht B{\grqq}.\\
$\neg(A\vee B) \Leftrightarrow (\neg A) \wedge (\neg B)$ & Das Gegenteil von
  {\glqq}A {\bf oder} B{\grqq} ist {\glqq}nicht A {\bf und} nicht B{\grqq}.
  \end{tabular}
\end{center}

\ul{Beispiel}: {\glqq}$(a=2)$ {\bf und} $(b\neq 3)$ ist falsch{\grqq} hei"st
{\glqq}$(a\neq 2)$ {\bf oder} $(b=3)${\grqq}.

\begin{attribution}
graebe (2004-09-02):\\ Dieses Material wurde vor einiger Zeit als
Begleitmaterial f�r den LSGM-Korrespondenzzirkel in der Klasse 7 erstellt und
nun nach den Regeln der KoSemNet-Literatursammlung aufbereitet.
\end{attribution}

\end{document}
