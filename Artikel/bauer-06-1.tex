\documentclass[a4paper,11pt]{article}
\usepackage{ko-math,kosemnet,ngerman}

\author{David Bauer, Freiberg}
\title{Spieltheorie\kosemnetlicensemark}
\date{}

\begin{document} 
\maketitle 

\section{Gewinnerzwingende Strategien}

In jedem Zweipersonenspiel, das den folgenden Regeln gen"ugt,
kann einer der beiden Spieler den Sieg erzwingen.
\begin{itemize}\setlength{\itemsep}{0pt}
\item Es gibt nur eine endliche Anzahl m"oglicher Stellungen. 
\item Das Spiel endet in jedem Fall nach einer endlichen Anzahl von Z"ugen.
\item Es gibt am Ende immer einen Sieger (kein Unentschieden).
\item Die m"oglichen Z"uge in einer Spielsituation sind f"ur beide Spieler
  gleich.
\end{itemize}
\paragraph*{"Ubung 1:} 
Welche der Regeln erf"ullt Schach?
\medskip

Die Endsituationen des Spiels, bei denen man gewonnen oder verloren hat,
bezeichnet man als Gewinner- bzw.\ Verlierersituation. Ferner hei"st auch jede
Spielsituation Gewinnersituation {\bf (GS)}, wenn man einen bestimmten Zug
finden kann, der den Gegner in eine Verlierersituation f"uhrt. Dagegen wird
jede Spielsituation als Verlierersituation {\bf (VS)} bezeichnet, wenn jeder
m"ogliche Zug dem Gegner zu einer Gewinnersituation verhilft. Auf diese Weise
l"asst sich durch R"uckw"artsschlie"sen f"ur jede Spielsituation entscheiden,
ob es sich um eine Gewinnersituation oder um eine Verlierersituation
handelt. 
\begin{quote}\it Der Spieler, der den ersten Zug macht, kann den Sieg genau
  dann erzwingen, wenn zu Beginn eine Gewinnersituation vorliegt.
\end{quote}

\paragraph*{"Ubung 2:} 
Anna und Bruno spielen mit einem St"uck Schokolade, dessen untere rechte Ecke
verschimmelt ist. Sie brechen die Tafel abwechselnd l"angs der Rippen entzwei
und essen eine der beiden H"alften auf. Wer das verschimmelte St"uck isst, hat
verloren. Anna macht den ersten Zug.

Wer kann den Sieg erzwingen, \ldots\
\begin{itemize}
\item[(a)] \ldots\ wenn sie mit einem St"uck Ritter-Sport-Schokolade ($4\times
  4$) beginnen?
\item[(b)] \ldots\ wenn sie mit einer Tafel Milka-Schokolade ($6\times 10$)
  beginnen?
\end{itemize}

\section{L"osungsmethoden f"ur Olympiadeaufgaben}

\subsection{R"uckw"artsarbeiten und Invarianzprinzip}

Dies ist wahrscheinlich die wichtigste Methode, um sukzessive alle Gewinner-
und Verlierersituationen zu ermitteln: 
\begin{quote}\it 
  Berechne durch R"uckw"artsschlie"sen einige GS und VS. Nach einiger Zeit
  erkennst du ein Muster (d.\,h.\ eine geeignete Invariante), die eine
  Darstellung von allen Gewinner- und Verlierersituationen liefert.
\end{quote}
\paragraph*{"Ubung 3:} Teile und Herrsche

Auf dem Tisch liegen zwei Haufen Kekse. Der eine Haufen besteht aus 31 Keksen,
der andere aus 43 Keksen. Ein Spielzug besteht darin, die Kekse eines Haufens
aufzuessen und den anderen in zwei Haufen zu zerteilen. Dabei darf kein Keks
zerbrochen werden, jedoch d"urfen die beiden neuen Haufen null Kekse
enthalten. Der Spieler, der bei seinem Zug keinen Keks essen kann, hat
verloren. Welcher Spieler kann den Sieg erzwingen?

\subsection{Vorw"artsarbeiten}
\begin{minipage}{.6\textwidth}
Manchmal gibt es zu viele Endpositionen, um durch R"uckw"artsschlie"sen das
Problem zu bearbeiten. Mitunter l"asst sich die Anzahl der (sinnvollen) Z"uge
des Gegners eingrenzen, so dass man ausgehend von der vorliegenden
Spielsituation durch Vorw"artsarbeiten zum Ziel gelangt. Ein typisches
Beispiel sind Schachr"atsel wie rechts (Matt in 2 Z"ugen).
\end{minipage}\hfill
\begin{minipage}{.35\textwidth}
\includegraphics[width=.8\textwidth]{bauer-06-1/bauer-06-1.eps}
\end{minipage}

\subsection{Zerlegung}
Eine wichtige Strategie ist es, das Spiel in kleine Bereiche zu zerlegen, um
dann in jedem dieser Bereiche den Sieg zu erzwingen.  Meistens ist folgende
Idee bereits f"ur die L"osungsfindung ausreichend: 
\begin{quote}\it
  Teile alle Spielpositionen in Paare ein, so dass es einen Zug zwischen
  beiden Elementen eines jeden Paares gibt. Immer wenn der Gegner die Position
  eines Paares annimmt, ziehst du auf die andere Position des Paares.
\end{quote}
\paragraph*{"Ubung 4:} 
Eine Tafel besteht aus sechs nebeneinander liegenden Feldern.  Zwei Spieler A
und B tragen abwechselnd so lange je eine der Ziffern $0,1,2,\ldots,9$ in ein
beliebiges noch freies Feld ein, bis alle Felder besetzt sind; dabei d"urfen
verschiedene Felder mit derselben Ziffer belegt werden. A f"angt an. Nachdem B
eine Ziffer in das letzte freie Feld eingetragen hat, werden die sechs
aneinandergereihten Ziffern als Dezimaldarstellung einer ganzen Zahl $z$
gedeutet.  B hat gewonnen, wenn $z$ durch eine vorher vereinbarte nat"urliche
Zahl $n$ teilbar ist. F"ur welche nat"urlichen Zahlen $n$ zwischen 1 und 15
kann B durch geschicktes Spiel den Gewinn erzwingen, f"ur welche nicht?

\subsection{Symmetrie}
H"aufig ist es n"utzlich, die symmetrischen Eigenschaften des Spielfelds
auszunutzen, um gegnerische Z"uge zu kontern. Dies macht sich unter anderem
auch Stefan in Northcotts Spiel zunutze (Aufgabe 451324).

\subsection{Indirektes Schlie"sen}
Es ist bekannt, dass bei einem Spiel nach unseren vier Regeln einer der beiden
Spieler eine Gewinnstrategie haben muss. Man kann daher indirekt beweisen,
dass ein Spieler bei optimalem Spiel verliert, indem man annimmt, er habe eine
Gewinnstrategie und dies zu einem Widerspruch f"uhrt.

\paragraph*{"Ubung 5:} Beim Doppelschach machen Schwarz und Wei"s abwechselnd
zwei g"ultige Z"uge nach den "ublichen Schachregeln, wobei Wei"s beginnt. Man
zeige, dass Wei"s bei optimalem Spiel Remis erzwingen kann!

\section{Aufgaben}

\paragraph*{Aufgabe 1:} 
Zwei Spieler spielen ein Spiel, bei dem sie abwechselnd Zahlen aus der Menge
$\{1,2,3,4,5,6,7,8,9\}$ nehmen.  Jede Zahl darf also nur einmal gew"ahlt
werden. Besitzt ein Spieler drei Zahlen, die sich zu 15 addieren, so hat er
gewonnen.  Wer gewinnt bei optimalem Spiel?  Oder gibt es ein Unentschieden?
\paragraph*{Aufgabe 2:} 
Beginnend bei 2 ersetzen zwei Spieler abwechselnd die aktuelle Zahl $N$ durch
$N+d$, wobei $d<N$ irgendein positiver Teiler von $N$ sein muss. Der Spieler,
bei dem die Zahl nach seinem Zug erstmals gr"o"ser oder gleich 2006 ist, hat
verloren. Welcher der beiden Spieler gewinnt?
\paragraph*{Aufgabe 3:} 
Zwei Personen P und Q spielen das folgende Spiel. In der Gleichung
$x^3+a\,x^2+b\,x+c=0$ belegt zun"achst P, danach Q und schlie"slich wieder P
je einen noch nicht belegten der drei Koeffizienten mit einer reellen
Zahl. Das Spiel ist genau dann f"ur P gewonnen, wenn die so entstandene
Gleichung drei paarweise verschiedene reelle L"osungen hat. Man untersuche, ob
P bei jeder Spielweise von Q den Gewinn erzwingen kann.
\paragraph*{Aufgabe 4:} 
Zwei Personen A und B machen folgendes Spiel. Sie nehmen aus der Menge
$\cbr{0, 1, 2, 3, \ldots, 1024}$ abwechselnd 512, 256, 128, 64, 32, 16, 8, 4,
2, 1 Zahlen weg, wobei A zuerst 512 Zahlen wegnimmt, B dann 256 Zahlen usw. Es
bleiben zwei Zahlen $a,b$ stehen. B zahlt an A den Betrag $|a-b|$. A m"ochte
m"oglichst viel gewinnen, B m"oglichst wenig verlieren. Welchen Gewinn erzielt
A, wenn jeder Spieler seiner Zielsetzung entsprechend optimal spielt?
\paragraph*{Aufgabe 5:} 
Clara und Dorothea spielen das Streckenspiel. Dieses Spiel wird mit 64 in
einem Quadratgitter angeordneten Punkten gespielt.  Clara beginnt das Spiel,
indem sie zwei senkrecht oder waagerecht unmittelbar benachbarte Punkte
miteinander verbindet. Im weiteren Spielverlauf besteht ein Zug darin, den
bestehenden Streckenzug um eine Strecke zu erweitern, indem einer der beiden
Endpunkte wiederum mit einem senkrecht oder waagerecht unmittelbar
benachbarten Punkt verbunden wird. Die einzige Einschr"ankung besteht darin,
dass man nicht zu einem Punkt ziehen darf, der bereits im Streckenzug
enthalten ist. Verloren hat diejenige, die keinen Zug mehr machen kann. Kann
eine der Spielerinnen den Sieg erzwingen? Wenn ja, gebe man eine Strategie an,
die mit Sicherheit zum Sieg f"uhrt.
\paragraph*{Aufgabe 6:} 
Wiebke und Stefan spielen mit einem Blatt K"astchenpapier.  Sie beginnen mit
einem 60 Zeilen und 40 Spalten gro"sen Rechteck und zerschneiden abwechselnd
in jedem Zug eines der Rechtecke auf dem Tisch entlang der Linien des
Karopapiers in zwei kleinere Rechtecke. Dabei darf Stefan bei seinen Z"ugen
nur senkrechte und Wiebke nur waagerechte Schnitte machen. Verloren hat, wer
keinen g"ultigen Zug mehr machen kann.
\begin{itemize}
\item[(a)] Wer kann den Sieg erzwingen, wenn Stefan beginnt?
\item[(b)] Wer kann den Sieg erzwingen, wenn Wiebke beginnt?
\end{itemize}

\begin{attribution}
Beitrag von David Bauer zum Begleitheft 2006 im Vorbereitungslehrgang der
S"achsischen Mannschaft auf die MO-Bundesrunde.
\end{attribution}
\end{document}

