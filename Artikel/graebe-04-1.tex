% Version: $Id: graebe-04-1.tex,v 1.1 2008/09/05 09:52:58 graebe Exp $
\documentclass[11pt]{article}  
\usepackage{kosemnet,ko-math,ngerman}  

\author{Hans-Gert Gr�be, Leipzig}
\title{Mathematische Aussagen und mathematische Beweise\kosemnetlicensemark\\ 
Arbeitsmaterial f�r Klasse 7}
\date{}

\begin{document} 
\maketitle         

In diesem Material geht es schwerpunktm"a"sig darum, wie mathematische
Aussagen bewiesen werden und wie solche Beweise aufzuschreiben sind. 

Mathematische S"atze bestehen in der Regel aus zwei Teilen, der {\em
Voraussetzung} und der {\em Behauptung} und haben die allgemeine Gestalt
\begin{center}
{\tt Wenn} die Voraussetzung erf"ullt ist, {\tt dann} gilt auch die
Behauptung.
\end{center}
Mathematische S"atze werden oft direkt mit diesen Worten formuliert:
\begin{quote}\it
{\tt Wenn} eine Zahl durch 9 teilbar ist, {\tt dann} ist sie auch durch 3
teilbar.

{\tt Wenn} zwei Dreiecke in zwei Seiten und dem eingeschlossenen Winkel
"ubereinstimmen, {\tt dann} sind sie kongruent.
\end{quote}
Aber auch anders formulierte S"atze kann man so umformulieren:
\begin{quote}\it
Die Innenwinkelsumme im Dreieck betr"agt $180^o$.
\begin{flushright}
  \parbox{12cm}{ {\tt Wenn} man die Winkelgr"o"sen der drei Innenwinkel eines
    Dreiecks addiert, {\tt dann} erh"alt man $180^o$.}
\end{flushright}

Stufenwinkel an geschnittenen Parallelen sind gleichgro"s. 
\begin{flushright}
  \parbox{12cm}{
    {\tt Wenn} zwei Winkel Stufenwinkel an geschnittenen Parallelen sind,
    {\tt dann} sind sie gleichgro"s.}
\end{flushright}
\end{quote}
Dasselbe gilt f"ur Aufgaben wie sie etwa bei der Matheolympiade gestellt
werden. Statt
\begin{quote}
Gegeben sei\ \ldots (V)\ldots \ Beweise, dass dann\ \ldots(B)\ldots\
gilt. 
\end{quote}
kann man auch
\begin{quote}
{\tt Wenn} \ldots (V)\ldots erf"ullt ist, {\tt dann} gilt auch
\ldots(B)\ldots\
\end{quote}
schreiben. 
\medskip

Aussagen dieser Art bezeichnet man als {\em Implikation} und schreibt auch
kurz $(V) \Rightarrow (B)$, wenn man die Struktur der Aussage besonders
betonen will. Hierbei stehen $(V)$ und $(B)$ als Kurzzeichen f"ur die
jeweilige konkrete Voraussetzung bzw. Behauptung.
\medskip

Eine zweite wichtige Aussageform ist die {\em "Aquivalenz}, die in dieser
allgemeinen Notation die Form
\begin{center}
(V) gilt {\tt genau dann, wenn} (B) gilt.
\end{center}
hat. Sie enth"alt einen Satz zusammen mit seiner Umkehrung, denn man kann eine
"Aquivalenz umformulieren zu den beiden Aussagen
\begin{quote}
{\tt Wenn} (V) gilt, {\tt dann} gilt auch (B)

{\bf und}

{\tt Wenn} (B) gilt, {\tt dann} gilt auch (V).
\end{quote}
F"ur eine solche "Aquivalenz schreiben wir auch kurz $(V) \Leftrightarrow
(B)$, denn sie bedeutet $(V) \Rightarrow (B)$ {\bf und} $(B) \Rightarrow (V)$.

\subsection*{Zum Beweisen mathematischer S"atze}

Wir hatten gesehen, dass mathematische S"atze wie kleine Bausteine
beschaffen sind, mit denen man die Verbindung zwischen zwei Aussagen
(V) und (B) herstellen kann. Wir k"onnen uns deshalb die Mathematik
als eine Welt kleiner Inseln, der verschiedenen Aussagen, vorstellen,
die durch Br"ucken, die S"atze, miteinander verbunden werden
k"onnen. Einen Satz (wie etwa Aufgabe 1) zu beweisen bedeutet deshalb
zuerst einmal, einen solchen Weg aus verschiedenen Br"ucken zu
finden. Dieser Weg startet auf der Insel (V) und muss auf der Insel
(B) enden. Als Br"ucken k"onnen wir nur bereits bekannte S"atze
verwenden. Einen solchen Weg bezeichnet man als {\em Schlusskette}.

Betrachten wir dazu die folgenden zwei Aufgaben:
\begin{quote}\it
Beweise den folgenden Satz: Es gibt keine Quadratzahl $n$, die bei Division
durch 3 den Rest 2 l"asst.
\end{quote}
\begin{aufgabe}
Formuliere diesen Satz als Wenn-Dann-Aussage.
\end{aufgabe}
\begin{beweis}
$n$ ist eine Quadratzahl $n=m^2$, wobei sich $m$ in der Form $m=3k$
oder $m=3k+1$ oder $m=3k+2$ darstellen l"asst (je nachdem, welchen
Rest $m$ bei Division druch 3 l"asst). 

1. Fall: Ist $m=3k$, so l"asst $n=m^2=9k^2$ bei Division durch 3 den
Rest 0.

2. Fall: Ist $m=3k+1$, so l"asst $n=m^2=9k^2+6k+1=3\cdot (3k^2+2k)+1$
bei Division durch 3 den Rest 1.

3. Fall: Ist $m=3k+2$, so l"asst $n=m^2=9k^2+12k+4=3\cdot
(3k^2+4k+1)+1$ bei Division durch 3 ebenfalls den Rest 1.
\end{beweis}

Diese Art des Beweises nennt man {\em Beweis durch vollst"andige
Fallunterscheidung}. Wir haben von unserer Insel Br"ucken zu {\em
allen} Nachbarinseln gebaut und von jeder aus einen Weg zu (B)
gefunden.
\begin{quote}\it
Beweise, dass f"ur positive reelle Zahlen $a,b$ stets
$\frac{a}{b}+\frac{b}{a}\geq 2$ gilt.
\end{quote}
\begin{beweis}
Wir formen die Ungleichung 
\begin{align*}\frac{a}{b}+\frac{b}{a}\geq 2\tag{1}\end{align*}
um, zuerst durch Multiplikation mit $ab$ zu
\begin{align*}a^2+b^2\geq 2ab\tag{2}\end{align*}
und schlie"slich durch Subtraktion von $2ab$ und Umformen zu 
\begin{align*}a^2-2ab+b^2 = (a-b)^2\geq 0.\tag{3}\end{align*}
Da das Quadrat einer reellen Zahl stets nichtnegativ ist, haben wir einen Weg
zu sicherem Festland gefunden. Allerdings haben wir den Weg vom
{\glqq}falschen{\grqq} Ende her gebaut, denn wir sind bei der Behauptung
gestartet. Unsere Schlusskette war $(1)\Rightarrow (2)\Rightarrow (3)$.  Diese
Art des Suchens nach einem Beweis bezeichnet man als {\em
R"uckw"artsarbeiten}.

Nachdem eine solche Stra"se aus mehreren Br"ucke gebaut, d.h.\ ein m"oglicher
Beweisweg gefunden ist, m"ussen wir nun pr"ufen, ob es sich auch wirklich um
einen Beweis handelt, denn Implikationen $(A_1)\Rightarrow (A_2)$ sind {\em
Einbahnstra"sen}-Br"ucken. Wir m"ussen also pr"ufen, ob unsere Br"ucken auch
so zusammenpassen, dass man von (V) nach (B) gelangen kann, d.h.\ ob die
Implikationen $(V)=(3) \Rightarrow (2)\Rightarrow (1)=(B)$ ebenfalls gelten.
Das ist aber offensichtlich der Fall, weil alle Umformungen r"uckg"angig
gemacht werden k"onnen, d.h wir {\em "aquivalent} umgeformt haben.
\end{beweis}

F"ur den eigentlichen Beweis h"atte es ausgereicht, die Beweiskette $(V)=(3)
\Rightarrow (2)\Rightarrow (1)=(B)$ aufzuschreiben.  Diese Richtung nennt man
{\em Vorw"artsarbeiten}. Oft, insbesondere bei geometrischen Aufgaben, ist es
zum Finden eines Beweisweges n"utzlich, zwischen Vorw"arts- und
R"uckw"artsarbeiten zu wechseln, also viele Br"ucken an verschiedenen Stellen
zu bauen, um einen Weg von der Voraussetzung zur Behauptung zu finden. F"ur
den gefundenen Weg muss aber stets gepr"uft werden, ob die gebauten Br"ucken
in der richtigen Richtung durchl"assig sind\,!
\medskip

"Ahnlich ist auch bei mathematischen {\em Bestimmungsaufgaben} vorzugehen. Der
einzige Unterschied besteht darin, dass (B) in diesem Fall keine Aussage,
sondern eine Bestimmungsfrage ist, die auf die Ermittlung einer L"osungsmenge
hinausl"auft.

\begin{attribution}
graebe (2004-09-02):\\ Dieses Material wurde vor einiger Zeit als
Begleitmaterial f�r den LSGM-Korrespondenzzirkel in der Klasse 7 erstellt und
nun nach den Regeln der KoSemNet-Literatursammlung aufbereitet.
\end{attribution}

\end{document}
