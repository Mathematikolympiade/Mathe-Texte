\documentclass[10pt]{article}  
\usepackage{kosemnet,ko-math,ngerman} 
\usepackage[utf8]{inputenc}   

\textheight21cm

\author{Dr. Peter Glatz (Erfurt) und Dr. Wolfgang Moldenhauer (Bad Berka)}
\title{Zum 150. Todestag des Princeps mathematicorum\kosemnetlicensemark}
\date{}

\newcommand{\Bild}[2]{
  \begin{center}
    \includegraphics*[width=#2\textwidth]{moldenhauer-06-1/#1}
  \end{center}
}

\begin{document} 
\maketitle 

\begin{minipage}{.4\textwidth}
  \Bild{gauss_bild.jpg}{.8}
\end{minipage}
\hfill
\parbox{6cm}{Carl Friedrich Gauß\footnotemark\\ $\ast$ 30. April 1777 in
  Braunschweig,\\ $\dag$ 23. Februar 1855 in Göttingen}

\footnotetext{Bild-Quelle:
\url{http://www.gaussjahr.de/images/gauss_bild.jpg}} 

\tableofcontents
\clearpage

\section{Carl Friedrich Gauß -- Fürst der Mathematiker}

{\large Zum 150. Todestag eines der bedeutendsten deutschen Gelehrten}

\subsection{Einleitung}

Mit der PISA-Studie ist der Mathematikunterricht in Deutschland wieder einmal
ins Gerede gekommen. Vom Lieblingsfach bis zum {\glqq}absoluten Horror{\grqq}
reichen die Sympathiebekundungen der Schüler -- und wohl auch der Eltern sowie
der gesamten Öffentlichkeit. Sicher kann diese wissenschaftliche Disziplin,
die ein wichtiger Bestandteil unserer Kulturgeschichte und auch des
naturwissenschaftlichen und technischen Fortschritts ist, den Menschen noch
ein wenig näher gebracht werden, wenn man bei aktuellen Gelegenheiten zeigt,
wie sie gewachsen ist und welche Personen sie unter konkreten historischen
Bedingungen geprägt haben. Nur wenige Namen stehen so für Reiz und Wert der
Mathematik wie der von Carl Friedrich Gauß. Man kennt die Gaußsche
Normalverteilung, die Gaußsche Zahlenebene, die Gaußsche Krümmung usw. Nach
ihm sind Schulen, Schüler-Wettbewerbe, Forschungsprojekte, Medaillen und
Preise, Straßen und Plätze benannt worden.

Anlass für alle diese Ehrungen ist der 150. Todestag des Mathematikers, der am
23. Februar 1855 in Göttingen verstorben ist. An seine Leistungen und seine
Person soll das {\glqq}Gauß-Jahr-2005{\grqq} erinnern, das neben anderen Orten
seine Geburtsstadt Braunschweig und die Stadt Göttingen mit zahlreichen
Veranstaltungen begehen. In einer zentralen Ausstellung im Braunschweigischen
Landesmuseum widmet man sich am Beispiel des Werdegangs des jungen Gauß dem
sehr aktuellen Thema {\glqq}Bildungsreform und Eliteförderung{\grqq}.

\subsection{Seine Ausbildung}

Von seiner frühzeitig erkennbaren mathematischen Begabung berichten einige
Anekdoten. So hat der Braunschweiger Schulmeister Büttner einmal den Schülern
seiner so genannten Rechenklasse die Aufgabe gestellt, die Zahlen von 1 bis
100 zusammenzuzählen. Während er erwartete, nun für den Rest der Stunde seine
Ruhe zu haben, legte der neunjährige Gauß nach ganz kurzer Zeit seine
Schiefertafel mit den Worten {\glqq}Ligget se.{\grqq} ({\glqq}Da liegt
sie.{\grqq}) auf den Lehrertisch. Der Junge hatte erkannt, dass man zur Lösung
des gestellten Problems die Zahlen umsortieren muss: $1+100$, $2+99$ usw. Man
bekommt so 50 Zahlenpaare, deren Summe immer 101 ist. Und das stand zum
Erstaunen des Lehrers auf der Tafel: $50\cdot 101 = 5050$. Ein Schüler hatte
das Prinzip der Summenformel für eine arithmetische Reihe erkannt! Durch
Fürsprache des Lehrers bei seinem Vater konnte Gauß ab Ostern 1788 ein
Gymnasium besuchen.

Für seine weitere Entwicklung spielte eine wichtige Begegnung eine
Rolle. Durch Vermittlung wurde das {\glqq}Wunderkind{\grqq} 1791 dem
Landesherrn Herzog Karl Wilhelm Ferdinand von Braunschweig vorgestellt (dessen
Schwester im Übrigen die Weimarer Großherzogin Anna Amalia war), der von
seinen Rechenkünsten beeindruckt war und für viele Jahre sein Gönner und
Förderer wurde. Nur so war es für Gauß finanziell möglich, ab 1792 das
Collegium Carolinum (die spätere Technische Hochschule Braunschweig) und ab
1795 die Georg-August-Universität Göttingen zu besuchen. Obwohl die
Universitätsstadt im {\glqq}Ausland{\grqq} lag, nämlich im Königreich
Hannover, verfügte der Herzog, dass man dem Studiosus Gauß einen Freitisch
(ein kostenloses Mittagessen) und jährlich 158 Taler gewähren solle.

Wegen seiner vielseitigen Interessen und Begabungen schwankte Gauß zunächst
noch in der Wahl seines Studienziels zwischen den alten Sprachen und der
Mathematik. Die Entscheidung fiel, als dem 19-jährigen Studenten der Nachweis
gelang, dass man mit Zirkel und Lineal ein regelmäßiges 17-Eck konstruieren
kann. Er fand die Lösung, indem er den Zusammenhang dieses schon sehr alten
geometrischen Problems mit der Zahlentheorie erkannte. Auch später war die
Zusammenschau anscheinend getrennter mathematischer Disziplinen eine der
Quellen seiner wissenschaftlichen Inspirationen. Die Beschäftigung mit der
Struktur von Zahlenmengen war zusätzlich angeregt worden, als ihm der Herzog
eine Logarithmentafel geschenkt hatte. Später sagte Gauß einmal in einem
Gespräch: {\glqq}Sie glauben gar nicht, wie viel Poesie in einer
Logarithmentafel steckt.{\grqq}

Die Promotion fand 1799 auf Wunsch des Herzogs an der damaligen
braunschweigischen Landesuniversität Helmstedt statt. Mit dem in der Arbeit
vorgelegten geschlossenen Beweis des Fundamentalsatzes der Algebra rückte Gauß
mit 22 Jahren unter die führenden Mathematiker seiner Zeit auf. 1801 erschien
seine erste große Monographie: {\glqq}Disquisitiones arithmeticae{\grqq}
(Arithmetische Untersuchungen). Mit dieser Schrift wurde die Zahlentheorie zur
{\glqq}Königin der Mathematik{\grqq}, wie Gauß sie gern bezeichnete. Die
Juwelen in der Krone waren für ihn die Primzahlen. Später erweiterte er den
Zahlenbegriff noch zu den komplexen Zahlen hin (Gaußsche Zahlenebene).

\subsection{Astronomie}

Ab Herbst 1801 widmete sich Gauß, der inzwischen mit Unterstützung des Herzogs
in Braunschweig als Privat-Gelehrter lebte, einem neuen Arbeitsgebiet. Da der
Landesherr in Braunschweig eine Sternwarte bauen wollte, befasste sich Gauß
zunehmend auch mit aktuellen astronomischen Problemen, z. B. mit der
Berechnung von Sternbahnen aus wenigen beobachteten Bahnelementen. Durch die
von ihm entwickelte Methode der kleinsten Quadrate zur Ausgleichung von
Beobachtungsfehlern konnte er ein brauchbares Verfahren zur Bahnbestimmung
entwickeln, das sogleich mit Erfolg angewendet wurde.

So hat der Astronom Franz Xaver von Zach (1754--1832) in der Neujahrsnacht
1801/02 in der Sternwarte auf dem Seeberg bei Gotha den Planetoiden Ceres
wiedergefunden, den der italienische Astronom Guiseppe Piazzi (1746--1826) in
Palermo bereits ein Jahr vorher entdeckt, aber dann wegen widriger
Lichtverhältnisse wieder aus seinem Beobachtungsfeld verloren hatte. Später
war Gauß zusammen mit seinem Schüler Johann Franz Encke (1791 1865), bekannt
durch den nach ihm benannten Kometen, auch einmal persönlich in der Seeberger
Sternwarte. Diese Entdeckung und einige astronomische Arbeiten hatten für Gauß
verlockende Angebote zur Folge. Zum Beispiel sollte er an der Sternwarte
St. Petersburg der Nachfolger des berühmten Leonhard Euler (1707--1783)
werden. Gauß wollte aber in Braunschweig bleiben.

Im Jahre 1809, genau 200 Jahre nach der Herausgabe von Keplers
{\glqq}Astronomia nova{\grqq}, erschien in Hamburg das Lehrbuch der rechnenden
Astronomie von Gauß: {\glqq}Theoria motus corporum coelestium in sectionibus
conibus solem ambientum{\grqq} (Theorie der Bewegung der in Kegelschnitten
sich um die Sonne bewegenden Himmelskörper). Der Autor stellte hierin die
Berechnung der kegelschnittförmigen Bahnen aus unterschiedlichen
Beobachtungsdaten dar und demonstrierte dabei weitreichende Methoden der
Störungsrechnung.

Einige Jahre später musste er seine Lebensplanung ändern, denn im Oktober 1806
wurde sein Gönner in der Schlacht von Jena und Auerstedt tödlich
verwundet. Die Pläne zur Errichtung einer Sternwarte in Braunschweig
zerschlugen sich und die Unterstützung aus den herzoglichen Kassen erlosch. So
folgte Gauß, der inzwischen auch eine Familie zu versorgen hatte, im November
1807 einem Ruf an die Universität Göttingen als Professor für Astronomie und
Direktor der dortigen Sternwarte. Dort blieb er bis zu seinem
Lebensende. Neben seinen Verpflichtungen als Hochschullehrer musste sich der
Gelehrte in ein für ihn völlig neues Aufgabenfeld einarbeiten.

\subsection{Geodäsie}

In der Zeit des Übergangs vom 18. zum 19. Jahrhundert beschäftigte man sich
international sehr intensiv mit der Bestimmung der genauen Größe und Gestalt
der Erde. Deshalb wurde von dem in Personalunion regierenden König von England
und Hannover, Georg IV., auch für das Königreich Hannover eine
Landesvermessung angeordnet, um damit den Anschluss an das entstehende
europäische Grad- und Vermessungsnetz herstellen zu können.

Mit der Gesamtleitung dieses Unternehmens hat Georg IV. im Jahr 1820 den
Göttinger Professor Gauß beauftragt. Das war für ihn eine riesige Aufgabe,
denn es mussten fast 2600 trigonometrische Punkte zum Teil erst eingerichtet
und dann vermessen werden, um dann das erhaltene Zahlenmaterial nummerisch
auswerten zu können. So arbeitete Gauß einige Jahre ganz praktisch im Gelände
mit und erfand zur Erleichterung der Arbeit ein geeignetes Instrument, das
Heliotrop, erfunden. Zeitweilig gehörte auch sein ältester Sohn Joseph zum
Vermessungstrupp. Das größte Dreieck in dem Triangulationsnetz wurde von den
Eckpunkten Brocken, Großer Inselsberg und Hoher Hagen bei Göttingen
gebildet. Zur Erinnerung an diese und weitere Vermessungsarbeiten wurde vor
zehn Jahren, am 17. Juni 1995, auf dem Großen Inselsberg ein Gedenkstein
enthüllt.\bigskip

\begin{minipage}{.3\textwidth}
  \Bild{Inselsberg_Trig.jpg}{1.0}
\end{minipage}\hfill
\begin{minipage}{.65\textwidth}
Trigonometrischer Punkt auf dem Inselsberg mit der
Inschrift\footnotemark:\\[4pt] Großer Inselsberg. Trigonometrischer Punkt
Erster Ordnung der Landesvermessung.\\[4pt] Seit dem zweiten Jahrzehnt des
19. Jahrhunderts wurde der Große Inselsberg wegen seiner exponierten Lage im
mitteldeutschen Raum für vermessungstechnische Grossaufgaben genutzt: für die
Hannoversche Gradmessung, die Landesvermessung des Königreiches Preussen, des
Kurfürstentums Hessen, der Herzogtümer Sachsen-Coburg-Gotha und
Sachsen-Meiningen, der Herrschaft Schmalkalden und für die Mitteleuropäische
Gradmessung. Mit diesen Arbeiten sind solche hervorragenden Persönlichkeiten
wie Carl Friedrich Gauss, Christian Ludwig Gerling und Peter Andreas Hansen
verknüpft.\\[4pt] Deutscher Verein für Vermessungswesen, Landesverein
Thüringen.  Bund der Öffentlich bestellten Vermessungsingenieure, Landesgruppe
Thüringen.  Im Juni 1995.
\end{minipage}

\footnotetext{Bild-Quelle:
  \url{http://lexikon.freenet.de/Bild:Inselsberg_Trig.jpg}}
\bigskip

\begin{minipage}{0.3\textwidth}
  \Bild{gauss_10DM.jpg}{1.0}
\end{minipage}\hfill
\begin{minipage}{0.65\textwidth}
Auch die Gestaltung der letzten deutschen 10-DM-Banknote war Gauß und seinen
geodätischen Leistungen gewidmet.

Die Vorderseite zeigt Carl Friedrich Gauß, der Hintergrund das Gebäude des
historischen Göttingen und eine Funktion, deren Graph links des Porträts
dargestellt ist. Diese wird als Gaußsche Normalverteilung bezeichnet.

Auf der Rückseite ist ein Sextant (Heliotrop) abgebildet, wie ihn Gauß für
Vermessungszwecke benutzt hat. Daneben ist eine Dreieckskette abgedruckt.
Diese Dreiecksketten bilden in ihrer Gesamtheit das Gerüst eines Netzes
topographischer Punkte 1. Ordnung. Sowohl Heliotrop als auch Dreieckskette
weisen auf die Beteiligung von C. F. Gauß bei den Vermessungsarbeiten des
Königreichs Hannover hin.
\end{minipage}

\footnotetext{Bild-Quelle: \scriptsize
  \url{http://www.didaktik.mathematik.uni-wuerzburg.de/history/ausstell/gauss/geldschein.html}} 
\bigskip

Seit dem Beginn der 20er Jahre waren diese Arbeiten aber auch sehr eng mit
seinen geometrischen Interessen verbunden. Schon während seiner Studienzeit
hat Gauß sehr ernsthaft über die Grundlagen der Geometrie nachgedacht. Im
vorliegenden praktischen Fall ging es ihm nun darum, die Abbildung der
gekrümmten Erdoberfläche auf ein ebenes Kartenblatt mit einer geeigneten
geometrischen Theorie zu fundieren. 1827 erschien sein Werk
{\glqq}Disquisitiones generales circa superficies curvas{\grqq} (Allgemeine
Untersuchungen über gekrümmte Flächen). Diese Flächentheorie von Gauß
begründete die Differentialgeometrie als selbstständiges mathematisches
Gebiet.

\subsection{Physik}

Im Jahr 1828 nahm Gauß als Gast von Alexander von Humboldt (1769--1859) an
der 7.~Versammlung deutscher Naturforscher und Ärzte in Berlin teil. Humboldt,
der in wissenschaftlichen Fragen am preußischen Hofe großen Einfluss hatte,
wollte den Gelehrten an die 1810 gegründete Universität Berlin, die heutige
Humboldt-Universität, holen. Das gelang ihm nicht. Er verstand es jedoch, Gauß
für ein damals sehr aktuelles Forschungsgebiet, den Erdmagnetismus, zu
interessieren. Gauß widmete sich auch sofort mit der ihm eigenen theoretischen
Tiefe dem neuen Arbeitsfeld.

Zusammen mit dem 1831 von Halle nach Göttingen berufenen Physiker Wilhelm
Weber (1804--1891) entwickelte er 1832 das absolute physikalische Maßsystem,
in dem die magnetischen wie auch die elektrischen Größen auf die mechanischen
Größen Länge, Zeit und Masse zurückgeführt werden. Vom Jahrgang 1836/37 an
wurde von Gauß und Weber die Heftreihe {\glqq}Resultate aus den Beobachtungen
des erdmagnetischen Vereins{\grqq} herausgegeben. Diese Reihe enthielt
u. a. 1838/39 die {\glqq}Gaußsche Allgemeine Theorie des
Erdmagnetismus{\grqq}.  Gauß gelangte durch Berechnungen zur Angabe der
ungefähren Lage der magnetischen Pole der Erde. Schiffsexpeditionen
bestätigten die Rechnungen wenig später. Heute noch widmet sich das deutsche
Forschungsschiff {\glqq}Gauß{\grqq} vielfältigen wissenschaftlichen Aufgaben.

Im Jahr 1831 entdeckte Michael Faraday (1791--1867) in England die
elektromagnetische Induktion. Bereits 1833/34 installierten Gauß und Weber den
ersten elektromagnetischen Telegraphen. Dazu wurde über die Dächer von
Göttingen hinweg eine doppelte Drahtverbindung vom Physikalischen Kabinett in
der Innenstadt zu der außerhalb der Stadtmauern liegenden Sternwarte gezogen,
eine Strecke von beinahe zwei Kilometern, über die sich in beide Richtungen
kräftige Stromstöße übertragen ließen. Die Anlage war zwölf Jahre lang in
Betrieb, bis sie 1845 durch einen Blitzschlag zerstört wurde.

Gauß erkannte sofort auch die bedeutenden möglichen Auswirkungen des
Telegraphen auf das sich rasch entwickelnde Eisenbahnwesen. 1835 wurde die
erste deutsche Strecke zwischen Nürnberg und Fürth in Betrieb genommen. Längs
dieser Linie legte der Münchener Professor Carl August Steinheil (1801 -
1870), der auch bei Gauß studiert hatte, eine Telegraphenverbindung an, zu
deren technischer Verbesserung Gauß beitragen konnte. Das Interesse des
Gelehrten am Eisenbahnbau wurde außerdem dadurch gefördert, dass sein ältester
Sohn nach 1846 in leitender Stellung beim Bau der Hannoverschen
Eisenbahnlinien tätig war. An der Einweihung der Linie von Kassel nach
Göttingen am 31. Juli 1854 nahm Gauß noch persönlich teil.

\subsection{Ehrungen}

In seinen letzten Lebensjahren hatte Gauß noch zwei weitere Schüler, die seine
mathematischen Arbeiten fortsetzen konnten und damit ganz wesentlich zum
Fortschritt der Mathematik im 19. Jahrhundert beigetragen haben. Richard
Dedekind (1831--1916), der 1852 bei Gauß promovierte, trug mit seiner Theorie
der Irrationalzahlen zum strengen Aufbau des Zahlensystems bei. 1888 erschien
seine Schrift {\glqq}Was sind und was sollen die Zahlen?{\grqq}, die bis heute
viele Auflagen erreicht hat. Bernhard Riemann (1826--1866), der 1851 bei Gauß
mit einer wegweisenden Arbeit zur Grundlegung der Funktionentheorie promoviert
hat, konnte mit seinem im Juni 1854 in Göttingen gehaltenen
Habilitationsvortrag {\glqq}Über die Hypothesen, welche der Geometrie zu
Grunde liegen{\grqq} wesentlich zum systematischen Aufbau der
nichteuklidischen Geometrie beitragen. Riemann gab damit wichtige Anregungen
zur Darstellung des Zusammenhangs von Raumstruktur und physikalischen
Gesetzmäßigkeiten, die dann schließlich bei Albert Einstein (1879--1955) in
die Grundlagen der Speziellen und der Allgemeinen Relativitätstheorie
einmündeten.

Die Leistungen von Gauß wurden weltweit gewürdigt. Die Royal Society in London
ehrte ihn 1838 mit der Copley-Medaille. Die russische Universität im Kasan
ernannte ihn zum Ehrenmitglied. Die Städte Braunschweig und Göttingen
verliehen ihm die Ehrenbürgerschaft. Er trug den Titel eines Geheimen
Hofrates.

Am 17. März 1796 hatte Gauß als junger Student begonnen, ein
wissenschaftliches Tagebuch in lateinischer Sprache zu führen, das man erst am
Ende des 19. Jahrhunderts wiederfand. Es gewährte interessante Einsichten in
sein gesamtes wissenschaftliches Denken und Arbeiten. Darunter waren auch
Entdeckungen, die Gauß selbst nicht publiziert hat, getreu seinem Wahlspruch
{\glqq}Pauca, sed matura{\grqq} (Weniges, aber Ausgereiftes). Teile des
Tagebuches wurden später veröffentlicht. Das Gesamtwerk von Gauß in zwölf
Bänden wurde in den Jahren von 1863 bis 1933 von der Gesellschaft der
Wissenschaften zu Göttingen herausgegeben. Auch einige Briefwechsel, z. B. der
mit Alexander von Humboldt, mit Friedrich Wilhelm Bessel und mit Wolfgang von
Bolyai sind in gedruckter Form erschienen.

Der König von Hannover ließ noch im Todesjahr 1855 eine Gedenkmedaille auf
Gauß prägen, auf der er als {\glqq}Mathematicorum princeps{\grqq} (Fürst der
Mathematiker) bezeichnet wird.

\section{Gauß als Namensgeber}

Nach Carl Friedrich Gauß sind Methoden, Verfahren oder Ideen benannt worden,
denen man in der Schule, im Studium oder in Anwendungen begegnet, wobei einige
auch auf Weiterentwicklungen beruhen. Die nachfolgende unvollständige
Zusammenstellung dokumentiert seine Leistungen nachhaltig:  
\begin{itemize}\setlength{\itemsep}{0pt}
\item das Gaußsche Eliminationsverfahren zur Lösung von linearen
  Gleichungssystemen,
\item der Gauß-Jordan-Algorithmus, eine Weiterentwicklung des Gaußschen
  Eliminationsverfahren,
\item das Gauß-Newton-Verfahren, ein Verfahren zur Lösung nichtlinearer
  Gleichungen,
\item das Gauß-Seidel-Verfahren, ein Verfahren zur Lösung von linearen
  Gleichungssystemen,
\item das Gaußsche Fehlerfortpflanzungsgesetz,
\item das Gaußsche Fehlerintegral,
\item der Gaußsche Integralsatz in der Vektoranalysis, auch als Satz von
  Gauß-Ostrogradski oder Divergenzsatz bezeichnet,
\item die Gaußsche Krümmung in der Differentialgeometrie,
\item der Satz von Gauß-Bonnet in der Differentialgeometrie,
\item die Gaußsche Osterformel zur Berechnung des Osterdatums,
\item das Gaußsche Prinzip des kleinsten Zwanges in der Mechanik,
\item die Gaußsche Methode der kleinsten Quadrate,
\item die Gaußsche Quadraturformeln, ein numerisches Integrations-Verfahren,
\item die Gaußsche Normalverteilung, auch Gaußsche Glockenkurve genannt,
\item die Gaußschen Zahlen, eine Erweiterung der ganzen Zahlen ins Komplexe,
\item die Gaußsche Zahlenebene als geometrische Darstellung der Menge der
  komplexen Zahlen,
\item die Gaußklammer, eine Funktion, die eine Zahl auf die nächstkleinere
  ganze Zahl abrundet (Beispiele: $[3,2] = 3$, $[-3,2] = -4$),
\item das Gauß-Geschütz, ein Geschütz, das ein ferromagnetisches Projektil
  mittels \mbox{(Elektro-)}\-Magneten beschleunigt, ähnlich einem Linearmotor,
\item der Gauß-Prozess, ein stochastischer Prozess das Gauß-Markov-Theorem
  über die Existenz eines BLUE-Schätzers in linearen Modellen,
\item das Gauß-Krüger-Koordinatensystem.
\end{itemize}

\section{Konstruktion regelmäßiger $n$-Ecke}

\begin{minipage}{.3\textwidth}
  \Bild{gauss3.jpg}{1.0}
\end{minipage}\hfill
\begin{minipage}{.65\textwidth}
Um Gauß anlässlich des 200. Geburtstags (30.~April 1977) zu ehren, wurde eine
Briefmarke{\footnotemark} herausgegeben. Sie stellt den Mathematiker als
junger Mann dar. Abgebildet sind ferner ein regelmäßiges 17-Eck, ein Zirkel
und ein Lineal (Zeichendreieck).
\end{minipage}
\footnotetext{Bild-Quelle: \url{http://members.tripod.com/jeff560/gauss3.jpg}}
\bigskip

Heute besteht das Logo des Mathematik-Olympiaden e.\,V.\ aus diesen
Bausteinen.  Eine aus\-führliche Konstruktionsbeschreibung findet man unter
\begin{center}
  \url{www.mathematik-olympiaden.de/17Eck.pdf},
\end{center}
eine darauf basierende verkürzte unter
\begin{center}
  \url{www.matheolympiade-thueringen.de/olympiade/daslogo.html}.
\end{center}

Zu bemerken ist, dass der Ausdruck 
\[\cos\br{\frac{360\grad}{17}} = -\frac{1}{16}+\sqrt{17} +
\frac{1}{16}\sqrt{34-2\sqrt{17}} +\frac18 \sqrt{17+3\sqrt{17}
  -\sqrt{34-2\sqrt{17}} -2\sqrt{34+2\sqrt{17}}}
\] 
mit Zirkel und Lineal zu konstruieren ist. Dies ist auch unter mehrfacher
Anwendung des Satzes von Pythagoras möglich. So ist z.\,B.\ $\sqrt{17}$ als
Hypotenuse eines rechtwinkligen Dreiecks mit den Katheten 4 und 1
konstruierbar. Natürlich kann man die Wurzel eines Ausdrucks auch unter
Ausnutzung des Katheten- oder Höhensatzes erhalten. Theoretische Informationen
über die Konstruktionen mit Zirkel und Lineal werden z.\,B.\ unter
\begin{center}
  \url{http://www.mathematik.uni-osnabrueck.de/script/zirkelundlineal/node2.html}
\end{center}
gegeben.

Mit Schülerinnen und Schüler kann man die Frage diskutieren, welche
regelmäßigen $n$-Ecke sich mit Zirkel und Lineal konstruieren lassen. Dies
gelingt beim Dreieck, Viereck (Quadrat) und Fünfeck. Die Konstruktionen
beruhen darauf, wie man die zugehörigen Innenwinkel von $60\grad$, $90\grad$
und $108\grad$ konstruiert. Äquivalent dazu ist die Konstruktion der
zugehörigen Mittelpunktswinkel $120\grad$, $90\grad$ und $>72\grad$. Eine
Konstruktionsbeschreibung für ein regelmäßiges Fünfeck wird unter
\begin{center}
  \url{http://mathworld.wolfram.com/Pentagon.html}
\end{center}
und eine weitere unter
\begin{center}
  \url{http://delphi.zsg-rottenburg.de/mathe.html}
\end{center}
gegeben.

Aus Dreieck, Viereck und Fünfeck gewinnt man die Konstruierbarkeit einer
Vielzahl anderer $n$-Ecke. Indem man die Mittelpunktswinkel von Dreieck und
Fünfeck überlagert, erhält man ihre Differenz mit $48\grad$, durch Halbierung
dieses Winkels $24\grad$ und auf diese Weise ein 15-Eck mit einem Innenwinkel
von $156\grad$. Durch (wiederholte) Halbierung der Mittelpunktswinkel lassen
sich (und dies ist schon seit der Antike bekannt) regelmäßige $n$-Ecke für
\begin{gather*}
  n = 3, 4, 5, 6, 8, 10, 12, 15, 16, 20, 24\ \text{usw.}
\end{gather*}
konstruieren. In dieser Aufzählung fehlen wichtige Fälle wie $n = 7, 9, 11$
usw.

Gauß bewies 1796:
\begin{quote}\em 
  Ist $n$ eine {\glqq}Fermatsche Zahl{\grqq}, d.\,h.\ von der Form
  $n=2^{2^k}+1$, wobei $k$ eine natürliche Zahl und $n$ eine Primzahl ist, so
  lässt sich ein regelmäßiges $n$-Eck mit Zirkel und Lineal konstruieren.
\end{quote}
Gauß war also der Erste, der die Möglichkeit einer solchen Konstruktion für $n
= 17,\ 257$ und 65537 (alles Fermatsche Primzahlen) nachweisen konnte. Man
kennt im Übrigen auch konkrete Konstruktionsvorschriften für diese $n$-Ecke,
die mit wachsendem $n$ aufwändiger werden. Wie dies für $n = 65537$ zu
bewerkstelligen ist, kann man bei J.\,Hermes\footnote{Siehe
  \url{http://de.wikipedia.org/wiki/65537-Eck}.} nachlesen.  Tatsächlich soll
J.\,Hermes die Konstruktion auch vollbracht haben, wozu er anscheinend 10
Jahre brauchte. In der Institutsbibliothek des Mathematischen Instituts der
Universität Göttingen soll es einen Koffer von J.\,Hermes mit den Zeichnungen
zu obiger Arbeit geben.

Analog zum 15-Eck gilt, dass man durch Multiplikation verschiedener
Fermatscher Primzahlen weitere Konstruktionen mit Zirkel und Lineal erhält,
also z.\,B.\ für  $n = 51\ (= 3\cdot 17)$  oder für  $n = 255\
(= 3\cdot 5\cdot 17)$. Denn man weiß aus der Zahlentheorie, dass es für zwei
teilerfremde natürliche Zahlen $n_1, n_2$  immer ganzzahlige Faktoren  $a_1$ 
und  $a_2$  gibt mit  $a_1 n_1  + a_2 n_2  = 1$, woraus $\frac{a_2}{n_1}  +
\frac{a_1}{n_2}  = \frac{1}{n_1 n_2}$ folgt. Kann man also das  $n_1$-Eck und
das  $n_2$-Eck konstruieren, so erhält man auf diese Weise mit Hilfe der
Mittelpunktswinkel das $(n_1\cdot n_2)$-Eck. Will man etwa das 51-Eck
konstruieren ($n_1 = 3,\ n_2 = 17$), so sind  $a_1 = 6,\ a_2 = -1$, also
bildet man die Differenz des 6-fachen Mittelpunktswinkels des 17-Ecks und des
Mittelpunktswinkels des Dreiecks: $\br{\frac{6}{17}-\frac{1}{3}}360\grad =
\frac{1}{51} 360\grad$. Dieses Verfahren lässt sich für mehr als zwei
teilerfremde Faktoren fortsetzen, z.\,B.\ für  $n = 255 = 5\cdot 51\
( = 5\cdot 3\cdot 17)$.

Weiterhin lassen sich die Mittelpunktswinkel fortgesetzt halbieren, also folgt
aus der Konstruierbarkeit des regelmäßigen $n$-Ecks auch die des  regelmäßigen
($2^k\cdot n$)-Ecks.

Pierre Laurent Wantzel bewies 1836, dass auch die Umkehrung der Gaußschen
Entdeckung gilt. Er zeigte: Aus der Konstruierbarkeit des
regelmäßigen $n$-Ecks folgt, dass die Primfaktorzerlegung von $n$ nur eine
Potenz von 2 und verschiedene Fermatsche Primzahlen enthält.

Es gilt also:
\begin{quote}\em
  Ein  regelmäßiges $n$-Eck lässt sich genau dann mit Zirkel und Lineal
  konstruieren, wenn sich $n$ als $n=2^k\cdot p_1\cdot\ldots\cdot p_m$ mit
  verschiedenen Fermatschen Primzahlen $p_i$ darstellen lässt.
\end{quote}

Damit ist aber die Frage immer noch nicht abschließend gelöst, denn man weiß
bis heute nicht, ob die Anzahl der Fermatschen Primzahlen endlich ist. Es ist
nicht einmal bekannt, ob es über die fünfte Fermatsche Primzahl
\[2^{2^4}+1=65537\]
hinaus weitere Fermatsche Primzahlen gibt. Die nächsten 28 folgenden
Fermatschen Zahlen
\[2^{2^k}+1\ \text{für}\ k=5,6,7,\dots,32\]
sind als zusammengesetzt nachgewiesen worden. Der Fall $k = 33$ scheint
ungeklärt zu sein. Von insgesamt 218 Fermatschen Zahlen weiß man, dass sie
zusammengesetzt sind (siehe auch
\url{http://de.wikipedia.org/wiki/Fermat-Zahl}). Ein Überblick über alle
$n$-Ecke, die mit Zirkel und Lineal konstruierbar sind, wird in \cite{Fuchs}
gegeben.

\section{Literatur}

\begin{thebibliography}{xxx}
\bibitem{GW} Gesellschaft der Wissenschaften zu Göttingen (Hrsg.): Gauß,
  C.\,F. Werke. 12 Bände, Göttingen. 1863--1933.
\bibitem{MTB} Gauß, C.\,F.: Mathematisches Tagebuch 1796--1814. In :
  Ostwalds Klassiker, Bd. 256 Leipzig 1985.
\bibitem{Wu-1} Wußing, H.: Vorlesungen zur Geschichte der Mathematik. Berlin
  1979. 
\bibitem{Wu-2} Wußing, H.: Carl Friedrich Gauß. Leipzig 1989.
\bibitem{Sau} Du Sautoy, M.: Die Musik der Primzahlen. Auf den Spuren des
  größten Rätsels der Mathematik. München 2004.
\bibitem{ZVW} Zeitschrift für Vermessungswesen 120 (1995) 9, S. VII.
\bibitem{Mar} Marwinski, T.: Zur Geschichte der Astronomie in Thüringen und
  Mitteilungen über Astronomen, die aus Thüringen stammen. Weimar 1992,
  Selbstverlag des Verfassers (überarbeitete Fassung eines am ThILLM Arnstadt
  gehaltenen Vortrags).
\bibitem{Her} Hermes, J.: Über die Teilung des Kreises in 65537 gleiche Teile.
  Nachrichten von der Gesellschaft der Wissenschaften zu Göttingen,
  Math.-Phys. Klasse, Bd. 1894, Heft 3, S. 170--186.
\bibitem{Fuchs} Fuchs, H.: C.\,F. Gauss und die regelmäßigen $n$-Ecke. Monoid
  25 (2005) 82, S. 10--11.
\end{thebibliography}


\begin{attribution}
moldenhauer (2006-01-04):\\ Text für KoSemNet freigegeben. Der Text wurde
zuerst 2005 als Manuskriptdruck herausgegeben vom Thüringer Institut für
Lehrerfortbildung, Lehrplanentwicklung und Medien (Thillm)

graebe (2006-05-02): Umsetzung in \LaTeX\ für das KoSemNet-Projekt
\end{attribution}

\end{document}
