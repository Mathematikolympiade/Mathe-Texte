% Version: $Id: graebe-04-2.tex,v 1.1 2008/09/05 09:52:58 graebe Exp $
\documentclass[11pt]{article}  
\usepackage{kosemnet,ko-math,ngerman}  

\def\gcd{{\rm ggT}}

\author{Hans-Gert Gr�be, Leipzig}
\title{Berechnung des gr"o"sten gemeinsamen Teilers (ggT)\\ mit
dem Euklidschen Algorithmus\kosemnetlicensemark\\
Arbeitsmaterial f�r Klasse 7}
\date{}

\begin{document} 
\maketitle         

Das hier besprochene Verfahren basiert auf dem {\em Satz von der
Division mit Rest}:
\begin{satz}
Zu zwei positiven nat"urlichen Zahlen $a,b$ gibt es eindeutig
bestimmte nat"urliche Zahlen $q,r$ mit $0\leq r<b$, so dass 
\[a=q\cdot b+r\]
gilt. $q$ nennt man den {\bf Quotienten} und $r$ den {\bf Rest} bei
Division von $a$ durch $b$. 
\end{satz}
Die beiden Gr"o"sen $q$ und $r$ kann man leicht bestimmen: $q$ ist
gerade der Vorkommaanteil, wenn man die Division $a/b$ auf dem
Taschenrechner ausf"uhrt. Die Mathematiker haben f"ur diesen
Vorkommateil einer (positiven reellen) Kommazahl $x$ das spezielle
Symbol $[x]$ eingef"uhrt (sprich: der ganze Teil von $x$). Wir k"onnen
also schreiben
\[q:=\left[\frac{a}{b}\right].\]
Der Rest $r$ ergibt sich dann eindeutig aus der Beziehung
\[r:=a-q\cdot b,\]
was man wieder auf dem Taschenrechner ausrechnen kann. 

Weil beide Funktionen so einfach sind, spielen sie auch in Computern
und Programmiersprachen eine wichtige Rolle. In Pascal kann man
z.B. $q$ und $r$ mit den Funktionen {\tt q:=a div b} und {\tt r:=a mod b}
berechnen und damit den Euklidschen Algorithmus sehr einfach
programmieren. 
\medskip

Zur ggT-Berechnung verwenden wir die folgende wichtige Eigenschaft,
die $a,\ b$ und $r$ verbindet: Es gilt $\gcd(a,b)=\gcd(b,r)$. Warum gilt
diese Beziehung?  Betrachten wir die Relation $a=q\,b+r$, so sehen
wir, dass {\em jeder} gemeinsame Teiler von $a$ und $b$ auch ein
Teiler von $r$ ist und umgekehrt jeder gemeinsame Teiler von $b$ und
$r$ auch ein gemeinsamer Teiler von $a$ ist. Die {\em Mengen} der
gemeinsamen Teiler von $a$ und $b$ bzw. von $b$ und $r$ stimmen also
"uberein und damit auch der {\em gr"o"ste} gemeinsame Teiler.
\medskip

Betrachten wir nun ein \ul{Beispiel}. Wir wollen $\gcd(99,69)$
berechnen. Nach dem aus der Schule bekannten Verfahren w"urden wir
dazu jede der beiden Zahlen in Faktoren $${99=3^2\cdot 11,\quad
69=3\cdot 23}$$ zerlegen und gew"annen durch Vergleich der Exponenten
die Antwort $ggT=3$.

Stattdessen k"onnen wir auch fortgesetzte Division mit Rest anwenden: 
\[\begin{array}{rcl}
99=1\cdot 69+30,&\mbox{also gilt}& \gcd(99,69)=\gcd(69,30)\\
69=2\cdot 30+9,&\mbox{also gilt}& \gcd(69,30)=\gcd(30,9)\\
30=3\cdot 9+3,&\mbox{also gilt}& \gcd(30,9)=\gcd(9,3)\\
9=3\cdot 3,&\mbox{also gilt}& \gcd(9,3)=3,
  \end{array}
\]
denn die letzte Division geht ja auf. Setzen wir die Beziehungen, die jeweils
hinter den Worten {\glqq}also gilt{\grqq} stehen, zusammen, so erhalten wir
schlie"slich $\gcd(99,69)=3$. Dieses Verfahren der fortgesetzten Division mit
Rest bezeichnet man als den {\bf Euklidschen Algorithmus}. Da die Reste in
jedem Schritt kleiner werden, landen wir nach endlicher Zeit {\em immer} bei
einer Division mit Rest 0.

Beachte, dass wir bei den Rechnungen keine Zerlegung in Faktoren
verwendet haben. In einfachen Beispielen wie dem hier betrachteten ist
eine solche Zerlegung schnell zu finden. Allgemein muss man daf"ur
aber ziemlichen Aufwand betreiben.

\begin{aufgabe}
  Bestimme $\gcd(2021027,3028009)$. Versuche, eine Zerlegung in Faktoren zu
  finden, ohne das Ergebnis unserer weiteren Rechnung zu betrachten.
\end{aufgabe}

Der Euklidsche Algorithmus, den man bei diesen Zahlen noch ohne
Probleme mit dem Taschenrechner bew"altigen kann, liefert nacheinander
\begin{align*}
3028009=&1\cdot 2021027 +1006982,\\
&\qquad\text{also }\gcd(3028009,2021027)=\gcd(2021027,1006982)\\ 
2021027=&2\cdot 1006982 + 7063,\\
&\qquad\text{also }\gcd(2021027,1006982)=\gcd(1006982,7063)\\ 
1006982=&142\cdot 7063 + 4036,\\&\qquad\text{ also }
\gcd(1006982,7063)=\gcd(7063,4036)\\ 
7063 =& 1\cdot 4036 + 3027,\\&\qquad\mbox{  also  }
\gcd(7063,4036)=\gcd(4036, 3027)\\  
4036 =& 1\cdot 3027 + 1009,\\&\qquad\mbox{  also  } 
\gcd(4036, 3027)=\gcd(3027, 1009)\\  
3027 =& 3\cdot 1009,\\&\qquad\mbox{  also  }\gcd(3027, 1009) = 1009\\ 
\end{align*}
Mit Hilfe des Euklidschen Algorithmus kann man sogar ganze Klassen von
Aufgaben wie zum Beispiel die folgende l"osen:
\begin{aufgabe}
  Untersuche, welchen gr"o"sten gemeinsamen Teiler die Zahlen
  $2n+3$ und $3n+2$ f"ur verschiedene nat"urliche Zahlen $n$ haben. 
\end{aufgabe}
Um uns ein Gef"uhl f"ur die Aufgabenstellung zu verschaffen, wollen wir zuerst
einmal den ggT f"ur einige Werte von $n$ bestimmen. Die entsprechenden
Aufgaben und L"osungen lauten ($n=0,\ldots,10$):
\[\begin{array}{ccc}\gcd(3,2)=1,& \gcd(5,5)=5,& \gcd(7,8)=1\\
\gcd(9,11)=1,& \gcd(11,14)=1,& \gcd(13,17)=1\\ \gcd(15,20)=5,&
\gcd(17,23)=1,& \gcd(19,26)=1\\ \gcd(21,29)=1,& \gcd(23,32)=1 \end{array}
\]
Wir erkennen, dass die beiden Zahlen meist teilerfremd sind, aber
manchmal auch einen gemeinsamen Teiler 5 haben.

Der Euklidsche Algorithmus hilft uns weiter, diese Beobachtung in eine
Gesetzm"a"sigkeit zu gie"sen: Egal, welchen Wert die Zahl $n$ annimmt,
es gilt immer 
\[\begin{array}{ccc}
3n+2 = 1\cdot (2n+3) + (n-1),&\mbox{  also  }& 
\gcd(3n+2,2n+3)=\gcd(2n+3,n-1)\\ 
2n+3 = 2\cdot (n-1) + 5,&\mbox{  also  }& 
\gcd(2n+3,n-1)=\gcd(n-1,5)
  \end{array}
\]
Das ist zwar nicht genau der Euklidsche Algorithmus, denn f"ur $n=3$
steht etwa in der zweiten Zeile $9=2\cdot 2+5$ und nicht $9=4\cdot
2+1$. Aber die Tatsache, dass $r<b$ sein soll, haben wir ja beim
Beweis, dass $\gcd(a,b)=\gcd(b,r)$ gilt, nicht verwendet, sondern erst,
als es darum ging, zu zeigen, dass der Euklidsche Algorithmus immer
auf eine exakte Division st"o"st und damit terminiert. Wir k"onnen
also f"ur unsere Aufgabe wenigstens sagen, dass f"ur jedes $n$
\[\gcd(3n+2,2n+3)=\gcd(n-1,5)\]
gilt. Da 5 aber nur die Teiler 1 und 5 hat, erkennen wir: Wenn $n-1$
nicht durch 5 teilbar ist, dann sind die beiden Ausgangszahlen
teilerfremd. Ist dagegen $n-1$ durch 5 teilbar, so ist der gesuchte
$ggT=5$. Die Antwort lautet also 
\[\gcd(3n+2,2n+3) =
  \begin{cases}
    5 & \text{wenn } 5\,|\,n-1 \\ 1 & \text{sonst}
\end{cases}\] 

\begin{attribution}
graebe (2004-09-02):\\ Dieses Material wurde vor einiger Zeit als
Begleitmaterial f�r den LSGM-Korrespondenzzirkel in der Klasse 7 erstellt und
nun nach den Regeln der KoSemNet-Literatursammlung aufbereitet.
\end{attribution}

\end{document}

