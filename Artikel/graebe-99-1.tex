% Beitrag f"ur das Heft zum Landesseminar M"arz 1999

\documentclass[11pt]{article}
\usepackage{ko-math,kosemnet,ngerman,graphicx}

\newcommand{\Bild}[3]{
\begin{center}
\includegraphics[#1]{graebe-99-1/#2}\nopagebreak\\[12pt] #3
\end{center}
}

\title{Eulersche Gerade und Feuerbachscher Kreis\kosemnetlicensemark}
\author{Hans-Gert Gr"abe, Leipzig} 
\date{6. Januar 1999}

\begin{document}
\maketitle

Tripel von Geraden, wie etwa die H"ohen, Seitenhalbierenden oder die
Winkelhalbierenden eines Dreiecks $\ktriangle{ABC}$, fasst man unter dem
Oberbegriff der {\em Ecktransversalen} zusammen. Neben den genannten gibt es
noch eine Reihe anderer Ecktransversalen, die ebenfalls oft bemerkenswerte
Eigenschaften besitzen.

Bekanntlich schneiden sich die drei H"ohen, Seitenhalbierenden,
Winkelhalbierenden und Mittelsenkrechten (letztere geh"oren allerdings nicht
zu den Ecktransversalen) jeweils in einem Punkt.  Drei solche Geraden, die
durch einen gemeinsamen Punkt gehen, bezeichnet man als {\em konkurrent}. F"ur
die Winkelhalbierenden und Mittelsenkrechten ist dies leicht einzusehen, indem
man ihre Eigenschaften als {\em geometrischer Bestimmungsort} heranzieht.  Sie
sind bekanntlich die Menge aller Punkte, die von zwei gegebenen Geraden
(Winkelhalbierende) bzw. zwei gegebenen Punkten (Mittelsenkrechte) gleichweit
entfernt sind.  Aus diesen Beweisen folgt zugleich die Charakterisierung des
jeweiligen Schnittpunkts als Mittelpunkt des In- bzw. Umkreises.

Auch f"ur die Seitenhalbierenden ist ein Beweis der Konkurrenz nicht besonders
schwierig, wenn man den Strahlensatz und seine Umkehrungen zu handhaben wei"s.
Der Schnittpunkt ist der Schwerpunkt, was sich besonders einfach aus einer
vektorgeometrischen Interpretation ergibt, auf die wir hier aber nicht weiter
eingehen wollen.

\begin{aufgabe}
  "Uberlege Dir die Details der jeweiligen Beweise.
\end{aufgabe}

\section{Das H"ohenfu"spunktdreieck}

Schwieriger ist es schon zu beweisen, dass sich die drei H"ohen eines Dreiecks
in einem Punkt schneiden.

\parbox[c]{8cm}{Ein direkter Beweis k�nnte etwa wie folgt gef�hrt werden:\\
  Die H"ohen aus $B$ und aus $C$ m"ogen sich im Punkt $H$ schneiden.  Die
  beiden H"ohenfu"spunkte $E$ und $F$ liegen dann auf zwei Thaleskreisen,
  einem "uber dem Radius $\ksegment{BC}$ und einem zweiten "uber dem Radius
  $\ksegment{AH}$.
  
  Zeige, dass die jeweils durch $a$ bzw. $b$ markierten Peripheriewinkel in
  nebenstehender Figur gleichgro"s sind und leite daraus ab, dass
  $\ksegment{AH}$ die dritte H"ohe des Dreiecks $\ktriangle{ABC}$ ist.
\begin{aufgabe}
  "Uberlege Dir die Details dieses Beweises.
\end{aufgabe}}\hfill \parbox[c]{7cm}{\Bild{width=4cm}{fb3}{Bild 1}}

Neben diesem direkten Beweis gibt es noch zwei weitere Beweise, die
Eigenschaften der H"ohen nutzen.
  
\parbox[c]{8cm}{ Auf den drei H"ohen des Dreiecks $\ktriangle{ABC}$ sei in den
  Eckpunkten jeweils die Senkrechte errichtet (Bild 2). Diese drei Geraden
  bilden ihrerseits ein Dreieck $\ktriangle{A''B''C''}$.
 \begin{aufgabe} Zeige (ohne die Konkurrenz der H"ohen zu verwenden), dass die
   H"ohen im alten Dreieck die Mittelsenkrechten im neuen Dreieck und somit
   konkurrent sind.
\end{aufgabe}}\hfill
\parbox[c]{7cm}{\Bild{width=5cm}{fb2}{Bild 2}} \vskip1cm

Ein "ahnlich interessantes Dreieck, das sogenannte {\em
  H"ohenfu"spunktdreieck}, erh"alt man, wenn man die H"ohenfu"spunkte $D,\,E$
und $F$ miteinander verbindet.

\parbox{8cm}{\begin{aufgabe} Zeige, dass die H"ohen die Winkelhalbierenden im
    H"ohenfu"spunktdreieck (und damit konkurrent) sind.\end{aufgabe}
    
    Zeige dazu, dass die in nebenstehender Figur mit $a$ markierten Winkel
    alle gleichgro"s sind. (Hinweis: Betrachte die Thaleskreise "uber
    $\ksegment{AB}$ und $\ksegment{AC}$.)}\hfill
\parbox{7cm}{\Bild{width=6cm}{fb1}{}}

Dieses H"ohenfu"spunktdreieck hat eine Reihe weiterer bemerkenswerter
Eigenschaften:
\begin{enumerate}
\item Die Dreiecke $\ktriangle{AEF}$, $\ktriangle{BDF}$ und $\ktriangle{CDE}$
  sind einander "ahnlich.
\item Die Lote aus $A$ auf $\kline{EF}$, aus $B$ auf $\kline{DF}$ und aus $C$
  auf $\kline{DE}$ sind konkurrent. (Sie schneiden sich im Umkreismittelpunkt
    $O$ des Dreiecks $\ktriangle{ABC}$.)
  \item Bezeichnen wir mit $\alpha,\,\beta,\,\gamma$ die Gr"o"sen der
    Innenwinkel im Dreieck $\ktriangle{ABC}$, so gilt
\[\mangle{OAH}=\abs{\beta-\gamma},\ \mangle{OBH}=\abs{\alpha-\gamma},\
\mangle{OCH}=\abs{\alpha-\beta}.\]
\end{enumerate}

\begin{aufgabe}
  Beweise diese Beziehungen.
\end{aufgabe}

\section{Das Mittendreieck und die Eulersche Gerade}

\noindent\parbox{10cm}{Ein weiteres interessantes Dreieck, das in
  enger Beziehung zum Dreieck $\ktriangle{ABC}$ steht, erh"alt man, wenn man
  die drei Seitenmitten $A',B',C'$ miteinander verbindet. Dieses sogenannte
  {\em Mittendreieck} $\ktriangle{A'B'C'}$ ist dem urspr"unglichen Dreieck
  nicht nur "ahnlich, sondern kann aus diesem sogar durch eine Streckung um
  den Faktor $-\frac12$ gewonnen werden, wie man sofort erkennt, wenn man die
  einander entsprechenden Eckpunkte verbindet. In der Tat, diese
  Verbindungsgeraden sind die Seitenhalbierenden, die durch einen gemeinsamen
  Punkt gehen und von diesem im Verh"altnis $2:1$ geteilt werden, woraus die
  Behauptung nach der Umkehrung des Strahlensatzes folgt. }\hfill
\parbox{5cm}{\Bild{width=5cm, bblly=4cm, bburx=20cm, bbury=20cm}{fb4}{Das
    Mittendreieck}}

Zwei Figuren hei"sen {\em in "Ahnlichkeitslage}, wenn man zwischen den Punkten
der beiden Figuren eine solche Zuordnung treffen kann, dass einander
entsprechende Strecken parallel sind.  Die beiden Dreiecke $\ktriangle{ABC}$
und $\ktriangle{A'B'C'}$ befinden sich also in "Ahnlichkeitslage.
Interessanterweise folgt aus der Parallelit"at homologer Strecken bereits die
Existenz eines Streckungszentrums wie in unserem Fall (das man "uberdies durch
Verbinden einander entsprechender Punkte finden kann).

\begin{satz}[Satz von Desargue] Befinden sich die Dreiecks $\ktriangle{ABC}$
  und $\ktriangle{RST}$ in "Ahnlichkeitslage, d.h. gilt $\kline{AB}\parallel
  \kline{RS},\ \kline{AC}\parallel \kline{RT}$ und $\kline{BC}\parallel
  \kline{ST}$, so gehen die drei Geraden $\kline{AR},\ \kline{BS}$ und
  $\kline{CT}$ durch einen gemeinsamen Punkt (und werden von diesem im selben
  Verh"altnis geteilt).
\end{satz}
\begin{aufgabe}
  Leite diesen Satz aus dem Strahlensatz her. (Hinweis: Die Geraden
  $\kline{AR}$ und $\kline{BS}$ m"ogen sich in $O$ schneiden. Zeige, dass dann
  $\kline{OC}$ durch $T$ geht.)
\end{aufgabe}
\vskip1cm

\noindent
\parbox{7cm}{\Bild{width=7cm, bblly=6cm, bburx=20cm,
bbury=20cm}{fb5}{}}\hfill
\parbox{7cm}{\Bild{width=7cm, bblly=6cm, bburx=20cm,
bbury=20cm}{fb6}{}}
\begin{center}
Die Eulersche Gerade
\end{center}

Wenden wir uns nach diesem kleinen Einschub wieder unserem Mittendreieck zu.
Um einzusehen, dass sich $\ktriangle{ABC}$ und $\ktriangle{A'B'C'}$ in
"Ahnlichkeitslage befinden, ben"otigen wir die oben genannte Eigenschaft der
Seitenhalbierenden nicht. Um etwa die Parallelit"at von $\kline{AB}$ und
$\kline{A'B'}$ zu zeigen, reicht eine einfache Argumentation in der
Strahlensatzfigur mit Zentrum $C$ usw.  Hieraus und aus dem Satz von Desargue
{\em folgt} nun, dass sich die Seitenhalbierenden in einem Punkt $S$ schneiden
und von diesem im Verh"altnis $2:1$ geteilt werden. Wir erhalten einen neuen
Beweis f"ur die Konkurrenz der Seitenhalbierenden eines Dreiecks.

Aber auch die Verbindungsstrecke anderer Paare homologer Punkte geht durch
dieses Streckungszentrum und wird von ihm im selben Verh"altnis geteilt.
Betrachten wir die Verbindungsstrecke der H"ohenschnittpunkte $H$ und $H'$ der
beiden Dreiecke. Da die H"ohen im Mittendreieck gerade die Mittelsenkrechten
im Ausgangsdreieck sind, f"allt $H'$ mit dem Umkreismittelpunkt $O$ des
Dreiecks $\ktriangle{ABC}$ zusammen. Wir erhalten folgenden

\begin{satz}
  Die Verbindungsstrecke $\ksegment{OH}$ von Umkreismittelpunkt $O$ und
  H"ohenschnittpunkt $H$ des Dreiecks $\ktriangle{ABC}$ geht durch dessen
  Schwerpunkt $S$ und wird von diesem im Verh"altnis $2:1$ geteilt.
    
  Die drei Punkte $H,\ S$ und $O$ liegen also auf einer gemeinsamen Geraden,
  die man als die {\bf Eulersche Gerade} bezeichnet.
\end{satz}

Den Umkreismittelpunkt $O'$ des Mittendreiecks finden wir auf dieselbe Weise:
Da die Verbindungsstrecke $\ksegment{OO'}$ ebenfalls durch das gemeinsame
Streckungszentrum $S$ verlaufen muss und von diesem im Verh"altnis $2:1$
geteilt wird, liegt $O'$ ebenfalls auf der Eulerschen Geraden und f"allt mit
dem Mittelpunkt $N$ der Strecke $\ksegment{OH}$ zusammen.  Die Details
erschlie"sen sich leicht aus nebenstehender Figur.

\section{Der Feuerbachsche Kreis}

Das Mittendreieck entstand aus $\ktriangle{ABC}$ durch Streckung mit dem
Zentrum $S$. Strecken wir es um den Faktor $\frac12$ mit dem Zentrum $H$, so
gehen die Eckpunkte $A,B,C$ in die {\em Mitten der oberen H"ohenabschnitte}
$U,V,W$ "uber. Diese Punkte sind bemerkenswert, weil sich dort die Zentren der
Thaleskreise befinden, die sich z.B.\ f"ur die L"osung der Aufgabe 2 als
n"utzlich erwiesen.

Offensichtlich befinden sich das Dreieck $\ktriangle{UVW}$ und das
Mittendreieck $\ktriangle{A'B'C'}$ in "Ahnlichkeitslage mit dem Faktor
$\frac12\,:\,-\frac12=-1$, d.h.\ beide Dreiecke gehen durch eine
Punktspiegelung an ihrem gemeinsamen Streckungszentrum, dessen Lage wir
sogleich bestimmen wollen, ineinander "uber. Da wir den Streckungsfaktor schon
kennen, finden wir das Zentrum als Mittelpunkt jeder Strecke, die Original-
und Bildpunkt miteinander verbinden.  Offensichtlich ist $H$ der
H"ohenschnittpunkt im Dreieck $\ktriangle{UVW}$. Da wir schon wissen, dass $O$
der H"ohenschnittpunkt im Dreieck $\ktriangle{A'B'C'}$ ist, erhalten wir also
als Streckungszentrum in diesem Fall den uns bereits wohlbekannten Punkt $N$.
$\ktriangle{UVW}$ entsteht also durch Drehung von $\ktriangle{A'B'C'}$ um $N$
um einen Winkel von $180\grad$.  \medskip

Wir erinnern uns, dass $N$ zugleich der Umkreismittelpunkt von
$\ktriangle{A'B'C'}$ ist. Der Umkreis dieses Dreiecks ist damit zugleich
Umkreis des Dreiecks $\ktriangle{UVW}$. Ein Kreis, der durch 6 derart
bemerkenswerte Punkte in einem Dreieck verl"auft, verdient einen eigenen
Namen. Man nennt diesen Kreis den {\em Feuerbachschen Kreis}.  Auf
nebenstehender Figur erkennen wir, dass er (in dieser Figur) drei weitere
bemerkenswerte Punkte passiert, die H"ohenfu"spunkte. Das ist kein Zufall: Da
$A'U$ ein Durchmesser des Feuerbachkreises ist, muss nach dem Thalessatz der
H"ohenfu"spunkt $D$ auf dessen Peripherie liegen.  "Ahnliches gilt f"ur die
beiden anderen H"ohenfu"spunkte $E$ und $F$.

\Bild{ width=12cm, bblly=4cm, bburx=20cm, bbury=20cm}{fb7}{Der
Feuerbachsche Kreis}

\begin{satz}
  Der Feuerbachkreis ist der gemeinsame Umkreis des Mittendreiecks und des
  H"ohenfu"spunktdreiecks. Er verl"auft weiterhin durch die drei Mittelpunkte
  der oberen H"ohenabschnitte.
\end{satz}

Dass der Feuerbachkreis der gemeinsame Umkreis des Mittendreiecks und des
H"ohenfu"spunktdreiecks ist, wusste schon Euler (1765), weshalb man diesen
Kreis oft auch den Eulerschen Kreis nennt.  Wahrscheinlich war diese
Eigenschaft aber bereits vorher bekannt.  Einen vollst"andigen Beweis aller
Aussagen obigen Satzes findet man erstmals bei Poncelet (1821).

Feuerbach fand eine weitere wirklich bemerkenswerte Eigenschaft dieses
Kreises, weshalb man heute seinen Namen mit diesem Kreis verbindet: Der
Feuerbachkreis ber"uhrt den Inkreis und die drei Ankreise des Dreiecks
$\ktriangle{ABC}$. Ein Beweis kann an dieser Stelle nicht gef"uhrt werden,
weil dazu weitere (elementargeometrische) Hilfsmittel erforderlich sind.

Da der Feuerbachkreis des Dreiecks $\ktriangle{ABC}$ aber zugleich der
Feuerbachkreis der Dreiecke $\ktriangle{ABH},\ \ktriangle{ACH}$ und
$\ktriangle{BCH}$ ist (warum?), ber"uhrt er auch deren In- und Ankreise, also
insgesamt 16 spezielle Kreise. Ein wirklich bemerkenswertes geometrisches
Objekt~!

\begin{attribution}
graebe (2004-09-09): Contributed to KoSemNet
\end{attribution}


\end{document}

