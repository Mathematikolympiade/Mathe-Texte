\documentclass[11pt]{article}
\usepackage{ngerman,schueler,url}
\usepackage{kosemnet,ko-math}

\title{Polynome\kosemnetlicensemark} 
\author{Axel Sch�ler, Mathematisches Institut, Univ. Leipzig\\[8pt]
\url{mailto:schueler@mathematik.uni-leipzig.de}}

\date{M�rz 2000} 

\begin{document}
\maketitle

{\em Polynome}, auch {\em ganzrationale Funktionen} genannt, sind die
einfachsten Funktionen "uberhaupt. Sie treten sowohl in der Analysis, eben als
Funktionen, als auch als auch in der Algebra, als Elemente von
{\glqq}abstrakten Ringen{\grqq}, auf. In Olympiadeaufgaben tauchen sie
entweder direkt bei der Restberechnung in der Polynomdivision, bei der
Nullstellenbestimmung oder als symmetrisches Gleichungssystem auf.  Man
benutzt Polynome in Verbindung mit dem {\sc Vieta}schen Wurzelsatz auch zur
Berechnung von Summen von Binomialkoeffizienten oder von trigonometrischen
Summen.

\subsubsection*{Division mit Rest, Nullstellen und der Identit"atssatz}
\begin{definition} Eine Funktion $f\colon\R\to\R$ der Gestalt
\[f(x)=a_nx^n+a_{n-1}x^{n-1}+\cdots +a_1x+a_0,\quad a_n\ne 0\]
hei"st {\em Polynom vom Grad} $n$ (Schreibweise: $\deg(f)=n$). Die festen
Zahlen $a_0,\dots, a_n$ hei"sen {\em Koeffizienten} von $f$ und k"onnen
komplex, reell, rational oder ganzzahlig sein.
\end{definition}

\begin{beispiel} Das (lineare) Polynom $f(x)=mx+n$, $m\ne0$, hat den Grad
  $1$. F"ur $f(x)=(x+1)^n-x^n-1$, $n\ge2$, gilt $\deg(f)=n-1$.
  Vereinbarungsgem"a"s hat das Nullpolynom $f\equiv 0$ den Grad $-1$. Die
  Polynome vom Grad $0$ sind genau die konstanten Polynome $f(x)=c$, $c\ne0$.
\end{beispiel}

Der folgende einfache Satz geh"ort zu den wichtigsten Aussagen "uber Polynome.
Aus ihm lassen sich viele interessante S"atze ableiten.

%%%%%%%%%%%%%%%%%%%%%%%%%%%%%%%%%%%%%%%%%%%%%%%%%%%%%%%%%%%%%%%%%%%%%%
\begin{satz}[Division mit Rest] Es seien $f(x)$ und $g(x)$ Polynome. Dann
  existieren Polynome $q(x)$ und $r(x)$ mit
\begin{equation}\label{div}
f(x)=g(x)q(x)+r(x),\quad \deg(r)<\deg(g).
\end{equation}
Die Polynome $q(x)$ und $r(x)$ hei"sen {\em Quotient} bzw.\ {\em Rest} bei der
Division von $f(x)$ durch $g(x)$. Ist $r(x)\equiv 0$, dann sagen wir {\em
  $g(x)$ teilt $f(x)$} und schreiben $g(x){\teilt}f(x)$.
\end{satz}
%%%%%%%%%%%%%%%%%%%%%%%%%%%%%%%%%%%%%%%%%%%%%%%%%%%%%%%%%%%%%%

\begin{beispiel}
  Die aus der Schule bekannte Methode der Division mit Rest f"ur ganze Zahlen
  l"asst sich auf Polynome "ubertragen. Man erh"alt etwa mit $f(x)=x^7-1$ und
  $g(x)=x^3+x+1$
$$
x^7-1=(x^3+x+1)(x^4-x^2-x+1)+2x^2-2.
$$
Wenn $\deg(f)\ge \deg(g)$ gilt, dann ist  offensichtlich
$\deg(f)=\deg(g)+\deg(q)$.  
\end{beispiel}

\begin{aufgabe}
  Gesucht sind alle reellen Zahlen $a$ und $b$ mit $x^2-2ax+2\teilt
  x^4+3x^2+ax+b$.
\end{aufgabe}
% L. a=0, b=1 und $a=\pm\frac{1}{2\sqrt{2}}$, $b=3$


\begin{aufgabe}
\label{division} Es seien $a,\,b,\,r$ und $s$ reelle Zahlen mit
$a\ne b$. Welchen Rest l"asst das Polynom $f(x)$ bei der Division durch
$(x-a)(x-b)$, wenn bekannt ist, dass $f(a)=r$ und $f(b)=s$ gilt?
\end{aufgabe}

{\em L"osung.} Da durch ein quadratisches Polynom dividiert wird, ist der Rest
h"ochstens linear, etwa $r(x)=Ax+B$. Wir wollen $A$ und $B$ ermitteln.  Nach
(\ref{div}) gilt also
$$
f(x)=(x-a)(x-b)q(x)+Ax+B
$$
mit einem gewissen Polynom $q(x)$ als Quotient.  Setzt man in diese
Gleichung nacheinander $x=a$ und $x=b$ ein, so erh"alt man das lineare
Gleichungssystem in den Variablen $A$ und $B$:
$$r=Aa+B,\qquad s=Ab+B.$$ Es besitzt die eindeutig bestimmte L"osung
$$A=\frac{s-r}{b-a},\qquad B=\frac{rb-sa}{b-a}.$$

Nun ein konkretes Beispiel: 
\begin{aufgabe}
  Welchen Rest l"asst $x^{100}-2x^{51}+1$ bei der Division durch $x^2-1$?
\end{aufgabe}

{\em L"osung.} Wegen $x^2-1=(x-1)(x+1)$ hat man $a=1$ und $b=-1$. Somit ist
$r=f(1)=1-2+1=0$ und $s=f(-1)=1+2+1=4$. Damit ergibt sich nach den obigen
Formeln $A=-2$ und $B=2$. Der Rest bei der Division durch $x^2-1$ ist also
gleich $-2x+2$.
\medskip

Die Zahl $\alpha$ hei"st {\em Nullstelle} des Polynoms $f$, wenn $f(\alpha)=0$
gilt.

Es sei $f$ ein Polynom vom Grad $n$ und $a\in\R$. Die Division mit Rest von
$f$ durch $x-a$ f"uhrt auf
\begin{align}\label{lin1}
f(x)=(x-a)q(x)+r(x),
\end{align}
wobei $r(x)$ ein konstantes Polynom ist, da $\deg(r)<\deg(x-a)=1$, also
$\deg(r)\le 0$.  Setzt man $x=a$ in (\ref{lin1}) ein, so erh"alt man
$f(a)=r(a)$ also $r(x)\equiv f(a)$ und somit
\begin{equation}\label{lin2}
f(x)=(x-a)q(x)+f(a).
\end{equation}

\begin{aufgabe}
  Es sei $f(x)$ ein Polynom mit ganzzahligen Koeffizienten und $f(0)$ und
  $f(1)$ seien beide ungerade. Dann hat $f(x)$ keine ganzzahlige Nullstelle.
\end{aufgabe}

Nach (\ref{lin2}) gilt
\begin{align*}
  f(x)=x q(x)+f(0)\quad\text{ und }\quad f(x)=(x-1)p(x)+f(1).
\end{align*}
Dabei sind $q(x)$ und $p(x)$ wieder Polynome mit ganzzahligen Koeffizienten,
da die Divisoren $x$ bzw.{} $x-1$ Polynome mit h"ochstem Koeffizienten gleich
Eins sind.  Ist nun $f(a)=0$ f"ur ein $a\in\Z$, so liefern die obigen
Gleichungen modulo $2$:
$$
0\equiv a q(a) +1\quad\text{ und }\quad 0\equiv (a-1)p(a) +1.
$$
Da jedoch entweder $a$ oder $a-1$ gerade ist, ergibt sich hier ein
Widerspruch. Das Polynom $f(x)$ hat also keine ganzzahligen Nullstellen.


%%%%%%%%%%%%%%%%%%%%%%%%%%%%%%%%%%%%%%%%%%%%%%%%%%%%%%%%%%%%%%%%%%%%%%%
\begin{folgerung}\label{ab} Es sei $f(x)$ ein Polynom, das nicht identisch Null
  ist.  Die Zahl $a$ ist genau dann Nullstelle von $f(x)$, wenn $f(x)$ ohne
  Rest durch $x-a$ teilbar ist, d.\,h.{}, wenn es ein Polynom $q(x)$ gibt mit
\begin{equation}\label{abspalten}
f(x)=(x-a)q(x).
\end{equation}
Dabei gilt $\deg(q)=\deg(f)-1$.
\end{folgerung}

%%%%%%%%%%%%%%%%%%%%%%%%%%%%%%%%%%%%%%%%%%%%%%%%%%%%%%%%%%%%%%%
\begin{aufgabe}
  (a) F"ur welche nat"urlichen Zahlen $n\in\N$ ist $x^2+x+1$ ein Teiler von
  $f_n(x):=x^{2n}+x^n+1$?
  
  (b) F"ur $n\in\N$ mit $3\not\teilt (n+1)$ gilt $37\teilt
  1\underbrace{0\dots0}_{n}1\underbrace{0\dots 0}_n 1$.
\end{aufgabe}

{\em L"osung.} (a) Es gilt die folgende Zerlegung in Linearfaktoren
$x^2+x+1=(x-\omega)(x-\omega^2)$, wobei $\omega=\half(-1+\ii \sqrt{3})$
($\ii^2=-1$ ist die imagin"are Einheit). Nach Folgerung\,\ref{ab} ist $f_n(x)$
genau dann durch $x-\omega$ und $x-\omega^2$ teilbar, wenn $f_n(\omega)=0$ und
$f_n(\omega^2)=0$. Es sei \mbox{$n=3k\pm1$}. Dann gilt wegen $\omega^3=1$
zun"achst
$f_n(\omega)=\omega^{6k\pm2}+\omega^{3k\pm1}+1=\omega^{\pm2}+\omega^{\pm1}+1=0$
und analog $f_n(\omega^2)=0$. Jedoch f"ur $n=3k$ ist
$f_n(\omega)=f_n(\omega^2)=3$. Somit teilt $x^2+x+1$ genau dann $f_n(x)$, wenn
$n$ nicht durch $3$ teilbar ist.

(b) Es gilt $1\underbrace{0\dots0}_{n}1\underbrace{0\dots 0}_n
1=10^{2n+2}+10^{n+1}+1$. Setzt man in (a) $x=10$ ein und beachtet
$10^2+10+1=3\cdot 37$, so ergibt sich die Behauptung.
\medskip

Man sagt {\em $f$ hat bei $x=a$ eine $m$-fache Nullstelle}, falls es ein
Polynom $q(x)$ mit 
$$f(x)=(x-a)^mq(x),\quad \text{und}\quad q(a)\ne0$$
 gibt. 

\begin{beispiel} Das Polynom $f(x)=x^2(x-1)^2$ hat eine doppelte Nullstelle bei
  $x=0$   und eine weitere doppelte Nullstelle bei $x=1$.
\end{beispiel}

\ul{Bemerkung} (zum Beispiel~\ref{division}): Treten im Divisor mehrfache
Nullstellen auf, so kommt man mit dem obigen Ansatz nicht weiter. Ist etwa $a$
eine $m$-fache Nullstelle von $g(x)$, so muss man die Werte
$f(a),\,f^\prime(a),\dots,f^{(m-1)}(a)$ kennen, um den Rest von $f(x)$ bei der
Division durch $g(x)$ zu ermitteln. Dabei bezeichnet $f^{(k)}(a)$ die {\em
  $k$-te Ableitung} von $f(x)$ an der Stelle $a$. Die erste Ableitung von
$f(x)=a_nx^n+\cdots a_1x+a_0$ ist definiert als
$f^\prime(x)=na_nx^{n-1}+(n-1)a_{n-1}x^{n-2}+\cdots +a_1$. Der neue Ansatz
ergibt sich aus der {\em Taylorentwicklung} von $f(x)$ in $a$:
\begin{align*}\label{taylor}
f(x)=f(a)+(x-a)f^\prime(a)+\frac{f^{\prime\prime}(a)}{2!}(x-a)^2 +\dots+
\frac{f^{(m-1)}(a)}{(m-1)!} (x-a)^{m-1}+ q(x)(x-a)^m.
\end{align*}
Wir bestimmen etwa den Rest von $f(x)=x^{100}-2x^{51}+1$ bei der Division
durch $g(x)=(x-1)^2$. Der Ansatz lautet nun mit $a=1$ als doppelter Nullstelle
von $g(x)$:
$$
f(x)=f(1)+(x-1)f'(1)+(x-1)^2q(x).
$$
Die ersten beiden Summanden auf der rechten Seite liefern den Rest bei der
Division durch $(x-1)^2$.  Wir bestimmen den Rest von $f(x)$ bei der Division
durch $(x-a)^2(x-b)$. Der Ansatz liefert ein quadratisches Polynom $Ax^2+Bx+C$
als Rest:
$$
f(x)=(x-a)^2(x-b)q(x)+Ax^2+Bx+C.
$$
Setzt man hier $x=a$ und $x=b$ ein, so erh"alt man 2 lineare Gleichungen
f"ur die 3 Variablen $A$, $B$ und $C$. Die fehlende dritte Gleichung
verschafft man sich durch Differentiation der obigen Gleichung und erneutes
Einsetzen von $x=a$:
$$ f'(a)=2A a+B a.
$$
Kennt man also $f(a),\,f'(a)$ und $f(b)$, so kann man den gesuchten Rest
ermitteln.

%%%%%%%%%%%%%%%%%%%%%%%%%%%%%%%%%%%%%%%
\begin{folgerung}\label{maximal} Es sei $f(x)$ ein Polynom vom  Grad $n$,
  $n\in\N$. Dann besitzt $f(x)$ h"ochstens $n$ Nullstellen gez"ahlt in ihrer
  Vielfachheit.
\end{folgerung}
%%%%%%%%%%%%%%%%%%%%%%%%%%%%%%%%%%%%%%%%
\begin{beweis} Wir benutzen vollst"andige Induktion "uber den Grad des
  Polynoms. F"ur Polynome vom Grad Null ist die Aussage sicher richtig, denn
  die konstanten, von Null verschiedenen Polynome haben keine Nullstellen. Nun
  gelte der Satz bereits f"ur alle Polynome vom Grad $n-1$. Es sei $x=a$ eine
  Nullstelle von $f(x)$. Dann gibt es nach Folgerung\,\ref{ab} ein Polynom
  $q(x)$ mit (\ref{abspalten}). Ferner gilt $\deg(q)=n-1$. Also hat nach
  Induktionsvoraussetzung $q(x)$ h"ochstens $n-1$ Nullstellen, gez"ahlt in
  ihrer Vielfachheit. Nun sind aber alle Nullstellen von $f(x)$ entweder
  Nullstellen von $q(x)$ oder vom $x-a$ oder von beiden. War $x=a$ bereits
  Nullstelle von $q(x)$, so erh"oht sich deren Vielfachheit (beim "Ubergang zu
  $f(x)$) um genau Eins. War $x=a$ keine Nullstelle von $q(x)$, so hat $f(x)$
  genau eine Nullstelle mehr als $q(x)$, also h"ochstens $n$ Nullstellen.
\end{beweis}

Aus Folgerung\,\ref{ab} ergibt sich durch schrittweises Abspalten von
Linearfaktoren die

\begin{folgerung}\label{zerlegung}
  Es sei $f(x)$ ein Polynom $n^{\mathrm{ten}}$ Grades mit den Nullstellen
  $a_1,\dots, a_n$, gez"ahlt in ihrer Vielfachheit. Dann existiert eine
  Konstante $c$, $c\ne0$, so dass
\begin{align*}
f(x)=c(x-a_1)\cdots (x-a_n).
\end{align*}
\end{folgerung}

\begin{folgerung}[Identit"atssatz]\label{ident} 
  {\em (i)} Es sei $f(x)$ ein Polynom mit $\deg(f)\le n$. Besitzt $f(x)$ mehr
  als $n$ paarweise verschiedene Nullstellen, so gilt $f(x)\equiv0$.

  {\em (ii)} Stimmen zwei Polynome $f(x)$ und $g(x)$, beide vom Grade kleiner
  oder gleich $n$, an $n+1$ verschiedenen Stellen "uberein, so stimmen sie an
  allen Stellen "uberein, d.\,h., sie sind identisch $f(x)\equiv g(x)$.
\end{folgerung}

\begin{beweis}
  (i) Das Polynom kann nicht den Grad $k$, $k\in\{0,\dots,n\}$ besitzen, da es
  sonst nach Folgerung\,\ref{maximal} h"ochstens $k$ verschiedene Nullstellen
  bes"a"se. Da $\deg(f)\le n$ bleibt nur noch $f(x)\equiv 0$.

  (ii) Wir setzen $h(x):=f(x)-g(x)$. Dann ist $\deg(h)\le n$, jedoch hat
  $h(x)$ mindestens $n+1$ verschiedene Nullstellen. Nach (i) ist dann
  $h(x)\equiv0$, also $f(x)\equiv g(x)$.
\end{beweis}

\begin{beispiel} Es seien $a$, $b$ und $c$ paarweise verschiedene reelle
  Zahlen. Die quadratische Gleichung  
\begin{equation}\label{bsp}
\frac{(x-a)(x-b)}{(c-a)(c-b)}+\frac{(x-b)(x-c)}{(a-b)(a-c)}+\frac{(x-c)(x-a)}{(b-c)(b-a)}=1
\end{equation}
hat offensichtlich die L"osungen $x_1=a$, $x_2=b$ und $x_3=c$. Was folgt
daraus?

Antwort: Auf der linken und rechten Seite von (\ref{bsp}) stehen Polynome
h"ochstens zweiten Grades, die aber in 3 verschiedenen Stellen
"ubereinstimmen. Nach Folgerung\,\ref{ident}\,(ii) stimmen sie dann f"ur alle
$x$ "uberein.
\end{beispiel}

Den Rahmen dieses Beitrages sprengen w"urde der Beweis des {\em
  Fundamentalsatzes der Algebra}, der streng genommen gar kein algebraischer
Satz ist, sondern in die Analysis geh"ort.

\begin{satz}[Fundamentalsatz der Algebra] 
  Jedes Polynom $n^{\mathrm{ten}}$ Grades, $n\ge1$, mit komplexen
  Koeffizienten besitzt mindestens eine komplexe Nullstelle.
\end{satz}

Zusammen mit der Folgerung\,\ref{maximal} ergibt sich damit

\begin{satz} 
  Jedes Polynom $n^{\mathrm{ten}}$ Grades, $n\ge1$, mit komplexen
  Koeffizienten besitzt genau $n$ komplexe Nullstellen.
\end{satz}

\subsubsection*{Der Vietasche Wurzelsatz}
Hat das Polynom
\begin{align}\label{v1}
f(x)=x^n+a_1x^{n-1}+a_2x^{n-2}+\cdots+a_{n-1}x+a_n
\end{align}
die Nullstellen $x_1,\dots,x_n$, gez"ahlt in ihrer Vielfachheit, dann liefert
Folgerung\,\ref{zerlegung}
\[f(x)=(x-x_1)(x-x_2)\cdots(x-x_n).\]
Multipliziert man dies aus und fasst die $2^n$ Summanden nach Potenzen von
$x^k$, $k=0,\dots,n$,  zusammen, so liefert der Koeffizientenvergleich mit
(\ref{v1})
\begin{align*}
a_1&=-(x_1+x_2+\dots+x_n),\\
a_2&=x_1x_2+x_1x_3+\cdots +x_{n-1}x_n,\\
&\,\vdots\\
a_n&=(-1)^{n}x_1x_2\cdots x_n.
\end{align*}
Die auf der rechten Seite auftretenden Polynome $\sigma_k$, $k=1,\dots,n$, in
den Ver"anderlichen $x_1,\dots, x_n$
\begin{align*}
\sigma_k(x_1,\dots,x_n):=\sum_{1\le i_1<\dots< i_k\le n}x_{i_1}\cdots x_{i_k}
\end{align*}
hei"sen {\em elementarsymmetrische Funktionen} der $x_1,\dots, x_n$. Wir haben
also gesehen

%%%%%%%%%%%%%%%%%%%%%%%%%%%%%%%%%%%%%%%%%%%%%%%%%%%%%%%%
\begin{satz}[{\sc Vieta}scher Wurzelsatz] 
  Sind $x_1,\dots, x_n$ die Nullstellen des Polynoms
  $f(x)=x^n+a_1x^{n-1}+\cdots +a_n$, so gilt f"ur $k=1,\dots,n$
\begin{align*}
a_k=(-1)^k\sigma_k(x_1,\dots,x_n).
\end{align*}
Dabei sind die $\sigma_k$ die elementarsymmetrischen Funktionen. 
\end{satz}
%%%%%%%%%%%%%%%%%%%%%%%%%%%%%%%%%%%%%%%%%%%%%%%%%%%%%%

\begin{definition}
Eine Funktion $f(x_1,\dots,x_n)$ hei"st {\em symmetrisch}, falls sie unter
allen Vertauschungen der Variablen $x_1,\dots, x_n$ invariant,
d.\,h. unver"andert, bleibt. Mit anderen Worten, es gilt f"ur alle
$k\in\{1,\dots,n-1\}$
\begin{align*}
f(x_1,\dots,x_n)=f(x_1,\dots,x_{k-1},x_{k+1},x_k,x_{k+2},\dots,x_n).
\end{align*}
\end{definition}

\begin{beispiel} Symmetrisch sind etwa die Funktionen $f(x,y)=\cos(x-y)$,
  $f(x,y,z)=1+xyz+x^2+y^2+z^2$, $g(x,y,z)=(x-y)^2(x-z)^2(y-z)^2$. Nicht
  symmetrisch dagegen sind $f(x,y)=\sin(x-y)$, $g(x,y,z)=xy+xz$ und
  $h(x,y,z)=(x-y)(x-z)(y-z)$, weil $\sin(y-x)=-\sin(x-y)$, zwar
  $g(x,y,z)=g(x,z,y)$ aber $g(x,y,z)\ne g(y,x,z)$ und weil
  $h(x,y,z)=-h(y,x,z)$.
\end{beispiel}

Eine weitere wichtige Klasse von symmetrischen Funktionen sind die {\em
Potenzsummen} $$p_k(x_1,\dots,x_n)=x_1^k+\cdots+x_n^k,\ k\in\N.$$ Der
Hauptsatz "uber die symmetrischen Funktionen besagt, dass sich jedes
symmetrische Polynom ausdr"ucken l"asst durch die elementarsymmetrischen
Funktionen.  Insbesondere lassen sich die Potenzsummen durch die
elementarsymmetrischen Funktionen ausdr"ucken:

%%%%%%%%%%%%%%%%%%%%%%%%%%%%%%%%%%%%%%%%%%%%%%%%%%%%
\begin{satz}[{\sc Newton}sche Relationen]  
  Zwischen den elementarsymmetrischen Funktionen\linebreak
  $\sigma_k(x_1,\dots,x_n)$ und den Potenzsummen $p_k(x_1,\dots,x_n)$ gelten die
  folgenden Relationen
\begin{alignat}{2}
  0&=p_k-\sigma_1p_{k-1}+-\cdots+(-1)^{k-1}\sigma_{k-1}p_1+(-1)^kk\sigma_k
  &\quad\text{f"ur} \quad k&\le  n,\label{n1}\\
  0&=p_k-\sigma_1p_{k-1}+-\cdots+(-1)^{n}\sigma_{n}p_{k-n}&\quad\text{f"ur} \quad
  k&> n.\label{n2}
\end{alignat}
\end{satz}
%%%%%%%%%%%%%%%%%%%%%%%%%%%%%%%%%%%%%%%%%%%%%%%%%%%%%%

\begin{beweis} Wir zeigen (\ref{n2}). Dazu sei $f(x)=(x-x_1)(x-x_2)\cdots
  (x-x_n)$. Nach dem {\sc Vieta}schen Wurzelsatz gilt dann
\[
f(x)=x^n-\sigma_1 x^{n-1}+-\cdots +(-1)^n\sigma_n.
\] 
Setzt man in die obige Gleichung $x=x_i$ ein und  multipliziert mit
$x_i^{k-n}$, so erh"alt man, da $x_i$ eine Nullstelle von $f(x)$ ist,
$$
0=x_i^k-\sigma_1x_i^{k-1}+\dots+(-1)^n\sigma_nx_i^{k-n}.
$$
Summiert man schlie"slich "uber $i=1,\dots,n$, so erh"alt man die
Behauptung.

Der Beweis von (\ref{n1}) ist etwa  schwieriger. Wir zeigen daher nur den
Spezialfall $k=2$. Offenbar ist $\sigma_1=p_1$ und daher
$\sigma_1p_1=(x_1+\cdots+x_n)^2=x_1^2+x_2^2+\cdots x_n^2+2(x_1x_2+x_1x_3+\cdots
x_{n-1}x_n)=p_2+2\sigma_2$. Dies ist die Behauptung.
\end{beweis}

\begin{aufgabe}
  Es seien $x_1\,x_2$ und $x_3$ die Nullstellen von $x^3+3x^2-7x+1$. Man
  berechne $x_1^2+x_2^2+x_3^2$.
\end{aufgabe}

{\em L"osung.}  Nach dem {\sc Vieta}schen Wurzelsatz ist $\sigma_1=-3$ und
$\sigma_2=-7$. Nach dem letzten Teil des obigen Beweises gilt
$p_2=\sigma_1^2-2\sigma_2=9+14=23$.

\begin{aufgabe}
  Es seien $x_1$ und $x_2$ die L"osungen der Gleichung $x^2-6x+1=0$. Man
  zeige, dass $x_1^n+x_2^n$ stets ganzzahlig und nicht durch 5 teilbar ist.
\end{aufgabe}

{\em L"osung.} Es ist $\sigma_1=x_1+x_2=6$, $\sigma_2=x_1x_2=1$ und nach obigem
$p_2=\sigma_1^2-2\sigma_2=36-2=34$. Nun gilt
\[
p_1p_n=(x_1^n+x_2^n)(x_1+x_2) =x_1^{n+1}+x_2^{n+1}+x_1x_2(x_1^{n-1}+x_2^{n-2})
=p_{n+1}+\sigma_2p_{n-1}. 
\]
Wir erhalten also die Rekursion $p_{n+1}=\sigma_1 p_n-\sigma_2 p_{n-1}$. Hieraus
folgt zun"achst die Ganzzahligkeit von $p_n$. Wegen $\sigma_1\equiv \sigma_2\equiv
1 \pmod{5}$ ergibt sich daraus $p_{n+1}\equiv p_n-p_{n-1} \pmod 5$. Als
periodische Folge der Reste  $(p_n \pmod{5})$ erhalten wir also
$1$, $-1$, $-2$, $ -1$, $1$, $2$, $1$, $-1, \dots$.

\begin{aufgabe}
  Man finde alle reellen L"osungen des Gleichungssystems $x+y+z=1$,
  $x^3+y^3+z^3+xyz=x^4+y^4+z^4+1$.
\end{aufgabe}

{\em L"osung.} Nach (\ref{n2}) mit $n=3$ und $k=4$ ist
\[p_4=\sigma_1 p_3-\sigma_2p_2+\sigma_3 p_1.\]
Beachtet man $\sigma_1=p_1=1$ und setzt dies in die zweite gegebene Gleichung
ein, so hat man $p_3+\sigma_3=p_4+1=p_3-\sigma_2p_2+\sigma_3+1$ bzw.{} $\sigma_2
p_2=1$. Andererseits war aber $1=p_1^2=p_2+2\sigma_2$, also
$p_2=1-2\sigma_2$. Setzt man dies wiederum oben ein, so hat man
$1=\sigma_2(1-2\sigma_2)$ bzw.{} $\sigma_2^2-\half \sigma_2+\half=0$, was keine reelle
L"osung $\sigma_2$ besitzt. Folglich gibt es auch keine reelle L"osung $(x,y,z)$
des gegebenen Gleichungssystems.

%\newpage
\subsection*{Anwendungen}

\begin{aufgabe}
  Dem Einheitskreis sei ein regul"ares $n$-Eck $P_1P_2\cdots P_n$
  einbeschrieben. Man zeige, dass das Produkt aller Streckenl�ngenquadrate
  $\msegment{P_iP_j}^2$, $1\le i<j\le n$, gleich $n^n$ ist!
\end{aufgabe}

{\em L"osung.} Wir legen das regul"are $n$-Eck so in die komplexe Ebene, dass
$P_i=\nu_{i-1}$, $i=1,\dots,n$, wobei die $\nu_i$ gerade die $n$-ten
Einheitswurzeln sind; $\nu_0=1$. Wegen $\overline{\nu_i}=\nu_{-i}$ gilt dann
f"ur $D=\prod_{1\le i<j\le n}(\nu_i-\nu_j)$
\begin{align}
|D|^2&=D\overline{D}=\prod_{i<j}(\nu_i-\nu_j)(\nu_{-i}-\nu_{-j})\notag\\
&=\prod_{i<j}\nu_i(1-\nu_{j-i})\nu_{-i}(1-\nu_{i-j})\notag\\
&=\prod_{i\ne j}(1-\nu_{i-j})=\prod_{k=1}^{n-1}\prod_{i=1}^n(1-\nu_k)\notag\\
&=\left(\prod_{k=1}^{n-1}(1-\nu_k)\right)^n\label{gl}
\end{align}
Nun gilt aber
$f(x)=x^{n-1}+x^{n-2}+\cdots+x+1=(x-\nu_1)(x-\nu_2)\cdots(x-\nu_{n-1})$ wegen
$x^n-1=(x-1)(x^{n-1}+\cdots+1)$. Setzt man hier $x=1$ ein und vergleicht dies
mit (\ref{gl}), so erh"alt man $|D|^2=f(1)^n=n^n$.

\subsubsection*{Aufgaben}
\begin{enumerate}
\item Welchen Rest l"asst $p(x)=x^7-1$ bei der Division durch $q(x)=x^3+x+1$?

%L $r=2x^2-2$
  
\item Es seien $x_1,x_2,x_3$ die Nullstellen des Polynoms $x^3+3x^2-7x+1$. Man
  ermittle $x_1^3+x_2^3+x_3^3$!

%L 23

\item  
  \begin{itemize}
  \item [(a)] F"ur welche nat"urlichen Zahlen $n\in\N$ ist $x^2+x+1$ ein
    Teiler von $x^{2n}+x^n+1$?
  \item [(b)] F"ur welche $n\in\N$ gilt
    $37\teilt 1\underbrace{0\dots0}_{n}1\underbrace{0\dots 0}_n 1$ ?
\end{itemize}

%L (a) $3\not| n$  (b) $ n\not\equiv 2\mod 3$

\item Es seien $P(x)$, $Q(x)$, $R(x)$ und $S(x)$ Polynome mit 
$$
P(x^5)+xQ(x^5)+x^2R(x^5)=(x^4+x^3+x^2+x+1)S(x).
$$
Zeigen Sie, dass $x-1$ ein Teiler von $P(x)$ ist!

\item Es sei $P(x)$ ein Polynom vom Grad $n$ mit $P(k)=\frac{k}{k+1}$,
  $k=0,\dots,n$. Man finde $P(n+1)$!

%L
%$$
%P(n+1)=\begin{cases}1&\text{f"ur ungerades }\quad n\\
%\frac{n}{n+2}& \text{f"ur gerades } \quad n
%\end{cases}
%$$
  
\item Es seien $a,\,b$ und $c$ paarweise verschiedene ganze Zahlen und $P$ ein
  Polynom mit ganzzahligen Koeffizienten.  Zeigen Sie, dass die Gleichungen
$$
P(a)=b,\quad P(b)=c,\quad P(c)=a
$$
nicht gleichzeitig erf"ullt werden k"onnen!

\item Man l"ose das folgende Gleichungssystem
$$ x^5+y^5=33,\quad x+y=3.$$

%L Wir setzen $\sigma_1=x+y$ und $sig_2=xy$. Dann gilt
%$p_5=\sigma_1^5-5\sigma_1^3\sigma_2+5\sigma_1\sigma_2^2$. Setzt man $p_5=33$
%und $\sigma_103$ ein, so hat man $\sigma_2^2-9\sigma_2+14=0$. mit den
%L"osungen $\sigma_2=2$ und $\sigma_2=7$. Also m"ussen wir die Systeme
%$x+y=3,\quad xy=2$ und $x+y=3\quad xy=7$ l"osen. Dies liefert $(2,1)$,
%$(1,2)$, $x,y=\half(3\pm\sqrt{19}\ii)$

\item Welche reellen L"osungen $x$ hat die Gleichung
$$
\sqrt[4]{97-x}+\sqrt[4]{x}=5?
$$

%L Wir setzen $\sqrt[4]{x}=y$ und $\sqrt[4]{97-x}=z$ und erhalten
%$y^4+z^4=x+97-x=97$ also 
%$$
%y+z=5,\quad y^4+z^4=97
%$$
%Dies f"uhrt auf $\sigma_2\in\{6,44\}$ mit den L"osungen $x_1=16$, $x_2=81$
%und 2 weiteren komplexen L"osungen.

\item Welcher Relation m"ussen $a,\,b$ und $c$ gen"ugen, damit das System
$$
x+y=a,\quad x^2+y^2=b, \quad x^3+y^3=c
$$
eine (komplexe) L"osung besitzt.

%L $a^3-3ab+2c=0$

\item Man l"ose das Gleichungssystem
$$
x+y+z=a,\quad x^2+y^2+z^2=b^2, \quad x^3+y^3+z^3=a^3!
$$

\item Man ermittle alle L"osungen des Systems $x+y+z=1$,
$x^3+y^3+z^3+xyz=x^4+y^4+z^4+1$!

%L keine L"osung.

\item Gegeben seien $2n$ paarweise verschiedene Zahlen $a_1,\dots,a_n$ und
  $b_1,\dots,b_n$. Ein $n\times n$-Feld wird nun mit den Zahlen $a_i+b_j$ in
  der $i^{\rm ten}$ Zeile und $j^{\rm ten}$ Spalte gef"ullt. 
  
  Beweise: Stimmen die Produkte der Zahlen in jeder Spalte "uberein, so
  stimmen auch die Zeilenprodukte "uberein!
  
\item Man dr"ucke $x^3+y^3+z^3-3xyz$ durch elementarsymmetrische Funktionen
  aus!
  
\item F"ur welche $a\in\R$ ist die Quadratsumme der Nullstellen von
  $x^2-(a-2)x-a-1$ minimal?
  
\item Sind $x_1$ und $x_2$ die Nullstellen des Polynoms $x^2-6x+1$, dann ist
  f"ur jede nat"urliche Zahl $n$ die Zahl $x_1^n+x_2^n$ ganzzahlig und nicht
  durch $5$ teilbar.
  
\item Das Polynom $x^{2n}-2x^{2n-1}+3x^{2n-2}-\cdots-2nx+2n+1$ hat keine
  reellen Wurzeln.
  
\item Ein Polynom $f(x,y)$ hei"st {\em antisymmetrisch}, falls
  $f(x,y)=-f(y,x)$. 
  
  Zeigen Sie, dass jedes antisymmetrische Polynom $f$ die Gestalt
  $f(x,y)=(x-y)g(x,y)$ mit einem symmetrischen Polynom $g$ hat!
  
\item Ist $f(x,y)$ symmetrisch und $(x-y)\teilt f(x,y)$, dann gilt
  $(x-y)^2\teilt f(x,y)$.

\item Man l"ose die Gleichung $z^8+4z^6-10z^4+4z^2+1=0$!

\item Man l"ose die Gleichung
$$
4z^{11}+4z^{10}-21z^9-21z^8+17z^7+17z^6+17z^5+17z^4-21z^3-21z^2+4z+4=0!
$$

\item Welchen Rest l"asst $x^{100}-2x^{51}+1$ bei der Division durch $x^2-1$?

\item Zeigen Sie:  $(x^4+x^3+x^2+x+1)\teilt (x^{44}+x^{33}+x^{22}+x^{11}+1)$.

\item Man l"ose die Gleichung $x^4+a^4-3ax^3+3a^3x=0$ bei gegebenem $a$.

\item Man bestimme $a$ und $b$ so, dass $(x-1)^2\teilt (ax^4+bx^3+1)$!

\item F"ur welche $n\in\N$ gilt 
  \begin{itemize}
  \item[(a)] $(x^2+x+1)\teilt \left((x+1)^n-x^n-1\right)$?
  \item[(b)] $(x^2+x+1)\teilt \left((x+1)^n+x^n+1\right)$?
\end{itemize}

\item Man beweise, dass $(x^k-a^k)\teilt (x^n-a^n)$ genau dann, wenn $k\teilt
  n$!
  
\item Man ermittle $a$, so dass $-1$ eine mehrfache Nullstelle von
  $x^5-ax^2-ax+1$ ist!
\end{enumerate}

\begin{attribution}
schueler (2004-09-09): Contributed to KoSemNet

graebe (2004-09-09): Prepared along the KoSemNet rules
\end{attribution}


\end{document}





