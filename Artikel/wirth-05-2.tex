\documentclass[11pt,a4paper]{article}
\usepackage{kosemnet,ko-math,ngerman,url}
\usepackage[utf8]{inputenc}  
\newtheorem{uebung}{\"{U}bung}
\author{Jens Wirth, Freiberg\\ \url{wirth@math.tu-freiberg.de}}
\title{Von Winkelfunktionen zur Dreiecksgeometrie\kosemnetlicensemark}
\date{}

\begin{document}
\maketitle

\section{Definition}
\parbox{6cm}{\input{fig1.pstex_t}} 
\hfill
\parbox{6cm}{Es sei $P$ ein Punkt auf dem Einheitskreis, $\angle{10P}=\phi$.
  Dann besitzt $P$ die Koordinaten $(\cos(\phi),\sin(\phi))$. Dies kann man
  nutzen, um durch periodische Fortsetzung auf ganz $\R$ die Funktionen
  $\sin(\phi)$ und $\cos(\phi)$ zu definieren.}

Mit dem Satz des Pythagoras gilt offensichtlich
$\msegment{0P}^2=1=\sin^2(\phi)+\cos^2(\phi$). Diese Formel wird als
trigonometrischer Pythagoras bezeichnet.

\begin{uebung}
  Man skizziere die Funktionen $\sin(\phi)$ und $\cos(\phi)$. Dabei verwende
  man zum Messen von Winkeln die Konvention, dass ein rechter Winkel das Ma"s
  $\frac\pi2$ hat.
\end{uebung}
\begin{uebung}\label{u2}
  Man {\glqq}beweise{\grqq}, dass
  \[\cos(\phi)=\sin(\phi+\frac\pi2),\qquad
  \cos(\phi)=\cos(-\phi)\qquad\text{und}\qquad\sin(\phi)=-\sin(-\phi)\]  
  gelten.
\end{uebung}

\section{Dreiecksberechnung mit Winkelfunktionen}
Wie der Name schon vermuten l"asst, eignen sich trigonometrische Funktionen in
besonderer Weise zur Berechnung von und in Dreiecken. In rechtwinkligen
Dreiecken $\ktriangle*{ABC}$ mit $\mangle{ABC}=\frac\pi2$ gilt (schon allein
wegen der "Ahnlichkeit zu einem Dreieck mit $\msegment{AC}=1$ und der
Definition der Winkelfunktionen)
\[c=b\,\cos(\alpha)\quad\text{und}\quad a=b\,\sin(\alpha), \]
wobei wir wie "ublich die Bezeichnungen $a=\msegment{BC}$ und
$\alpha=\mangle{CAB}$ usw.\ verwenden.  Uns interessieren aber Formeln die in
allen Dreiecken gelten.

\subsection{Erweiterter Sinussatz}
\parbox{6cm}{\input{fig3.pstex_t}}
\hfill 
\parbox{6cm}{Ein allgemeines Dreieck wird durch die H"ohe in rechtwinklige
  Teildreiecke zerlegt. Es gilt also insbesondere
\[ b\,\sin(\alpha)=h_c=a\,\sin(\beta)\]
und damit
\[ \frac{a}{\sin(\alpha)}=\frac{b}{\sin(\beta)}=\frac{c}{\sin(\gamma)}\,. \]
}

{\bf Frage:} Welchen Wert hat $\frac{a}{\sin(\alpha)}$ am allgemeinen Dreieck?
Wir suchen eine geometrische Interpretation.

\parbox{6cm}{\input{fig4.pstex_t}}
\hfill 
\parbox{6cm}{ Sei o.B.d.A.\ $\alpha<\frac\pi2$. Dann k"onnen wir nach dem
  Peripheriewinkelsatz $A$ auf dem Umkreis des Dreiecks $\ktriangle*{ABC}$
  verschieben, ohne $\alpha$ (und $a$) zu "andern. W"ahlen wir $A'$ so, dass
  $\mangle{BCA}=\frac{\pi}{2}$ ein rechter Winkel ist. Dann gilt mit der
  Umkehrung vom Satz des Thales f"ur den Umkreismittelpunkt
  $M\in\ksegment{BA'}$ und somit
\[\frac{a}{\sin(\alpha)}=\msegment{A'B}=2\,R\,,\] 
wobei $R$ der Umkreisradius des Dreiecks ist.}

Es gilt also der erweiterte Sinussatz
\begin{center}
  \framebox{$ \dfrac{a}{\sin(\alpha)} =\dfrac{b}{\sin(\beta)}
    =\dfrac{c}{\sin(\gamma)} =2\,R.$}
\end{center}

\subsection{Additionstheoreme I}
\label{sec.3.add1}
Die Innenwinkel im Dreieck erf"ullen $\alpha+\beta+\gamma=\pi$. Damit ergeben
sich auf elementare Weise „Additionstheoreme“ f"ur Winkelfunktionen
\begin{align*}
  \sin(\alpha+\beta)=\sin(\pi-\alpha-\beta)=\sin(\gamma)
\intertext{und entsprechend}
\cos(\alpha+\beta)=-\cos(\gamma)
\end{align*}
unter der Nebenbedingung $\alpha+\beta<\pi$. Um dies sch"oner zu gestalten,
wenden wir den erweiterten Sinussatz an. Es gilt
\[  2\,R\,\sin(\gamma)=c =\msegment{AH_c}+\msegment{BH_c}
=b\,\cos(\alpha)+a\,\cos(\beta)
=2\,R\,(\sin(\beta)\,\cos(\alpha)+\sin(\alpha)\,\cos(\beta))
\] 
wobei $H_c$ der entsprechende H"ohenfu"spunkt ist. Wir haben also
\begin{center}
  \framebox{$\sin(\alpha+\beta)
    =\sin(\alpha)\,\cos(\beta)+\sin(\beta)\,\cos(\alpha)$}
\end{center}
bewiesen.

Insbesondere ergibt sich die Doppelwinkelformel
\begin{center}
  \framebox{$\sin(2\,\alpha)=2\,\sin(\alpha)\,\cos(\alpha)$.}
\end{center}

Um entsprechende Beziehungen f"ur den Cosinus zu bekommen, m"ussen wir
entweder verstehen, warum das Additionstheorem f"ur \emph{alle}
$\alpha,\beta\in\R$ gilt, oder eine bessere geometrische Interpretation f"ur
den Cosinus finden. Wir werden letzteres tun. An der Stelle soll nur vorab auf
die Doppelwinkelformel f"ur den Cosinus hingewiesen werden. Es gilt
\[ \framebox{$\cos(2\,\alpha) =\cos^2(\alpha)-\sin^2(\alpha)
  =2\cos^2(\alpha)-1$.} \] 
Ein Beweis erfolgt in Abschnitt \ref{sec.3.5}.

\subsection{Fl"acheninhalt}
\label{sec.3.2}
Auch der Fl"acheninhalt ist eine Invariante des Dreiecks. Wir wollen die
„unsymmetrischen“ Formel $A=\frac12\,a\,h_a$ umformen. Es gilt
$A=\frac12\,a\,h_a=\frac12\,a\,b\,\sin(\gamma)$ und wegen dem erweiterten
Sinussatz $a\,b=4\,R^2\,\sin(\alpha)\,\sin(\beta)$.  Somit ergibt sich
\begin{center}
  \framebox{$\displaystyle A =2\,R^2\sin(\alpha)\,\sin(\beta)\,\sin(\gamma)
    =\frac{a\,b\,c}{4\,R}.$}
\end{center}

\subsection{Zusammenhang zu Inkreisradius und Umfang}
\label{sec.3.3}
\parbox{6cm}{Der Inkreismittelpunkt ist der Schnittpunkt der
  Winkelhalbierenden.  Diese zerlegen wie in der Skizze das Dreieck in drei
  Teilfl"achen $\ktriangle*{ABW}$, $\ktriangle*{BCW}$ und $\ktriangle*{CAW}$,
  die H"ohen der Teildreiecke sind jeweils die Radien des Inkreises. Damit
  ergibt sich eine einfache Fl"achenformel } 
\hfill
\parbox{6cm}{ \input{fig5.pstex_t}}

\[\framebox{$A=p\,r$,}\]
in der $p=\frac12\,(a+b+c)$ der halbe Umfang des Dreiecks ist. 

Mit dem Sinussatz 
\[ a=2\,R\,\sin(\alpha), \quad b=2\,R\,\sin(\beta), \quad
c=2\,R\,\sin(\gamma) \] 
folgt
\[\framebox{$\displaystyle p=R\,(\sin(\alpha)+\sin(\beta)+\sin(\gamma)),$}\]
was sich mit den Doppelwinkelformeln 
\[ \sin(\alpha)=2\,\sin\br{\frac\alpha2}\,\cos\br{\frac\alpha2}\quad
\text{und} \quad\cos(\alpha)=2\,\cos^2\br{\frac\beta2}-1\] 
"uber die Zwischenschritte
\begin{align*}
 &=R\,(\sin(\alpha) +\sin(\beta)+\sin(\alpha+\beta))\\ 
 &=R\,(\sin(\alpha)+\sin(\beta) +\sin(\alpha)\,\cos(\beta)
  +\sin(\beta)\,\cos(\alpha))\\ 
 &=R\,(\sin(\alpha\,(1+\cos(\beta)+\sin(\beta\,(1+\cos(\alpha))\\
 &=4\,R\,\br{\sin\br{\frac\alpha2}\,\cos\br{\frac\alpha2}\,
    \cos^2\br{\frac\beta2} +\sin\br{\frac\beta2}\, \cos\br{\frac\beta2}\,
    \cos^2\br{\frac\alpha2}}\\  
 &=4\,R\,\cos\br{\frac\alpha2}\,\cos\br{\frac\beta2}\,
  \br{\sin\br{\frac\alpha2}\,\cos\br{\frac\beta2}
    +\sin\br{\frac\beta2}\,\cos\br{\frac\alpha2}}\\ 
 &=4\,R\,\cos\br{\frac\alpha2}\,\cos\br{\frac\beta2}\,
  \sin\br{\frac{\alpha+\beta}2}
  =4\,R\,\cos\br{\frac\alpha2}\,\cos\br{\frac\beta2}\,\cos\br{\frac\gamma2} 
\end{align*}
umformen l"asst in
\[\framebox{$\displaystyle  p=4\,R\,\cos\br{\frac\alpha2}\,
  \cos\br{\frac\beta2}\, \cos\br{\frac\gamma2}$} 
\] 
und mit $A=p\,r$ in
\[\framebox{$\displaystyle   r=4\,R\,\sin\br{\frac\alpha2}\,
  \sin\br{\frac\beta2}\,\sin\br{\frac\gamma2}.$} \] 

\subsection{Cosinussatz}
Etwas aus der Rolle f"allt der Cosinussatz, er ist unsymmetrisch, soll aber
trotzdem nicht unerw"ahnt bleiben.

\parbox{6cm}{\input{fig6.pstex_t}}\hfill\parbox{6cm}{
Es gilt mit dem Satz des Pythagoras
\begin{align*}
  a^2&=h_c^2+\msegment{H_cB}^2\\
  &=b^2-\msegment{AH_c}^2+\msegment{H_cB}^2\\
  &=b^2+(c-\msegment{AH_c})^2-\msegment{AH_c}^2\\
  &=b^2+c^2-2c\msegment{AH_c}
\end{align*}}

und damit der Cosinussatz
\[\framebox{$\displaystyle a^2=b^2+c^2-2\,b\,c\cos(\alpha)$.}\]

Der Cosinussatz ist nichts wirklich Neues. Er ergibt sich wie so vieles aus
dem Sinussatz, wie folgende "Ubung zeigt (zeigen soll).

\begin{uebung}
  Man folgere den Cosinussatz aus dem Sinussatz und dem trigonometrischen
  Pythagoras (als {\glqq}Definition{\grqq} der Cosinus-Funktion).
\end{uebung}

\subsection{H"ohen und H"ohenabschnitte}

Die H"ohen eines Dreiecks erf"ullen 
\[ h_c=b\,\sin(\alpha)=2\,R\,\sin(\alpha)\,\sin(\beta). \]
F"uhrt man die halbe H"ohensumme als neue Hilfsgr"o"se ein, so ergibt sich
damit
\[q =\frac12\,\br{h_a+h_b+h_c} =R\,\br{\sin(\alpha)\,\sin(\beta)
  +\sin(\beta)\,\sin(\gamma) +\sin(\gamma)\,\sin(\alpha)}. \] 

Mit den Formeln aus Abschnitt \ref{sec.3.2} und \ref{sec.3.3} erh"alt man
damit
\begin{align*}
   \frac{A}{2\,R^2}&=\sin(\alpha)\,\sin(\beta)\,\sin(\gamma),\\
   \frac{q}{R}&=\sin(\alpha)\,\sin(\beta)+\sin(\beta)\,\sin(\gamma)
   +\sin(\gamma)\,\sin(\alpha),\\ 
   \frac{p}{R}&=\sin(\alpha)+\sin(\beta)+\sin(\gamma).   
\end{align*}

"Ahnliche Beziehungen gelten auch f"ur die Cosini der Winkel.  Aufgabe 1 a)
macht deutlich, dass f"ur die H"ohenabschnitte die Beziehungen
\[ \msegment{AH}=2\,R\,\cos(\alpha),\quad
   \msegment{BH}=2\,R\,\cos(\beta),\quad
   \msegment{CH}=2\,R\,\cos(\gamma),\]
und 
\[ \msegment{HH_a}=2\,R\,\cos(\beta)\,\cos(\gamma),\quad
   \msegment{HH_b}=2\,R\,\cos(\gamma)\,\cos(\alpha),\quad
   \msegment{HH_c}=2\,R\,\cos(\alpha)\,\cos(\beta)
\]
gelten.

Insbesondere ist das Produkt der H"ohenabschnitte konstant,
\[
\framebox{$
\msegment{AH}\,\msegment{HH_a} =\msegment{BH}\,\msegment{HH_b}
=\msegment{CH}\,\msegment{HH_c}
=4\,R^2\,\cos(\alpha)\,\cos(\beta)\,\cos(\gamma).$} \] 

\subsection{Additionstheoreme II}
\label{sec.3.5}
Wir gehen wieder vor wie in Abschnitt \ref{sec.3.add1}, ersetzen nur den
erweiterten Sinussatz durch die Formeln aus dem vorigen Abschnitt. Es gilt
\begin{align*}
2\,R\,\cos(\gamma)&=\msegment{CH}=\msegment{CH_c}-\msegment{HH_c}\\
&=b\,\sin(\alpha)-2\,R\,\cos(\alpha)\,\cos(\beta)\\
&=2\,R\,\br{\sin(\alpha)\,\sin(\beta)-\cos(\alpha)\,\cos(\beta)}
\end{align*}
und damit 
\[\framebox{$\cos(\alpha+\beta)
  =\cos(\alpha)\,\cos(\beta)-\sin(\alpha)\,\sin(\beta)$,}\] 
vorausgesetzt, dass $\alpha+\beta<\pi$ ist. 

\begin{uebung}
  Man komplettiere den Beweis durch jeweils eine Skizze f"ur den Fall
  $\gamma<\frac\pi2$ und $\gamma>\frac\pi2$.
\end{uebung}

\section{Aufgaben}
\begin{aufgabe}
 Man zeige in einem Dreieck $\ktriangle*{ABC}$ (mit den "ublichen
 Bezeichnungen) die folgenden Beziehungen:
 \begin{enumerate}
 \item[a)] 
\[\msegment{AH} =2\,R\,\cos(\alpha),\qquad\qquad\msegment{HH_a}
=2\,R\,\cos(\beta)\,\cos(\gamma)\]
   f"ur den H"ohenschnittpunkt $H$ und den H"ohenfu"spunkt $H_a\in\kline{BC}$.
 \item[b)] Der Fu"spunkt der Winkelhalbierenden $\kline{AW_a}$ teilt die Seite
   $\ksegment{BC}$ im Verh"altnis $\sin(\gamma):\sin(\beta)$.
 \item[c)] 
\[ \frac{\sin(\alpha)+\sin(\beta)+\sin(\gamma)}
  {\sin(\alpha)\,\sin(\beta)\,\sin(\gamma)}=2\,\frac{R}{r} \] 
 \item[d)] ($\sqrt{\mbox{WURZEL}}$, $\iota31$)
   \[ R\,(\cos(\alpha)+\cos(\beta)+\cos(\gamma))=R+r\,. \]
 \end{enumerate}
\end{aufgabe}
\begin{aufgabe} (A411345,\cite{MO})\\
  Man beweise, dass ein Dreieck genau dann rechtwinklig ist, wenn f"ur seine
  Innenwinkel $\alpha$, $\beta$ und $\gamma$
  \[ \frac{\sin^2(\alpha)+\sin^2(\beta)+\sin^2(\gamma)}
     {\cos^2(\alpha)+\cos^2(\beta)+\cos^2(\gamma)}=2
  \]
  gilt.
\end{aufgabe}
\begin{aufgabe}
  In einem Dreieck $\ktriangle*{ABC}$ gelten stets die folgenden drei
  Ungleichungen
  \begin{enumerate}
  \item[a)] 
\[\sin\br{\frac\alpha2}\cdot \sin\br{\frac\beta2}\cdot \sin\br{\frac\gamma2}
\leq\frac18\] 
  \item[b)] 
\[\frac12 \leq\cos\br{\frac\alpha2}\cdot \cos\br{\frac\beta2}\cdot
\cos\br{\frac\gamma2} \leq\frac34\,\sqrt3\]
  \item[c)] 
\[0<\sin(\alpha)+\sin(\beta)+\sin(\gamma)\leq\frac32\,\sqrt{3}\]
  \end{enumerate}
\end{aufgabe}
\begin{aufgabe} (A171335, \cite{Lesebogen} A85)\\
  Man beweise folgenden Satz:\\
  Sind $u$ den Umfang, $R$ der Umkreis- und $r$ der Inkreisradius des Dreiecks
  $\ktriangle*{ABC}$, so gilt
  \[ R>\frac13\,\sqrt{3}\,\sqrt{u\,r}.\]
  Ist das Dreieck insbesondere rechtwinklig, so gilt sogar
  \[ R\geq\frac12\,\sqrt2\,\sqrt{u\,r}. \]
\end{aufgabe}
\begin{aufgabe}
  Die durch die Fu"spunkte der Dreieckstransversalen gebildeten Dreiecke
  werden als Fu"spunktdreiecke bezeichnet. Die Transversalen des
  Ausgangsdreicks sind dann wieder (andere) Transversalen des
  Fu"spunktdreiecks. So sind die Seitenhalbierenden eines Dreiecks
  gleichzeitig Seitenhalbierende seines Mittendreiecks.
  
  Man zeige: Die H"ohen eines Dreiecks $\ktriangle*{ABC}$ bilden
  Winkelhalbierende seines H"ohenfu"spunktdreiecks $\ktriangle*{H_aH_bH_c}$.
\end{aufgabe}
\begin{aufgabe}
  Der Umkreis des H"ohenfu"spunktdreiecks hat den Radius $\frac12\,R$.
\end{aufgabe}
\begin{thebibliography}{99}
\bibitem{Lesebogen} Mathematischer Lesebogen {\glqq}Junge Mathematiker{\grqq},
  Heft 80\\ Bezirkskabinett f"ur au"serunterrichtliche T"atigkeit, Rat des
  Bezirkes Leipzig, 1987
\bibitem{WURZEL} WURZEL, 5/97, \url{http://www.wurzel.org}
\bibitem{MO} \url{http://www.mathematik-olympiaden.de}
\end{thebibliography}

\begin{comment}
  to do: convert pictures
\end{comment}

\begin{attribution}
wirth (Dec 2004): Contributed to KoSemNet

graebe (2005-02-11): Prepared along the KoSemNet rules
\end{attribution}
\end{document}
