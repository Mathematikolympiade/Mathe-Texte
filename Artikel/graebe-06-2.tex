% Version: $Id: graebe-06-2.tex,v 1.2 2008/09/30 10:41:15 graebe Exp $
\documentclass[11pt]{article}  
\usepackage{kosemnet,ko-math,ngerman}  

\author{Hans-Gert Gr�be, Leipzig}
\title{Beweisen von Ungleichungen\kosemnetlicensemark\\ 
Arbeitsmaterial f�r Klasse 8}
\date{}

\newcommand{\merke}[1]{
  \begin{center}\fboxsep8pt
    \fbox{\begin{minipage}{.9\textwidth}#1\end{minipage}}
  \end{center}
}

\begin{document} 
\maketitle 

\merke{Statt die Ungleichung $A(x)\ge B(x)$ zu beweisen, ist es oft einfacher,
  die Ungleichung $A(x)-B(x)\ge 0$ zu beweisen.}

\begin{quote}\it 
  Zeige f�r reelle Zahlen $a,b\ge 0$, dass $\frac{a+b}{2}\ge \sqrt{a\,b}$
  gilt.
\end{quote}
{\bf Vor�berlegung}: Die Ungleichung kann in die Form 
\[\br{\frac{a+b}{2}}^2-a\,b = \frac{a^2+2\,a\,b+b^2}{4}-a\,b\ge 0\] 
gebracht werden. Umformen der linken Seite ergibt den Ausdruck 
\[\frac{a^2+2\,a\,b+b^2}{4}-a\,b=\frac{a^2-2\,a\,b+b^2}{4}
=\br{\frac{a-b}{2}}^2\] 
Dieser ist als Quadrat einer rellen Zahl stets nicht-negativ.  Mehr noch,
Gleichheit gilt genau dann, wenn $a=b$ ist.
\medskip

Dies ist jedoch erst die Analyse der Aufgabe durch R�ckw�rtsarbeiten, wobei
wir von der Behauptung ausgegangen sind.  Ein {\bf mathematischer Beweis} ist
in umgekehrter Richtung zu f�hren, also von einer wahren Aussage ausgehend
durch eine logisch einwandfreie Schlusskette die behauptete Aussage
abzuleiten.   Hier k�nnte wie folgt argumentiert werden: 
\medskip

{\bf Beweis}: Da das Quadrat einer rellen Zahl stets nichtnegativ ist, gilt
nacheinander
\begin{align*}
  &\br{\frac{a-b}{2}}^2\ge 0\\
  \yields\quad&\frac{a^2-2\,a\,b+b^2}{4}\ge 0\\
  \yields\quad&\frac{a^2+2\,a\,b+b^2}{4}-a\,b\ge 0\\
  \yields\quad&\br{\frac{a+b}{2}}^2\ge a\,b\\
  \yields\quad&\frac{a+b}{2}\ge \sqrt{a\,b}\,,\\
\end{align*}
wobei die letzte Umformung wegen $a,b\ge 0$ m�glich ist.\quad$\Box$ 
\medskip

Die eben bewiesene Ungleichung ist eine spezielle Form der {\em Ungleichung
  vom arithmetisch-geometrischen Mittel} (a.-g. M.).  F�r Zahlen
$a_1,\dots,a_n\ge 0$ bezeichnet man
\[A(a_1,\dots,a_n) = \frac{a_1+\dots+a_n}{n}\]
als {\em arithmetisches Mittel} und 
\[G(a_1,\dots,a_n) = \sqrt[n]{a_1\cdot\ldots\cdot a_n}\]
als {\em geometrisches Mittel}.  

Offensichtlich liegt jedes dieser Mittel zwischen der gr��ten
$\max(a_1,\dots,a_n)$ und der kleinsten $\min(a_1,\dots,a_n)$ dieser Zahlen.
Andere Quellen bezeichnen $A(a_1,\dots,a_n)$ als Mittel vom Grad 1 und
$G(a_1,\dots,a_n)$ als Mittel vom Grad 0.

Die eben bewiesene Ungleichung kann als $A(a,b)\ge G(a,b)$ angeschrieben
werden. 

Allgemein gilt f�r $a_1,\dots,a_n\ge 0$ die Ungleichung
\[A(a_1,\dots,a_n)\ge G(a_1,\dots,a_n)\]
und Gleichheit tritt genau f�r den Fall $a_1=a_2=\dots=a_n$ ein.  Diese
Aussage werden wir hier nicht beweisen\footnote{Ein Beweis dieser Ungleichung
wird etwa im KoSemNet-Aufsatz [graebe-97-1] gegeben.}.

\merke{Eine Ungleichung $A(x)\ge 0$ kann oft durch Zerlegung von $A(x)$ in
  Faktoren auf andere Ungleichungen zur�ckgef�hrt werden.}

\begin{quote}\it 
  Beweise f�r $a,b,c\ge 0$ die G�ltigkeit der Ungleichung
\[a^3+b^3+c^3\ge 3\,a\,b\,c\tag{1}\]
\end{quote}
Die Umformung in $a^3+b^3+c^3-3\,a\,b\,c\ge 0$ legt eine Faktorzerlegung von
$a^3+b^3+c^3-3\,a\,b\,c$ nahe.  Diese ist nicht offensichtlich, aber man kann
etwa durch Ausmultiplizieren pr�fen, dass  
\[a^3+b^3+c^3-3\,a\,b\,c = (a + b + c)\cdot (a^2 - a\,b - a\,c + b^2 - b\,c + 
c^2)
\] 
gilt. Folglich reicht es aus, die Ungleichung 
\[a^2+b^2+c^2\ge a\,b + a\,c +  b\,c\tag{2}\]
zu beweisen.  Diese l�sst sich durch eine weitere Methode herleiten:

\merke{Viele Ungleichungen lassen sich durch geschickte Zerlegungen auf eine
  der grundlegenden Ungleichungen, vor allem die Ungleichung vom
  arithmetisch-geometrischen Mittel, zur�ckf�hren. }

In der Tat gilt $a^2+b^2\ge 2\,a\,b$, $a^2+c^2\ge 2\,a\,c$ und  $b^2+c^2\ge
2\,b\,c$ nach der Ungleichung vom a.-g. M. Addieren wir die drei
Ungleichungen, so ergibt sich
\[2\br{a^2+b^2+c^2}\ge 2\,a\,b + 2\,a\,c + 2\,b\,c\]
und damit die Ungleichung (2).  Da wegen $a,b,c\ge 0$ auch $a+b+c\ge 0$ gilt,
k�nnen wir die Ungleichung 
\[a^2+b^2+c^2- \br{a\,b + a\,c +  b\,c}\ge 0\tag{3}\]
mit $(a+b+c)$ multiplizieren, erhalten
\[(a + b + c)\cdot (a^2 - a\,b - a\,c + b^2 - b\,c + c^2) =
a^3+b^3+c^3-3\,a\,b\,c \ge 0\]
und schlie�lich einen Beweis der Ungleichung (1).

\merke{Oft lassen sich Aufgaben, die auf den ersten Blick gar nicht danach
  aussehen, auf bekannte Ungleichungen zur�ckf�hren.}

\begin{quote}\it 
  F�r zwei reelle Zahlen $x,y>0$ sei $\frac1x+\frac1y=3$.  Zeige, dass dann
  $x\,y\ge \frac49$ gilt.
\end{quote}
Diese Aufgabe {\glqq}riecht{\grqq} nach der Ungleichung vom a.-g.\,M., denn
f�r $x=y\,(=\frac23)$  gilt gerade $x\,y=\frac49$.  Und in der Tat erhalten
wir nacheinander 
\begin{align*}
  3&=1/x+1/y\\
  \yields\quad\frac32&=\frac{1/x+1/y}{2}\ge \sqrt{\frac1x\,\frac1y} =
  \frac{1}{\sqrt{x\,y}}
\intertext{Wir bilden nun das Reziproke beider Seiten, wobei sich das
  Relationszeichen umkehrt (hier ist auch wesentlich, dass beide Seiten der
  Ungleichung positive Zahlen sind).}
  \yields\quad\frac23&\le{\sqrt{x\,y}}\\
\intertext{Schlie�lich quadrieren wir diese Ungleichung und haben die
  Behauptung bewiesen.}
  \yields\quad\frac49&\le{x\,y}
\end{align*}

\begin{attribution}
  graebe (2006-03-16):\\ Begleitmaterial f�r den LSGM-Korrespondenzzirkel in
  der Klasse 8
\end{attribution}
\newpage \mbox{}
\end{document}
