\documentclass[11pt,a4paper]{article}
\usepackage{ngerman,url}
\usepackage{kosemnet,ko-math}

\title{Schnipsel} 
\author{Hans-Gert Gräbe}
\date{24. Oktober 2023}

\begin{document}
\maketitle

Neben den Fibonacci-Zahlen umfassen andere konstant rekursive Sequenzen die
Lucas-Zahlen und Lucas-Sequenzen, die Jacobsthal-Zahlen, die Pell-Zahlen und
allgemeiner die Lösungen für die Pell-Gleichung.

Quelle: \url{https://de.wikibrief.org/wiki/Recurrence_relation}

Pell-Zahlen sind eine Approximation von $\sqrt{2}$: $a_1=b_1=1$,
$a_{n+1}=a_n+2b_n$, $b_{n+1}=a_n+b_n$. Es gilt $a_n^2-2b_n^2=(-1)^n$

\begin{gather*}
  \frac12, \frac32, \frac75, \frac{17}{12} \to \sqrt{2}
\end{gather*}

\paragraph{BWM 2001, Aufgabe 1.2:}
Von einer Folge $(a_0, a_1, a_2, \dots$) reeller Zahlen sei bekannt:
\begin{gather*}
  a_0 =1\ \text{und} a_{n+1}=a_n+\sqrt{a_{n+1}+a_n}\ \text{für alle
    natürlichen Zahlen $n$}. 
\end{gather*}
Man beweise, dass nur eine einzige Folge mit diesen Eigenschaften existiert,
und gebe eine explizite Formel für $a_n$ an.

\begin{loesung}
  Wir brauchen zunächst eine Bildungsvorschrift, mit der $a_{n+1}$ aus bereits
  bekannten Folgengliedern berechnet werden kann. diese soll zunächst
  hergeleitet werden.

  Erfüllt eine Folge $(a_n)$ die Bedingungen, so muss insbesondere
  \begin{gather*}
    \br{a_{n+1}-a_n}^2=a_{n+1}+a_n
  \end{gather*}
  sein.  Das führt auf eine quadratische Gleichung in $a_{n+1}$
  \begin{gather*}
    a_{n+1}^2-(2a_n+1)a_{n+1}+a_n^2-a_n=0
  \end{gather*}
  mit den zwei Lösungen
  \begin{gather*}
    a_{n+1}=a_n+\frac12 \pm \frac12\sqrt{8a_n+1}. 
  \end{gather*}
  Durch das Quadrieren ist hier eine Scheinlösung hinzugekommen, denn aus
  \begin{gather*}
    a_{n+1}=a_n+\sqrt{a_{n+1}+a_n}
  \end{gather*}
  folgt $a_{n+1}>a_n$, so dass die Folge monoton wächst.  Damit bleibt als
  einziger Lösungskandidat die Folge
  \begin{gather*}
    a_{n+1}=a_n+\frac12\br{1+\sqrt{8a_n+1}},\ a_0=1. \tag{L} 
  \end{gather*}
  Berechne erste Folgenglieder:
  \begin{code}
    a(n):=if n=0 then 1\\
    else a(n-1)+1/2+1/2*sqrt(8*a(n-1)+1);\\[4pt]
    makelist(a(n),n,0,10);
  \end{code}
  \begin{gather*}    
    [1, 3, 6, 10, 15, 21, 28, 36, 45, 55, 66]
  \end{gather*}
  Interessanterweise sind das alles ganze Zahlen und die OEIS kennt diese
  Folge als A000217 und den Hintergrund dazu: Es handelt sich um Folge der
  Dreieckszahlen, also $a_n=\binom{n+2}{2}=\dfrac{(n+1)(n+2)}{2}$. 
\end{loesung}

Für andere Startwerte (etwa $n=2$) erhalten wir geschachtelte Wurzelausdrücke
als Folgenglieder. Die Näherungswerte scheinen aber ebenfalls quadratisch zu
wachsen.  Dazu berechnen wir die zweite Differenzenfolge
\begin{code}
  a(n):=if n=0 then 1\\
  else a(n-1)+1/2+1/2*sqrt(8*a(n-1)+1);\\[4pt]
  D2(n):=a(n+2)-2*a(n+1)+a(n);\\[4pt]

  l:makelist(a(n),n,0,3);\\
  ratsimp(l);\\
  d:makelist(D2(n),n,0,8),float;
\end{code}
\begin{align*}
  l&=\sbr{2,\frac{5+\sqrt{17}}{2},
    \frac{\sqrt{4\sqrt{17} + 21} + \sqrt{17} + 6}{2}, a_3,\dots}\tag{$L_1$}\\
  &\qquad \text{mit}\
  a_3=\frac{\sqrt{4\sqrt{4\sqrt{17} + 21}+\sqrt{17}+25} + \sqrt{4\sqrt{17} + 21}
      + \sqrt{17} + 7}{2}\\
  d&=\sbr{1.0, 1.0, 1.0, 1.0, 1.0, 1.0, 1.0, 1.0, 1.0} 
\end{align*}
Wir sehen, dass die zweite Differenzenfolge konstant ist, was auch bei anderen
Startwerten als $a_0=1$ auf eine Rekursion 
\begin{gather*}
  a_{n+2}-2\m a_{n+1}+a_n=1\tag{R}
\end{gather*}
hindeutet mit der allgemeinen Lösung $a_n=A\m n^2+B\m n+C$.  Setzen wir $n=0$,
so erhalten wir $C=a_0$.  Setzen wir $A\m n^2$ in (R) ein, so erhalten wir
$A=\frac12$.  Für die oben betrachtete Folge mit $a_0=2$ ergibt sich für $n=1$
\begin{gather*}
  a_1=\frac{5+\sqrt{17}}{2}=\frac12+B+2 
\end{gather*}
und damit $B=\frac12\sqrt{17}$ sowie die allgemeine Formel $a_n=\frac12
n^2+\frac12\sqrt{17} n+2$. Vergleichen wir das mit den Werten der
ursprünglichen Folge 
\begin{code}
  a1(n):=1/2*(n\pw 2+sqrt(17)*n+4);\\[4pt]
  l1:makelist(a1(n),n,0,9);\\
  makelist(a1(n)-a(n),n,0,9),float;
\end{code}
\begin{gather*}
  \sbr{2,\frac{5+\sqrt{17}}{2},\frac{8+2\sqrt{17}}{2},\frac{13+3\sqrt{17}}{2},\dots}\\  
  \sbr{0.0, 0.0, 0.0, - 1.776E-15, 0.0, - 3.55E-15, - 7.105E-15, - 7.105E-15,
    0.0, 0.0}
\end{gather*}
so sehen wir Übereinstimmung. Es liegt also die Vermutung nahe, dass sich die
geschachtelten Wurzelausdrücke in $(L_1)$ durch einfachere Wurzeln ausdrücken
lassen.  Insbesondere ergibt der Vergleich der beiden Darstellungen von $a_2$
die Vermutung, dass 
\begin{gather*}
  \sqrt{4\sqrt{17} + 21}=2+\sqrt{17}
\end{gather*}
gilt, was durch Quadrieren leicht zu verifizieren ist.  Für $a_3$ sind die
Rechnungen schon komplizierter.

Bleibt zu zeigen, dass (L) mit beliebigen Startwerten (R) erfüllt.  Mit (L)
ist $a_{n+1}-a_n=\frac12\br{1+\sqrt{8a_n+1}}$ und damit
\begin{gather*}
  a_{n+2}-2\m a_{n+1}+a_n=\br{a_{n+2}-a_{n+1}}-\br{a_{n+1}-a_n}
  =\frac12\br{\sqrt{8a_{n+1}+1}-\sqrt{8a_n+1}}.
\end{gather*}
Weiter haben wir wieder mit (L)
\begin{gather*}
  8a_{n+1}+1=8a_n+5+4\sqrt{8a_n+1}=\br{\sqrt{8a_n+1}+2}^2
  \intertext{und so}
  \sqrt{8a_{n+1}+1}-\sqrt{8a_n+1}=2.
\end{gather*}

\paragraph{Wurzel, Okt. 1990:}
\begin{quote}
  $a_0,\dots, a_n$ sei eine Folge nichtnegativer ganzer Zahlen mit $a_0=0$ und
  $\abs{a_i-a_{i+1}}=1, i=0,\dots,n-1$.

  Zeige, dass es genau $\binom{n}{\floor{\frac{n}{2}}}$ solcher Folgen gibt.
\end{quote}
\begin{loesung}
  Stellt man die entsprechenden Pfade bis zu $a_k=i$ ($k+1$ ist immer gerade)
  dar, so ergibt sich ein „halbes“ Pascalsches Dreieck
  
  Setze $f(k,i)$ gleich Zahl der Folgen bis $a_k=i$, dann ergibt sich die
  Rekursion $f(k,i)=f(k-1,i+1)+f(k-1,i-1)$, wobei $f(0,0)=1$ und $f(k,i)=0$
  für $i<0$ oder $i>k$ gesetzt wird. Weiter?  
\end{loesung}

\paragraph{(Wurzel 9/1972):} 
\begin{quote}
  Haben von $n>2$ beidseitig unendlichen arithmetischen Folgen \emph{ganzer
  Zahlen} je zwei ein gemeinsames Glied, so haben alle Folgen ein gemeinsames
  Glied.
\end{quote}
\begin{loesung}
  Zwei Folgen $a_1+n\m d_1$ und $a_2+n\m d_2$ haben genau dann ein gemeinsames
  Folgenglied $A$, wenn $a_1-a_2$ durch $\gcd(d_1,d_2)$ teilbar ist, wenn also
  $a_1-a_2\equiv 0\pmod{\gcd(d_1,d_2)}$ ist.  Die beiden Folgen haben dann sie
  die gemeinsame Teilfolge $A+n\m d$ mit $d=\mathrm{kgV}(d_1,d_2)$.

  Weiter?

\end{loesung}

\paragraph{Aufgabe:} 
\begin{quote}
  Haben von $n$ beidseitig unendlichen arithmetischen Folgen \emph{reeller
  Zahlen} je drei ein gemeinsames Glied, so haben alle Folgen ein gemeinsames
  Glied.
\end{quote}
\begin{loesung}
  Zwei Folgen $a_1+n\m d_1$ und $a_2+n\m d_2$ mehr als ein Folgenglied
  gemeinsam, so ist $\frac{d_1}{d_2}$ rational, denn aus 
  \begin{gather*}
    a_1+n_1d_1=a_2+n_2d_2\ \text{und}\ a_1+n_1'd_1=a_2+n_2'd_2
  \end{gather*}
  folgt $\br{n_1-n_1'}d_1=\br{n_2-n_2'}d_2$.  Ist also ein solcher Quotient
  nicht rational, so haben diese beiden Folgen höchstens ein Glied gemeinsam,
  und das ist auch das einzig mögliche Glied, das dann eine dritte Folge mit
  diesen beiden Folgen gemeinsam haben kann.

  Weiter? Müssen die Folgen beidseitig unendlich sein? 
\end{loesung}

\paragraph{\cite{S}, Aufgabe 15:}
Für die Mersennezahlen gilt $\gcd\br{2^a-1,2^b-1}=2^{\gcd(a,b)}-1$.  Das
ergibt sich wie folgt: Mit $a=q\m b+r$ ist 
\begin{gather*}
  2^a=\br{2^b}^q\m 2^r\equiv 1\m 2^r \pmod{2^b-1}
\end{gather*}
und damit $\gcd\br{2^a-1,2^b-1}=\gcd\br{2^b-1,2^r-1}$.  Weiter wie im
Euklidschen Algorithmus.

Vermutung: Die Glieder der  Folge $a_n=2^n-3$, deren Index eine Fibonaccizahl
ist, sind paarweise teilerfremd.

\begin{thebibliography}{Wor77}

\bibitem[S]{S} A.~Schüler.  \newblock \emph{Rekursive Folgen}.  Text
  \emph{schueler-05-1} im KoSemNet.
\end{thebibliography}

\end{document}
