\documentclass[11pt,a4paper]{article}
\usepackage{ngerman,url}
\usepackage{kosemnet,ko-math}

\title{Schnipsel} 
\author{Hans-Gert Gräbe}
\date{16. Oktober 2023}

\begin{document}
\maketitle
 
\paragraph{Wurzel, Okt. 1990:}
\begin{quote}
  $a_0,\dots, a_n$ sei eine Folge nichtnegativer ganzer Zahlen mit $a_0=0$ und
  $\abs{a_i-a_{i+1}}=1, i=0,\dots,n-1$.

  Zeige, dass es genau $\binom{n}{\floor{\frac{n}{2}}}$ solcher Folgen gibt.
\end{quote}
\begin{loesung}
  Stellt man die entsprechenden Pfade bis zu $a_k=i$ ($k+1$ ist immer gerade)
  dar, so ergibt sich ein „halbes“ Pascalsches Dreieck
  
  Setze $f(k,i)$ gleich Zahl der Folgen bis $a_k=i$, dann ergibt sich die
  Rekursion $f(k,i)=f(k-1,i+1)+f(k-1,i-1)$, wobei $f(0,0)=1$ und $f(k,i)=0$
  für $i<0$ oder $i>k$ gesetzt wird. Weiter?  
\end{loesung}

\paragraph{(Wurzel 9/1972):} 
\begin{quote}
  Haben von $n>2$ beidseitig unendlichen arithmetischen Folgen \emph{ganzer
  Zahlen} je zwei ein gemeinsames Glied, so haben alle Folgen ein gemeinsames
  Glied.
\end{quote}
\begin{loesung}
  Zwei Folgen $a_1+n\m d_1$ und $a_2+n\m d_2$ haben genau dann ein gemeinsames
  Folgenglied $A$, wenn $a_1-a_2$ durch $\gcd(d_1,d_2)$ teilbar ist, wenn also
  $a_1-a_2\equiv 0\pmod{\gcd(d_1,d_2)}$ ist.  Die beiden Folgen haben dann sie
  die gemeinsame Teilfolge $A+n\m d$ mit $d=\mathrm{kgV}(d_1,d_2)$.

  Weiter?

\end{loesung}

\paragraph{Aufgabe:} 
\begin{quote}
  Haben von $n$ beidseitig unendlichen arithmetischen Folgen \emph{reeller
  Zahlen} je drei ein gemeinsames Glied, so haben alle Folgen ein gemeinsames
  Glied.
\end{quote}
\begin{loesung}
  Zwei Folgen $a_1+n\m d_1$ und $a_2+n\m d_2$ mehr als ein Folgenglied
  gemeinsam, so ist $\frac{d_1}{d_2}$ rational, denn aus 
  \begin{gather*}
    a_1+n_1d_1=a_2+n_2d_2\ \text{und}\ a_1+n_1'd_1=a_2+n_2'd_2
  \end{gather*}
  folgt $\br{n_1-n_1'}d_1=\br{n_2-n_2'}d_2$.  Ist also ein solcher Quotient
  nicht rational, so haben diese beiden Folgen höchstens ein Glied gemeinsam,
  und das ist auch das einzig mögliche Glied, das dann eine dritte Folge mit
  diesen beiden Folgen gemeinsam haben kann.

  Weiter? Müssen die Folgen beidseitig unendlich sein? 
\end{loesung}

\paragraph{\cite{S}, Aufgabe 15:}
Für die Mersennezahlen gilt $\gcd\br{2^a-1,2^b-1}=2^{\gcd(a,b)}-1$.  Das
ergibt sich wie folgt: Mit $a=q\m b+r$ ist 
\begin{gather*}
  2^a=\br{2^b}^q\m 2^r\equiv 1\m 2^r \pmod{2^b-1}
\end{gather*}
und damit $\gcd\br{2^a-1,2^b-1}=\gcd\br{2^b-1,2^r-1}$.  Weiter wie im
Euklidschen Algorithmus.

Vermutung: Die Glieder der  Folge $a_n=2^n-3$, deren Index eine Fibonaccizahl
ist, sind paarweise teilerfremd.

\begin{thebibliography}{Wor77}

\bibitem[S]{S} A.~Schüler.  \newblock \emph{Rekursive Folgen}.  Text
  \emph{schueler-05-1} im KoSemNet.
\end{thebibliography}

\end{document}
