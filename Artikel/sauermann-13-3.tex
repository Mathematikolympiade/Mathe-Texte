\documentclass[11pt,a4paper]{article}
\usepackage{kosemnet,ko-math,ngerman,url}
\usepackage[utf8]{inputenc}

\title{Kombinatorik für Klasse 8\kosemnetlicensemark}
\author{Lisa Sauermann}
\date{März 2013}
\begin{document}
\maketitle

Kombinatorik ist ein Gebiet, das oft bei Mathematikolympiaden vorkommt. Es
gibt dabei Aufgaben in beliebigen Schwierigkeitsstufen. In unteren
Klassenstufen werden oft Aufgaben vom Typ der Aufgaben 2--4 gestellt. Solche
Aufgaben löst man wie weiter unten erklärt wird, aber wie euch sicherlich
schon bekannt ist. Deshalb sollen diese Aufgabentypen hier nicht weiter
besprochen werden, wer sich darin unsicher fühlt, findet unter den
Bundesrundenaufgaben Klasse 8 aus den vergangenen Jahren zahlreiche
Beispiele. Oft sind diese noch mit Wahrscheinlichkeitsüberlegungen bei
irgendwelchen Spielen mit bunten Kugeln in verschiedenen Töpfen
verbunden. Dies macht die Aufgaben aber auch nicht interessanter.

Für einen zweiten, bei Bundesrunden Klasse 8 oft vorkommender Aufgabentyp ist
Aufgabe~1 ein Beispiel. Auch diese Logikrätsel könnt ihr bei Bedarf anhand von
alten Aufgaben üben.

Hier sollen dagegen Kombinatorikaufgaben, die nicht so standardmäßig mit
diesem Lösungsprinzipien durchgehen, besprochen werden. Denn auch solche
Aufgaben findet man in Bundesrunden Klasse 8. Es gibt einige Lösungstricks,
die dafür und auch für Olympiadeaufgaben in den kommenden Jahren wichtig und
nützlich sind.

\subsection*{Logik-Rätsel}

\paragraph{Aufgabe 1 (400841)} 
Beim großen Preis von Schönheide wurde unter den Springreitern ein Stechen
erforderlich, an dem nur noch Alex, Boris, Chris und Danny teilnahmen. Bei
einem solchen Stechen erreicht jeder der vier Reiter genau eine Platznummer
(Erster, Zweiter, Dritter, Vierter). Einige Fans machten Vorhersagen, etwas
nebulös, wie sie es gewohnt waren. Sie sagten:
\begin{itemize}
\item[(1)] Jede der vier Platznummern wird genau einmal erreicht.
\item[(2)] Wenn Alex nicht erster wird, dann wird Chris Vierter.
\item[(3)] Und wenn Chris Dritter wird, dann wird Alex sogar Letzter.
\item[(4)] Nun, jedenfalls wird Alex, verglichen mit Danny, einen besseren
  Platz erreichen.
\item[(5)] Immerhin: Wenn Boris nicht Erster wird, dann wird Alex Dritter.
\item[(6)] Wenn Chris Zweiter wird, dann wird Danny gewiss nicht Vierter.
\item[(7)] Wenn Chris sogar Erster wird, dann wird Danny Zweiter.
\item[(8)] Wenn aber Danny nicht Zweiter wird, dann wird auch Boris nicht
  Zweiter.
\end{itemize}
Ein weiterer Fan, der das hörte, meinte: Wenn alle Vorhersagen wahr sind, dann
gibt es ja höchstens eine Möglichkeit, wie sich die Plätze verteilen! Zeige,
dass er recht hat! Wie lautet diese Platzverteilung? Weise auch nach, dass in
der Tat bei dieser Platzverteilung alle Vorhersagen (1) bis (8) wahr
sind!

\textit{Hinweis:} Eine Aussage vom Typ „Wenn-dann“, in der die mit „Wenn“
eingeleitete --- und ohne dieses „Wenn“ betrachtete --- Aussage falsch ist,
ist wahr.

Bei solchen Aufgaben empfiehlt es sich, eine Tabelle mit den Zeilen Alex,
Boris, Chris, Danny und den Spalten Erster, Zweiter, Dritter, Vierter
anzulegen und die einzelnen Aussagen dort einzutragen. Mit verschiedenen
Farben (deshalb immer Buntstifte oder Fineliner zu Klausuren mitnehmen!) kann
man dann die Aussagen vom Typ „Wenn-dann“ eintragen. So behält man die
Übersicht und kommt zügig zur Lösung.

\subsection*{Permutationen, Variationen, Kombinationen}

\paragraph{Aufgabe 2} 
Die 16 Bundesländer nehmen bei der Bundesrunde in der (inoffiziellen)
Länderwertung eine eindeutige Rangfolge von 1 bis 16 ein. Wie viele
Möglichkeiten gibt es für die Reihenfolge der Platzierung von Sachsen,
Sachsen-Anhalt, Thüringen, Brandenburg, Berlin und Mecklenburg-Vorpommern? Wie
viele Möglichkeiten gibt es, wenn Sachsen die Bundesrunde gewinnt?

\paragraph{Aufgabe 3} 
Sachsen und NRW sind bei der Bundesrunde besondere Konkurrenten. Wie viele
Möglichkeiten für die Platzierungen von Sachsen und NRW innerhalb der 16
Bundesländer gibt es insgesamt?

\paragraph{Aufgabe 4 (360941, Aufgabentext geändert)} 
Ein Fotoklub möchte eine Serie von Blättern zum Verteilen herstellen. Auf
jedem Blatt sollen 6 Bilder sein. Für solche Bilder stehen insgesamt 10
verschiedene Motive zur Verfügung. Jede mögliche Zusammenstellung von 6
verschiedenen dieser Motive soll in einer beliebigen Reihenfolge auf genau
einem Blatt vorkommen. Wie viele solche Blätter gibt es dann? Wieviele
Exemplare (kleinstmöglihe Anzahl) von jedem der Motive reichen aus?  

In Aufgabe 2 geht es um die möglichen Reihenfolgen der 6 genannten
Bundesländer. Für Sachsen gibt es 6 Möglichkeiten (Erster dieser 6, Zweiter
dieser 6, \dots). Für Sachsen-Anhalt verbleiben noch 5 Möglichkeiten (eine hat
Sachsen ja schon belegt). Für Thüringen gibt es also noch 4, für Brandenburg
3, für Berlin 2 und für Mecklenburg-Vorpommern 1 Möglichkeit. Das macht
insgesamt $6\cdot 5\cdot 4\cdots 1=720$ Möglichkeiten.

Wissen wir schon, dass Sachsen Erster ist, müssen wir die Anzahl der möglichen
Reihenfolgen der anderen 5 Bundesländer festlegen. Nach dem gleichen Prinzip
ergeben sich dafür $5\cdot 4\cdots 1=120$ Möglichkeiten.

Diese Aufgabe kann man natürlich auch für jede beliebige positive ganze Zahl
$n$ lösen. Das Produkt der ganzen Zahlen von 1 bis $n$ bezeichnet man als $n!$
($n$ Fakultät). Die Anzahl der Möglichkeiten, $n$ Objekte in eine Mögliche
Reihenfolge zu bringen, ist also $n!$. Diese möglichen Reihenfolgen bezeichnet
man auch als \textit{Permutationen}. Es gibt also $n!$ verschiedene
Permutationen einer Menge aus $n$ Objekten.

Die Zahl $0!$ bezeichnet das leere Produkt (es gibt ja keinen Faktor). Deshalb
wird $0!=1$ definiert.

In Aufgabe 3 kann man die gleiche Lösungsidee anwenden: Für die Platzierung
von Sachsen gibt es 16 Möglichkeiten und für die Platzierung von NRW dann noch
15.  Also gibt es insgesamt $16 \cdot 15=240$ Möglichkeiten.

Es wird also der Reihenfolge nach erst für Sachsen eine der 16 Platznummern
ausgewählt und dann für NRW eine davon verschiedene. Unter Beachtung der
Reihenfolge aus $n$ Objekten $k$ verschiedene auszuwählen, gibt es also
$n\cdot (n-1)\dots (n-k+1)=\frac{n!}{(n-k)!}$ Möglichkeiten. Diese heißen
\textit{Variationen}. In Aufgabe 3 war $n=16$ und $k=2$.

In Aufgabe 4 geht es um \textit{Kombinationen} von 6 verschiedenen Bildern aus
10. Beachten wir ihre Reihenfolge auf den Blättern, erhalten wir die
$\frac{10!}{4!}$ verschiedenen Variationen. Jede Kombination kommt in $6!$
verschiedenen Reihenfolgen vor, die Reihenfolge ist uns aber egal. Also müssen
wir die Anzahl der Variationen durch $6!$ teilen und erhalten
$\frac{10!}{4!\cdot 6!}=210$ verschiedene Blätter.

Nun ist die Anzahl der Exemplare von jedem Bild gefragt. Jedes Bild soll mit
jeder Kombination aus 5 der übrigen 9 Bilder genau einmal auftauchen. Für die
Anzahl dieser Kombinationen gibt es nun $\frac{9!}{4!\cdot 5!}=126$
Möglichkeiten, also braucht man von jedem Bild 126 Exemplare.

Aus $n$ Objekten $k$ verschiedene ohne Beachtung der Reihenfolge auszuwählen,
gibt es also $\frac{n!}{k!\cdot (n-k)!}$ Möglichkeiten.

\paragraph{Aufgabe 5 (American High School Mathematics Examination 1989)} 
Herr und Frau Zeta wollen ihrem Baby zwei Vornamen geben, sodass die drei
Initialien paarweise verschieden und in alphabetischer Reihenfolge sind. Wie
viele Möglichkeiten gibt es für das Monogramm ihres Babys?

\paragraph{Aufgabe 6 (American High School Mathematics Examination 1988)} 
Am Ende eines Bowling-Turniers gibt es ein Endspiel der besten 5 Bowler der
Vorrunde. Zuerst spielt der Fünfte gegen den Vierten. Der Verlierer bekommt
den 5. Preis und der Gewinner spielt ein Spiel gegen den Dritten. Dabei
bekommt der Verlierer den 4. Preis und der Gewinner spielt gegen den
Zweiten. Dabei bekommt der Verlierer den 3. Preis und der Gewinner spielt
gegen den Ersten. Der Gewinner dieses Spiels bekommt den 1. Preis und der
Verlierer den 2. Preis. Wie viele verschiedene Verteilungen der Preise gibt es
auf die 5 Besten der Vorrunde?

\paragraph{Aufgabe 7 (American Mathematics Contest 12 2001)} 
Eine Spinne hat einen Strumpf und einen Schuh für jedes ihrer 8 Beine. In wie
vielen verschiedenen Reihenfolgen kann sich die Spinne ihre Strümpfe und
Schuhe anziehen, wenn an jedem Bein der Strumpf vor dem Schuh angezogen werden
muss?

\paragraph{Aufgabe 8 (American Invitational Mathematics Examination 1996)} 
Zwei Felder eines $7\times 7$-Quadrates sind gelb gefärbt, die anderen grün.
Zwei Färbungen werden als gleich angesehen, wenn man sie durch Drehung
auseinander erhalten kann. Wie viele verschiedene Färbungen gibt es?

\subsection*{Binomialkoeffizienten}
Die Anzahl $\frac{n!}{k!\cdot (n-k)!}$ der Kombinationen von $k$ Objekten aus
$n$ gegebenen bezeichnet man als $\binom{n}{k}$ (lies: $n$ über $k$). Wir
definieren also
\[\binom{n}{k}=\frac{n!}{k!\cdot (n-k)!}\,.\]
Diese Schreibweise $\binom{n}{k}$ wird als Binomialkoeffizient bezeichnet.

Wieviel ist $\binom{n}{0}$, $\binom{n}{1}$, $\binom{n}{k}$,
$\binom{n}{n-2}$, $\binom{n}{n-1}$ und $\binom{n}{n}$?

Diese Schreibweise wollen wir nun bei der Beantwortung folgender Frage
benutzen: Was ergibt sich beim Ausmultiplizieren von $(a+b)^{n}$?

Es ergibt sich sicherlich eine Summe lauter Terme der Form $a^{i}b^{n-i}$,
denn jeder Faktor $(a+b)$ der Potenz steuert entweder ein $a$ oder ein $b$
bei.  Wie viele Summanden $a^{i}b^{n-i}$ gibt es für eine feste ganze Zahl
$0\leq i\leq n$? Es muss bei jedem solchen Summanden genau $i$ Faktoren
$(a+b)$ der Potenz geben, die ein $a$ beisteuern. Wie viele Möglichkeiten gibt
es für diese $i$ Faktoren? Nun, das ist eine beliebige Kombination von $i$ der
insgesamt $n$ Faktoren. Es gibt also $\binom{n}{i}$ Möglichkeiten. Damit
taucht der Summand $a^{i}b^{n-i}$ genau $\binom{n}{i}$ mal in der Summe
auf. Es gilt
also 
\begin{gather*}
  (a+b)^{n}=\binom{n}{0}\,a^{0}b^{n-0} +\binom{n}{1}\,a^{1}b^{n-1}
  +\binom{n}{2}\,a^{2}b^{n-2}+\dots +\binom{n}{n}\,a^{n}b^{n-n}\,.
\end{gather*}
Dies können wir auch schreiben als
\begin{gather*}
  (a+b)^{n}=\sum_{i=0}^{n} \binom{n}{i}\,a^{i}b^{n-i}\,.
\end{gather*}

Das Summenzeichen ist die Abkürzung einer Summe mit vielen Summanden. Es wird
für die Laufvariable, hier $i$, der Reihe nach 0 (siehe unter dem
Summenzeichen) bis $n$ (über dem Summenzeichen) eingesetzt und jedesmal der
Term $\binom{n}{i}\,a^{i}b^{n-i}$ gebildet. All diese Terme werden
summiert. Ausgeschrieben heißt das Summenzeichen also tatsächlich
\begin{gather*}
  \sum_{i=0}^{n} \binom{n}{i}\,a^{i}b^{n-i} =\binom{n}{0}\,a^{0}b^{n-0}
  +\binom{n}{1}\,a^{1}b^{n-1} +\binom{n}{2}\,a^{2}b^{n-2} +\dots
  +\binom{n}{n}\,a^{n}b^{n-n}\,.
\end{gather*}
Somit haben wir den Binomischen Lehrsatz hergeleitet.
\begin{satz}
  Für reelle Zahlen $a$ und $b$ und eine nicht negative ganze Zahl $n$
  gilt
  \begin{gather*}
    (a+b)^{n}=\sum_{i=0}^{n} \binom{n}{i}\,a^{i}b^{n-i}\,.
  \end{gather*}
\end{satz}
Der Spezialfall $n=2$ des Binomischen Lehrsatzes ist die erste binomische
Formel.  Wie erhält man die zweite binomische Formal aus dem Binomischen
Lehrsatz?

Zwischen den Binomialkoeffizienten gibt es zahlreiche Identitäten und
Beziehungen, von denen hier nur sehr wenige behandelt werden. Solche
Beziehungen kann man durch eine kombinatorische Interpretation oder durch
Herumrechnen mit den Fakultätsausdrücken nachweisen.

Eine direkt aus der Definition der Binomialkoeffizienten folgende Identität
ist diese: Für $n$ nichtnegativ, ganz und $0\leq k\leq n$ ganz gilt
\[\binom{n}{k}=\binom{n}{n-k}\]

Setzen wir im Binomischen Lehrsatz $a=b=1$, erhalten wir folgende sehr hübsche
Identität für alle nichtnegativen ganzen Zahlen $n$
\begin{gather*}
  \binom{n}{0}+\binom{n}{1}+\binom{n}{2}+\dots +\binom{n}{n}
  =\sum_{i=0}^{n} \binom{n}{i}=2^{n}\,.
\end{gather*}

Für $n\geq 1$ ganz und $0\leq k\leq n-1$ ganz gilt
\[\binom{n+1}{k+1}=\binom{n}{k}+\binom{n}{k+1}\,.\]
Dies Formel lässt sich am einfachsten durch eine kombinatorische
Interpretation zeigen. Tipp: Stelle dir $n$ Schüler beim Landesseminar vor,
von denen einer zu spät zum Vortrag kommt.

\paragraph{Aufgabe 9 (Rumänische Mathematikzeitschrift)} 
Sei $n$ eine ungerade Zahl größer~1. Zeige, dass die Folge
\[\binom{n}{1}, \binom{n}{2}, \dots ,\binom{n}{\frac{n-1}{2}}\]
eine ungerade Anzahl ungerader Zahlen enthält.

Binomialkoeffizienten kommen auch vor, wo man sie gar nicht vermutet:

\paragraph{Aufgabe 10} 
Finde einen Ausdruck als Binomialkoeffizienten für die Anzahl der
Möglichkeiten, 2009 als Summe von 48 positiven ganzzahligen Summanden
darzustellen (zwei Darstellungen gelten hier auch als verschieden, wenn sie
sich nur in der Reihenfolge der Summanden unterscheiden)!

\paragraph{Aufgabe 11 (American Invitational Mathematics Examination 1998)} 
Finde die Anzahl geordneter Quadrupel $\br{x_{1}, x_{2}, x_{3}, x_{4}}$
positiver ungerader Zahlen mit $x_{1}+x_{2}+x_{3}+x_{4}=98$.

\paragraph{Aufgabe 12 (American High School Mathematics Examination 1994)} 
Neun Stühle in einer Reihe sollen von 6 Schülern und den drei Lehrern Herrn
Alpha, Herrn Beta und Herrn Gamma belegt werden. Die drei Lehrer betreten den
Raum zuerst und wollen sich so hinsetzen, dass jeder neben zwei Schüler sitzen
wird. Wie viele Möglichkeiten haben die drei Lehrer, sich hinzusetzen?

\paragraph{Aufgabe 13} 
Finde die Anzahl der Möglichkeiten, aus den ganzen Zahlen von 1 bis 18 fünf
Zahlen auszuwählen, so dass sich je zwei davon um mindestens 2 unterscheiden.

\subsection*{Doppeltes Abzählen und Permutationen}

Eine sehr wichtige Beweismethode bei Kombinatorikaufgaben ist doppeltes
Abzählen. Die Idee ist einfach: Wir zählen die Anzahl bestimmter Dinge auf
zwei verschiedene Weisen. Dann müssen beide Anzahlen gleich sein. Dies kann
man beispielsweise benutzen um Identitäten zu zeigen, die man dazu
kombinatorisch interpretiert. Oder aber man kann zeigen, dass beispielsweise
irgendwelche Anzahlen gerade sein müssen, weil sie sich auf eine andere Art
und Weise abzählen lassen, bei der offensichtlich eine gerade Anzahl
herauskommt.

Man benutzt doppeltes Abzählen oft schon intuitiv ohne es zu merken,
z.\,B.\ wenn man einen Haufen Bonbons in Fünferhäufchen zerlegt, um leichter
nachzählen zu können. Um das Prinzip also gut zu verdeutlichen, behandeln wir
eine schwere Beispielaufgabe, die man ohne bewusstes Anwenden dieser Tehnik
vermutlich nur sehr umständlich lösen kann.

Beschäftigen wir uns aber zunächst noch einmal etwas näher mit Permutationen,
die ja ein wichtiges Thema in der Kombinatorik sind. Betrachten wir eine
Permutation $P$ der Menge $\cbr{1,2,3,\dots ,n}$. Diese können wir nun
folgendermaßen interpretieren: Die Zahlen von 1 bis $n$, die anfangs geordnet
sind, vertauschen sich nun und stellen sich in der von $P$ vorgegebenen
Reihenfolge auf. Dabei laufen sie wild durcheinander. Aber betrachten wir nun
die Art und Weise ihrer Vertauschung anhand eines Beispiels:

Es sei $n=7$ und $P=3524176$. Die 1 geht zum Platz der 5. Die 5 hat diesen
Platz verlassen und ist nun auf dem Platz der 2. Die 2 ist auf dem Platz der 3
und die 3 ist auf dem Platz der 1. So haben die Zahlen 1,5,2,3 zyklisch
getauscht (also einen Ringtausch gemacht). Auch 6,7 bilden einen Zyklus (in
Form einer einfachen Vertauschung dieser zwei Zahlen). 4 bildet alllein einen
Zyklus, diese Zahl ist stehen geblieben. Das nennt man auch Fixpunkt.

Auf diese Art und Weise lässt sich jede Permutaion in Zyklen zerlegen. Wie
sieht eine Permutation von $\cbr{1,2,3,\dots ,n}$ aus, bei der jede Zahl
allein in einem solchen Zyklus liegt? Finde ein Beispiel einer Permutation,
die insgesamt ein Zyklus ist (sich also nicht in noch kleinere zerlegen
lässt)!

Von besonderem Interesse sind die Fixpunkte, d.\,h.\ die Zahlen $k$, die bei
der Permutation an der $k$-ten Stelle stehen (also ihren Platz nicht ändern).
Dazu folgende Aufgabe, die man mit doppeltem Abzählen lösen kann.

\paragraph{Aufgabe 14 (1. Aufgabe IMO 1987)} 
Sei $S$ die Menge $\cbr{1,2,3,\dots ,n}$. Wir bezeichnen die Anzahl der
Permutationen von $S$, die genau $k$ Fixpunkte haben, mit $p_{n}(k)$. Zeige,
dass
\[\sum_{i=0}^{n} k\,p_{n}(k)=n!\]
gilt.  Tipp: Interpretiere die Summanden als Anzahlen.

\subsection*{Graphen}

Graphen sind ein gutes Mittel zur Veranschaulichung von Freundschafts- oder
Feinschaftsbeziehungen zwischen Personen. Grob gesagt besteht ein Graph aus
einer endlichen Anzahl von Ecken (auch Knoten gennnt) und Kanten als Linien
zwischen zwei Ecken. Beispielsweise könnten wir einen Graphen mit den gerade
anwesenden Personen als Ecken zeichnen, bei dem je zwei Ecken genau dann
verbunden sind, wenn sich die entsprechenden Personen schon vor dem
Landesseminar kannten.

Folgende Aufgabe lässt sich leicht mithilfe von Graphen und doppeltem Abzählen
lösen

\paragraph{Aufgabe 15} 
In einem Raum befinden sich gewisse Personen, von denen sich einige per
Handschlag begrüßt haben und einige nicht. Zeige, dass die Anzahl der
Personen, die ungerade vielen anderen die Hand gegeben hat, gerade ist.

\paragraph{Aufgabe 16} 
Auf einer Party sind $2\,n\geq 2$ Personen und jede Person hat eine gerade
Anzahl an Freunden. (Wenn A mit B befreundet ist, dann ist auch B mit A
befreundet; keiner ist mit sich selbst befreundet). Zeige, dass es zwei
Personen mit einer geraden Anzahl gemeinsamer Freunde auf der Party gibt.

In einem Graphen kann man auch Kanten färben, wie in dieser Aufgabe:

\paragraph{Aufgabe 17} 
6 Punkte in der Ebene sind paarweise durch Strecken verbunden. Jede dieser 15
Strecken ist rot oder blau gefärbt. Zeige, dass es ein einfarbiges Dreieck
gibt.

\paragraph{Aufgabe 18} 
17 Mathematiker korrespondieren paarweise miteinander über jeweils genau eines
der folgenden drei Medien: Post, Fax, Email. Zeige, dass es drei dieser 17
Mathematiker gibt, die paarweise mit dem gleichen Medium korrespondieren.

\subsection*{Invarianzprinzip}

Das Invarianzprinzip als wichtige Methode soll hier nicht fehlen. Das Finden
von Invarianten (Größen, die während eines Prozesses gleich bleiben) oder
Halbinvarianten (Größen, die sich bei einem Prozess stets verkleinern oder
stets vergrößern) ist bei manchen Aufgabe zur Lösung unumgänglich.

\paragraph{Aufgabe 19} 
In einem $4\times 4$-Quadrat steht auf 15 Feldern 1 und auf einem Feld $-1$.
In einem Schritt darf man in einer Zeile, einer Spalte oder einem $2\times
2$-Quadrat alle Vorzeichen ändern. Kann man dadurch erreichen, dass überall
Einsen stehen?

\paragraph{Aufgabe 20 (Klausur beim Landesseminar 2004 Klasse 8)} 
Auf einer abgelegenen Insel leben 50 braune, 57 grüne, 62 gelbe und 68 rote
Frösche. Immer, wenn sich drei Frösche unterschiedlicher Farbe begegnen,
verwandeln sie sich in zwei Exemplare der vierten Farbe. Irgendwann hat man
festgestellt, dass alle verbleibenden Frösche die gleiche Farbe haben. Welche
Farbe ist das?

\subsection*{Kombinatorische Geometrie}

In den letzten Jahren kamen in den Bundesrunden Klasse 8 erstaunlich viele
Aufgaben aus dem Bereich der kombinatorischen Geometrie vor, deshalb wollen
wir uns auch damit noch ein wenig beschäftigen.

\paragraph{Aufgabe 21} 
In der Ebene seien $n$ Geraden in allgemeiner Lage (d.\,h.\ keine zwei
parallel und keine drei schneiden sich in einem Punkt). Wie viele
Schnittpunkte gibt es? In wie viele Teilstücke wird die Ebene zerlegt?

\paragraph{Aufgabe 22 (420845)} 
Im Innern eines Quadrats seien genau 187 Punkte markiert. Es sollen Dreiecke
gezeichnet werden, die einander nicht überdecken und folgende Forderungen
erfüllen:
\begin{itemize}
\item[(1)] Eckpunkte eines Dreiecks sind entweder markierte Punkte oder
  Eckpunkte des Quadrats.
\item[(2)] Mindestens ein Eckpunkt eines Dreiecks muss einer der markierten
  Punkte sein.
\end{itemize}
Wie viele Dreiecke lassen sich unter diesen Voraussetzungen höchstens bilden?

\paragraph{Aufgabe 23 (430845)} 
In einem Land seien die gegenseitigen Entfernungen aller Städte verschieden
groß. Eines Tages startet in jeder Stadt ein Flugzeug und fliegt nach der
nächstgelegenen Stadt. Nach der Landung aller Flugzeuge stellt sich heraus,
dass in keiner Stadt mehr als fünf Flugzeuge gelandet sind. Zeige, dass das
kein Zufall ist.

\emph{Hinweis:} Es ist zu beweisen, dass unter den genannten Voraussetzungen
höchstens fünf Flugzeuge in dersellben Stadt landen können.

\paragraph{Aufgabe 24 (440842)} 
Wir betrachten in einer Ebene rote und blaue Punkte, von denen niemals mehr
als zwei Punkte auf einer Geraden liegen.
\begin{itemize}
\item[a)] Zeichne 4 rote und 2 blaue Punkte so, dass jedes aus 3 roten Punkten
  gebildete Dreieck, wir werden es rotes Dreieck nennen, genau einen der
  blauen Punkte in seinem Innern enthält.
\item[b)] Zeichne 4 rote und 4 blaue Punkte so, dass jedes rote Dreieck genau
  einen blauen Punkt und jedes blaue Dreieck genau einen roten Punkt in seinem
  Innern enthält.
\item[c)] Welche Bedingung müssen die 4 roten Punkte erfüllen, damit die in
  Teil a) und Teil b) genannten Forderungen erfüllbar sind?
\end{itemize}

\paragraph{Aufgabe 25 (470845)} 
Der Mathematiker Dr. Eieck veranstaltet eine Denkerparty. Dazu treib er in
jede Ecke seines dreieckigen Gartens einen Pflock und schlägt zusätzlich
insgesamt $n$ weitere Pflöcke am Rand oder im Innern der Rasenfläche ein.
Innerhalb des Rasens seien genau $k$ Pflöcke ($0\leq k\leq n$) eingesetzt, und
von ihnen liegen keine drei auf einer gemeinsamen Geraden. Nun befestigt er
möglichst viele, nicht unbedingt gleich lange Hängematten an den Pflöcken, die
einander nicht überschneiden dürfen. Auf diese Weise wird das Rasendreieck in
Teildreiecke zerlegt, in die er jeweils einen Stehtisch mit Papier,
Schreibzeug und Getränken stellt.
\begin{itemize}
\item[a)] Ermittle die Anzahl $s$ der Stehtische und die Anzahl $h$ der
  Hängematten für folgende konkrete Fälle:
  \begin{itemize}
  \item $n=3$ und $k=$0, 1, 2 oder 3 
  \item $n=4$ und $k=0$
  \end{itemize}
  \emph{Hinweis:} Dabei reicht es, jeweils einen solchen Fall zu betrachten,
  obwohl es verschiedene Möglichkeiten für die Platzierung der Pflöcke gibt.
\item[b)] Gib jeweils eine Formel für die Anzahl $s(n,k)$ von Stehtischen und
  die Anzahl $h(n,k)$ von Hängematten in Abhängigkeit von $n$ und $k$ an und
  berechne $s(33,22)$ und $h(33,22)$.
\item[c)] Beweise die Richtigkeit dieser beiden Formeln.
\end{itemize}

\subsection*{Weiteres Training}

Die beste Vorbereitung auf die Bundesrunde oder kommende Olympiaden ist meiner
Meinung nach das Training anhand von alten Aufgaben. Auf der Internetseite des
Mathematikolympiadevereins findet sich das Aufgabenarchiv mit den Aufgaben der
vergangenen zehn Jahre (www.mathematik-olympiaden.de).

Ein Klassiker zum Olympiadetraining ist „Problem Solving Strategies“ von
Arthur Engel, Springer-Verlag. Vor dem Englisch sollte man keine Angst haben,
da der Autor ein Deutscher ist, ist es sehr verständlich geschrieben.

Als gutes Buch voller Kombinatorikaufgaben kann ich „102 Combinatorial
Problems“ von Titu Andreescu, Zuming Feng (Birkhäuser) empfehlen, viele
Aufgaben dieses Vortrags entstammen diesem Buch (frei übersetzt).

\begin{attribution}
sauermann (März 2013): Für KoSemNet freigegeben.

graebe (2014-01-01): Nach den KoSemNet Regeln aufbereitet.
\end{attribution}
\end{document}
