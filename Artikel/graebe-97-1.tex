% Beitrag f"ur das Heft zum Landesseminar M"arz 1997

\documentclass[11pt]{article}
\usepackage{kosemnet,ko-math,ngerman,url}

\title{Einige wichtige Ungleichungen\kosemnetlicensemark}
\author{Hans-Gert Gr"abe, Leipzig\\[8pt]
\url{http://www.informatik.uni-leipzig.de/~graebe}}
\date{1.~Februar 1997}

\begin{document}
\maketitle

Ziel dieser kurzen Note ist es, einige wichtige Ungleichungen, die in
verschiedenen Olympiadeaufgaben immer wieder zur Anwendung kommen,
einschlie"slich ihrer Beweise zusammenzustellen.

\section{Die Ungleichung vom arithmetisch-geometrischen Mittel}

Gut bekannt d"urfte den meisten Sch"ulern mit einigem Olympiadetraining die
Ungleichung
\[\frac{a}{b}+\frac{b}{a}\geq 2\]
sein, die bekanntlich f"ur alle $a,b>0$ gilt und leicht aus der Beziehung
$(a-b)^{2}\geq 0$ hergeleitet werden kann. Diese Herleitung zeigt auch, dass
in obiger Ungleichung Gleichheit genau dann gilt, wenn $a=b$ ist.

Eine schon weniger triviale Ungleichung ist die folgende:
\begin{satz} F"ur alle $a,b,c>0$ gilt stets die Ungleichung
\[\frac{a}{b}+\frac{b}{c}+\frac{c}{a}\geq 3.\]
\end{satz}

\begin{aufgabe}
  Zeige die G"ultigkeit dieser Ungleichung, ohne auf die folgende allgemeine
  Theorie zur"uckzugreifen.
\end{aufgabe}

Diese beiden Ungleichungen sind Spezialf"alle der folgenden allgemeinen
Ungleichung
\begin{satz} Sind $x_{1},\ldots,x_{n}$ nichtnegative reelle Zahlen, deren
  Produkt gleich 1 ist, so gilt f"ur ihre Summe \[\sum_{i=1}^{n}{x_{i}} = x_{1}
  + x_{2} + \ldots + x_{n}\geq n.\]
Gleichheit gilt genau dann, wenn $x_{1}= \ldots = x_{n} =1$ ist.
\end{satz}

\begin{beweis} (nach \cite{Kor}) Wir f"uhren den Beweis mit vollst"andiger
Induktion. F"ur $n=2$ ist seine Aussage Gegenstand der oben betrachteten
Ungleichung. Sei nun die folgende Aussage richtig: F"ur beliebige $k$
nichtnegative reelle Zahlen $y_{1},\ldots,y_{k}$ gilt
\[y_{1}\cdot\ldots\cdot y_{k}=1\quad\Rightarrow\quad y_{1}+ \ldots + y_{k}\geq
k. \] 

Wir betrachten $k+1$ Zahlen $x_{1},\ldots,x_{k+1}$ (von denen wir ohne
Beschr"ankung der Allgemeinheit annehmen wollen, dass sie der Gr"o"se nach
geordnet sind), deren Produkt ebenfalls gleich 1 ist und wollen zeigen, dass
dann $x_{1}+\ldots +x_{k+1}\geq k+1$ gilt.

Fassen wir die Faktoren $x_{1}$ und $x_{k+1}$ zu einem gemeinsamen Faktor
$y_{1} = x_{1}x_{k+1}$ zusammen, so haben wir ein Produkt aus $k$ Faktoren und
nach Induktionsvoraussetzung gilt 
\begin{align*}
  k\leq &\ y_{1} + x_{2} + x_{3} + \ldots + x_{k} \\ &\ = x_{1}x_{k+1} + x_{2}
  + x_{3} + \ldots + x_{k} \\ &\ = x_{1}+\ldots +x_{k+1} + x_{1}x_{k+1} -x_{1}
  - x_{k+1}, 
\intertext{also (wegen $x_{1}x_{k+1} -x_{1} - x_{k+1} +1 =
(x_{1}-1)(x_{k+1}-1)$)} 
 k+1 \leq &\ x_{1}+\ldots +x_{k+1} + (x_{1}-1)(x_{k+1}-1)
\intertext{Da $x_{1}$ die kleinste und $x_{k+1}$ die gr"o"ste der $k+1$ Zahlen
ist, ihr Produkt aber gleich 1, so folgt, dass $x_{1}\leq 1\leq x_{k+1}$ sein
muss.  Damit ist der Korrektursummand in obiger Ungleichung nichtpositiv und
es gilt erst recht}
k+1 \leq &\ x_{1}+\ldots +x_{k+1}.
\end{align*}
Gleichheit erhalten wir h"ochstens, wenn $(x_{1}-1)(x_{k+1}-1)=0$, also
entweder $x_{1}=1$ oder $x_{k+1}=1$ gilt. Dann ergibt das Produkt der
restlichen $k$ Zahlen aber bereits 1 und der zweite Teil der Behauptung folgt
ebenfalls aus der Induktionsvoraussetzung.
\end{beweis}

Als Folgerung erhalten wir insbesondere die G"ultigkeit der Ungleichung
\[\frac{x_{1}}{x_{2}} + \frac{x_{2}}{x_{3}} + \ldots + \frac{x_{n-1}}{x_{n}} +
\frac{x_{n}}{x_{1}}\geq n\]
f"ur positive reelle Zahlen $x_{1},\ldots,x_{n}$.
\medskip

\begin{aufgabe} Beweise die Ungleichungen
\begin{align*}
\textbf{(a)}\ \frac{x^{2}+2}{\sqrt{x^{2}+1}}\geq 2\qquad
\textbf{(b)}\ \log_{a}(b) + \log_{b}(c) + \log_{c}(a) \geq 3\qquad
\textbf{(c)}\ \frac{x^{2}}{1+x^{4}}\leq \frac12
\end{align*}
\end{aufgabe}

Seien im folgenden $x_{1},\ldots,x_{n}$ stets nichtnegative reelle
Zahlen. Als ihr {\em arithmetisches Mittel} bezeichnet man die Zahl
\[A(x_{1},\ldots,x_{n}):=\frac{x_{1}+\ldots +x_{n}}{n},\]
als ihr {\em geometrisches Mittel} die Zahl
\[G(x_{1},\ldots,x_{n}):=\sqrt[n]{x_{1}\cdot\ldots\cdot x_{n}}.\]
Beide Zahlen liegen zwischen der kleinsten und der gr"o"sten der $n$ Zahlen
und fallen mit diesen zusammen, wenn alle $n$ Zahlen gleich sind.

Eine weitere Folgerung aus der oben bewiesenen Ungleichung ist der 
\begin{satz}[Ungleichung zwischen arithmetischem und geometrischem Mittel] 
\mbox{}\\[8pt] Es gilt stets
\[A(x_{1},\ldots,x_{n})\geq G(x_{1},\ldots,x_{n})\]
und Gleichheit tritt genau dann ein, wenn alle $n$ Zahlen "ubereinstimmen. 
\end{satz}

\begin{beweis} Wir betrachten die $n$ reellen Zahlen
  $\frac{{x_1}}{G}, \ldots, \frac{{x_n}}{G}$, deren Produkt ganz
  offensichtlich gleich 1 ist.  Die Aussage des Satzes ergibt sich dann durch
  ein einfaches Umstellen der Ungleichung
\[\frac{{x_1}}{G} + \ldots + \frac{{x_n}}{G}\geq n.\]
\end{beweis}

\begin{aufgabe} Zeige, dass unter allen Quadern mit gleicher Summe der
  Kantenl"angen der W"urfel das gr"o"ste Volumen hat.
\end{aufgabe}

\begin{aufgabe}$^{*}$ Zeige, dass unter allen Quadern mit gleicher Oberfl"ache
  der W"urfel das gr"o"ste Volumen hat.
\end{aufgabe}

\section{Andere Mittel zwischen poitiven reellen Zahlen}

Es gibt noch weitere Mittel der Zahlen $x_{1},\ldots,x_{n}$, die man
betrachten kann. Auch f"ur sie gilt die Aussage, dass sie zwischen der
kleinsten und der gr"o"sten der $n$ Zahlen liegen und mit diesen
zusammenfallen, wenn alle $n$ Zahlen gleich sind.

So bezeichnet man f"ur $\alpha\neq 0$ die Zahl
\[M_{\alpha}(x_{1},\ldots,x_{n}):=\left(\frac{x_{1}^{\alpha} + \ldots +
    x_{n}^{\alpha }}{n}\right)^\frac{1}{\alpha}\]
als {\em das Mittel vom Grad $\alpha$} der Zahlen $x_{1},\ldots,x_{n}$. F"ur
$\alpha=1 $ erhalten wir wieder das arithmetische Mittel. Das Mittel vom Grad
2 nennt man das {\em quadratische Mittel} der genannten Zahlen und das Mittel
vom Grad $\alpha=-1$ das harmonische Mittel. Beide spielen in Anwendungen eine
wichtige Rolle. 

Als Folgerung aus unserem Satz k"onnen wir die folgende Beziehung zwischen 
dem geometrischen Mittel $G(x_{1},\ldots,x_{n})$ und dem entsprechenden Mittel
vom Grad $\alpha$ herleiten:

\begin{satz} F"ur positive reelle Zahlen $x_{1},\ldots,x_{n}$ gilt
\begin{align*}
&M_{\alpha}(x_{1},\ldots,x_{n})\geq G(x_{1},\ldots,x_{n})\quad \text{f"ur
 $\alpha >0$}
\intertext{und }
&M_{\alpha}(x_{1},\ldots,x_{n})\leq G(x_{1},\ldots,x_{n})\quad \text{f"ur
 $\alpha <0$}.
\end{align*}
Gleichheit gilt in beiden Ungleichungen genau f"ur $x_{1}=\ldots = x_{n}$.
\end{satz}
\begin{beweis} Wende den Satz "uber das arithmetisch-geometrische Mittel auf
  $x_{1}^{\alpha},\ldots,x_{n}^{\alpha}$ an.
\end{beweis}

Die bewiesenen Ungleichungen legen es nahe, das geometrische Mittel als das
{\em Mittel vom Grad 0} zu bezeichnen. Eine weitere Folgerung aus diesen
beiden Ungleichungen ist in der folgenden Aufgabe zu beweisen:

\begin{aufgabe} Zeige, dass f"ur positive reelle Zahlen $x_{1},\ldots,x_{n}$
  stets 
\[\left(x_{1}+\ldots+x_{n}\right)
\left(\frac{1}{x_{1}}+\ldots+\frac{1}{x_{n}}\right)\geq n^{2}\] gilt. 
\end{aufgabe}

\section{Die Bernoullischen Ungleichungen}

EIne weitere wichtige Gruppe von Ungleichungen wollen wir nun herleiten.

\begin{satz} Ist $x\geq -1$ und $\alpha>0$ eine reelle Zahl, so gilt
\begin{align*}
&(1+x)^{\alpha} \leq 1+\alpha x\qquad \text{wenn $0<\alpha<1$} 
\intertext{und}
&(1+x)^{\alpha} \geq 1+\alpha x\qquad \text{wenn $\alpha>1$.}
\end{align*}
In beiden Ungleichungen gilt wieder Gleichheit genau dann, wenn $x=0$.
\end{satz}

\begin{beweis} F"ur diese Ungleichung wollen wir eine Beweistechnik verwenden,
  die oft in der Analysis angewendet wird und die wir hier nicht bis ins
  letzte Detail exakt begr"unden k"onnen. Wir beweisen die erste Ungleichung
  zuerst f"ur rationale Exponenten $\alpha=\frac{p}{q}$. Wegen $\alpha<1$ ist
  dann $p<q$. Es ist
\[(1+x)^\frac{p}{q}=\sqrt[q]{(1+x)^{p}\cdot 1^{q-p}},\]
wobei wir den zweiten Faktor hinzugef"ugt haben, um zu verdeutlichen, dass wir
die Ungleichung vom arithmetisch-geometrischen Mittel anwenden k"onnen, da
unter dem Wurzelzeichen genau $q$ Faktoren stehen. Wir erhalten
\[(1+x)^\frac{p}{q}\leq \frac{p\cdot (1+x)+(q-p)\cdot 1}{q} = 1+\frac{p}{q}x,\]
das hei"st, die zu beweisende Ungleichung einschlie"slich der Zusatzaussage
"uber die G"ultigkeit des Gleichheitszeichens.

Nun argumentiert man in der Analysis, dass man jede reelle Zahl durch
rationale Zahlen beliebig genau approximieren kann. Da sich andererseits alle
in der Ungleichung vorkommenden Ausdr"ucke stetig "andern, bleibt die
Ungleichung auch f"ur reelle Zahlen richtig.

Die zweite Ungleichung bekommt man nun aus der ersten, indem man $y=\alpha x$
setzt.
\end{beweis}

\begin{aufgabe} F"uhre diesen Teil des Beweises aus.
\end{aufgabe}

Diese Gruppe von Ungleichungen nennt man die {\em Bernoullischen
  Ungleichungen}. \medskip

Damit k"onnen wir die Ungleichung vom arithmetisch-geometrischen Mittel in der
folgenden Weise verallgemeinern:
\begin{satz} 
  F"ur positive reelle Zahlen $x_{1},\ldots,x_{n}$ gilt
\[M_{\alpha}(x_{1},\ldots,x_{n})\leq M_{\beta}(x_{1},\ldots,x_{n}),\]
wenn $\alpha < \beta$ (und Gleichheit gilt genau dann, wenn $x_{1}=\ldots =
x_{n}$). 
\end{satz}

\begin{beweis} Haben $\alpha$ und $\beta$ unterschiedliches Vorzeichen, so
  liegt zwischen ihnen das geometrische Mittel. Wir wollen also im Weiteren
  $\alpha, \beta >0$ annehmen und "uberlassen den Beweis im verbleibenden Fall
  $\alpha, \beta <0$ dem Leser. Da au"serdem mit $\gamma=\frac{\beta}{\alpha }$
\[M_{\beta}(x_{1},\ldots,x_{n}) =
M_{\alpha}(x_{1}^{\gamma},\ldots,x_{n}^{\gamma})^\frac{1}{\gamma}\] 
gilt, k"onnen wir statt der zu beweisenden Ungleichung auch
\[M_{1}(x_{1}^{\alpha},\ldots,x_{n}^{\alpha}) \leq
M_{\gamma}(x_{1}^{\alpha},\ldots,x_{n}^{\alpha})\] 
zeigen, d.h. uns auf den Beweis der Tatsache beschr"anken, dass
$M_{\gamma}\geq M_{1}$ ist f"ur $\gamma>1$, d.h. ($y_{i}:=x_{i}^{\alpha}$)
\[\left(\frac{y_{1}+\ldots +y_{n}}{n}\right)^{\gamma}\leq
\frac{y_{1}^{\gamma} +\ldots +y_{n}^{\gamma}}{n}\] gilt oder mit
$A=\dfrac{y_{1}+\ldots + y_{n}}{n}$ 
\[n\leq \left(\frac{{y_1}}{A}\right)^{\gamma} +\ldots
+\left(\frac{{y_n}}{A}\right)^{\gamma}.\] 
Setzen wir in der rechten Seite $z_{i}:=\frac{{y_i}}{A}-1$, so k"onnen wir die
eben bewiesene Benoullische Ungleichung anwenden:
\[\left(\frac{{y_1}}{A}\right)^{\gamma} +\ldots
+\left(\frac{{y_n}}{A}\right)^{\gamma} = (1+z_{1})^{\gamma} +\ldots
+(1+z_{n})^{\gamma} \geq (1+\gamma\,z_{1}) +\ldots + (1+\gamma\,z_{n}) =n,\]
weil ja $z_{1} +\ldots + z_{n} = 0$ ist. Damit haben wir die Aussage des
Satzes bewiesen.
\end{beweis}

\section{Gewichtete Mittel}

Eine noch weitergehende Verallgemeinerung stellt die Betrachtung gewichteter
Summen dar, in die die einzelnen Zahlen nicht wie bisher gleichberechtigt,
sondern mit unterschiedlichen Gewichten eingehen. Daf"ur betrachten wir
positive reelle Zahlen $w_{1},\ldots, w_{n}$, die {\em Gewichte}, und
definieren 
\[M_{\alpha}^{(\bf w)}:=\left( \frac{w_{1}\,x_{1}^{\alpha} + \ldots +
  w_{n}\,x_{n}^{\alpha}}{w_{1} + \ldots + w_{n}}\right)^\frac{1}{\alpha} \] 
als das {\em mit ({\bf w}) gewichtete Mittel vom Grad $\alpha$} der
positiven reellen Zahlen $x_{1},\ldots,x_{n}$. Sind die Gewichte
wieder s"amtlich rational, wobei wir oBdA einen gemeinsamen
Hauptnenneer voraussetzen k"onnen, so ist leicht einzusehen, dass alle
bisher bewiesenen Ungleichungen "uber Mittel auch in dieser
Verallgemeinerung richtig bleiben. F"ur reelle Zahlen als Gewichte
k"onnen wir denselben Grenz"ubergangsprozess heranziehen wie bereits
im Beweis der Bernoullischen Ungleichungen.

\begin{thebibliography}{xxx}
\bibitem{Kor} P.P.Korowkin: Ungleichungen. Dt. Verlag der Wissenschaften,
  Berlin 1970, Math. Sch"ulerb"ucherei, Band 9.
\end{thebibliography}


\begin{attribution}
graebe (2004-09-03): Contributed to KoSemNet\\
graebe (2005-02-02): Revision\\
graebe (2016-06-27): Small bug fix
\end{attribution}

\end{document}

