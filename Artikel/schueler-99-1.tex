\documentclass[11pt]{article}
\usepackage{kosemnet,ko-math,ngerman,url} 

\title{Das Skalarprodukt und seine Anwendungen\kosemnetlicensemark}  
\author{Axel Sch�ler, Mathematisches Institut, Univ. Leipzig\\[8pt]
\texttt{mailto:schueler@mathematik.uni-leipzig.de}}
\date{Schmalzgrube, M�rz 1999}


% emlines.sty  M�rz 1990, Georg Horn / Eberhard Mattes.
%
% Makros zum Zeichnen von Linien mit beliebiger Steigung.
% Nur bei Verwendung der DVI-Treiber von Eberhard Mattes.
%
% Der Makro \emline#1#2#3#4#5#6 setzt an die Koordinaten (#1,#2) den
% Punkt #3 und an die Koordinaten (#4,#5) den Punkt #6. Diese beiden 
% Punkte werden dann verbunden.
%
% TeXcad erzeugt Aufrufe dieses Makros f�r mit eingeschalteter
% EMLines-Option gezeichnete Linien.
% (z.B. \emline{0.0}{0.0}{1}{17.0}{4.0}{2}).
%
% Der Makro \newpic#1 definiert den Makro \emline so, dass den
% Punktnummern ##3 und ##6 die Ziffer #1 vorangestellt wird.
% Dies ist notwendig, wenn mehr als ein Bild auf einer Seite
% des Dokuments eingefuegt wird.
%
\def\newpic#1{%
   \def\emline##1##2##3##4##5##6{%
      \put(##1,##2){\special{em:point #1##3}}%
      \put(##4,##5){\special{em:point #1##6}}%
      \special{em:line #1##3,#1##6}}}
%
% Standarddefinition von \emline herstellen
%
\newpic{}
%
% Beispiel: 
% \input bild1.pic
% \newpic{1} \input bild2.pic
%
% Hier ist emlines.sty zu Ende

\newcommand{\ppp}{\bullet}
\newcommand{\rf}[1][]{\textup{\eqref{#1}}}
\newcommand{\bd}{\boldsymbol{\cdot}}
\newcommand{\ft}{{\tt t}}
\newcommand{\dvec}[3]{{(#1, #2, #3)^\ft}}
\newcommand{\evec}[2]{{(#1,#2)^\ft}}
\newcommand{\norm}[1]{{\|\vec #1 \|}}
\newcommand{\vvec}{\overrightarrow}
\newcommand{\nnorm}[1]{{\|\vvec #1 \|}}

\begin{document}
\maketitle

\subsection*{Das Skalarprodukt}
Das Skalarprodukt von Vektoren kann ein elegantes und n"utzliches Hilfsmittel
beim L"osen von geometrischen Aufgaben sein. In den folgenden Situationen
k"onnte die Benutzung des Skalarproduktes erfolgversprechend sein:
\begin{enumerate}
\item Nachweis von Identit"aten, in denen {\em Quadrate von Streckenl"angen}
auftauchen,
\item Nachweis der {\em Orthogonalit"at} von Vektoren,
\item Berechnung oder Vergleich von {\em Winkelgr"o"sen}.
\end{enumerate}
In Hinblick auf Formel \rf[skp1] macht das Skalarprodukt Aussagen "uber den
Winkel zwischen den beiden Vektoren und "uber deren L"ange. 
\\
Die {\em "Ubungen} sind gedacht zur Festigung des Umgangs mit dem
Skalarprodukt w"ahrend die {\em Aufgaben} einen L"osungsansatz bzw.{} eine
Idee erfordern, wie es in der Olympiade "ublich ist. 
%Mit den "Ubungen erlernt man das Handwerk.
Zu den Aufgaben mit $\ppp$ ist die L"osung selbst zu finden.

Das Skalarprodukt  ist eine Abbildung $\R^3\times\R^3\to \R$
und wird f"ur Vektoren $\vec{x}=
\left(\begin{matrix}x_1\\ x_2\\x_3\end{matrix}\right)=
x_1\vec{e_1}+x_2\vec{e_2}+x_3\vec{e_3}$ 
und $\vec{y}=\left(\begin{matrix}y_1\\ y_2
\\y_3\end{matrix}\right)=
y_1\vec{e_1}+y_2\vec{e_2}+y_3\vec{e_3}$ wie folgt definiert
\begin{align}\label{skp}
\boxed{
\vec{x}\bd\vec{y}=x_1y_1+x_2y_2+x_3y_3.}
\end{align}
Dabei sind $\vec{e_1}=\left(\begin{matrix}1\\ 0\\0\end{matrix}\right)
$, $\vec{e_2}=\left(\begin{matrix}0\\ 1\\0\end{matrix}\right)
$ und $\vec{e_3}=\left(\begin{matrix}0\\ 0\\1\end{matrix}\right)$ die Einheitsvektoren im 
$\R^3$.
\\
Vereinbarung: Vektoren werden bei uns prinzipiell als {\em Spalten}vektoren 
geschrieben. Da dies  viel Platz in Anspruch nimmt, benutzen 
wir die Transposition $^\ft$, also $\left(\begin{matrix}a\\b\\c 
\end{matrix}\right)=(a,b,c)^\ft$. Entsprechende Formeln mit nur zwei
Koordinaten gelten f"ur das Skalarprodukt von Vektoren im $\R^2$.


{\bf "Ubung 1$\ppp$} Berechnen Sie  die Skalarprodukte der folgenden Vektoren
\\
\begin{tabular}{llll}
a) $\dvec{1}{2}{3}\bd\dvec{1}{1}{-1}$ &
b) $\dvec{1}{1}{1}\bd\dvec{4}{5}{6}$&
c) $\dvec{1}{1}{1}\bd\dvec{1}{1}{1}$&
d) $\evec{2}{3}\bd\evec{1}{2}$
\end{tabular}

\subsection*{Eigenschaften des Skalarproduktes}
\begin{enumerate}
\item Bilinearit"at (Distributivgesetze)
\begin{align*}
(\vec{x}+\vec{y})\bd\vec{z}&=\vec{x}\bd\vec{z}+\vec{y}\bd\vec{z}\\
\vec{x}\bd(\vec{y}+\vec{z})&=\vec{x}\bd\vec{y}+\vec{x}\bd\vec{z}\\
(\alpha\vec{x})\bd\vec{y}&=\vec{x}\bd(\alpha\vec{y})=\alpha(\vec{x}\bd\vec{y})
\end{align*}
\item Symmetrie (Kommutativit"at)
$$
\vec{x}\bd\vec{y}=\vec{y}\bd\vec{x}.
$$
\item{Positivit"at}
\begin{align*}
\vec{x}\bd\vec{x}&\ge0,\\
\vec{x}\bd\vec{x}&=0\quad\text{ genau dann, wenn }\quad \vec{x}=\vec{0}.
\end{align*}
Bezeichnet $\norm{x}$ die {\em L"ange} (oder auch {\em Norm}) des Vektors 
$\vec{x}$, dann gilt $\norm{x}^2=\vec{x}\bd\vec{x}$.
\item Die Cauchy-Schwarzsche-Ungleichung
$$
|\vec{x}\bd\vec{y}|\le\norm{x}\norm{y}.
$$
\item  Es sei $\gamma$ der Winkel zwischen den (von Null verschiedenen) Vektoren  
$\vec{x}$ und $\vec{y}$. Dann gilt 
\begin{align}
\cos \gamma&=\frac{ \vec{x}\bd\vec{y} } {\norm{x}\norm{y}},
\\
\vec{x}\bd\vec{y}&=\cos\gamma\norm{x}\norm{y}.
\label{skp1}
\end{align}
\end{enumerate}

{\bf "Ubung 2$\ppp$} Berechnen Sie die L"angen der folgenden Vektoren

\begin{tabular}{lllll}
a) $\dvec{1}{2}{3}$& b) $\dvec{1}{1}{-1}$& c) $\dvec{1}{1}{1}$ &d) 
$\evec{1}{2}$
 &e) $\evec{2}{3}$.
\end{tabular}

Berechnen Sie  den Kosinus des von den Vektoren $\vec{x}$ und $\vec{y}$ 
eingeschlossenen Winkels. Entscheiden Sie, ob der Winkel kleiner, gleich oder
gr"o"ser als  $90^\circ$ ist!

a) $\vec{x}=\dvec{1}{2}{3}$, $\vec{y}=\dvec{1}{1}{-1}$, b) $\vec{x}=
\evec{1}{0}$, 
$\vec{y}=\evec{-\half}{\half\sqrt{3}}$, c) $\vec{x}=\evec{1}{2}$, 
$\vec{y}=\evec{2}{3}$.

{\bf Aufgabe 1.} Gegeben sei ein Einheitsw"urfel $ABCDEFGH$ sowie ein Punkt $P$ 
auf $\overline{AB}$ mit $|\overline{PB}|=p$, $0<p<1$, und ein Punkt $Q$ auf $\overline{DE}$, der 
von der Kante $\overline{AE}$ ebenfalls den Abstand $p$ hat. Ermittle alle Werte von 
$p$, f"ur die die Strecken $\overline{PQ}$ und $\overline{ED}$ aufeinander senkrecht 
stehen! Hinweis: Zwei Vektoren sind genau dann orthogonal, wenn ihr 
Skalarprodukt gleich Null ist.

{\em L"osung.}  Wir plazieren den W"urfel in einem rechtwinkligen 
Koordinatensystem, so dass $A=(0,0,0)$, $B=(1,0,0)$, $D=(0,1,0)$ und 
$E=(0,0,1)$ gilt. Nach Voraussetzung sind  dann $P=(1-p,0,0)$ und 
$Q=(0,p,1-p)$.\\
Jetzt berechnen wir das Skalarprodukt der Vektoren 
$\vvec{{PQ}}\bd\vvec{{DE}}=(Q-P)\bd(E-D)=(p-1,p,1-p)^\ft\bd(0,-1,1)^\ft=
-p+1-p=1-2p$. Die Vektoren sind demnach genau dann senkrecht, wenn $1-2p=0$ 
bzw.\ $p=\half$.


{\bf Aufgabe 2.} Gegeben sei das untenstehende Quadrat $ABCD$ mit den 
Seitenmittelpunkten $E$, $F$, $G$ und $H$. Verbindet man die Seitenmittelpunkte 
jeweils mit den gegen"uberliegenden Eckpunkten des Quadrates, so entsteht als 
symmetrische Schnittpunktefigur ein Achteck. Entscheiden Sie, ob dieses Achteck 
regelm"a"sig ist! (Hinweis: Ein $n$-Eck hei"st regelm"a"sig, wenn alle Seiten 
gleichlang und alle Innenwinkel gleichgro"s sind.)

\special{em:linewidth 0.4pt}
\unitlength 1mm
\linethickness{0.4pt}
\begin{picture}(54.00,53.67)
\emline{10.00}{10.00}{1}{50.00}{10.00}{2}
\emline{50.00}{10.00}{3}{50.00}{49.67}{4}
\emline{50.00}{49.67}{5}{10.33}{49.67}{6}
\emline{10.33}{49.67}{7}{10.33}{10.33}{8}
\emline{10.33}{10.33}{9}{50.33}{29.67}{10}
\emline{50.33}{29.67}{11}{10.33}{49.67}{12}
\emline{10.33}{49.67}{13}{30.00}{10.00}{14}
\emline{30.00}{10.00}{15}{50.00}{49.67}{16}
\emline{50.00}{49.67}{17}{10.33}{29.67}{18}
\emline{10.33}{29.67}{19}{50.00}{10.00}{20}
\emline{50.00}{10.00}{21}{30.00}{49.67}{22}
\emline{30.00}{49.67}{23}{10.33}{9.67}{24}
\emline{10.33}{9.67}{25}{10.33}{9.67}{26}
\emline{10.33}{9.67}{27}{10.33}{9.67}{28}
\emline{10.33}{9.67}{29}{10.33}{9.67}{30}
\emline{10.33}{9.67}{31}{10.33}{9.67}{32}
\put(7.33,5.67){\makebox(0,0)[cc]{$A$}}
\put(53.00,5.67){\makebox(0,0)[cc]{$B$}}
\put(54.00,53.33){\makebox(0,0)[cc]{$C$}}
\put(7.33,53.67){\makebox(0,0)[cc]{$D$}}
\put(30.00,5.67){\makebox(0,0)[cc]{$E$}}
\put(54.00,29.33){\makebox(0,0)[cc]{$F$}}
\put(29.67,53.67){\makebox(0,0)[cc]{$G$}}
\put(7.33,30.00){\makebox(0,0)[cc]{$H$}}
\end{picture}

{\em L"osung.}  Aus Symmetriegr"unden ist das entstandene 8-Eck gleichseitig. Wir 
pr"ufen die Gleichheit der Winkel "uber den Vergleich ihrer Kosinus. Da der 
Winkel zwischen Vektoren definitionsgem"a"s immer zwischen $0^\circ$ und 
$180^\circ$ liegt und die Kosinusfunktion in diesem Bereich eineindeutig ist,
stimmen die Winkel genau dann "uberein, wenn ihre Kosinus "ubereinstimmen. Wir 
bezeichnen $\angle(DE,AG)=\alpha$ und $\angle(HC,GA)=\beta$. Ferner seien
$A=(0,0)$, $B=(1,0)$ und $D=(0,1)$. Dann gilt
\\
$\cos\alpha=\vvec{{DE}}\bd\vvec{{AG}}\nnorm{{DE}}^{-1}\nnorm{{AG}}^{-1}=
(E-D)\bd(G-A)(\frac{
5}{4})^{-1}=(\half,-1)^\ft\bd(\half,1)^\ft\frac{4}{5}=
-\frac{3}{4}{\cdot}\frac{4}{5}=-\frac{3}{5}.$ Weiter hat man
\\
$\cos\beta=\vvec{{HC}}\bd\vvec{{GA}}\frac{4}{5}=
(C-H)^\ft\bd (A-G)^\ft\frac{4}{5}=
(1,\half)^\ft\bd(-\half,-1)^\ft\frac{4}{5}=-\frac{4}{5}$. 
Folglich ist das 8-Eck {\em nicht} regelm"a"sig.

{\bf Aufgabe 3$\ppp$} Gegeben sei ein Rechteck $ABCD$ und ein Punkt $P$. Beweisen Sie, 
dass gilt
$$\overline{PA}^2+\overline{PC}^2=\overline{PB}^2+\overline{PD}^2.
$$
(Verwenden Sie, dass $\overline{XY}^2=\vvec{{XY}}\bd\vvec{XY}$ und
$\vvec{{AB}}\bd\vvec{{BC}}=0$.)

{\bf Aufgabe 4.} "Uber der Seite $\overline{AB}$ eines Quadrates $ABCD$ wird nach 
innen ein gleichseitiges Dreieck $ABE$ errichtet. Ermittlen Sie die Gr"o"se des 
Winkels $\angle EDC$! 

{\em L"osung.} Das Koordinatensystem liege so wie in Aufgabe 2. Dann gilt 
$E=(\half,\half\sqrt{3})$. Dann gilt 
$\vvec{{DE}}=E-D=(\half,\half\sqrt{3}-1)^\ft$, $\vvec{{DC}}=(1,0)^\ft$
und weiter
\begin{align*}
\cos\angle EDC&=
\frac{
      \vvec{{DE}}\bd\vvec{{DC}}
      }  
      { 
          \nnorm{{DE}} \nnorm{{DC}}
      }
=\frac{\half}{ 
               \sqrt{
                      \frac{1}{4}+\frac{3}{4}-\sqrt{3}+1
                    }
              }
\\
&=\frac{1}
       {
           2\sqrt{  2-\sqrt{3}  }
       }
 =\half\sqrt{ 2+\sqrt{3}      }.              
\end{align*}

Der Taschenrechner liefert die Vermutung $\angle EDC=15^\circ$. Zum Beweis 
benutzt man $\cos 30^\circ=\half\sqrt{3}$ sowie den Doppelwinkelsatz 
$\cos(2\alpha)=2\cos^2\alpha-1$. Also $\cos 
15^\circ=\sqrt{\frac{1+\half\sqrt{3}}{2}}=\half\sqrt{2+\sqrt{3}}$.

{\bf Aufgabe 5$\ppp$} Gegeben sei ein Rhombus $ABCD$ mit $\angle BAC=60^\circ$ und 
$k$ sei der Inkreis von $ABCD$ mit dem Mittelpunkt $M$. Ferner gelte 
$\overline{MD}=1$. Man zeige, dass f"ur jeden Punkt $P$ auf $k$ gilt
$$
\overline{PA}^2+\overline{PB}^2+\overline{PC}^2+\overline{PD}^2=11.
$$

{\bf "Ubung 3.} Ein Vektor hei"st {\em Einheitsvektor} oder {\em normiert}, wenn er die L"ange 
$1$ hat. Berechnen Sie die Koordinaten desjenigen Einheitsvektors $\vec{a}$, der 
senkrecht auf den Vektoren $\vec{b}=(1,1,0)^\ft$ und $\vec{c}=(0,1,1)^\ft$ 
steht!

{\em L"osung.} 
Die Koordinaten des gesuchten Vektors seien 
$\vec{a}=(a_1,a_2,a_3)^\ft$. Die Orthogonalit"at von $\vec{a},\vec{b}$ bzw.\ 
$\vec{a},\vec{c}$ sowie die Normiertheit von $\vec{a}$ (L"ange gleich 1) lassen 
sich unter Verwendung des Skalarprodukts wie folgt schreiben
$$
\vec{a}\bd\vec{b}=0,\quad \vec{a}\bd\vec{c}=0,\quad\vec{a}\bd\vec{a}=1.
$$
Nach Einsetzen der Koordinaten hat  man
$$
a_1+a_2=0,\quad a_2+a_3=0,\quad a_1^2+a_2^2+a_3^2=1.
$$
Die ersten beide Gleichungen liefern $a_2=-a_3=-a_1$. Einsetzen in die letzte 
Gleichung liefert {\em zwei} L"osungen, 
$\vec{a}=(\frac{1}{\sqrt{3}},-\frac{1}{\sqrt{3}},\frac{1}{\sqrt{3}})$ 
sowie $\vec{a}=(-\frac{1}{\sqrt{3}},\frac{1}{\sqrt{3}},-\frac{1}{\sqrt{3}})$.
Bemerkung: Auf einer Ebene des Raumes gibt es immer zwei senkrechte Vektoren 
der L"ange Eins, die entgegengesetzt gerichtet sind.

{\bf "Ubungen$\ppp$}
\\
4. Die Vektoren $\vvec{AB}=(3,-2,2)^\ft$ und $\vvec{BC}=(-1,0,-2)^\ft$ sind die 
benachbarten Seiten eines Parallelogramms $ABCD$. Bestimmen Sie  den Winkel 
zwischen den Diagonalen!


5. F"ur welchen Wert $z$ stehen die Vektoren $\vec{a}=(6,0,12)^\ft$ und 
$\vec{b}=(-8,13,z)^\ft$ senkrecht aufeinander?

6. Vom Parallelogramm $ABCD$ sind die folgenden Koordinaten bekannt: 
$A=(3,2,1)$, $B=(0,-1,-1)$ und $C=(-1,1,0)$. Ermitteln Sie die L"ange der 
Diagonalen $\overline{BD}$!

%7. Beweisen Sie, dass die Punkte $A=(1,-1,1)$, $B=(1,3,1)$, $C=(4,3,1)$ und 
%$D=(4,-1,1)$ die Eckpunkte eines Rechtecks bilden.
%
%8. Gegeben seien die Punkte $B=(1,-3)$ und $D=(0,4)$ eines Rhombus' $ABCD$. 
%Berechnen Sie die Koordinaten der Eckpunkte $A$ und $C$, wenn $\angle 
%BAD=60^\circ$!


{\bf Aufgabe 6.}  Gegeben seien eine nat"urliche Zahl $n\ge3$ und ein 
regelm"a"siges $n$-Eck $P_1P_2\cdots P_n$ mit Umkreis $k$ 
vom Radius $r$. Zeigen Sie, dass f"ur alle Punkte $P$ auf $k$ die Summe
$$                                       
\overline{PP_1}^2+\overline{PP_2}^2+\cdots +\overline{PP_n}^2
$$
einen konstanten Wert hat, der nur von $r$ und $n$ nicht aber von der Lage des
Punktes $P$ auf $k$ abh"angt, und ermittlen Sie diesen Wert!

{\em L"osung.}  Der Mittelpunkt des regul"aren $n$-Ecks 
$P_1P_2\cdots P_n$ sei $O$.
Wir schreiben $\vec{p_i}=\vvec{OP_i}$, $i=1,\dots,n$ und $\vec{p}=\vvec{OP}$.
Dann gilt wegen der  Regelm"a"sigkeit des $n$-Ecks 
\begin{align}\label{sum}
\vec{p_1}+\vec{p_2}+\cdots+\vec{p_n}=\vec{0}.
\end{align}
 Ferner ist $\vvec{PP_i}=\vec{p_i}-\vec{p}$. Wegen 
$\overline{AB}^2=\vvec{AB}\bd\vvec{AB}$ hat die  gesuchte Summe hat nun den Wert
\begin{align*}
s=&(\vec{p_1}-\vec{p})\bd(\vec{p_1}-\vec{p}) 
+(\vec{p_2}-\vec{p})\bd(\vec{p_2}-\vec{p}) +\cdots 
+(\vec{p_n}-\vec{p})\bd(\vec{p_n}-\vec{p})
\\
=&(\vec{p_1}^2-2\vec{p_1}\bd\vec{p}+\vec{p}\,^2)+
\cdots+(\vec{p_n}^2-2\vec{p_n}\bd\vec{p}+\vec{p}\,^2) 
\\
=&(\vec{p_1}^2+\cdots+\vec{p_n}^2)-2(\vec{p_1}+\cdots+\vec{p_n})\bd \vec{p}+n\vec{p}\,^2.
\end{align*}
Nun verwenden wir in der ersten Klammer, dass $\vec{p_i}^2=r^2=\vec{p}\,^2$ gilt, und in der 
zweiten Klammer (\ref{sum}). Damit ergibt sich $s=nr^2+nr^2=2nr^2$. Aus der 
letzten Formel erkennt man die Unabh"angigkeit der Summe $s$ von der Lage des 
Punktes $P$ auf dem Kreis $k$. 

{\bf Aufgabe 7$\ppp$} Gegeben sei ein Dreieck $ABC$ mit $\overline{BC}^2+\overline{CA}^2=5\overline{AB}^2$. Zeigen 
Sie, dass die Seitenhalbierenden durch $A$ und $B$ aufeinander senkrecht 
stehen!

{\bf Aufgabe 8$\ppp$} Gegeben seien zwei kongruente Kreise $k_1$, $k_2$ mit den Mittelpunkten 
$M_1$ und $M_2$ vom Radius $1$, wobei $M_2$ auf $k_1$ und $M_1$ auf $k_2$ 
liegen. Ferner seien 
$A$ ein Punkt auf $k_1$ und $B$ und $C$ liegen auf $k_2$ bez"uglich $M_1M_2$
symmetrisch zueinander.
\\
Zeigen Sie, dass $\overline{AB}^2+\overline{AC}^2\ge 2$!

{\bf Aufgabe 9$\ppp$} Die Seitenhalbierenden eines Dreiecks $ABC$ schneiden sich im Punkte $M$. 
Beweisen Sie, dass 
$$
\overline{AB}^2+\overline{BC}^2+\overline{AC}^2=3(\overline{MA}^2+\overline{MB}^2+\overline{MC}^2).
$$

{\bf Aufgabe 10.} Es sei $O$ der Mittelpunkt des Umkreises des Dreiecks $ABC$, $D$ der 
Mittelpunkt der Seite $\overline{AB}$ und $E$ der Schnittpunkt der Seitenhalbierenden  
des Dreiecks $ACD$. Beweisen Sie, dass $OE\perp CD$ genau dann, wenn $\overline{AB}=\overline{AC}$.

{\em L"osung.} 
Es sei $O$ der Koordinatenursprung; $\vec{a}=\vvec{OA}$, 
$\vec{b}=\vvec{OB}$, $\vec{c}=\vvec{OC}$, $\vec{d}=\vvec{OD}$
und $\vec{e}=\vvec{OE}$. Ohne Beschr"ankung der Allgemeinheit sei der Radius des Umkreises gleich $1$. Dann 
gilt also $\vec{a}\,^2=\vec{b}\,^2=\vec{c}\,^2=1$. Ferner haben wir
$\vec{d}=\half(\vec{a}+\vec{b})$. Als 
Schwerpunkt des Dreiecks $ADC$ hat $E$ den Ortsvektor 
$\vec{e}=\frac{1}{3}(\vec{a}+\vec{c}+\vec{d})=\frac{1}{3}\vec{a}+
\frac{1}{3}\vec{c}+\frac{1}{6}(\vec{a}+\vec{b})=\half\vec{a}+\frac{1}{3}\vec{c}
+\frac{1}{6}\vec{b}$.
Nun gilt $OE\perp CD$ genau dann, wenn $\vec{e}\bd\vvec{CD}=0$. 
Das hei"st,
\begin{align*}
0&=\vec{e}(\vec{d}-\vec{c})
\\
&=(\half\vec{a}+\frac{1}{3}\vec{c}+\frac{1}{6}\vec{b}\,)\bd(\half\vec{a}+\half\vec{b}
-\vec{c}\,)
\\
&=(\frac{1}{4}\vec{a}\,^2+\frac{1}{4}\vec{a}\vec{b}-
\half\vec{a}\vec{c})+(\frac{1}{6}
\vec{a}\vec{c}+\frac{1}{6}\vec{b}\vec{c}-\frac{1}{3}\vec{c}\,^2)+
(\frac{1}{12}\vec{a}
\vec{b}+\frac{1}{12}\vec{b}\,^2-\frac{1}{6}\vec{b}\vec{c})
\\
&=\frac{1}{4}-\frac{1}{3}+\frac{1}{12}+\frac{1}{3}\vec{a}\vec{b}
-\frac{1}{3}\vec{a}\vec{c}
\\
&=\frac{1}{3}\vec{a}(\vec{b}-\vec{c}).
\end{align*}
Andererseits gilt $\overline{AB}=\overline{AC}$ genau dann, wenn 
$(\vec{b}-\vec{a})^2=(\vec{c}-\vec{a})^2$. Da die Vektoren $\vec{a}$, $\vec{b}$ 
und $\vec{c}$ die L"ange $1$ haben, ist die letzte Gleichung "aquivalent zu
$-2\vec{a}\bd\vec{b}=-2\vec{c}\bd\vec{a}$ bzw.\ zu $\vec{a}(\vec{b}-\vec{c}\,)=0$.
Damit ist der Beweis erbracht.

{\bf Aufgabe 11$\ppp$} Die H"ohen des  spitzwinkligen Dreiecks $ABC$ schneiden sich im 
Punkte $H$. Auf den Strecken $\overline{HB}$ und $\overline{HC}$ sind die Punkte $B_1$ und 
$C_1$ derart gew"ahlt, dass 
$$
\angle AB_1C=\angle AC_1B=90^\circ.
$$
Beweisen Sie, dass $\overline{AB_1}=\overline{AC_1}$!

\begin{comment}
  todo: geometric markup
\end{comment}

\begin{attribution}
schueler (2004-09-09): Contributed to KoSemNet

graebe (2004-09-09): Prepared along the KoSemNet rules
\end{attribution}

\end{document}

