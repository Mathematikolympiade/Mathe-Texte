\documentclass[11pt]{article}
\usepackage{ngerman,schueler,url}
\usepackage{kosemnet,ko-math}

\title{Zerlegung von nat�rlichen Zahlen in die Summe von
  Quadratzahlen\kosemnetlicensemark}  
\author{Axel Sch�ler, Mathematisches Institut, Univ. Leipzig\\[8pt]
\url{mailto:schueler@mathematik.uni-leipzig.de}}
\date{Februar 2001}

\begin{document}
\maketitle

Die Zerlegung von nat"urlichen Zahlen in die Summe von Quadratzahlen ist eine
alte, abgeschlossene Theorie, die schon von {\sc Fermat} im 17. Jahrhundert
und sp"ater von {\sc Euler}, {\sc Lagrange} und {\sc Jacobi} bearbeitet wurde;
die wichtigsten Resultate gehen auf die oben genannten zur"uck.

\subsection*{Die pythagor"aischen Tripel}
Wir suchen L"osungen der Gleichung
\begin{align}\label{p}
x^2+y^2=z^2,\quad\text{mit}\quad x,y,z\in \N.
\end{align}
Diese Tripel $(x,y,z)$ nat"urlicher Zahlen hei"sen {\em pythagor"aische
  Zahlentripel}. Sicherlich kennt jeder das Tripel $(3,4,5)$, es gilt die
  Gleichung $3^2+4^2=9+16=25=5^2$. Vielleicht kennen ja auch einige das Tripel
  $(5, 12, 13)$ --- es ist $5^2+12^2=25+144=169=13^2$. Es stellt sich nun die
  Frage nach {\em allen} L"osungen der obigen Gleichung. Als erstes beobachtet
  man, dass mit $(x,y,z)$ auch $(tx, ty, tz)$, $t\in \N$ eine L"osung
  von \rf[p] ist. Haben umgekehrt zwei der Zahlen $x,y,z$ einen gemeinsamen
  Teiler $t$, so ist auch die dritte Zahl durch $t$ teilbar und man kann
  \rf[p] duch $t^2$ dividieren und erh"alt eine kleinere L"osung. Das kann man
  so lange machen, bis $\ggT(x,y,z)=1$. Solche L"osungen hei"sen {\em
  primitiv}. Als zweites stellt man fest, dass in einer primitiven L"osung
$  (x,y,z)$ immer genau eine der Zahlen $x$, $y$ gerade ist, die andere
  und $z$ sind stets ungerade. W"aren n"amlich  $x$ und $y$ gerade, dann hat
  man keine primitive L"osung mehr, da man das Tripel durch $2$ teilen kann, w"aren beide ungerade, etwa $x=2a+1$,
  $y=2b+1$, dann w"are $z^2=4(a(a+1)+b(b+1))+2$. Eine Quadratzahl kann aber
  nicht den Rest $2$ bei der Division durch $4$ lassen.

Gibt es endlich oder unendlich viele primitive pythagor"aische Tripel? Eine
einfache Beobachtung der Folge der Quadratzahlen und ihrer Differenzenfolge
beantwortet die Frage sofort.
\begin{center}
\begin{tabular}{l|llllllll}
$n$&1&2&3&4&5&6&7&8
\\ \hline
$n^2$&1&4&9&16&25&36&49&64
\\ \hline
Differenz&&3&5&7&9&11&13&15
\end{tabular}
\end{center}
Die Differenzenfolge der  aufeinanderfolgenden Quadratzahlen durchl"auft also
alle ungeraden nat"urlichen Zahlen. Jede ungerade Quadratzahl $(2n+1)^2$ in der
Differenzenfolge liefert also ein  pythagor"aisches Tripel $(2n(n+1), 2n+1,
2n(n+1)+1)$. Dies ist primitiv, da $x$ und $z$ aufeinanderfolgend und damit
teilerfremd sind. Unsere obigen beiden Beispiel sind genau von dieser
Gestalt. Gibt es aber noch andere primitive Tripel? Mit etwas Geschick oder mit dem
Computer findet man  das primitive Tripel $(8,15,17)$, das offenbar keine
aufeinanderfolgenden Quadratzahlen enth"alt. Nun ist es an der Zeit eine
vollst"andige L"osung anzugeben.

\begin{satz}\label{s-p}
F"ur beliebige paarweise teilerfremde nat"urliche Zahlen $m$ und $n$ mit
$m>n$, wobei eine von beiden gerade und die andere ungerade ist, liefern die
Formeln
\begin{align}
\begin{split}
\label{los-p}
x&=2mn,
\\
y&=m^2-n^2,
\\
z&=m^2+n^2
\end{split}
\end{align}
eine primitive L"osung von \rf[p]. Umgekehrt ist jede primitive L"osung von
\rf[p] mit geradem $x$ von dieser Gestalt.
\end{satz}

\begin{beweis} (nach \cite[\S 2]{b-Postnikov}) Die leicht nachzurechnende Identit"at
$$ (2mn)^2 +(m^2-n^2)^2=(m^2+n^2)^2 $$
beweist, dass \rf[los-p] ein pythagor"aisches Tripel von  positiven
ganzen Zahlen $(x,y,z)$ liefert. H"atten die drei Zahlen einen gemeinsamen
Teiler $t\ge 2$, so h"atten auch
\begin{align*}
2m^2&=y+z=(m^2-n^2)+(m^2+n^2)\quad \text{und}\quad 2n^2=z-y=(m^2+n^2)-(m^2-n^2)
\end{align*}
diesen gemeinsamen Teiler. Da aber $m$ und $n$ teilerfremd sind, muss $t=2$
gelten. Dann ist aber $y=m^2-n^2$ gerade. Also sind $m$ und $n$ beide gerade oder
beide ungerade, was unserer Voraussetzung widerspricht. Es gibt daher keinen
echten gemeinsamen Teiler; die angegebene L"osung ist primitiv.
\\
Sei umgekehrt $(x,y,z)$ eine primitive L"osung von \rf[p] mit positiven ganzen
Zahlen. O.\,B.\,d.\,A. sei $x=2a$ die gerade Zahl und $y$ und $z$
ungerade. Dann sind $z+y=2b$ und $z-y=2c$ gerade Zahlen mit
$a,b,c\in\N$. Jeder gemeinsame Teiler $d$ von $b$ und $c$ ist auch gemeinsamer
Teiler von $y$ und $z$, also sind $b$ und $c$ teilerfremd. Andererseits ist
nach \rf[p]
$$
4a^2=x^2=z^2-y^2=(z+y)(z-y)=4bc\quad\text{also}\quad a^2=bc.
$$
Wegen der eindeutigen Primfaktorzerlegung der nat"urlichen Zahlen und wegen
der Teilerfremdheit von $b$ und $c$, m"ussen $b$ und $c$ bereits f"ur
sich Quadratzahl sein, also $b=m^2$ und $c=n^2$. Dann ergibt sich $a^2=m^2n^2$
und somit $a=mn$. Wir erhalten insgesamt $x=2mn$, $z=b+c=m^2+n^2$ und
$y=b-c=m^2-n^2$. Wegen $b>c$ gilt auch $m>n$.
\end{beweis}

\begin{aufgabe} Man zeige, dass jedes primitive pythagor"aische Tripel $(m,n)$ 
mit $m>n$ genau 3 primitive Tripel als Nachfolger hat und, falls es von
$(1,0,0)$ veschieden ist, genau einen Vorg"anger. Keine zwei Nachfolger fallen
dabei zusammen.
\end{aufgabe}

\begin{loesung} Es sei $M=\{(m,n)\in\N\times\N\,|\,m>n, \ggT(m,n)=1, m\not\equiv n\mod 2\}$
die Menge der Paare, die die primitiven pythagor"aischen Tripel
parametrisiert. Man "uberzeugt sich leicht davon, dass die drei Funktionen
$f(m,n):=(2n+m,n)$, $g(m,n):=(2m+n,m)$ und $h(m,n):=(2m-n,m)$ die Menge $M$ in
sich selbst abbilden, denn der $\ggT{}{}$ bleibt erhalten, die verschiedene
Parit"at und die erste Komponente ist noch immer gr"o"ser als die zweite.
Zur Eindeutigkeit des Nachfolgers. Es sei $(m',n')$ der Nachfolger von $(m,n)$
unter $f,\,g$ bzw.{} $h$. F"ur die Abbildung $f$ gilt $m'>3n'$, f"ur $g$ gilt
$3n'>m'>2n'$ und f"ur $h$ gilt schlie"slich $2n'>m'$. Die
Gr"o"senverh"altnisse von $m'$ und $n'$ bestimmen also den Vorg"anger
eindeutig. Man "uberzeugt sich auch davon, dass diese R"uckabbildung stets
m"oglich ist.

Im Falle $a>3b$ w"ahle man $\tilde{a}:=a-2b$ und $\tilde{b}:=b$; im Falle
$3b>a>2b$ w"ahle man $\tilde{a}:=b$ und $\tilde{b}:=a-2b$; im Falle $2b>a$
w"ahle man $\tilde{a}:=b$ und $\tilde{b}:=2b-a$.

\end{loesung}


\subsection*{Die Methode des unendlichen Abstiegs und die L"osungen der
  Gleichung $x^4+y^4=z^4$}

Schon {\sc Fermat} war 1637 in der Lage zu zeigen, dass diese Gleichung
keine L"osung besitzt. Er benutzte dabei die {\em Methode des unendlichen
  Abstiegs}, die {\em Descendenzmethode}, die in der Olympiademathematik auch
einfach unter dem Namen {\em Extremalprinzip} bekannt ist. Ein weiterer Name ist  --- {\em die Suche nach dem kleinsten Verbrecher}. Diese Prinzip beruht ganz
einfach auf der Tatsache, dass jede Teilmenge nat"urlicher Zahlen ein 
kleinstes Element besitzt. In folgender Gestalt wird es meist verwendet:
\begin{quote}
Wir wollen zeigen, dass eine Aufgabe keine L"osung besitzt. Dazu nehmen wir an,
sie h"atte eine. Wir  w"ahlen uns unter allen hypothetischen  L"osungen eine
minimale (bez"uglich einer geeigneten Ordnung). Wenn es uns dann gelingt, eine
kleinere L"osung zu konstruieren, dann sind wir offensichtlich an einem
Widerspruch angelangt. Die Aufgabe hat keine L"osung.
\end{quote}

Als Beispiel geben wir einen recht ungew"ohnlichen Beweis f"ur die Tatsache,
dass $\sqrt{2}$ irrational ist, siehe \cite[Kapitel~3]{b-Engel}. Dazu sei $M=\{n\sqrt{2}\,|\, n\in\N\}\cap\N$
die Menge der Vielfachen von $\sqrt{2}$, die nat"urliche Zahlen liefern. Wir
m"ussen zeigen, dass $M$ leer ist. Ist $M$ nichtleer, so gibt es eine {\em
  kleinste } nat"urliche Zahl $k\in M$. Wir betrachten die Zahl
$n=(\sqrt{2}-1)k$. Dann gilt
$$
n\sqrt{2}=(\sqrt{2}-1)k \sqrt{2}=2k-k\sqrt{2}.
$$
Nach Definition der Menge $M$ und wegen $k\in M$ sind  sowohl $(\sqrt{2}-1)k$
als  auch $2k-k\sqrt{2}$ nat"urliche Zahlen. Folglich, wieder nach Definition
von $M$, ist auch $(\sqrt{2}-1)k\in M$. Nun ist aber $(\sqrt{2}-1)k<k$, was der
Minimalit"at von $k$ widerspricht. Also ist $M$ leer und $\sqrt{2}$
irrational. 

\begin{satz}  Die Gleichung
\begin{align}\label{4}
x^4+y^4=z^2
\end{align}
hat keine von Null verschiedene ganzzahlige L"osung.
\end{satz}

\begin{beweis}[(nach \cite{b-Postnikov})] Angenommen, es gibt L"osungen der
  obigen Gleichung. Dann  k"onnen wir wieder unsere Betrachtungen auf
  teilerfremde (primitive) Tripel von nat"urlichen Zahlen beschr"anken. Unter
  allen primitiven Tripeln w"ahlen wir dasjenige $(x,y,z)$, wo $z$ am
  kleinsten (aber von Null verschieden) ist. Wie schon bei der L"osung von
  \rf[p] schlie"st man, dass genau eine der Zahlen $x$  und $y$ gerade, die
  andere ungerade sein muss. O.\,B.\,d.\,A.{} sei $x$ gerade. Wegen
  $(x^2)^2+(y^2)^2=z^2$ bilden $(x^2,y^2,z)$ ein primitives pythagor"aisches
  Tripel. Also gibt es teilerfremde nat"urliche Zahlen $m,n$, $n>m$, verschiedener
  Parit"at (gerade/ungerade) mit
$$ x^2=2mn,\quad y^2=m^2- n^2,\quad z=m^2+n^2.
$$
Falls $m$ gerade und $n$ ungerade, dann ist $y^2\equiv -1\mod 4$, was nicht
m"oglich ist. Also ist $m$ ungerade und $n=2q$ gerade.
Also gilt $x^2=4qm$ bzw.{} $(x/2)^2=qm$. Wegen der Teilerfremdheit von $m$ und
$n$ sind auch $q$ und $m$ teilerfremd, also m"ussen sie bereits beide einzeln
Quadratzahlen sein, etwa
$$
m=s^2,\quad q=t^2,
$$
wobei $s$ und $t$ gewisse teilerfremde positive ganze Zahlen sind. Wir
sehen, dass
$$
y^2=(s^2)^2-(2t^2)^2\quad \text{bzw.}\quad (2t^2)^2+y^2=(s^2)^2
$$
gilt. Da $t$ und $s$ teilerfremd sind, k"onnen wir erneut Satz\,\ref{s-p}
anwenden und folglich gibt es teilerfremde nat"urliche Zahlen $a$ und $b$
verschiedener Parit"at mit $a>b$ und 
$$
2t^2=2ab,\quad y=a^2-b^2,\quad s^2=a^2+b^2.
$$
 Da $a$ und $b$ teilerfremd sind, folgt aus der ersten Gleichung wieder
 $a=u^2$ und $b=v^2$ und somit aus der letzten Gleichung
$$ s^2=u^4+v^4,$$
wobei $u$, $v$ und $s$ paarweise teilerfremde nat"urliche Zahlen sind.
Dabei ist
$$z=m^2+n^2>m=s^2>s.$$ Dies widerspricht aber der minimalen Wahl von
$z$. Somit hat Gleichung \rf[4] keine L"osung.
\end{beweis}

Mit Hilfe des Extremalprinzips l"ose man die folgende Aufgaben

\begin{aufgabe} Es gibt kein Quadrupel $(x,y,z,u)$ von nat"urlichen Zahlen mit 
$$x^2+y^2=3(z^2+u^2).$$
\end{aufgabe}

\begin{aufgabe} Finde alle Paare $(x,y)$ von nat"urlichen Zahlen mit
  $2x^2-3y^2=0$!
\end{aufgabe}

\begin{aufgabe} Zeige, dass die Gleichung $x^2+y^2+z^2=2xyz$ keine ganzzahlige
  L"osung bis auf $x=y=z=0$ besitzt.
\end{aufgabe}

\begin{aufgabe} 
  Die Menge $\Z\times \Z$ hei"st {\em ebenes Gitter}. Man zeige, dass es f"ur
  $n\ne 4$ kein regul"ares $n$-Eck gibt, dessen Ecken auf dem ebenen Gitter
  liegen!
\end{aufgabe}

\begin{loesung} 
  Die Drehung um $90^\circ$ um einen Punkt aus $\Z/2\times \Z/2$ "uberf"uhrt
  das Gitter $\Gamm=\Z\times \Z$ in sich. Es sei $P_0\cdots P_{n-1}$ ein
  regul"ares $n$-Eck minimaler Kantenl"ange mit $P_i\in \Gamm$. Wir betrachten
  die $n$ Drehungen um die Mittelpunkte $(P_i+P_{i+1})/2$, $i=0,\dots, n-1$ um
  $+90^\circ$. Das Bild von $P_{i+1}$ bei der zugeh"origen Drehung sei
  $P_{i+1}^\prime$ und liegt echt im Innern von $P_0\cdots P_{n-1}$. Dann ist
  das $n$-Eck $P'_0\cdots P'_{n-1}$ wieder regul"ar und liegt auf dem Gitter
  $\Gamm $. Dies ist aber ein Widerspruch zur Minimalit"at der Seitenl"ange.
  Nur f"ur $n=4$ fallen die 4 Bildpunkte zusammen und es ergibt sich kein
  Widerspruch.
\end{loesung}

%%%%%%%%%%%%%%%%%%%%%%%%%%%%%%%%%%%%%%%%%%%%%%%%%%%%%%%%%%%
\subsection*{Unm"oglichkeitss"atze zu Zerlegungen}
Wir werden sp"ater sehen, dass jede nat"urliche Zahl als Summe von h"ochstens
vier Quadratzahlen darstellbar ist. Dies wurde schon von {\sc Fermat}
vermutet und sp"ater von {\sc Lagrange} bewiesen. Die Anzahl dieser
Darstellungen bestimmte {\sc Jacobi}, siehe Satz\,\ref{j2}.
\nocite{b-Pieper}

\begin{satz} 
{\em (a)} Eine Primzahl der Form $4k+3$ l"asst sich nicht als Summe von zwei
Quadratzahlen schreiben
\\
{\em(b)} Eine Zahl der Form $4^n(8k+7)$ l"asst sich nicht als Summe von
  drei Quadratzahlen schreiben.
\end{satz}


\begin{beweis} (a) Die quadratischen Reste modulo 4 sind 0 und 1. Somit l"a"st
  sich 3 nicht als Summe zweier solcher Reste schreiben.
\\
(b) Mit vollst"andiger Induktion "uber $n$. Im Falle $n=0$ betrachten wir die
quadratischen Reste modulo 8; das sind 0, 1 und 4. Die Summe dreier solcher
Reste kann aber niemals den Rest 7 ergeben. Nehmen wir jetzt an, die Zahl
$4^a(8k+7)$ ist nicht als Summe von drei Quadraten darstellbar. Wir haben zu
zeigen, dass dann auch $4^{a+1}(8k+7)$ nicht als Summe von drei Quadratzahlen
darstellbar ist. Angenommen, es gibt doch eine derartige Darstellung
$$
4^{a+1}(8k+7)=u^2+v^2+w^2.
$$
Dann folgt aus $u^2+v^2+w^2\equiv 0\mod 4$ sofort $u\equiv v\equiv w\equiv
0\mod 2$, denn der Rest $0$ l"asst sich nur als $0=0+0+0\mod 4$ mit drei
quadratischen Resten modulo $4$ darstellen. Dann kann man aber die obige
Gleichung durch $4$ dividieren und man erh"alt einen Widerspruch zur
Induktionsannahme, \cite[Abschnitt 6.2]{b-Kraetzel}.
\end{beweis}

\begin{bemerkung} Es ist erw"ahnenswert, dass alle Zahlen, die nicht von der
  Form $4^n(8k+7)$ sind, als Summe von 3 Quadratzahlen darstellbar sind. Dies
  ist  schwierig zu zeigen. Den Beweis findet man etwa in \cite[Band
  I, Teil III, Kap. 4]{b-Landau}
\end{bemerkung}

\begin{aufgabe} Die Gleichung $x_1^4+ x_2^4+\cdots +x_{14}^4=1599$ hat keine
  ganzzahlige L"osung.
\end{aufgabe}

\begin{aufgabe} Zahlen der Form $8k+6$ sind nicht als Summe von zwei Quadratzahlen
  darstellbar.
\end{aufgabe}

\begin{loesung} Die quadratischen Reste modulo 8 sind 0,1,4. Die Summe zweier
  solcher Reste ist niemals gleich 6.
\end{loesung}

\subsection*{Die Darstellung nat"urlicher Zahlen als Summe von Quadraten}

Im folgenden Abschnitt werden wir die Frage beleuchten, wann eine nat"urliche
Zahl als Summe von zwei bzw.{} vier Quadratzahlen darstellbar ist. Zum Schluss
werden wir --- allerdings  ohne Beweis --- auch Formeln f"ur die Anzahl
solcher Zerlegungen angeben. Haben wir im vorigen Abschnitt einfache
Negativ-Resultate bewiesen, so wollen wir uns nun den etwas schwierigeren
Existenz- und Eindeutigkeitss"atzen f"ur Zerlegungen zuwenden.

\begin{satz} {\em (a)} Es seien $m=a^2+b^2$ und $n=x^2+y^2$. Dann ist
$$mn=(ax+by)^2 +(ay-bx)^2=(ax-by)^2+(ay+bx)^2.$$
Ferner gilt $2m=(a-b)^2+(a+b)^2$.
\\
{\em (b)} Es seien $m=a^2+b^2+c^2+d^2$ und $n=x^2+y^2+z^2+u^2$. Dann gilt
\begin{xalignat*}{2}
mn&=A^2+B^2+C^2+D^2,\,\text{wobei}&&
\\
A&=ax+by+cz+du,& B&=ay-bx-cu+dz,
\\
C&=az+bu-cx-dy,& D&=au-bz+cy-dx.
\end{xalignat*}
Ferner gilt $2m=(a-b)^2+(a+b)^2+(c-d)^2+(c+d)^2$.
\end{satz}

\begin{beweis} Man erh"alt die Identit"aten unmittelbar durch Ausmultiplizieren
  (binomische Formel). Nat"urlich gibt es auch bei (b) mehrere M"oglichkeiten,
  der Darstellung des Produkts. Wie viele eigentlich? 
\end{beweis}

Somit kann man sich in beiden F"allen auf die Zerlegung von Primzahlen
zur"uckziehen. Oben haben wir gesehen, dass sich die Primzahlen der Form
$4k+3$ nicht als Summe von zwei Quadraten schreiben lassen.

\begin{satz}\label{2qs} Jede Primzahl der Form $4n+1$ l"asst sich eindeutig als Summe von
  zwei Quadratzahlen schreiben.
\end{satz}

Bevor wir diesen Satz beweisen, m"ussen wir noch einige Hilfsmittel zur
Verf"ugung stellen.

\begin{satz}[ (Wilson)] F"ur jede Primzahl $p$ ist
$$ (p-1)!\equiv -1 \mod p.  $$
Wenn umgekehrt diese Kongruenz besteht, dann ist $p$ eine Primzahl.
\end{satz}

\begin{beweis} F"ur $p=2$ und $p=3$ ist der Satz sofort einzusehen. Es sei also
  $p>3$. Keine der Zahlen
$$2,3,\dots, p-2$$ gen"ugt der Kongruenz $x^2\equiv 1\mod p$. Denn diese
 Kongruenz ist gleichwertig mit der Aussage ${p\vert (x-1)(x+1)}$ und, da $p$
 Primzahl ist, sind $x\equiv \pm 1\mod p$ die einzigen beiden L"osungen.  In
 der oben genannten Folge von Resten gibt es also zu jedem $x$ ein $x'$ mit $x
 x'\equiv 1\mod p$, wobei $x'\not\equiv x\mod p$. Die obigen $p-3$ Reste
 lassen sich also zu Paaren anordnen, deren Produkt immer kongruent $1$ modulo
 $p$ ist. Somit gilt
$$ (p-2)!\equiv 1\mod p,\quad\text{bzw.}\quad (p-1)!\equiv -1\mod p.
$$
F"ur jede zusammengesetzte Zahl $n=ab$ ist $(n-1)!\equiv 0\mod n$, da die
Faktoren $a$ und $b$ beide in  den Zahlen $1,\dots, n-1$ als Faktoren
aufgehen. 
\end{beweis}

\begin{satz} \label{hs} Ist $p$ eine Primzahl der Form $4n+1$, so ist
$$\left[ \left( \frac{p-1}{2}\right)!\right]^2\equiv -1 \mod p.
$$
\end{satz}

\begin{beweis} Nach dem {\sc Wilson}schen Satz gilt
\begin{align*}
-1&\equiv (p-1)!\equiv 1{\cdot} 2\cdots 2n{\cdot}(2n+1)\cdots 4n
\equiv 1\cdots 2n (-2n)(-2n+1)\cdots (-1)
\\
&\equiv (2n)!(-1)^{2n}
\equiv\left[ \left( \frac{p-1}{2}\right)!\right]^2\mod  p
\end{align*}
\end{beweis}

Dieser Satz zeigt insbesondere, dass f"ur die Primzahlen der Form $4n+1$ die
Kongruenz ${x^2\equiv -1\mod p}$ stets eine L"osung hat.

\begin{satz}[ (Thue)] Es sei $p$ eine Primzahl, $e$ und $f$ zwei ganze Zahlen mit
  $1<e,f<p$ und $p<ef$. Dann lassen sich alle Reste $r$ modulo $p$ auf die
  folgende Gestalt bringen:
$r\equiv 0\mod p$ oder 
$$ r\equiv\pm \frac{x}{y}\mod p,\quad\text{ wobei }\quad
0<x<e\quad\text{und}\quad 0<y<f.
$$
\end{satz}

\begin{beweis} Es sei  $r\not\equiv 0\mod p$. Wir betrachten die $ef$ Reste
  $v+rw$, wobei $0\le v<e$ und $0\le w <f$ gelte. Weil $ef>p$, m"ussen
  mindestens zwei dieser Reste "ubereinstimmen, etwa
$$ v_1+rw_1\equiv v_2+rw_2 \mod p.$$
Der Fall $w_1=w_2$ ist aber unm"oglich, da sonst auch $v_1=v_2$ gelten w"urde,
und die Paare sind gleich. Es gilt also
$$
r\equiv \frac{v_2-v_1}{w_1-w_2}\equiv \pm\frac{v_1-v_2}{w_1-w_2}\mod p
$$
und $\abs{v_1-v_2}<e$ und $\abs{w_1-w_2}<f$.
\end{beweis}

\begin{beweis}[(von Satz\,\ref{2qs})] Wir richten uns nach \cite[Kapitel VII,
  Abschnitt 3]{b-Neiss}.
Nach Satz\,\ref{hs} gibt es eine L"osung der Kongruenz $z^2\equiv -1\mod
p$. Wir wenden den Satz von {\sc Thue} mit $e=f$ an, so dass $e^2>p$
gilt. Dabei sei $e$ die kleinste derartige Zahl. Es gibt also zwei nat"urliche
Zahlen $x$ und $y$ mit $0<x,y<e$, so dass $z\equiv \pm x/y\mod p$ gilt. Dann
ist aber
$$ \left(\frac{x}{y}\right)^2\equiv z^2\equiv-1\mod p$$
und somit $x^2+y^2=pr$ f"ur eine gewisse nat"urlichen Zahl $r$. Wegen $x,y<e$
ist $x^2<p$ und auch $y^2<p$, denn sonst w"are $e$ nicht die kleinste Zahl
mit $e^2>p$. Somit ist $x^2+y^2=pr<2p$. Also gilt $r=1$ und somit $x^2+y^2=p$.
\\
Zur Eindeutigkeit. Angenommen, $p=x^2+y^2=u^2+v^2$ sind zwei Darstellungen
f"ur $p$. Dann gilt $-1\equiv x^2/y^2\equiv u^2/v^2\mod p$. Hieraus folgt
$$\frac{x}{y}\equiv \pm \frac{u}{v}\equiv \mp \frac{v}{u}\mod p.$$
Durch Vertauschung von $u$ und $v$ kann man jedenfalls erreichen, dass
$x/y\equiv u/v\mod p$ bzw.{} $xv-yu\equiv 0\mod p$ gilt.
Nun ist aber 
$$ p^2=(x^2+y^2)(u^2+v^2)=(xu+yv)^2+(xv-yu)^2.$$
Da der letzte Summand durch $p^2$ teilbar sein muss, muss er sogar gleich 0
sein, also $xv=yu$. Wegen $\ggT(x,y)=1$ und $\ggT(u,v)=1$ folgt hieraus $x=u$
und $y=v$.
\end{beweis}

In sch"onster Allgemeinheit lautet der 2-Quadrate Satz dann
 
\begin{satz}\label{d1} Ist $n=2^ep_1^{f_1}\cdots p_k^{f_k}q_1^{h_1}\cdots
  q_{s}^{h_s}$ mit Primzahlen $p_1, \dots, p_k$ der Form $4m+1$ und Primzahlen
  $q_1,\dots, q_s$ der Form $4m+3$ und gilt
$$
2\vert h_1,\dots, 2\vert h_s,
$$
so kann man $n$ als Summe zweier Quadrate --- Null ist auch ein Quadrat ---
darstellen.
\\
Gilt f"ur ein $j$ unter gleichen Voraussetzungen ${2\nmid h_j}$, so l"asst
sich $n$ nicht als Summe zweier Quadrate darstellen.
\end{satz}


\subsubsection*{Potenzreihen} 

In diesem Abschnitt soll ganz knapp angedeutet
werden, wie man Potenzreihen zum Abz"ahlen von L"osungen nutzen kann.
\renewcommand{\d}[2][n]{\mathrm{d}_{#2,4}(#1)}

\begin{satz}[ (Jacobi, 1828)]\label{j1} Die Anzahl der Darstellungen  einer nat"urlichen
  Zahl $n$ als Summe von 2 Quadraten ist gleich
$$ 4(\d{1}-\d{3}).$$
Dabei ist $\d{r} $ die Anzahl der Teiler von $n$ (einschlie"slich $1$ und
$n$), die bei der Division durch $4$ den Rest $r$ lassen.
\end{satz}

\begin{satz}[ (Jacobi, 1829)] \label{j2} Die Anzahl der Darstellungen einer nat"urlichen
  Zahl $n$ als Summe von 4 Quadraten ist gleich 
$$ 8\sum_{d\vert n,\, 4\nmid d} d.$$
\end{satz}

\begin{bemerkung} In beiden S"atzen z"ahlen die Darstellungen
  $5=1^2+2^2=(-1)^2+2^2=1^2+(-2)^2=(-1)^2+(-2)^2=2^2+1^2=2^2+(-1)^2=(-2)^2+1^2=(-2)^2+(-1)^2$
  alle  als verschiedene Darstellungen. Tats"achlich ist $\d[5]{1}=2$, da 1
  und 5 beides Teiler  von 5 sind, die den Rest 1 lassen. Ferner ist
  $\d[5]{3}=0$ und somit kommt man auf 8 Darstellungen. Darstellungen mit 0
  als Summand werden ebenfalls mitgez"ahlt: $4=(\pm 1)^2 +(\pm 1)^2 +(\pm 1)^2
  +(\pm 1)^2$ (16 M"oglichkeiten) und $4=(\pm 2)^2+0^2+0^2+0^2= 0^2+(\pm
  2)^2+0^2+0^2=0^2+0^2+(\pm 2)^2+0^2= 0^2+0^2+0^2+(\pm 2)^2$ (8
  M"oglichkeiten). Tats"achlich ist $\sum_{d\vert 4,\, 4{\nmid} d}
  d=1+2=3$ und es gibt 24 Zerlegungen von 4 in 4 Quadrate.
\end{bemerkung}  

\begin{bemerkung} Der Satz\,\ref{j2} hat den Satz von {\sc Lagrange} zur Folge: Jede
  nat"urliche Zahl l"asst sich als Summe von 4 Quadratzahlen schreiben. Denn
  die im Satz angegebene Anzahl von Zerlegungen ist f"ur alle $n$ eine {\em
  positive} nat"urliche Zahl, da $d=1$ als Teiler stets mitgez"ahlt
  wird.
\end{bemerkung}

Der Ausgangspunkt f"ur unseren Beweis ist dabei der folgende
Satz.  Einen elementaren Beweis dieses Satzes --- durch reines Abz"ahlen von
Partitionen ---  findet
man in \cite[Chapter~2.2]{b-Bressoud}.

\begin{satz}[ Jacobi-Tripelprodukt-Identit"at] \label{s-jacobi}
F"ur $\abs{q} <1$ und alle $x$ gilt:
\begin{align}\label{e-jacobi}
\prod_{i=1}^\infty (1+q^i
x)(1+q^{i-1}x^{-1})(1-q^i)=\sum_{n\in\Z}q^{n(n+1)/2}x^n.
\end{align}
\end{satz}
 
Durch trickreiche Umformungen \cite{a-Hirschhorn} leitet man hieraus  die
folgenden beiden Identit"aten ab
\begin{align}
\left(\sum_{n\in\Z}q^{n^2}\right)^2&=1+4\sum_{k\ge1, l\ge
  0}(q^{k(4l+1)}-q^{k(4l+3)})=1+4\sum_{n\ge 1}(\d{1}-\d{3})q^n,
\label{e2}
\\
\left(\sum_{n\in\Z}q^{n^2}\right)^4&=
1+8\sum_{n\ge 1}\left(\sum_{d\vert n,\, 4{\nmid} d}d \right) q^n.\label{e4}
\end{align}

Schauen wir uns die linke Seite von \rf[e2] einmal genauer an. Nach formalem 
Ausmultiplizieren der beiden unendlichen Reihen lautet der allgemeine Summand 
$a_rq^r$, wobei f"ur ein festes $r$ alle Summanden ${q^r=q^{n_1^2} q^{n_2^2}}$ mit
$n_1,n_2\in \Z$ zu ber"ucksichtigen sind. Jede L"osung $(n_1, n_2)$ der
Gleichung $r=n_1^2+n_2^2$ liefert also einen Summanden $q^r$. Also ist  $a_r$
die gesuchte Anzahl.

%\bibliographystyle{../hab/mybst}
%\bibliography{../hab/bibdat}
\begin{thebibliography}{1}
\bibitem{b-Bressoud}
Bressoud, D.~M.: \emph{Proofs and confirmations. The story of the alternating
  sign matrix conjecture}, MAA Spectrum, Cambridge University Press, Cambridge,
  1999

\bibitem{b-Engel}
Engel, A.: \emph{Problem-solving strategies}, Springer, New York, 1998

\bibitem{a-Hirschhorn}
Hirschhorn, M.~D.: Partial fractions and four classical theorems of number
  theory, \emph{Amer. Math. Monthl.} \textbf{107} (2000), 260--264

\bibitem{b-Kraetzel}
Kr{\" a}tzel, E.: \emph{Zahlentheorie}, Nummer~19 in Studienb{\" u}cherei.
  Mathematik f{\" u}r Lehrer, VEB Deutscher Verlag der Wissenschaften, Berlin,
  1981

\bibitem{b-Landau}
Landau, E.: \emph{Vorlesungen {\" u}ber Zahlentheorie}, Chelsea Publishing Co.,
  New York, 1969

\bibitem{b-Neiss}
Nei{\ss }, F.: \emph{Einf{\" u}hrung in die Zahlentheorie}, S. Hirzel Verlag,
  Leipzig, 1952

\bibitem{b-Pieper}
Pieper, H.: \emph{{D}ie komplexen {Z}ahlen. {T}heorie -- {P}raxis --
  {G}eschichte}, Nummer 110 in {M}athematische {S}ch{\" u}lerb{\" u}cherei,
  Deutscher Verlag der Wissenschaften, Berlin, 1991

\bibitem{b-Postnikov}
Postnikov, M.~M.: \emph{{Vvedenie v teoriyu algebraicheskikh chisel} (Russian)
  [Introduction to algebraic number theory]}, Nauka, Moscow, 1982

\end{thebibliography}

\begin{attribution}
schueler (2004-09-09): Contributed to KoSemNet

graebe (2004-09-09): Prepared along the KoSemNet rules
\end{attribution}

\end{document}
