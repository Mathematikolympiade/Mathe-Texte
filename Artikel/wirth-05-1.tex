% Version: $Id: wirth-05-1.tex,v 1.1 2008/09/05 09:52:58 graebe Exp $
\documentclass[11pt]{article}  
\usepackage{kosemnet,ko-math,ngerman,url}  
\usepackage{curves}
\newtheorem{uebung}{\"{U}bung}

\author{Jens Wirth, Freiberg\\ \url{wirth@math.tu-freiberg.de}}
\title{Trigonometrische Funktionen\kosemnetlicensemark}
\date{}

\def\Re{{\sf Re}\,}
\def\Im{{\sf Im}\,}

\begin{document} 
\maketitle         
%
\section{Bezeichnungen, komplexe Zahlen}
Im folgenden bezeichnet $\C$ die Menge der komplexen Zahlen $z=x+\ii y$ mit
$x,y\in\R$, $\ii^2=-1$. Die Zahl $x=\Re z$ bezeichnet den Realteil, $y=\Im z$
den Imagin"arteil von $z$.

\begin{minipage}{5cm}
\unitlength1mm
\begin{picture}(50,40)
  \put(5,10){\vector(1,0){40}}
  \put(42,5){$\Re$}
  \put(15,5){\vector(0,1){30}}
  \put(8,32){$\Im$}
  \put(15,10){\vector(3,2){15}}
  \put(32,22){$z$}
  \put(30,8){\line(0,1){4}}
  \put(30,5){$x$}
  \put(13,20){\line(1,0){4}}
  \put(10,20){$y$}
  \put(15,10){\arc(10,0){33.7}}
  \put(21,11.5){$\phi$}
\end{picture}
\end{minipage}
\hfill
\begin{minipage}{8cm}
Weitere wichtige Begriffe sind 
\begin{itemize}\setlength{\itemsep}{0pt}
\item $\phi=\arg z$, das Argument von $z$
\item $\overline z=x-\ii y$, die zu $z$ konjugierte Zahl
\item $|z|=\sqrt{x^2+y^2}=\sqrt{z \overline z}$, der Betrag von $z$ 
\end{itemize}
\end{minipage}

Ein wichtiges Hilfsmittel ist die f"ur alle komplexen Zahlen definierte
Exponentialfunktion 
\begin{equation}
  e^{z_1+z_2}=e^{z_1}e^{z_2},\qquad e^{1}=e,\qquad e^{\ii\frac\pi2}=\ii.
\end{equation}
Damit gilt f"ur jede komplexe Zahl $z$ 
\begin{equation}
  z=|z|\,e^{\ii\arg z},
\end{equation}
beziehungsweise dazu "aquivalent der folgende Satz.
\begin{satz}{\bf[Eulersche Relation]}
\label{satz1}
\begin{equation}
  e^{\ii\phi}=\cos\phi+\ii\sin\phi,\qquad \phi\in\R
\end{equation}
\end{satz}
%
\section{Anwendung 1: Additionstheoreme f"ur Winkelfunktionen}
Satz \ref{satz1} ist ein wichtiges Hilfsmittel um Beziehungen zwischen
Winkelfunktionen zu verstehen. Zum Aufw"armen deshalb eine erste Anwendung.

Wir versuchen die bekannten Additionstheoreme f"ur $\sin$ und $\cos$
abzuleiten, der Bequemlichkeit halber beide gleichzeitig. Es gilt
\begin{align}
  \cos(\phi+\psi)&+\ii\sin(\phi+\psi)=e^{\ii(\phi+\psi)}
  =e^{\ii\phi}e^{\ii\psi}\notag\\
  =&(\cos\phi+\ii\sin\phi)(\cos\psi+\ii\sin\psi)\\
  =&(\cos\phi\,\cos\psi-\sin\phi\,\sin\psi)
  +\ii(\sin\phi\,\cos\psi+\cos\phi\,\sin\psi)\notag
\end{align}
und ein Vergleich der Real- und Imagin"arteile ergibt die gesuchten
Beziehungen. 

\begin{uebung}
  Man nutze die Additionstheoreme um die Beziehungen
  \begin{align}
    \sin\phi+\sin\psi&=2\sin\frac{\phi+\psi}2\,\cos\frac{\phi-\psi}2\notag\\
    \cos\phi+\cos\psi&=2\cos\frac{\phi+\psi}2\,\cos\frac{\phi-\psi}2\\
    \cos\phi-\cos\psi&=-2\sin\frac{\phi+\psi}2\,\sin\frac{\phi-\psi}2\notag
  \end{align}
  abzuleiten!
\end{uebung}

Komplexe Zahlen sind auch gut geeignet, Probleme der ebenen Geometrie zu 
behandeln. 
\begin{uebung}
  Man nutze komplexe Zahlen um einen Beweis des Cosinussatzes
  \begin{equation*}
    c^2=a^2+b^2-2ab\cos\angle(a,b)
  \end{equation*}
  f"ur ein Dreieck mit den Seiten $a,b,c$ zu finden.
\end{uebung}
%
\section{Anwendung 2: Winkelvielfache}
Aufgrund der Additionstheoreme gilt
\begin{align}
  &\cos 2\phi=2\cos^2\phi-1\\
  &\sin 2\phi=2\sin\phi\,\cos\phi.
\end{align}
Wir wollen eine entsprechende Relation f"ur allgemeine ganzzahlige Vielfache
suchen. Auch hier hilft wiederum Satz \ref{satz1}. Es gilt
\begin{align*}
  cos(n\phi)+\ii\sin(n\phi)
  &=\left(e^{\ii\phi}\right)^n=(\cos\phi+\ii\sin\phi)^n\\
  &=\sum_{k=0}^n\binom{n}{k} \ii^k \cos^{n-k}\phi\,\sin^k\phi.
\end{align*}   
Interessiert man sich nur f"ur $\cos(n\phi)$, so haben wir den Realteil zu
bestimmen. Dazu setzen wir $k=2l$ und erhalten
\begin{align*}
  \cos(n\phi)&=\sum_{l=0}^{\lfloor\frac n2\rfloor}
  \binom{n}{2l}(-1)^l\cos^{n-2l}\phi\,\sin^{2l}\phi\\
  &=\sum_{l=0}^{\lfloor\frac n2\rfloor}
  \binom{n}{2l}(-1)^l\cos^{n-2l}\phi\,(1-\cos^2\phi)^l\\
  &=\sum_{l=0}^{\lfloor\frac n2\rfloor}\sum_{k=0}^l
  \binom{n}{2l}\binom{l}{k}(-1)^{l+k}\cos^{n-2l+2k}\phi.
\end{align*}
Die Funktion $\cos(n\phi)$ ist also insbesondere ein Polynom in
$\cos\phi$. Dieses Polynom $T_n$ hat den Grad $n$ und wird als
Tschebyscheff-Polynom $n$-ter Ordnung bezeichnet. Es gilt
\begin{equation}
  \cos(n\phi)=T_n(\cos\phi).
\end{equation}
Die Tschebyscheff-Polynome bis zur Ordnung 5 sind
\begin{equation}
  1,\;\;x,\;\;2x^2-1,\;\;4x^3-3x,\;\;8x^4-8x^2+1,\;\;16x^5-20x^3+5x.
\end{equation}
\begin{uebung}
  Eine "ahnliche Betrachtung f"ur Vielfache des Sinus ist m"oglich. Allerdings
  sind nur die ungeraden Vielfachen als Polynome in $\sin\phi$ darstellbar, die
  geraden enthalten $\cos\phi$ als Faktor.

  "Ublich ist eine andere Vorgehensweise. Man eliminiert so viele
  $\sin$-Potenzen wie m"oglich (d.h. alle bis auf eine) und schreibt
  \begin{equation}
    \sin(n\phi)=\sin(\phi)\,U_{n-1}(\cos\phi)
  \end{equation}
  mit dem Tschebyscheff-Polynom $n$-ter Ordnung und zweiter Art $U_n$.
\end{uebung}
\begin{uebung}
  Der h"ochste Koeffizient der Tschebyscheff-Polynome $T_{n+1}$ und $U_n$ 
  ist $2^n$ f"ur alle $n\geq0$. 
\end{uebung}
%
\section{Regelm"a"sige $n$-Ecke und Nullstellen von $T_n$}
Wir suchen nach einer Interpretation der Nullstellen des Polynoms $T_n$. Das
sind gerade die Werte $\cos\phi$, f"ur die $\cos(n\phi)=0$ gilt,
d.h. $n\phi=\frac\pi2 +k\pi$. Es gen"ugt nach denjenigen $\phi$ zu suchen, die
in einem Monotonie-Intervall der Cosinus-Funktion liegen. Das sind im
Intervall $[0,\pi]$
\begin{equation}
  \frac\pi{2n},\;\;
  \frac{3\pi}{2n},\;\;
  \frac{5\pi}{2n},\;\;
  \dots,\;\;
  \frac{(2n-1)\pi}{2n}.
\end{equation}
Diese sind paarweise verschieden. Da es $n$ St"uck sind, handelt sich bei den
betreffenden Cosinus-Werten
\begin{equation}
  \cos\left(\frac1{2n}\,\pi\right),\;\;
  \cos\left(\frac{3}{2n}\,\pi\right),\;\;
  \dots,\;\;
  \cos\left(\frac{2n-1}{2n}\,\pi\right)
\end{equation}
um alle Nullstellen von $T_n$.

\begin{minipage}{.45\textwidth}
  \begin{uebung}
    Diese Werte lassen sich als $x$-Koor\-dinaten der Eckpunkte eines
    regel\-m"a"sigen $2n$-Ecks deuten.  Wie und warum? (vgl. Skizze)
  \end{uebung}
\end{minipage}
\begin{minipage}{.52\textwidth}\centering
\unitlength2mm
\begin{picture}(40,40)
  \put(10,20){\vector(1,0){30}}
  \put(38,17){$\Re$}
  \put(25,5){\vector(0,1){30}}
  \put(22,34){$\Im$}
  \put(16.3,10){\line(0,1){20}}
  \put(16.5,21){$\zeta_1$}
  \put(23,21){$\zeta_2$}
  \put(33.6,10){\line(0,1){20}}
  \put(31.8,21){$\zeta_3$}
  \thinlines
  \put(25,10){\curve(0,0,8.6,15)}
  \put(25,10){\curve(0,0,-8.6,15)}
  \put(16.3,25){\line(1,0){17.2}}
  \put(25,20){\bigcircle{20}}
  \thicklines
  \put(25,30){\curve(0,0,8.6,-15)}
  \put(25,30){\curve(0,0,-8.6,-15)}
  \put(16.3,15){\line(1,0){17.2}}
\end{picture}
\end{minipage}
%
\section{Trigonometrische Polynome}
Ein Ausdruck der Form
\begin{equation}\label{trigPol}
  f(x)=a+\sum_{k=1}^n \big(b_k\sin kx+c_k\cos kx\big)
\end{equation}
wird als trigonometrisches Polynom bezeichnet. 
\begin{satz}
  \label{satz2}
  Eine Funktion $f(x)$ ist ein trigonometrisches Polynom genau dann, wenn 
  sie sich als Polynom $f(x)=P(\cos x,\sin x)$ schreiben l"asst.
\end{satz}
\begin{uebung}
  Man schreibe $f(x)=\cos^5x$ als trigonometrisches Polynom!
\end{uebung}
\begin{uebung}
  Man beweise Satz \ref{satz2}, indem man einen Algorithmus zur Bestimmung des
  Polynoms $P$ angibt!
\end{uebung}
\begin{uebung}
  Das Polynom $P$ in Satz \ref{satz2} ist nicht eindeutig bestimmt. Vielmehr
  l"asst sich immer die (dann eindeutig bestimmte) Struktur
  \begin{equation}
    f(x)=P_1(\cos x)+\sin x\,P_2(\cos x)
  \end{equation}
  erzeugen. Welche Forderung muss man an die Koeffizienten $a,b_k,c_k$ in
  (\ref{trigPol}) Stellen, dass $P_2\equiv0$ gilt?
\end{uebung}
%
\section{Aufgabensammlung}
\begin{aufgabe}(391333A, \cite{MO})\\
  Es seien $\alpha$,$\beta$ und $\gamma$ nichtnegative Zahlen mit
  $\alpha+\beta+\gamma\leq2\pi$. \\
  Beweisen Sie
  \begin{equation*}
    \cos\alpha+\cos\beta+\cos\gamma
    \geq\cos(\alpha+\beta)+\cos(\beta+\gamma)+\cos(\gamma+\alpha) .
  \end{equation*}
  In welchen F"allen gilt Gleichheit?
\end{aufgabe}
\begin{aufgabe}(61236, \cite[A 24]{Lesebogen})\\
  Die Zahl 
  $\;\sin\left(\frac\pi{18}\right)\;$%$\sin 10^\circ$ 
  gen"ugt einer Gleichung dritten Grades mit
  ganzzahligen Koeffizienten.
  
  Man stelle diese (bis auf einen gemeinsamen Teiler aller Koeffizienten
  eindeutig bestimmte) Gleichung auf und ermittle ihre beiden anderen Wurzeln.
\end{aufgabe}
\begin{aufgabe}\hspace*{1cm}\\
  a) Man berechne den Wert von $\cos\frac\pi5$.\\
  b) Gesucht ist ein Algorithmus zur Konstruktion eines regelm"a"sigen 10-Ecks
  mit Zirkel und Lineal.
\end{aufgabe}
\begin{aufgabe}(71232, \cite[A 25]{Lesebogen})\\
  Es ist das Produkt 
  \begin{equation*}
    \sin %\left(\frac\pi{36}\right)
    5^\circ  
    \sin %\left(\frac{3\pi}{36}\right)\,
    15^\circ  
    \sin %\left(\frac{5\pi}{36}\right)\,
    25^\circ
    \sin %\left(\frac{7\pi}{36}\right)\,
    35^\circ
    \sin %\left(\frac{9\pi}{36}\right)\,
    45^\circ  
    \sin %\left(\frac{11\pi}{36}\right)\,
    55^\circ  
    \sin %\left(\frac{13\pi}{36}\right)\,
    65^\circ 
    \sin %\left(\frac{15\pi}{36}\right)\,
    75^\circ  
    \sin %\left(\frac{17\pi}{36}\right)\,
    85^\circ
  \end{equation*}
  in einen Ausdruck umzuformen, der aus nat"urlichen Zahlen lediglich durch
  Anwendung der Rechenoperationen des Addierens, Subtrahierens,
  Multiplizierens, Dividierens sowie des Radizierens mit nat"urlichen
  Wurzelexponenten gebildet werden kann.\\
  (Beispiel daf"ur $\sin30^\circ\,\sin60^\circ=\frac14\sqrt3$)
\end{aufgabe}
\begin{aufgabe}($\sqrt{\mbox{WURZEL}}$, $\delta 19$)\\
  Man bestimme alle reellen $x$ mit
  \begin{equation*}
    4\cos x\,\cos 2x\,\cos 5x+1=0.
  \end{equation*}
\end{aufgabe}
\begin{aufgabe}\hspace*{1cm}\\
  Man beweise die Relation
  \begin{equation*}
    8 \cos\left(\frac\pi8\right)\,
    \cos\left(\frac{3\pi}8\right)\,
    \cos\left(\frac{5\pi}8\right)\,
    \cos\left(\frac{7\pi}8\right)=1.
  \end{equation*}
\end{aufgabe}
% \begin{aufgabe}\hspace*{1cm}\\
%   Sei $f(x)$ ein trigonometrisches Polynom ohne Absolutglied
%   (d.h. (\ref{trigPol}) mit $a=0$). Man zeige
%   $$
%   f(x)\geq0\qquad\Longrightarrow\qquad f\equiv0
%   $$
%   auf elementare Weise.
% \end{aufgabe}
%
\begin{thebibliography}{99}
\bibitem{} Herbert Pieper: Die komplexen Zahlen\\ VEB Deutscher Verlag der
  Wissenschaften, Berlin 1988
\bibitem{Lesebogen} Mathematischer Lesebogen "`Junge Mathematiker"', Heft 80\\
  Bezirkskabinett f"ur au"serunterrichtliche T"atigkeit, Rat des Bezirkes
  Leipzig, 1987
\bibitem{WURZEL} $\sqrt{\mbox{WURZEL}}$, 5/97, \url{http://www.wurzel.org}
\bibitem{MO} \url{http://www.mathematik-olympiaden.de}
\end{thebibliography}

\begin{attribution}
wirth (Dec 2004): Contributed to KoSemNet

graebe (2005-01-05): Prepared along the KoSemNet rules
\end{attribution}
\end{document}
