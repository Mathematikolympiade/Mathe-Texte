% Version: $Id: graebe-04-4.tex,v 1.1 2008/09/05 09:52:58 graebe Exp $
\documentclass[11pt]{article}  
\usepackage{kosemnet,ko-math,ngerman}  

\author{Hans-Gert Gr�be, Leipzig}
\title{Rechnen mit Kongruenzen\kosemnetlicensemark\\
Arbeitsmaterial f�r Klasse 7}
\date{}

\begin{document} 
\maketitle         

Kongruenzen oder Restklassen sind ein sehr wichtiges Hilfsmittel, mit
dem sich viele "Uberlegungen, in denen in der einen oder anderen Form
Teilbarkeitsaussagen auftreten, besonders elegant formulieren
lassen. Hat man einmal die grundlegenden Prinzipien dieser {\em
Modulrechnung} verstanden, dann ist sie auch ein wichtiges Hilfsmittel
zum Auffinden von L"osungen entsprechender Aufgaben.
\medskip

Im weiteren sei eine ganze Zahl $m>1$ fixiert, der {\em Modul},
bez"uglich welcher wir Teilbarkeitsaussagen untersuchen wollen.

Wir sagen, dass zwei ganze Zahlen $a,b\in {\Z}$ {\em kongruent
modulo $m$} sind, und schreiben
\[a\equiv b\pmod{m}\qquad \mbox{oder kurz}\qquad a\equiv b\ (m),\]
wenn $a$ und $b$ {\glqq}bei Division durch $m$ denselben Rest
lassen{\grqq}. So gilt etwa $73\equiv 38\ (7)$, denn beide Zahlen lassen
bei Division durch 7 den Rest 3. "Ahnlich gilt $71\equiv 23\ (8)$,
weil beide Zahlen bei Division durch 8 den Rest 7 lassen.

Diese Definition, unter der ihr euch hoffentlich etwas vorstellen
k"onnt, ist zwar sehr einpr"agsam und f"ur positive $a,b$ auch
verst"andlich, aber entbehrt doch der f"ur exakte mathematische
Argumentation notwendigen Strenge. Eine dem urspr"unglichen Anliegen
entsprechende Aussage, die dem Anspruch an eine solche Strenge
gen"ugt, ist die "Uberlegung, dass zwei Zahlen bei Division durch $m$
genau dann denselben Rest lassen, wenn deren Differenz durch $m$
teilbar ist. Auf diese Weise verbinden wir den neuen Begriff
{\glqq}kongruent{\grqq} mit dem bereits bekannten Begriff der Teilbarkeit:
\[a\equiv b\pmod{m}\ :\Leftrightarrow\ m\,|\,(a-b)\]
Eine solche Beziehung zwischen zwei Zahlen (und allgemeiner mathematischen
Gr��en) bezeichnet man auch als {\em Relation} und $\equiv$ als die {\em
Kongruenzrelation}.  Die Kongruenzrelation hat drei grundlegende
Eigenschaften; sie ist 
\begin{itemize}
\item reflexiv (das hei�t $a\equiv a\pmod{m}$), 
\item transitiv (das hei�t $a\equiv b\pmod{m}, b\equiv c\pmod{m} \Rightarrow
  a\equiv c\pmod{m}$)   
\item und symmetrisch (das hei�t $a\equiv b\pmod{m}\Rightarrow b\equiv
  a\pmod{m}$). 
\end{itemize}
Relationen mit diesen drei Eigenschaften bezeichnet man auch als {\em
�quivalenzrelationen}. Wir wollen diese drei Eigenschaften hier beweisen,
indem wir die jeweilige Aussage "uber Kongruenzen in eine solche "uber
Teilbarkeit umformulieren und dann unser Wissen "uber Teilbarkeitsaussagen
anwenden:
\begin{quote}
{\em Reflexivit"at}: Es gilt stets $a\equiv a\pmod{m}$, denn $m\,|\,(a-a)=0$
(bekanntlich ist jede Zahl Teiler der Zahl 0).

{\em Symmetrie}: Wenn $a\equiv b\pmod{m}$ gilt, so gilt auch $b\equiv a\
(m)$: Ist $m$ ein Teiler von $(a-b)$, so ist $m$ auch ein Teiler von
$(b-a)$.

{\em Transitivit"at} oder {\em Drittengleichheit}: Wenn $a\equiv b\
(m)$ und $b\equiv c\pmod{m}$ gilt, so gilt auch $a\equiv c\pmod{m}$: Ist $m$
sowohl ein Teiler von $(a-b)$ als auch von $(b-c)$, so ist $m$ auch
ein Teiler von $(a-c)=(a-b)+(b-c)$.
\end{quote}
An dieser Stelle sei daran erinnert, wie Teilbarkeit definiert ist:
Eine ganze Zahl $u\in {\Z}$ hei"st {\em Teiler} einer Zahl $v\in
{\Z}$, wenn es eine dritte Zahl $t\in {\Z}$ gibt, so dass
$v=u\cdot t$ gilt (z.B. gilt $3\,|\,12$, weil es die Zahl $t=4$ gibt mit
$3\cdot 4=12$). $m$ ist also genau dann Teiler der Zahl $(a-b)$, wenn
es eine Zahl $t\in {\Z}$ mit $a-b=m\cdot t$ gibt, oder anders
\[a\equiv b\ (mod\ m)\ \Leftrightarrow\ m\,|\,(a-b)\ \Leftrightarrow\
\exists\,t\in {\Z}\ :\ a=b+m\cdot t 
\] 
Oft ist es wichtig, zwischen diesen drei M"oglichkeiten, die
Kongruenzeigenschaft zu formulieren, zu wechseln. So sind etwa die
drei folgenden Aussagen "aquivalent:
\[z\equiv 5\ (8)\ \Leftrightarrow\ 8\,|\,(z-5)\ \Leftrightarrow\
\exists\,t\in {\Z}\ :\ z=8t+5 
\]
Mit Kongruenzen kann man fast genauso wie mit Gleichungen rechnen. Es
gilt
\[a\equiv b\pmod{m},\ c\equiv d\pmod{m}\ \Rightarrow\ \left\{\ 
\parbox{8cm}{$\begin{array}{cccc} a+c &\equiv& b+d& \pmod{m}\\ a-c&\equiv&
b-d& \pmod{m}\\a\cdot c&\equiv& b\cdot d& \pmod{m} \end{array}$}\right.
\] 
Nur bei der Division muss man vorsichtig sein~!

Wir wollen die erste Aussage beweisen: Ist $a\equiv b\pmod{m}$, also
$m\,|\,(a-b)$, so gilt auch $a+c\equiv b+c\pmod{m}$, denn die Differenz
$(a+c) -(b+c)$ beider Seiten ist genau $(a-b)$, also durch $m$
teilbar. Genauso zeigen wir, dass aus $c\equiv d\pmod{m}$ die Kongruenz
$b+c\equiv b+d\pmod{m}$ folgt, womit sich schlie"slich $a+c \equiv b+d\
\pmod{m}$ nach der Drittengleichheit ergibt.
\begin{aufgabe}
Beweise auch die anderen beiden Aussagen sowie die vierte wichtige
Beziehung 
\[a\equiv b\pmod{m}\ \Rightarrow\ \forall\,n\in {\N}\ a^n\equiv b^n\
\pmod{m}. \]
\end{aufgabe}
Wir k"onnen also in jedem {\em arithmetischen Ausdruck}, d.h. in einem
solchen, wo die einzelnen Gr"o"sen nur durch die vier Grundrechenarten
verbunden sind, und in dem keine Division vorkommt, Zahlen durch
andere Zahlen mit demselben Rest $\pmod{m}$ ersetzen, ohne dass sich
der Rest des Ausdrucks "andert. Insbesondere kann man eine Zahl stets
durch ihren {\em kleinsten nichtnegativen Rest} ersetzen, d.h. durch
eine Zahl im Intervall $[0,m-1]$. Es spielen aber auch negative Reste
mit kleinem Absolutbetrag (z.B. der Rest $m-1\equiv (-1)\pmod{m}$) eine
wichtige Rolle.
\medskip

Wir k"onnen damit Aufgaben der folgenden Art einfach l"osen:
\begin{aufgabe}
    Zeige, dass $z=43^7-87^{13}$ durch 44 teilbar ist.
  \end{aufgabe}
Zum Beweis dieser Aussage m"ussen wir die Zahl $z$ zum Gl"uck nicht
ausrechnen\footnote{Es gilt $z=-16358756351530025699161940$}, sondern
nur $z\equiv 0\ (44)$ zeigen. Dazu k"onnen wir alle Summanden und
Faktoren durch einfachere Zahlen ersetzen, wenn diese nur bei Division
durch 44 denselben Rest lassen. Nun gilt aber $87\equiv 43\equiv (-1)\
(44)$ (letzteres, weil 44 ein Teiler von $(43-(-1))=(43+1)$ ist) und
folglich
\[z\equiv (-1)^7-(-1)^{13} = (-1)-(-1) = 0\ (44).\]
Beachte den Wechsel von $\equiv$ und $=$ in dieser Kette~! $\equiv$
wird verwendet, wenn die Ausdr"ucke links und rechts des Zeichens nur
denselben Rest lassen, $=$ dagegen, wenn die Ausdr"ucke wirklich
gleich sind.

\begin{aufgabe}
    Auf welche 3 Ziffern endet die Zahl $2^{100}$~?
  \end{aufgabe}
Rechnet man diese 30-stellige Zahl auf einem Taschenrechner aus, so
erh"alt man je nach Anzeige die {\em ersten} 8 -- 12 Ziffern, aber
keine Information "uber die {\em letzten} Ziffern. Informationen "uber
diese Ziffern erh"alt man aber aus der Modulrechnung, denn {\em die
letzten drei Ziffern einer Zahl sind gerade deren Rest bei Division
durch 1000}. Bei den folgenden Rechnungen leistet ein Taschenrechner
trotzdem gute Dienste. Wir schreiben zuerst
$2^{100}=(2^{10})^{10}=1024^{10}$ (gruppiere die 100 Faktoren 2 zu 10
Gruppen zu je 10 Faktoren) und ersetzen $1024\equiv 24\ (1000)$. Dies
liefert
\[2^{100}\equiv 24^{10}=(24^3)^3\cdot 24\ (1000).\]
Der Taschenrechner hilft weiter: $24^3=13\,824\equiv 824\ (1000)$,
also 
\[2^{100}\equiv 824^3\cdot 24=(824^2)\cdot(824\cdot 24)\ (1000).\]
Weiter mit dem Taschenrechner: $824^2=678\,976\equiv 976\ (1000)$ und
$824\cdot 24= 19\,776\equiv 776\ (1000)$, also
\[2^{100}\equiv 976\cdot 776 = 757\,376\equiv 376\ (1000).\]
Die Zahl endet also auf die drei Ziffern 376. 

Nat"urlich ist es heute nicht schwer, Software f"ur einen Computer zu
finden, die eine solche Zahl exakt berechnet. Man erh"alt dann
\[2^{100} = 1\,267\,650\,600\,228\,229\,401\,496\,703\,205\,376\]

\begin{aufgabe}
Finde die letzten drei Ziffern der beiden Zahlen $2^{1000}$ und
$3^{1000}$~!
\end{aufgabe}
(Antwort, zum Vergleich: 376 und 001)
\medskip

Eine weitere Besonderheit des Rechnens mit Kongruenzen beruht auf der
Tatsache, {\em dass es nur endlich viele verschiedene Klassen von
Resten gibt.}  Teilbarkeitsaussagen kann man deshalb oft durch eine
Fallunterscheidung beweisen, wie in der folgenden Aufgabe.
\begin{aufgabe}
    Zeige, dass eine Quadratzahl bei Division durch 4 nur den Rest 0 oder
    1 lassen kann~!
  \end{aufgabe}
Eine Quadratzahl hat immer die Gestalt $a^2$ mit einer nat"urlichen
Zahl $a\in {\N}$. Da es bei ihrem Rest $(mod\ 4)$ nur auf den Rest
von $a$ ankommt, k"onnen wir die Aussage durch vollst"andige
Fallunterscheidung (in Tabellenform) l"osen:
\[\begin{array}{c|c}
a\pmod{4} & a^2\pmod{4}\\\hline 0&0\\ 1&1\\ 2&4\equiv 0\\
3&9\equiv 1\\ \end{array}
\]
\begin{aufgabe}
Zeige, dass die Summe zweier ungerader Quadratzahlen niemals eine
Quadratzahl sein kann.
\end{aufgabe}
\begin{aufgabe}
Beweise folgende Aussage: Ist die Summe zweier Quadratzahlen durch 3
teilbar, so auch jeder der beiden Summanden.
\end{aufgabe}

\begin{attribution}
graebe (2004-09-02):\\ Dieses Material wurde vor einiger Zeit als
Begleitmaterial f�r den LSGM-Korrespondenzzirkel in der Klasse 7 erstellt und
nun nach den Regeln der KoSemNet-Literatursammlung aufbereitet.
\end{attribution}

\end{document}
