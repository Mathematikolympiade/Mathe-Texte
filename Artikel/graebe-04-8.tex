% Version: $Id: graebe-04-8.tex,v 1.1 2008/09/05 09:52:58 graebe Exp $
\documentclass[11pt]{article}  
\usepackage{kosemnet,ko-math,ngerman}  

\author{Hans-Gert Gr�be, Leipzig}
\title{"Ahnlichkeit von Dreiecken\kosemnetlicensemark\\ 
Arbeitsmaterial f�r Klasse 8}
\date{}

\begin{document} 
\maketitle 


Zwei Dreiecke $\ktriangle{ABC}$ und $\ktriangle{A'B'C'}$ hei"sen {\em
"ahnlich}, wenn sie, grob gesprochen, {\glqq}in ihrer Form, aber nicht
notwendig in ihrer Gr"o"se "ubereinstimmen{\grqq}.

Genauer: Die beiden Dreiecke hei"sen "ahnlich, wenn man das Dreieck
$\ktriangle{ABC}$ durch eine Bewegung und eine nachfolgende Streckung in das
andere Dreieck $\ktriangle{A'B'C'}$ "uberf"uhren kann.  Da bei einer Streckung
Geraden in parallele Geraden "ubergehen, muss die anf"angliche Bewegung (also
etwa eine Drehung) das Dreieck $\ktriangle{ABC}$ in ein Dreieck
$\ktriangle{A''B''C''}$ "uberf"uhren, dessen Seiten zu denen des Dreiecks
$\ktriangle{A'B'C'}$ jeweils parallel sind.

Zwei Dreiecke, in denen einander entsprechende Seiten zueinander
parallel sind, hei"sen {\em in "Ahnlichkeitslage liegend}.  Es stellt
sich heraus, dass man zwei solche Dreiecke stets durch eine Streckung
ineinander "uberf"uhren kann.  Das Streckungszentrum findet man, indem
man einander entsprechende Eckpunkte der beiden Dreiecke verbindet.
Die drei so entstehenden Verbindungsgeraden gehen alle durch einen
gemeinsamen Punkt (was man nat"urlich beweisen muss; das werde ich
hier aber nicht tun).
\medskip

Zwei Dreiecke sind insbesondere genau dann "ahnlich, wenn sie in den
drei Innenwinkelgr"o"sen "ubereinstimmen. Da die Innenwinkelsumme im
Dreieck $180\grad$ betr"agt, gen"ugt es zu zeigen, dass zwei der drei
Innenwinkel dieselbe Gr"o"se haben. Die dritten Innenwinkel stimmen
dann automatisch "uber\-ein. Diesen "Ahnlichkeitssatz bezeichnet man
auch kurz als (ww). 

Da "ahnliche Dreiecke durch eine Bewegung und eine Streckung ineinander
"ubergehen, unterscheiden sich die L"angen von Original und Bild einer Strecke
alle um denselben Faktor (den Streckungsfaktor) $v$. Es gilt also
\[\frac{\msegment{AB}}{\msegment{A'B'}} =
\frac{\msegment{AC}}{\msegment{A'C'}} =
\frac{\msegment{BC}}{\msegment{B'C'}} = v.
\] 

\begin{attribution}
graebe (2004-09-03):\\ Dieses Material wurde vor einiger Zeit als
Begleitmaterial f�r den LSGM-Korrespondenzzirkel in der Klasse 8 erstellt und
nun nach den Regeln der KoSemNet-Literatursammlung aufbereitet.
\end{attribution}

\end{document}
