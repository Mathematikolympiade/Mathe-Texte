\documentclass[12pt]{article}
\usepackage{kosemnet,ko-math,ngerman,url}
\title{Aufgabensammlung zur vollst\"andigen Induktion \\ Winterschule
  Colditz, Februar 2004\kosemnetlicensemark} 
\author{Jelena Djoki\'c, MPI MIN Leipzig,
\url{mailto:Jelena.Djokic@mis.mpg.de}}
\date{25.2.2004}

\begin{document}
\maketitle
\begin{enumerate}
\item Man beweise, dass  f�r alle $n\in \N$ gilt:
\begin{enumerate}
\item $1+2+3+\ldots+n = \frac{n(n+1)}{2}$
\item $1+3+5+\ldots+(2n-1)=n^2$
\item $1+3+6+\ldots+\frac{n(n+1)}{2}=\frac{n(n+1)(n+2)}{6}$
\item $1^2+2^2+3^2+\ldots+n^2=\frac{n(n+1)(2n+1)}{6}$
\item $1^3+2^3+3^3+\ldots+n^3=\left[\frac{n(n+1)}{2}\right]^2$
\item $1^4+2^4+3^4+\ldots+n^4=\frac{n(n+1)(2n+1)(3n^2+3n-1)}{30}$
\item $2+7+14+\ldots+(n^2+2n-1)=\frac{n(2n^2+9n+1)}{6}$
\item $1\cdot 2\cdot 3 + 2\cdot 3\cdot 4 +\ldots+ n(n+1)(n+2)
  =\frac{n(n+1)(n+2)(n+3)}{4}$ 
\end{enumerate}
\item Man beweise, dass f�r alle $ n \in \N$ gilt:
\begin{enumerate}
\item $\frac{1}{1\cdot 2}+\frac{1}{2\cdot 3}+\ldots+\frac{1}{n\cdot (n+1)} =
  \frac{n}{n+1}$ 

\item $\frac{1}{1\cdot 2\cdot 3}+\frac{1}{2\cdot 3\cdot 4}+ \ldots
  +\frac{1}{n\cdot (n+1) \cdot (n+2)} = \frac{1}{4}\frac{n(n+3)}{(n+1)(n+2)}$

\item $\frac{3}{1\cdot 2}+\frac{7}{2\cdot 3}+\ldots+\frac{n^2+n+1}{n\cdot
  (n+1)} = \frac{n(n+2)}{n+1}$ 

\item $\frac{5}{1\cdot 2}+\frac{13}{2\cdot 3}+\ldots+\frac{2n^2+2n+1}{n\cdot
  (n+1)} = \frac{n(2n+3)}{n+1}$ 

\item $\frac{2}{2+1}+\frac{2^2}{2^2+1}+\ldots+\frac{2^n}{2^{2^{n-1}}+1} =
  2-\frac{2^{n+1}}{2^{2^n-1}}$ 

\item $\frac{1^2}{1\cdot 3}+\frac{2^2}{3\cdot 5}+\ldots+\frac{n^2}{(2n-1)\cdot
  (2n+1)} = \frac{n(n+1)}{2(2n+1)}$ 

\item $\frac{3}{4}+\frac{5}{6}+\ldots+\frac{2n+1}{n^2(n+1)^2} = \frac{n}{n+1}$ 

\item $1-\frac{1}{2} + \frac{1}{3}-\frac{1}{4} +\ldots
  +\frac{1}{2n-1}-\frac{1}{2n} =\frac{1}{n+1} +\frac{1}{n+2} +\ldots
  +\frac{1}{2n}$
\end{enumerate}
\item Mit Hilfe der vollst\"andigen Induktion beweise man:
\begin{enumerate}
\item $1 \cdot 1! +2 \cdot 2!+ \ldots + n\cdot n! = (n+1)!-1$
\item $\frac{1}{2!}+\frac{2}{3!} + \frac{3}{4!} +\ldots +\frac{n-1}{n!} = 1 -
  \frac{1}{n!}$ 
\end{enumerate}

\item Mit Hilfe der vollst\"andigen  Induktion beweise man:
\begin{enumerate}
\item
  $\br{1-\frac{1}{4}}\cdot\br{1-\frac{1}{9}}\cdots\br{1-\frac{1}{n^2}}
  =\frac{n+1}{2n}$, $n \geq 2$  
\item
  $\br{1-\frac{4}{1}}\cdot \br{1-\frac{4}{9}}\cdots \br{1-\frac{4}{(2n+1)^2}}
  =\frac{1+2n}{1-2n}$, $n \in \N$
\item $\frac{7}{9} \cdot \frac{26}{28} \cdots \frac{n^3-1}{n^3+1} =
  \frac{2}{3}\br{1+ \frac{1}{n(n+1)}}$, $n \geq 2$
\end{enumerate}
\item Man beweise, dass f�r alle $n \in \N_0$ gilt:
\begin{enumerate}
\item $3|5^n+2^{n+1}$
\item $133|11^{n+2}+12^{2n+1}$
\item $19|7\cdot 5^{2n}+12 \cdot 6^n$
\item $17|6^{2n}+19^n-2^{n+1}$
\item $59|5^{n+2}+26\cdot 5^n+8^{2n+1}$ 
\item $11|30^{n}+4^n(3^n-2^n)-1$
\item $676| 3^{3n+1}-26n-27$
\item $19| 2^{2^{6n+2}}+3$
\item $9 | n4^{n+1}-(n+1)\cdot 4^{n}+1$
\item $84| 4^{2n}-3^{2n}-7$, $n \geq 1$
\item $11| 5^{5n+1}+4^{5n+2}+3^{5n}$
\end{enumerate}
\item Man beweise die folgenden Aussagen.
\begin{enumerate}
\item Alle Zahlen der Form $2^{2^n}+1$, $n \geq 2$ haben \textbf{7} als letzte
 Ziffer.
\item Alle Zahlen der Form $2^{4^n}-5$, $n \geq 1$ haben \textbf{1} als letzte
 Ziffer.
\end{enumerate}
\item Mit Hilfe der vollst\"andigen Induktion beweise man folgende
Ungleichungen:
\begin{enumerate}
\item $\frac{1}{n+1}+\frac{1}{n+2}+\ldots+\frac{1}{2n}>\frac{1}{2}$, $n \geq
  2$ 
\item$\frac{1}{n+1}+\frac{1}{n+2}+\ldots+\frac{1}{2n}>\frac{13}{24}$, $n \geq
  2$ 
\item$\frac{1}{n+1}+\frac{1}{n+2}+\ldots+\frac{1}{3n+1}>1$
\item $2+2^2+2^3+\ldots 2^{2n} <n(2^{n+1}+1)$
\end{enumerate}
\item Man beweise, dass die folgende Ungleichungen gelten:
\begin{enumerate}
\item $2^n>n^2$, $n \geq 5$
\item $n! > 2^n$, $n \geq 4$
\item $n! < n^{n-1}$, $n \geq 3$
\item $\frac{4^n}{n+1}<\frac{(2n)!}{(n!)^2}$, $n \geq 2$
\item $(1+h)^n\ge  1+nh$, $n \in \N$, $h \in \R$, $ h >-1$
\item $\frac{3\cdot7\cdot 11 \cdots (4n-1)}{5 \cdot 9 \cdot 13 \cdots (4n+1)}
  <\sqrt{\frac{3}{4n+3}}$
\item $\frac{1}{\sqrt{1}}+\frac{1}{\sqrt{2}}+\ldots+\frac{1}{\sqrt{n}}
  >2(\sqrt{n+1}-1)$, $n\ge 2$
\item $\frac{1}{\sqrt{1}}+\frac{1}{\sqrt{2}}+\ldots+\frac{1}{\sqrt{n}}
  >\sqrt{n}$, $n \geq 2$
\item $1+2+2^2+\ldots +2^n>\frac{n+1}{n-1}(2+2^2+\ldots+2^{n-1})$, $n \geq 2$
\end{enumerate}
\item Mit Hilfe der vollst\"andigen Induktion beweise man:
\begin{enumerate}
\item Wenn $x_1=1, x_2=2$ und $x_n=(n-1)(x_{n-1}+x_{n-2})$ f�r alle $n\ge 3$,
  dann gilt $x_n=n!$ f�r alle $n$.
\item Wenn $x_0=2, x_1=5$ und $x_{n+2}=5x_{n+1}-6x_{n}$ f�r alle $n\ge 0$,
  dann gilt $x_n=2^n+3^n$ f�r alle $n$.
\item Wenn $a_0=2, a_1=3$ und $a_{n+1}=a_1a_{n}-a_0a_{n-1}$ f�r alle $n\ge 1$,
  dann gilt $a_n=2^n+1$ f�r alle $n\ge 0$.

\item Wenn $a_1=5$, $a_2=7$ und $a_{n+1}=2a_{n}-a_{n-1}$ f�r alle $n\ge 2$,
 dann gilt $a_n=2n+3$ f�r alle $n$.

\item Wenn $a_0=1, a_1=4$ und $a_{n+2}=4a_{n+1}-4a_{n}$ f�r alle $n$, dann
  gilt $a_n=2^n+n2^n$ f�r alle $n$.
\end{enumerate}
\item Mit Hilfe der vollst\"andigen Induktion beweise man:
\begin{enumerate}
\item 
  $\sin \alpha +\sin 2\alpha+\ldots+\sin n\alpha =
  \frac{\sin\frac{(n+1)\alpha}{2} \sin \frac{n \alpha}{2}}{\sin
  \frac{\alpha}{2}}$, $\alpha \neq 2k \pi$, $k \in \Z$
\item 
  $\cos \alpha +\cos 2\alpha+\ldots+\cos n\alpha =
  \frac{\cos\frac{(n+1)\alpha}{2} \sin \frac{n \alpha}{2}}{\sin
  \frac{\alpha}{2}}$, $\alpha \neq 2k \pi$, $k \in \Z$
\item 
  $\frac{1}{\sin 2x}+\frac{1}{\sin 4x}+ \ldots + \frac{1}{\sin
  2^nx}=\frac{1}{\tan x}-\frac{1}{\tan 2^nx}$, $x \neq \frac{\lambda
  \pi}{2^k}$, $k \in \N_0$, $\lambda \in \Z$
\end{enumerate}
\item Man beweise, dass f�r alle $n\ge 2$ die Zahl $\cos \frac{\pi}{2^n}$
  irrational ist.
\item Sei $a_1=1, a_{n+1}= \frac{2xa_n}{a_n+x}$, $n \geq 1, x>0$. Man beweise,
  dass $a_{n+2}= \frac{2^{n+1}x}{2^{n+1}+x-1}$, f"ur $n \geq 0$.
\item Sei $f(x)=\frac{x}{\sqrt{1+x^2}}$. Wir definieren $f^{\circ n} = f
{\circ} f {\circ} \cdots {\circ} f$, ($n$ mal), als die $n$-fache
Hintereinanderausf�hrung (Komposition) von $f$ mit sich selbst.  So ist etwa
$f^{\circ 2}(x)=f(f(x))$.  

Man beweise, dass f�r alle $n\in\N$ gilt
\begin{gather*}
  f^{\circ n}(x) = \frac{x}{\sqrt{1+nx^2}}. 
\end{gather*}
\item In einer Ebene sind $n$ Geraden gegeben.  Man beweise, dass die Ebene
 durch diese Geraden in h\"ochstens $2^n$ Teile aufgeteilt wird.
\item Man beweise, dass $n$ Kreise eine Ebene in h\"ochstens $n^2-n+2$ Teile
  teilen.
\item Man beweise: Wird die Ebene durch Geraden in Gebiete aufgeteilt, so kann
  man diese Gebiete derart rot oder blau f�rben, dass benachbarte Gebiete
  unterschiedlich gef\"arbt sind.
\end{enumerate}

\begin{attribution}
schueler (2005-04-29): Contributed to KoSemNet
\end{attribution}

\end{document}
