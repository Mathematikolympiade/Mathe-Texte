%%%%%%%%%%%%%%%%%%%%%%%%%%%%%%%%%%%%%%%%%%%%%%%%%%%%%%%%%%%%%%%%%%%%%%%%%%%
%  Abzähltheorie in Anlehnung an Känguru der Mathematik 2006, Klasse 11/13,
%  Aufgabe 26: Werden die Seitenflächen eines Würfels mit 6 Farben gefärbt, so
%  gibt es 30 Möglichkeiten, die nicht durch Drehung ineinander über gehen.
%%%%%%%%%%%%%%%%%%%%%%%%%%%%%%%%%%%%%%%%%%%%%%%%%%%%%%%%%%%%%5

\documentclass[12pt,a4paper]{article}
\usepackage{schueler}
\usepackage{kosemnet,ko-math}
\usepackage[utf8]{inputenc}    

\newcommand{\D}{\mathrm{D}}

\title{Abzählen von Färbungen ---\\ Das Cauchy-Frobenius- oder
  Burnside-Lemma\kosemnetlicensemark} 
\author{Axel Schüler}
\date{Leipzig, 2006}

\begin{document}
\maketitle

Wir wollen hier das Burnside Lemma formulieren und beweisen.  Dann wollen wir
es anwenden auf das Abzählen von Färbungen eines symmetrischen Körpers.
Hierbei wollen wir nicht den Begriff des Zyklenzeigers verwenden.

\section{Bezeichnungen}
Im folgenden sei $X$ eine Menge, $G$ eine endliche Gruppe, die auf $X$ wirkt,
das heißt, jedem Gruppenelement $g\in G $ entspricht eine Abbildung,
bezeichnet durch $g.$, von $X$ in $X$ mit ${e.x=x}$ und $(gh).x=g.(h.x)$ für
alle $g,h\in G $ und alle $x\in X$.  Für jedes $g\in G $ bezeichne
${X^g=\{x\in X\mid g.x=x\}}$, die Teilmenge der Elemente von $X$, die unter
der Wirkung von $g$ fest bleiben.  Für jedes $x\in X$ sei $G_x=\{g\in G\mid
g.x =x\}$ der {\bf Stabilisator} von $x$ ($G_x $ ist eine Untergruppe in $G$)
und $G(x)=\{g.x\mid g\in G\}$ das {\bf Orbit} von $x$ unter $G$.  Schließlich
sei $X/G$ die Menge der Orbits der Gruppenwirkung.  Die Anzahl der Elemente
einer Menge $A$ sei mit $\abs{A} $ bezeichnet.
\begin{lemma}\label{l-1}
Es sei $G$ eine endliche Gruppe, die auf der Menge $X$ wirkt. Dann 
gilt für jedes $x\in X$
\begin{align*}
\abs{G}=\abs{G(x)} \cdot \abs{G_x}.
\end{align*}
\end{lemma}

\begin{beweis}
Wir benutzen das doppelte Abzählen der Relation $S=\{(g,y)\in G\times X\mid
y=g.x\}$. Klar ist, für jedes $g\in G$ gibt es genau ein $y\in X $ mit $y=g.x$
und somit ist $\abs{S}=\abs{G}$. Umgekehrt gibt für $y\in G(x)$, etwa $y=g_0.x
$, genau $\abs{G_x} $ Elemente $h\in G$ mit $h.x=y$. Das sind nämlich genau
die Elemente $h= g_0g $ mit $g\in G_x$. Liegt $y$ nicht im Orbit von $x$, so
gibt es kein $(g,y)\in S$. Folglich ist
\begin{gather*}
  \abs{S}=\abs{G} = \sum_{y\in G(x)} \abs{G_x} =\abs{G(x)}\,\abs{G_x}.
\end{gather*}
\end{beweis}

\begin{lemma}\label{l-stabilizer}
Unter den Voraussetzungen von Lemma\,\ref{l-1} gilt
\begin{align}\label{e-s-o}
\sum_{g\in G} \abs{X^g} =\sum_{x\in X} \abs {G_x}.
\end{align}
\end{lemma}

\begin{beweis}
Wir benutzen das doppelte Abzählen für die Relation $T=\{(g,x)\in G\times
X\mid g.x = x\}$.  Für ein festes $h\in G$ ist $\{(h,x)\mid x\in X^{h}\}$
gerade die Menge der Paare in $T$ mit erster Koordinate gleich $h$. Umgekehrt
ist für ein festes $z\in X$ die Menge der Paare in $T$ mit zweiter Koordinate
gleich $z$ gerade $\{(g,z)\mid g\in G_z\}$. Folglich gilt
\begin{gather*}
  \sum_{h\in G} \abs{X^h} =\abs{T} =\sum_{z\in X} \abs{ G_z}.
\end{gather*}
\end{beweis}

\begin{lemma}[Lemma von Burnside]
\label{l-burnside}
Für die Anzahl der Orbits $\abs{X/G}$ der Gruppenwirkung von $G$ auf $X$ gilt
\begin{align}\label{e-burn}
\abs{X/G} =\frac{1}{\abs{G} } \sum_{g\in G} \abs{X^g}.
\end{align}
\end{lemma}

\begin{beweis} 
Wir benutzen die Formel \rf[e-s-o] und sortieren die Summanden auf der rechten
Seite nach gleichen Stabilisatoren; insbesondere haben alle Elemente $y\in
G(x)$ eines Orbits gleich große Stabilisatoren $G_x$. In der Tat, wenn $g\in
G_x$ und $y=g_0. x$, dann ist $g.x=x $ und damit $(g_0 g).x =g_0.x$, also
$(g_0g g_0^{-1}).(g_0.x ) = g_0.x$. Durchläuft also $g$ den Stabilisator
$G_x$, so durchläuft $g_0 G_x g_0^{-1} $ den Stabilisator von $y=g_0.x$.
Diese Mengen sind aber gleich groß, sie haben $\abs{G_x} $ Elemente.

Nach Lemma\,\ref{l-1} ist $\abs{G_x}= \abs{G} /\abs{G(x)} $ und weiter:
\begin{gather*}
  \sum_{h\in G} \abs{X^ h} =\sum_{A\in X/G} \sum_{x\in A} \abs{G_x} = \sum
  _{A\in X/G} \abs{A}\,\frac{\abs{G}}{\abs{A}}= \sum_{A\in X/g} \abs{G}
  =\abs{X/G}{\cdot } \abs{G}.
\end{gather*}
Teilt man durch $\abs{G}$, so erhält man die Behauptung.
\end{beweis}

\begin{bemerkung}
William Burnside schrieb diese Formel um 1900 auf. Mathematikhistoriker jedoch
fanden diese Formel auch schon bei Cauchy (1845) und Frobenius (1887). Daher
wird diese Formel auch mitunter bezeichnet als die {\em Formel, die nicht von
  Burnside stammt}.
\end{bemerkung}


\section{Anwendungen}
\begin{beispiel}
Wir wählen als $G$ die drei Drehungen eines Würfels $ABCDEFGH$ um die
Raumdiagonale $AG$ und $X$ sei die Eckenmenge des Würfels. Dann ist $X^{e}=
E$, $X^{D_{2\pi/3}} = X^{D_{4\pi/3}} =\{A,G\}$. Folglich ist die Anzahl der
Orbits gleich
\begin{gather*}
  \abs{X/G} =\frac{1}{3} \left( 8 + 2 +2 \right)= 4.
\end{gather*}
In der Tat sind die vier Orbits gleich $\{A\}$, $\{G\}$, $\{B,E,D\} $ und
$\{C, F, H\}$.
\end{beispiel}

\begin{aufgabe}[Känguru-Wettbewerb 2006, Klasse 11/13, A26]
Auf wie viele Arten lassen sich die sechs Flächen eines Würfels mit sechs
verschiedenen Farben färben?  Jede Farbe tritt dabei genau einmal auf. Zwei
Färbungen heißen gleich, wenn sie durch Drehung ineinander überführt werden
können.
\end{aufgabe}

\begin{loesung} 
Es sei $G=S_4$ die Würfelgruppe bestehend aus den 24 Drehungen des Würfels und
$X$ die Menge der $6!$ verschiedenen Färbungen der Seitenflächen. Bei jeder
{\em echten} Drehung bleibt keine Färbung fixiert, also ist $X^g=\emptyset $
für alle $g\ne e$. Für das Einselement hingegen ist $X^e=X$. Somit gibt es
\begin{gather*}
  \abs{X/G} =\frac{\abs{X}}{\abs{G}} = \frac{6!}{4!} =30
\end{gather*}
verschiedene Färbungen.
\end{loesung}

\begin{aufgabe} 
Bestimme alle möglichen Färbungen der Würfelecken mit rot und blau, wobei zwei
Ecken rot und sechs Ecken blau gefärbt sein sollen.
\end{aufgabe}

\begin{loesung}
Es ist klar, dass es nur drei solche Färbungen geben kann, denn die beiden
roten Ecken können entweder benachbart sein, oder sie sind flächendiagonal
gegenüberliegend oder raumdiagonal gegenüberliegend.  Wir wollen trotzdem das
Burnside Lemma anwenden.  Wieder ist $G=S_4$ die Würfelgruppe und $X$ die
Menge der $\binom{8}{2} = 28$ rot-blau-Färbungen der Eckenmenge.  Bei jeder
$\pm 90^\circ$-Drehung bleibt keine Färbung fix. Bei jeder der $9$ Drehungen
um $180^\circ $ bleiben immer vier Färbungen fix und zwar gibt es immer vier
Paare von Ecken, die gegenseitig auf sich abgebildet werden, immer ein Paar
kann man als rotes Paar auswählen.  Schließlich gibt es bei den acht $\pm
120^\circ $-Drehungen jeweils eine fest bleibende Färbung, wobei die beiden
diagonalen Punkte auf der Drehachse rot gefärbt sind.  Folgich erhalten wir
als Anzahl der Orbits
\begin{gather*}
  \frac{1}{24}\left( 28 +  9\cdot 4 + 8\cdot 1\right) = 3.
\end{gather*}
\end{loesung}

\begin{aufgabe}  
Bestimme die Anzahl aller möglichen Färbungen der Würfelflächen mit drei
Farben.
\end{aufgabe}

\begin{loesung}
Es sei $G=S_4$ die Würfelgruppe und $X$ die Menge aller $6^3 $ Färbungen der
sechs Würfelflächen mit drei Farben. Wir betrachten die Fixelemente der
einzelnen Würfeldrehungen:
\begin{center}
\begin{tabular}{|l||cl|l|}
\hline {\bf Drehung $g$} & {\bf konj. El.}  & {\bf einfarbig} & {\bf
  $\abs{X^g}$} \\\hline\hline 
Identität & 1 & egal &$3^6$ \\ \hline 
Drehung um $90^\circ $ & 6 & Deckel(1) , Boden(1), Mantel(4) & $3^3$ \\\hline
Drehung um $180^\circ $ um Achsen & 3 & Deckel(1), Boden(1), M(2), M(2)& $3^4$
\\\hline Drehungen um $120^\circ$ & 8 & Deckel(3), Boden(3) & $3^2$\\\hline
Drehungen um $180^\circ $ & 6 & Deckel(2), Mantel(2) , Boden(2) & $3^3$
\\\hline\hline
\end{tabular}
\end{center}
Die Anzahl der Färbungen beträgt daher 
\begin{gather*}
  \frac{1}{24} \left( 3^6 +6\cdot 3^3 + 3\cdot 3^4 + 8\cdot 3^2 + 6\cdot
  3^3\right) =57.
\end{gather*}
\end{loesung}

\bilda{0.2}{0.7}{loch}{
\begin{aufgabe}[Lochkartenproblem] 
Aus quadratischen $3\times 3$-Kärtchen sollen durch Ausstanzen von genau zwei
Löchern Identitätskarten hergestellt werden.  \\ Wie viele unter Drehen und
Wenden verschiedene Karten lassen sich herstellen?
\end{aufgabe}
}

\medskip

\begin{loesung} 
Es ist $G=D_4$ -- die Diedergruppe mit $\abs{D_4}=8 $ Elementen.  Der
Konfigurationsraum $X$ besteht aus den $\binom{9}{2}= 36$ möglichen
Kombinationen von zwei Löchern auf $9$ Feldern.  In $G$ gibt es fünf
verschiedene Klassen von Elementen: die Identität $e$, zwei Drehungen $d_+$
und $d_-$, um $90^\circ $ bzw. um $-90^\circ$, eine Drehung $d_2$ um
$180^\circ$, zwei Spiegelungen an achsenparallenen Geraden und zwei
Spiegelungen an Diagonalen.

Wie immer lässt die Identität alle $\abs{X}=36 $ Elemente fix, wogegen die
beiden Drehungen $d_\pm $ gar keine Konfiguration fest lassen, da genau ein
Kästchen in sich geht und alle anderen 8 Kästchen erst nach vier Drehungen
wieder an ihrem Ort sind.

\smallskip

\begin{minipage}{0.4\textwidth}
\bilda{0.2}{0.7}{lochDrehung}{Bei der $180^\circ$-Drehung bleiben genau die
  nebenstehenden vier Muster fest}
\end{minipage}
\quad
\begin{minipage}{0.55\textwidth}
\bilda{0.4}{0.5}{lochSpiegelung}{Bei den Diagonal- und Achsenspiegelungen
  bleiben jeweils $6$ Muster fix.}
\end{minipage}

\medskip

\begin{minipage}{0.55\textwidth}
Damit ergeben sich nach dem Lemma von Burnside
\begin{gather*}
  \abs{X/G}= \frac{1}{8}\left(36 +4 +4\cdot 6\right) =\frac{64}{8}= 8
\end{gather*}
verschiedene Muster.  Dies sind genau die neben stehenden.
\end{minipage}
\quad 
\begin{minipage}{0.4\textwidth}
\bilda{0.6}{0.4}{loch1}{}
\end{minipage}
\end{loesung}

\begin{aufgabe}\label{a-kette1}
Wie viele verschiedene Perlenketten lassen sich aus zwei roten, 
drei blauen und vier grünen Perlen zusammenstellen, wenn
\begin{itemize}
\item [a)] die Ketten einen Anfang und ein Ende haben (lineare Kette) bzw.\
\item [b)] die Ketten geschlossen sein sollen (zyklisch). 
\end{itemize}
\end{aufgabe}

\begin{loesung}
Der Konfigurationsraum $X$ besteht aus allen 
\begin{gather*}
  \frac{ 9!}{2!3!4!}  =\frac{9\cdot 8\cdot 7\cdot 6\cdot 5}{2\cdot 3\cdot 2}=
  63 \cdot 20= 1260
\end{gather*}
Permutationen von 9 Elementen mit Wiederholung.

a) Die Symmetriegruppe ist $G=\Z_2$, denn man kann nur Anfang und Ende der
Kette vertauschen --- das sei $g\in \Z_2$.  Wir ermitteln die Mutser, die unter
$g$ fest bleiben.  Da eine ungerade Anzahl von blauen Perlen existieren, muss
die mittelste Perle blau sein und die vier ersten Positionen mit zwei grünen
und jeweils einer blauen und roten Perle besetzt sein. Dafür gibt es
$\frac{4!}{1!1!2!}= 12 $ Möglichkeiten. Nach Frobenius gibt es also
\begin{gather*}
  \half\left( 1260 + 12\right)= \frac{1272}{2} = 636
\end{gather*}
verschiedene Ketten.

b) Bei einer geschlossenen Kette ist die Symmetriegruppe die $D_9$ mit 18
Elementen -- 9 Drehungen und 9 Spiegelungen. Da es keine Drehung der Ordnung 2
gibt, kann keine Anordnung der zwei roten Kugeln stabilisiert werden.  Bei den
9 Spiegelungen muss in der Mitte (auf der Spielgelungsachse) wieder eine blaue
Perle sein und rechts von der Achse wieder zwei grüne, eine blaue und eine
rote Perle, was wie in a) 12 Anordnungen liefert. Demnach gibt es
\begin{gather*}
  \frac{1}{18} \left(1260 + 9 \cdot 12\right)=\frac{9\cdot 7\cdot 20}{2\cdot
    9} + 6= 76
\end{gather*}
verschiedene Muster für geschlossenen Ketten.
\end{loesung}

\nocite{b-PolyaRead}
\nocite{b-PolyaTarjan}
\nocite{a-Polya}
\nocite{b-Burnside}

\providecommand{\bysame}{\leavevmode\hbox to3em{\hrulefill}\thinspace}
\begin{thebibliography}{1}

\bibitem{b-Burnside}
W.~Burnside, \emph{Theory of Groups of Finite Order}, 2 ed., Dover
  Publications, New York, 1955.

\bibitem{b-PolyaTarjan}
R.~Tarjan G.~P{\'o}lya and D.~R. Woods, \emph{Notes on introductory
  combinatorics}, Progress in Computer Science, no.~4, Birkh{\"a}ser, Boston,
  1983.

\bibitem{a-Polya}
G.~P{\'o}lya, \emph{Kombinatorische {A}nzahlbestimmungen f{\"u}r {G}ruppen,
  {G}raphen und chemische {V}erbindungen}, Acta Math. \textbf{68} (1937),
  145--254.

\bibitem{b-PolyaRead}
G.~P{\'o}lya and R.~C. Read, \emph{Combinatorial enumeration of groups, graphs,
  and chemical compounds}, Springer-Verlag, New York, 1987.

\end{thebibliography}

Siehe auch 
\begin{itemize}
\item \url{http://lsgm.uni-leipzig.de/KoSemNet/pdf/graebe-05-2.pdf}
\item \url{http://encyclopedia.thefreedictionary.com/Cauchy-Frobenius%20lemma}
\end{itemize}

\end{document}



