\documentclass[11pt,a4paper]{article}  
\usepackage{kosemnet,ko-math,ngerman} 
\usepackage[utf8]{inputenc}  

\author{Hans-Gert Gräbe, Leipzig}
\title{Ein Van-der-Monde-artiges Theorem\kosemnetlicensemark}
\date{11. Januar 2025}

\begin{document} 
\maketitle 

\begin{abstract}
  Ziel der vorliegenden Arbeit ist die Berechnung einer Determinante, die der
  van-der-Mondeschen ähnelt. Zwei Anwendungen auf Gausszahlen und
  Binomialkoeffizienten werden diskutiert.  Der Text setzt Bekanntheit mit
  dem Determinantenbegriff voraus. 

  NB: Die Arbeit entstand wohl Ende der 1980er Jahre und fand sich 2025 in
  einer englischen Version beim Aufräumen wieder. Ich habe den Text nun
  digital erfasst und ins Deutsche übertragen.
\end{abstract}

\section{Der Satz von van der Monde und eine Verallgemeinerung}

$a_i$, $i=1,2,\dots,n$, seien $n$ Unbestimmte und $a_{ij}=a_j^{i-1},
$i,j=1,2,\dots,n$. 
\begin{theorem}{Satz von van der Monde}
  Für die Matrix $A_n=\begin{pmatrix} a_{ij} \end{pmatrix}$ gilt   
  \begin{gather*}
    \mathrm{det}(A_n)=\prod_{1\le i<j\le n}\br{a_j-a_i}.
  \end{gather*}
\end{theorem}
\emph{Beweis:} $\mathrm{det}(A_n)$ ist ein Polynom in $a_1,a_2,\dots,a_n$. Für
$a_i=a_j$ stimmen die Spalten $i$ und $j$ von $A_n$ überein und
$\mathrm{det}(A_n)$ verschwindet in diesem Fall.  Also ist $\mathrm{det}(A_n)$
durch $a_j-a_i$ teilbar für alle $1\le i<j\le n$. Da diese Faktoren (als
lineare Faktoren) paarweise teilerfremd sind, ergibt sich
\begin{gather*}
  \mathrm{det}(A_n)=\prod_{1\le i<j\le n}\br{a_j-a_i}\m P_n\tag{1}
\end{gather*}
für ein Polynom $P_n=P_n(a_1,a_2,\dots,a_n)$.  Vergleich der $a_i$-Grade auf
den beiden Seiten von (1) zeigen, dass $P_n$ ein konstantes Polynom ist.  Der
Vergleich der Koeffizienten vor $a_n^{n-1}$ auf den beiden Seiten von (1)
zeigt, dass $P_n=1$ ist.\hfill $\Box$

\subsection{Eine Verallgemeinerung des Satzes von van der Monde}

Seien nun $a_i,b_k$, $i,k=1,2,\dots,n$, $2n$ Unbestimmte und 
\begin{gather*}
  a_{ij}:=\prod_{k=1}^{i-j}\br{a_j-b_k},\quad i,j=1,2,\dots,n.
\end{gather*}
Das leere Produkt setzen wir dabei wie gewöhnlich per Definition gleich 1.

\begin{theorem}
  Für die Matrix $A_n=\begin{pmatrix} a_{ij} \end{pmatrix}$ gilt   
  \begin{gather*}
    \mathrm{det}(A_n)=\prod_{1\le i<j\le n}\br{a_j-a_i}.
  \end{gather*}
\end{theorem}
Der Satz von van der Monde ergibt sich hieraus sofort für
$b_1=b_2=\dots=b_n=0$, 

\emph{Erster Beweis:} Wir argumentieren wie oben. $\mathrm{det}(A_n)$ ist nun
ein Polynom in $a_1,a_2,\dots,a_n$ und $b_1,b_2,\dots,b_n$. Für $a_i=a_j$
stimmen die Spalten $i$ und $j$ von $A_n$ wieder überein und wir schließen wie
oben, dass $\mathrm{det}(A_n)$ für alle $1\le i<j\le n$ durch $a_j-a_i$
teilbar ist. Wie oben ergibt sich
\begin{gather*}
  \mathrm{det}(A_n)=\prod_{1\le i<j\le n}\br{a_j-a_i}\m P_n\tag{2}
\end{gather*}
für ein gewisses Polynom $P_n=P_n(a_1.\dots,a_n,b_1,\dots,b_n)$. Ein Vergleich
der $a_i$-Grade der beiden Seiten von (1) für $i=1,\dots,n$ zeigt wieder, dass
alle $a_i$-Grade von $P_n$ gleich null sind. Also ist diesmal $P_n$ zwar nicht
konstant, sondern ein Polynom nur in $b_1,\dots,b_n$.  Der Vergleich der
Koeffizienten von $a_n^{n-1}$ auf der linken und der rechten Seite von (1)
ergibt
\begin{gather*}
  P_n\m\prod_{1\le i<j<n}\br{a_j-a_i}=\mathrm{det}(A_{n-1}).
\end{gather*}
Ein Induktionsargument liefert $P_n=P_{n-1}$ und weiter $P_n=1$ für $n\ge 1$,
da offensichtlich $P_1=P_2=1$ gilt.  Das komplettiert den Beweis. \hfill
$\Box$

\emph{Zweiter Beweis:} Addiere $b_i$ Mal die Zeile $i$ von $\begin{pmatrix}
  a_{i,j}\end{pmatrix}$ zur Zeile $(i+1)$, $i=n-1,n-2,\dots,1$. Als Zeile
$(i+1)$ der neuen Matrix ergibt sich 
\begin{gather*}
  a_{i+1,j}^{(1)}=a_{i+1,j}+b_i\m a_{ij}
  =\prod_{k=1}^i\br{a_j-b_k}+b_i\m\prod_{k=1}^{i-1}\br{a_j-b_k} 
  =a_j\m \prod_{k=1}^{i-1}\br{a_j-b_k} = a_j\m a_{ij}.
\end{gather*}
Addiere nun $b_{i-1}$ Mal die Zeile $i$ der Matrix $\begin{pmatrix}
  a_{i,j}^{(1)}\end{pmatrix}$ zur Zeile $(i+1)$ $(i=n-1,n-2,\dots,2)$.  Als
Zeile $(i+1)$ der neuen Matrix ergibt sich
\begin{gather*}
  a_{i+1,j}^{(2)}=a_{i+1,j}^{(1)}+b_i\m a_{ij}^{(1)}
  =a_j\m\br{a_{ij}+b_{i-1}\m a_{i-1,j}} = a_j\m a_{ij}^{(1)} =a_j^2\m
  a_{i-1,j}\quad (i\ge 2).
\end{gather*}
Wiederholt man dasselbe Argument $(n-1)$ Mal, so wird die Matrix
$\begin{pmatrix} a_{i,j}\end{pmatrix}$ in die van-der-Mondesche Matrix
transformiert, ohne dass sich der Wert der Determinante ändert.  Damit folgt
die Behauptung aber aus dem originalen Satz von van der Monde. \hfill$\Box$

\section{Anwendungen}

\newcommand{\gauss}[2]{\ensuremath\sbr{\frac{#1}{#2}}}

Let be $a\in\R$, $n\in\N$ and $x$ an indeterminate. Define \emph{Gaussian
numbers} in the usual way as
\begin{gather*}
  \gauss{a,n}=\prod{i=1}^n\frac{x^{a+1-i}-1}{x^i-1}
\end{gather*}

\begin{folgerung}
  
\end{folgerung}


\end{document}
