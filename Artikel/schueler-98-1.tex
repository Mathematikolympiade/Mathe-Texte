\documentclass[11pt]{article}
\usepackage{kosemnet,ko-math,ngerman,url}

\title{Komplexe Zahlen und Geometrie\kosemnetlicensemark}
\author{Dr. Axel Sch"uler, Univ. Leipzig}
\date{M"arz 1998} 

% emlines.sty  M�rz 1990, Georg Horn / Eberhard Mattes.
%
% Makros zum Zeichnen von Linien mit beliebiger Steigung.
% Nur bei Verwendung der DVI-Treiber von Eberhard Mattes.
%
% Der Makro \emline#1#2#3#4#5#6 setzt an die Koordinaten (#1,#2) den
% Punkt #3 und an die Koordinaten (#4,#5) den Punkt #6. Diese beiden 
% Punkte werden dann verbunden.
%
% TeXcad erzeugt Aufrufe dieses Makros f�r mit eingeschalteter
% EMLines-Option gezeichnete Linien.
% (z.B. \emline{0.0}{0.0}{1}{17.0}{4.0}{2}).
%
% Der Makro \newpic#1 definiert den Makro \emline so, dass den
% Punktnummern ##3 und ##6 die Ziffer #1 vorangestellt wird.
% Dies ist notwendig, wenn mehr als ein Bild auf einer Seite
% des Dokuments eingefuegt wird.
%
\def\newpic#1{%
   \def\emline##1##2##3##4##5##6{%
      \put(##1,##2){\special{em:point #1##3}}%
      \put(##4,##5){\special{em:point #1##6}}%
      \special{em:line #1##3,#1##6}}}
%
% Standarddefinition von \emline herstellen
%
\newpic{}
%
% Beispiel: 
% \input bild1.pic
% \newpic{1} \input bild2.pic
%

\def\vp{\varphi}
\def\zq{\overline{z}}
\def\half{\frac{1}{2}}
\def\Re{{\rm Re}}
\def\Im{{\rm Im}}
\def\sqq{ \textstyle{  \frac{1}{\sqrt{2}} }}
\def\sqqq{ \textstyle{ \frac{1}{\sqrt{3}}  } }



\begin{document}

\maketitle

\begin{abstract}
  Ziel dieses Beitrages ist es, die komplexen Zahlen bei einfachen
  geometrischen Aufgaben einzusetzen. Besonderes Augenmerk gilt dabei der {\em
    Drehung} um einen Winkel $\vp$. Sie l"a"st sich durch Multiplikation mit
  $e^{\ii\vp}$ beschreiben.

Im ersten Teil wiederholen wir Grundeigenschaften der komplexen
Zahlen.

Im zweiten Teil "ubersetzen wir geometrische Begriffe, wie
Gerade, Kreis, L"ange einer Strecke, Teilungsverh"altnis und
Drehung um einen Winnkel in die Sprache der komplexen Zahlen.

Im dritten Teil werden Beispiele und Aufgaben betrachtet.
\end{abstract}

\subsection*{Komplexe Zahlen}

{\em Die kartesischen Koordinaten.}

Komplexe Zahlen sind Paare $(a,b)$ reeller Zahlen, kurz $a+b\ii$,
f"ur die unter Beachtung von $\ii^2=-1$ die folgenden
Rechenregeln gelten: $${(a+b\ii)\pm(c+d\ii)} ={a\pm c+(b\pm
d)\ii}$$ und gem"a"s dem Distributivgesetz
$$(a+b\ii){\cdot}(c+d\ii)=(ac-bd)+(ad+bc)\ii.$$ Es gelten die
Kommutativgesetze und Assoziativgesetze der Addition und der
Multiplikation. Man kann sich die komplexen Zahlen als Punkte der
Gau"sschen Zahlenebene vorstellen.  

\special{em:linewidth 0.4pt} \unitlength 1.00mm
\linethickness{0.4pt}
\begin{picture}(58.34,42.67)
\put(57.33,9.67){\vector(1,0){0.2}}
\emline{0.00}{9.67}{1}{57.33}{9.67}{2}
\put(52.00,26.00){\vector(3,1){0.2}}
\emline{10.67}{9.33}{3}{52.00}{26.00}{4}
\put(58.34,29.67){\makebox(0,0)[cc]{$a+b\ii$}}
\put(57.67,4.67){\makebox(0,0)[cc]{Re}}
\put(6.33,37.00){\makebox(0,0)[cc]{Im}}
\emline{52.00}{25.67}{5}{52.00}{9.33}{6}
\put(52.33,5.67){\makebox(0,0)[cc]{$a$}}
\put(6.00,25.33){\makebox(0,0)[cc]{$b\ii$}}
\put(29.00,19.67){\makebox(0,0)[cc]{$r$}}
\emline{25.00}{9.67}{7}{24.67}{12.67}{8}
\emline{24.67}{12.67}{9}{23.67}{14.33}{10}
\put(24.00,14.33){\vector(-1,2){0.2}}
\emline{24.67}{13.00}{11}{24.00}{14.33}{12}
\put(20.67,11.67){\makebox(0,0)[cc]{$\vp$}}
\put(10.00,42.67){\vector(0,1){0.2}}
\emline{10.00}{9.67}{13}{10.00}{42.67}{14}
\emline{10.00}{26.00}{15}{51.67}{26.00}{16}
\emline{10.00}{0.00}{17}{10.00}{9.33}{18}
\end{picture}
%\input gauss.pic

{\em Konjugation und Betrag.} 

Spiegelt man eine komplexe Zahl $z=a+b\ii$ an der reellen Achse,
so erh"alt man die {\em komplex konjugierte} Zahl
$\overline{z}=a-b\ii$.  Der {\em Betrag} von $z$ ist die L"ange der
Strecke vom Nullpunkt bis $z$.  Nach dem Satz des Pythagoras ist
also $|z|=\sqrt{a^2+b^2}$ oder $|z|^2= z\zq$. Offenbar ist $z$
reell gdw.{} $z=\zq$.  Den {\em Realteil} $a$ bzw.{} {\em
Imagin"arteil} $b$ der komplexen Zahl $z=a+b\ii$ erh"alt man mit
Hilfe der komplexen Konjugation wie folgt: 
$$\Re(z)=\half(z+\zq)\ \mbox{ bzw. }\
\Im(z)=\frac{1}{2\ii}(z-\zq).$$ Die {\em Division} komplexer
Zahlen l"a"st sich nun durch Erweitern mit $\zq_2$ auf die
Standardform $a+b\ii$ zur"uckf"uhren:
$$\frac{z_1}{z_2}=\frac{1}{|z_2|^2}z_1\zq_2.$$

{\em Polarkoordinaten.}  

Die komplexe Zahl $z=a+b\ii$ kann alternativ durch ihren Betrag
$r$ und ihr {\em Argument} $\vp$, den Winkel von der reellen
Achse bis zum \glqq Leitstrahl\grqq\ durch $z$ dargestellt
werden. Es ist dann $z=r(\cos \vp+\ii\sin\vp)$ bzw.{} in
Exponentialschreibweise $z=re^{\ii\vp}$. In der letzten Formel
wird $\vp$ im Bogenma"s gemessen, etwa
$60^\circ=\frac{\pi}{3}$. Die Zahl $e=2,7182818\dots$ ist die
Eulersche Konstante. Die Multiplikation komplexer Zahlen ist in
der Exponentialschreibweise besonders einfach, da das
Potenzgesetz gilt: 
$$r_1 e^{\ii\vp} {\cdot} r_2 e^{\ii\psi} = r_1 r_2
e^{\ii(\vp+\psi)}.$$ Die Argumente der beiden komplexen Zahlen
werden addiert, das hei"st, $z_1$ wird um dem Winkel $\psi$
(entgegen dem Uhrzeiger) um den Ursprung gedreht. Die Betr"age
werden wie gew"ohnliche positive reelle Zahlen miteinander
multipliziert.

\subsection*{Geometrie}

{\em Kreis} mit Radius $r$ um den Punkt $z_0$ der Ebene: $|z-z_0|=r$.

{\em Gerade} durch $z_1$ und $z_2$: $\frac{z-z_1}{z-z_2}\in{\bf R}$.

{\em Teilverh"altnis}. Die Strecke $\overline{AB}$ werde von innen und
von au"sen durch Punkte $P$ bzw.{} $Q$ im Verh"altnis $t$
geteilt, das hei"st,
$$
\overline{AP}:\overline{PB}=t=-\overline{AQ}:\overline{QB}.
$$
Die gerichtete Strecke $\overline{XY}$ ist, als Vektor im ${\bf
R}^2={\bf C}$, gleich $Y-X$. Daher hat man nach Umstellen
$P=\frac{1}{1+t}(A+tB)$, $t\ne-1$, und $Q=\frac{1}{1-t}(A-tB)$,
$t\ne1$.   

{\em Drehung} von $z$ um $z_0$ um den Winkel $\vp$ entgegen dem
Uhrzeiger:

\special{em:linewidth 0.4pt}
\unitlength 1.00mm \linethickness{0.4pt}
\begin{picture}(104.00,41.00)
\put(52.33,17.33){\vector(4,1){0.2}}
\emline{10.00}{10.00}{1}{52.33}{17.33}{2}
\put(40.67,37.67){\vector(1,1){0.2}}
\emline{10.00}{10.33}{3}{40.67}{37.67}{4}
\emline{22.33}{12.00}{5}{22.08}{14.48}{6}
\emline{22.08}{14.48}{7}{21.33}{16.25}{8}
\emline{21.33}{16.25}{9}{20.08}{17.32}{10}
\emline{20.08}{17.32}{11}{18.33}{17.67}{12}
\put(17.00,14.00){\makebox(0,0)[cc]{$\vp$}}
\put(7.00,6.33){\makebox(0,0)[cc]{$z_0$}}
\put(57.00,14.33){\makebox(0,0)[cc]{$z$}}
\put(46.00,41.00){\makebox(0,0)[cc]{$z^\prime$}}
\put(104.00,23.33){\makebox(0,0)[cc]
{\fbox{$z^\prime-z_0=(z-z_0)e^{\ii\vp}$}$\qquad (\ast)$}}
\put(18.00,17.67){\vector(-2,1){0.2}}
\emline{20.33}{16.67}{13}{18.00}{17.67}{14}
\end{picture}
%\input{dreh.pic}

\subsection*{Anwendung der Drehung bei geometrischen Beweisen}

\begin{quote}
{\bf Beispiel 1} : "Uber den Seiten eines spitzwinkligen Dreiecks
$ABC$ werden nach au"sen gleichseitige Dreiecke $ABC'$, $BCA'$
bzw.{} $ACB'$ mit den Mittelpunkten $D$, $E$ bzw.{} $F$ erichtet.

Zeigen Sie, da"s das Dreieck $DEF$ gleichseitig ist!
\end{quote}

\begin{beweis}

\special{em:linewidth 0.4pt}
\unitlength 1mm
\linethickness{0.4pt}
\begin{picture}(70.00,53.33)
\emline{20.00}{20.00}{1}{60.00}{20.00}{2}
\emline{60.00}{20.00}{3}{50.00}{50.00}{4}
\emline{50.00}{50.00}{5}{20.33}{20.00}{6}
\emline{20.33}{20.00}{7}{28.00}{44.67}{8}
\emline{28.00}{44.67}{9}{50.00}{49.67}{10}
\emline{50.00}{49.67}{11}{66.33}{38.67}{12}
\emline{66.33}{38.67}{13}{60.33}{20.33}{14}
\emline{60.33}{20.33}{15}{40.67}{9.33}{16}
\emline{40.67}{9.33}{17}{20.67}{20.00}{18}
\put(16.33,15.00){\makebox(0,0)[cc]{$A$}}
\put(66.00,15.33){\makebox(0,0)[cc]{$B$}}
\put(50.67,53.33){\makebox(0,0)[cc]{$C$}}
\put(40.67,4.67){\makebox(0,0)[cc]{$D$}}
\put(70.00,40.33){\makebox(0,0)[cc]{$E$}}
\put(23.33,47.67){\makebox(0,0)[cc]{$F$}}
\end{picture}
%\input{dreieck.pic}

Wir fassen $A$, $B$ und $C$ als Punkte der komplexen Ebene auf
und berechnen daraus $D$, $E$ und $F$. Es entsteht $D$ durch
Drehung von $B$ um $A$ um $-30^\circ$ (im Bogenma"s
$-\frac{\pi}{6}$) und anschlie"sender Stauchung mit dem Faktor
$\overline{AD}:\overline{AB}=1:\sqrt{3}$, denn die H"ohe $h$ im
gleichseitigen Dreieck $ABC^\prime$ ist nach Pythagoras
$\frac{1}{2}\sqrt{3}|AB|$ und $\overline{AD}= \frac{2}{3}h$. Nach
Formel $(\ast)$ gilt also
\begin{eqnarray}
D-A&=&(B-A)e^{-\ii\frac{\pi}{6}}\sqqq,  \label{1}\\
D-B&=&(A-B)e^{\ii\frac{\pi}{6}}\sqqq,   \label{2}  \\
E-B&=&(C-B)e^{-\ii\frac{\pi}{6}}\sqqq,  \label{3}\\
F-A&=&(C-A)e^{\ii\frac{\pi}{6}}\sqqq.   \label{4}
\end{eqnarray}
Dabei steht ein positives Vorzeichen vor dem Drehwinkel, wenn
entgegen dem Uhrzeiger gedreht wird, ein negatives Vorzeichen,
wenn mit dem Uhrzeiger gedreht wird. Wir m"ussen zeigen, da"s
gilt $(E-D) e^{\ii\frac{\pi}{3}}=F-D$, das hei"st, $F$ entsteht
durch Drehung von $E$ um $D$ um $60^\circ=\frac{\pi}{3}$ entgegen
dem Uhrzeiger.

Wir bilden die Differenzen der Gleichungen $(3)-(2)$ und
$(4)-(1)$:
\begin{eqnarray*}
E-D&\!\!=\!\!&\sqqq((-e^{-\ii\frac{\pi}{6}}+e^{\ii\frac{\pi}{6})}B
-e^{\ii\frac{\pi}{6}} A+e^{-\ii\frac{\pi}{6}}  C),
\\
F-D&\!\!=\!\!&\sqqq(-e^{-\ii\frac{\pi}{6}} B+
(-e^{\ii\frac{\pi}{6}} +e^{-\ii\frac{\pi}{6}} )A +e^{\ii\frac{\pi}{6}} C).
\end{eqnarray*}
Beachtet man $-e^{\ii\frac{\pi}{6}} + e^{-\ii\frac{\pi}{6}} =-\cos\frac{\pi}{6}
-\ii\sin\frac{\pi}{6}+\cos(-\frac{\pi}{6})+\ii\sin(-\frac{\pi}{6})=
-2\half\ii=-\ii$, so ergibt sich
\begin{eqnarray*}
e^{\ii\frac{\pi}{3}}{(E-D)}{-}{(F-D)}&\!\!\!=\!\!\!&
\textstyle{\sqqq}\bigl((-e^{\ii\frac{\pi}{2}}
{+}e^{\ii\frac{\pi}{6}}{-} e^{-\ii\frac{\pi}{6}} )A {+} 
({-}e^{\ii\frac{\pi}{6}}{+}e^{\ii\frac{\pi}{2}}{+}e^{-\ii\frac{\pi}{6}})B{+}
( e^{\ii\frac{\pi}{6}} {-} e^{\ii\frac{\pi}{6}} )C\bigr)\\
&\!\!\!=\!\!\!&\textstyle{\sqqq}((-\ii+\ii)A + (-\ii+\ii)B)  =0.
\end{eqnarray*}
Hieraus folgt die Behauptung.
\end{beweis}

\begin{quote}
{\bf Beispiel 2} : "Uber den Seiten $\overline{BC}$ und $\overline{CA}$ eines
Dreiecks $ABC$ werden nach au"sen Quadrate $BDEC$ und $ACFG$
errichtet. Ferner sei $M$ der Mittelpunkt von $\overline{AB}$.

Beweisen Sie, da"s die Strecken $\overline{FE}$ und $\overline{CM}$
aufeinander senkrecht stehen und da"s die erste doppelt so lang
ist wie die zweite Strecke!
\end{quote}

\begin{beweis}
Wir fassen die Punkte $A$, $B$, $C$, $M$, $D$, $E$ und $F$ als
Punkte der komplexen Ebene auf. Man erh"alt $F$ durch Drehung von
$A$ um $C$ um $-\frac{\pi}{2}$ und $E$ durch Drehung von $B$ um
$C$ um $\frac{\pi}{2}$.  Folglich gilt
\begin{displaymath}
F-C=e^{-\ii\frac{\pi}{2}}(A-C)  \quad  \mbox{ und }\quad
E-C=e^{\ii\frac{\pi}{2} }(B-C).
\end{displaymath}
Wegen $\ii=-e^{-\ii\frac{\pi}{2}}  = e^{\ii\frac{\pi}{2}} $ 
erhalten wir als Differenz der  beiden Gleichungen
$$
F-E=\ii(-A+C-B+C).
$$
Wegen $M=\half(A+B)$ ergibt sich hieraus $\half\ii(F-E)=M-C$. Das
ist aber gerade die zu zeigende Behauptung.
\end{beweis}

Die folgende Olympiadeaufgabe besitzt eine einfache
elementargeometrische L"osung, auf die man unter Zeitdruck aber
m"oglicherweise nicht kommt.  Die oben beschriebene Methode
ben"otigt hingegen {\em "uberhaupt keine} Idee, nur die Umsetzung
der geometrischen Figur in komplexe Zahlen. Wir formulieren die
Olympiadeaufgabe, beweisen jedoch ihre Verallgemeinerung
(Beispiel 4).

\begin{quote}
{\bf Beispiel 3} (351046): Gegeben sei eine Strecke $\overline{AB}$ und
auf ihr Punkte $X$ und $Y$. "Uber den Strecken $\overline{AX}$ und
$\overline{XB}$ werden \glqq nach oben\grqq\ Quadrate mit den
Mittelpunkten $K$ bzw.{} $L$ errichtet. "Uber den Strecken
$\overline{AY}$ und $\overline{YB}$ werden \glqq nach unten\grqq\ Quadrate
mit den Mittelpunkten $M$ bzw.{} $N$ errichtet.

Beweisen Sie, da"s die Strecken $\overline{LM}$ und $\overline{KN}$
orthogonal und gleichlang sind!
\end{quote}

\begin{quote}
{\bf Beispiel 4} : "Uber den Seiten eines Vierecks $ABCD$ werden
nach au"sen Quadrate mit den Mittelpunkten $E$, $F$, $G$ bzw.{}
$H$ errichtet.

Beweisen Sie, da"s die Strecken $\overline{EG}$ und $\overline{FH}$
orthogonal und gleichlang sind.
\end{quote}

\begin{beweis}
Der Punkt $E$ entsteht durch Drehung von $B$ um $A$ um
$-45^\circ$ und anschlie"sender Stauchung mit dem Faktor $\sqq$:
$$
E-A=\sqq(B-A)e^{-\ii\frac{\pi}{4}}.
$$
Analog erh"alt man
\begin{eqnarray*}
G-D&=&\sqq(C-D)e^{\ii\frac{\pi}{4}},   \\
H-A&=&\sqq(D-A)e^{\ii\frac{\pi}{4}},  \\
F-B&=&\sqq(C-B)e^{-\ii\frac{\pi}{4}}.
\end{eqnarray*}
Zu zeigen ist $E-G=\ii(H-F)$. Dann sind die beiden Strecken
orthogonal und gleichlang.  Aus den obigen Gleichungen ergibt
sich
\begin{eqnarray*}
E-G&=&(1- \sqq e^{-\ii\frac{\pi}{4}})A
   +\sqq e^{-\ii\frac{\pi}{4}}B
   -\sqq e^{\ii\frac{\pi}{4}} C
   +(-1+\sqq e^{\ii\frac{\pi}{4}})D,\\
H-F&=&(1- \sqq e^{\ii\frac{\pi}{4}} )A
   + (-1+\sqq e^{-\ii\frac{\pi}{4}})B
   -\sqq e^{-\ii\frac{\pi}{4}} C
   +\sqq e^{\ii\frac{\pi}{4}}D.
\end{eqnarray*}
Beachtet man $\ii=e^{\ii\frac{\pi}{2}}$, so ergibt sich
\begin{eqnarray*}
E-G-\ii(H-F)&=&A(1-\sqq e^{-\ii\frac{\pi}{4}} -\ii +
\sqq e^{\ii\frac{3\pi}{4}} )
+B(\sqq e^{-\ii\frac{\pi}{4}} +\ii 
-\sqq e^{\ii\frac{\pi}{4}})
\\
&&+C(-\sqq e^{\ii\frac{\pi}{4}} 
+\sqq e^{\ii\frac{\pi}{4}})
+D(-1+\sqq e^{\ii\frac{\pi}{4}}- \sqq
e^{\ii\frac{3\pi}{4}}).
\end{eqnarray*}
Nun gilt aber $-e^{-\ii\frac{\pi}{4}} +
e^{\ii\frac{3\pi}{4}} =2e^{\ii\frac{3\pi}{4}}=\sqrt{2}(-1+\ii)$
und $e^{-\ii\frac{\pi}{4}} - e^{\ii\frac{\pi}{4}} =\ii\sqrt{2}$.
Daher sind alle Koeffizienten vor $A$, $B$, $C$ bzw.{} $D$ 
Null, und es gilt die Behauptung.
\end{beweis}

\section*{Aufgaben}

\begin{itemize}
\item[1.] Gegeben sei ein Quadrat $ABCD$ und ein Punkt $P$ in
seinem Innern mit $\angle\, PAB=\angle\, PBA=15^\circ$.

Zeigen Sie, da"s das Dreieck $CPD$ gleichseitig ist!

\item[2.] "Uber den Seiten eines Parallelogramms $ABCD$ werden
nach au"sen Quadrate mit den Mittelpunkten $E$, $F$, $G$ bzw.{}
$H$ errichtet.

Beweisen Sie, da"s $EFGH$ ein Quadrat ist!

\item[3.] "Uber den Seiten eines spitzwinkligen Dreiecks $ABC$
sind nach au"sen Dreiecke $ABR$, $BCP$ und $CAQ$ errichtet. Dabei
sind $\angle\, RAB=\angle\, RBA=15^\circ$, $\angle\, PBC=\angle\,
QAC=45^\circ$ und $\angle\, PCB=\angle\, QCA=30^\circ$.

Zeigen Sie, da"s das Dreieck $PQR$ rechtwinklig gleichschenklig
ist!
\end{itemize}

{\em Hinweis} : Benutzen Sie zur Berechnung der ben"otigten
Streckungs\-ver\-h"altnisse den Sinussatz. Zeigen Sie zun"achst,
da"s $\overline{AR}:\overline{AB}=\overline{AQ}:\overline{AC}$ gilt.

\begin{comment}
  todo: geometric markup
\end{comment}

\begin{attribution}
schueler (2004-09-09): Contributed to KoSemNet

graebe (2004-09-09): Prepared along the KoSemNet rules
\end{attribution}

\end{document}





