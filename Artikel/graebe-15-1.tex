\documentclass[11pt,a4paper]{article}  
\usepackage{kosemnet,ko-math,ngerman} 
\usepackage[utf8]{inputenc}  

\newcommand{\cas}[1]{{\sc #1}}
\newcommand{\red}{\mathrm{red}}


\author{Hans-Gert Gräbe, Leipzig}
\title{Zur Vereinfachung von Wurzelausdrücken.\\ 
Eine Anmerkung zur Aufgabe $\varphi\ 28$\kosemnetlicensemark}
\date{11. Januar 2015}

\begin{document} 
\maketitle 

In [1] stellt Friedhelm Götze aus Jena die Aufgabe, den Ausdruck 
\begin{gather*}
  A=\frac{14}{\sqrt{78+52\,\sqrt[4]{2}+51\,\sqrt[4]{4}+34\,\sqrt[4]{8}}}
\end{gather*}
zu vereinfachen und in eine nennerfreie Darstellung zu überführen. Reiner
Möwald gibt in [2] durch geeignet „trickreiche“ Umformungen eine Lösung an, die
allerdings die Schönheit der dahinter verborgenen Mathematik nicht erkennen
lässt -- dass sich nämlich \emph{jeder} derartige Ausdruck als nennerfreie
Linearkombination darstellen lässt und dafür auch ein einfaches algorithmisches
Vorgehen bekannt ist. Diese beiden Aspekte möchte ich im folgenden kleinen
Aufsatz darstellen.  Eine ausführlicher Besprechung der Thematik ist in [4] zu
finden.

\section{Die Problemstellung}

Rationale Ausdrücke wie 
\begin{gather*}
  \frac{a^2}{\br{a-b}\,\br{a-c}} + \frac{b^2}{\br{b-a}\,\br{b-c}} +
  \frac{c^2}{\br{c-a}\,\br{c-b}}\tag{1}
\end{gather*}     
lassen sich durch einfache Termumformungen oft substanziell vereinfachen.
Ähnliches gilt für Wurzelausdrücke, wie zum Beispiel 
\begin{gather*}
  \sqrt{2\sqrt{3}+4}=1+\sqrt{3}\tag{2}
\end{gather*}
oder 
\begin{gather*}
  \sqrt{11+6\sqrt{2}}+\sqrt{11-6\sqrt{2}} = 6\tag{3}.
\end{gather*}

Entsprechende Algorithmen sind auch in den meisten Computeralgebrasystemen
(CAS) implementiert, allerdings aus hier nicht weiter zu besprechenden Gründen
oft in Spezialpakete ausgelagert.  Während \cas{Maple} derartige
Vereinfachungen automatisch vornimmt, sind andere CAS oft nur mit viel gutem
Zureden bereit, diese Simplifikation vorzunehmen. Selbst das Normalformproblem,
also jeweils die Vereinfachung der Differenz beider Seiten der Identitäten zu
null, wird weder in \cas{MuPAD} noch in \cas{Mathematica} oder \cas{Maxima}
allein durch den Aufruf von \texttt{simplify} gelöst. \cas{MuPAD}
(\texttt{radsimp}) und \cas{Mathematica} (\texttt{RootReduce}) stellen
spezielle Simplifikationsroutinen für geschachtelte Wurzelausdrücke zur
Verfügung, was bei \cas{Mathematica} im stärkeren \texttt{FullSimplify}
integriert ist. \cas{Maxima} [3] bietet eine solche Simplifikationsroutine
\texttt{sqrtdenest} im Paket \texttt{sqdnst}.  

Solche Wurzelausdrücke sind stets Nullstellen von Polynomen. Für
$a=\sqrt{2\sqrt{3}+4}$ etwa rechnet man $(a^2-4)^2=4\cdot 3=12$ leicht nach,
also ist $a$ Nullstelle des Polynoms 
\begin{gather*}
  P(x)=\br{x^2-4}^2-12=x^4-8\,x^2+4,
\end{gather*}
welches sich in die algebraische Ersetzungsregel $a^4\to 8\,a^2-4$
transformieren lässt.  Wegen
\begin{gather*}
  P(x)=\br{x^2-2\,x-2}\cdot\br{x^2+2\,x-2}\tag{4}
\end{gather*}
ist $a$ aber auch Nullstelle eines der beiden Faktorpolynome und damit
existiert sogar eine algebraische Ersetzungsregel für $a^2$.  

Damit ist auch schon die Gültigkeit von (2) fast gezeigt, es bleibt allein zu
begründen, welche der vier Nullstellen von (4) auf der rechten und welche auf
der linken Seite steht. Dies kann man aber in diesem Fall durch eine einfache
numerische Abschätzung entscheiden, da die vier Nullstellen genügend weit
auseinander liegen.

Jede Zahl $a$, die Nullstelle eines Polynoms $P(x)\in\Q[x]$ ist, ist auch
Nullstelle eines irreduziblen monischen Polynoms $Q(x)=x^d-r(x)\in\Q[x]$, also
eines Polynoms, das sich (über $\Q$) nicht weiter in Faktoren zerlegen lässt
und dessen Koeffizient vor der höchsten $x$-Potenz gleich eins ist.  Hierbei
ist $r(x)\in\Q[x]$ ein Polynom vom Grad kleiner als $d$.

Derartige Zahlen $a$ bezeichnet man als \emph{algebraische Zahlen}, das Polynom
$Q(x)$ als \emph{charakteristisches Polynom} der Zahl $a$ und $d$ als den
\emph{Grad} der algebraischen Zahl.

Ein erstes kleines Lemma beweist, dass dieses charakteristische Polynom und
damit auch dessen Grad eindeutig bestimmt sind.
\begin{lemma}
Unter allen Polynomen
\[P:=\{q(x)\in \Q[x]\,:\,q(a)=0 \text{ und } lc(q)=1\}\]
mit Leitkoeffizient 1 und Nullstelle $a$ gibt es genau ein Polynom $p(x)$
kleinsten Grades.  Dieses ist irreduzibel und jedes andere Polynom $q(x)\in P$
ist ein Vielfaches von $p(x)$.
\end{lemma}

\begin{beweis} Division mit Rest in $\Q[x]$ ergibt 
\[q(x)=s(x)\cdot p(x) + r(x)\]
mit $r=0$ oder $\deg(r)<\deg(p)$. Wäre $r\neq 0$, so ergäbe sich wegen
$q(a)=p(a)=0$ auch $r(a)=0$ und $\frac{1}{lc(r)}r(x)\in P$ im Gegensatz zur
Annahme, dass $p(x)\in P$ minimalen Grad hat.

Wäre $p(x)$ reduzibel, so wäre $a$ auch Nullstelle eines der Faktoren.  Dieser
würde also zu $P$ gehören im Widerspruch zur Auswahl von $p(x)$.
\end{beweis}

\section{Rechnen mit algebraischen Zahlen}

Ist das Minimalpolynom $Q(x)=x^d-r(x)$ einer algebraischen Zahl $a$ bekannt, so
gilt $a^d=r(a)$ und es lassen sich alle arithmetischen Ausdrücke ohne Division,
in denen (nur) $a$ vorkommt, also $A(a)\in\Q[a]$, wie folgt vereinfachen:
Multipliziere den Ausdruck vollständig aus, wende die algebraische
Ersetzungsregel $a^d\to r(a)$ an, und fasse danach die Ausdrücke mit gleicher
Potenz $a^k$ zusammen.

Anwenden der algebraischen Ersetzungsregel bedeutet dabei, alle Potenzen $a^k$
mit $k\ge d$ zu ersetzen, so dass im Ergebnis nur noch Potenzen $a^k$ mit $0\le
k\le d-1$ vorkommen.  Die Menge dieser Potenzen bezeichnen wir als
\emph{reduzierte Terme} $T_{\red}$, die so berechnete Darstellung 
\begin{gather*}
  A(a)=c_0+c_1\,a+c_2\,a^2+\dots+c_{d-1}\,a^{d-1}
\end{gather*}
mit $c_i\in\Q$ als \emph{reduzierte Darstellung} von $A(a)$.

\begin{satz}
  Die reduzierte Darstellung von $A(a)$ ist eindeutig bestimmt.
\end{satz}
\begin{beweis}
  Sind 
\begin{gather*}
  A(a)=c_0+c_1\,a+c_2\,a^2+\dots+c_{d-1}\,a^{d-1}=c_0'+c_1\,a+c_2'\,a^2+\dots+c_{d-1}'\,a^{d-1}
\end{gather*}
zwei Darstellungen des Ausdrucks $A(a)$, so gilt 
\begin{gather*}
  (c_0-c_0')+(c_1-c_1')\,a+(c_2-c_2')\,a^2+\dots+(c_{d-1}-c_{d-1}')\,a^{d-1}=0.
\end{gather*}
und damit $c_0=c_0',\ \dots,\ c_{d-1}=c_{d-1}'$, denn sonst wäre $a$ Nullstelle
eines Polynoms vom einem Grad kleiner als $d$.
\end{beweis}

Dieses Vorgehen lässt sich unmittelbar auf das Rechnen mit mehreren
algebraischen Zahlen verallgemeinern: Sind $\alpha_1,\ldots,\alpha_s$
algebraische Zahlen vom Grad $d_1,\ldots,d_s$, so ergeben sich aus den
entsprechenden Minimalpolynomen
\[p_i(x)=x^{d_i}-q_i(x)\]
Ersetzungsformeln 
\[\{\alpha_i^{d_i}\to q_i(\alpha_i),\ i=1,\ldots,s\},\] 
die es erlauben, jeden polynomialen Ausdruck aus $R=
\Q[\alpha_1,\ldots,\alpha_s]$ in dessen \emph{reduzierte Form} zu
transformieren, d.\,h.\ ihn als Linearkombination der $D:=d_1\cdots d_s$
Produkte aus der Menge $T_{\red} := \cbr{ \alpha_1^{j_1} \cdots
  \alpha_s^{j_s}\ : \ 0 \leq j_i<d_i}$ zu schreiben.

Allerdings sind diese reduzierten Formen nur im Falle $s=1$ eindeutig, da
zwischen den verschiedenen algebraischen Zahlen $\alpha_1,\ldots,\alpha_s$
algebraische Abhängigkeitsrelationen bestehen können, die lineare
Abhängigkeitsrelationen in der Menge $T_{\red}$ nach sich ziehen.

\emph{Beispiel:} $\alpha_1=\sqrt{2}+\sqrt{3},\ \alpha_2=\sqrt{6}$. Es gilt
$\alpha_1^2-2\,\alpha_2-5=0$.

\section{Der Ring der algebraischen Zahlen}

\begin{satz}
  Jedes Element aus $R= \Q[\alpha_1,\ldots,\alpha_s]$ ist wieder eine
  algebraische Zahl. 
\end{satz}

Statt eines Beweises demonstrieren wir das Vorgehen am Beispiel des Ausdrucks
$c=a+b\in\Q[a,b]$ für die algebraischen Zahlen $a=\sqrt{2}$ und
$b=\sqrt[3]{5}$.  Der allgemeine Beweis lässt sich ähnlich führen. 

Um zu zeigen, dass $c$ eine algebraische Zahl ist, müssen wir ein Polynom
$p(x)=\sum_{i=0}^n{r_i\,x^i}$ finden, für das $p(c)=\sum_{i=0}^n{r_i\,c^i}=0$
gilt.  

Wir setzen $p(x)$ mit unbestimmten Koeffizienten an und berechnen zunächst die
Potenzen $c^i=(a+b)^i$ als Ausdrücke in $a$ und $b$ wie oben beschrieben. Dazu
sind die binomischen Formeln sowie die algebraischen Ersetzungen $\{a^2\to 2,
b^3\to 5\}$ anzuwenden, die sich aus den charakteristischen Polynomen von $a$
bzw.\ $b$ unmittelbar ergeben. 

Diese Rechnungen lassen sich mit Bleistift und Papier ausführen, aber auch mit
einem CAS, das algebraische Ersetzungen beherrscht. Für die folgenden
Rechnungen habe ich das freie CAS \cas{Maxima} [3] verwendet.
\begin{code}
tellrat(a\pw2=2,b\pw3=5);\\
l:makelist(ratsimp((a+b){\pw}n),n,0,10),algebraic;
\end{code}
%% tellrat(a^2=2,b^3=5);
%% l:makelist(ratsimp((a+b)^n),n,0,10),algebraic;
\begin{align*}
\Big[\quad & 1,\\ 
& b+a,\\ 
& b^2+2\,a\,b+2,\\ 
& 3\,a\,b^2+6\,b+2\,a+5,\\ 
& 12\,b^2+(8\,a+5)\,b+20\,a+4,\\ 
& (20\,a+5)\,b^2+(25 \,a+20)\,b+4\,a+100,\\
& (30\,a+60)\,b^2+(24\,a+ 150)\,b+200\,a+33,\\
& (84\,a+210)\,b^2+(350\,a+ 81)\,b+183\,a+700,\\
& (560\,a+249)\,b^2+(264\,a+ 1400)\,b+1120\,a+1416,\\
& (513\,a+2520)\,b^2+(2520\,a+1944)\,b+4216\,a+3485,\\
& (5040\,a+2970)\,b^2 +(6160\,a+8525)\,b+6050\,a+21032\quad \Big]
\end{align*}
Wir sehen, dass sich alle Potenzen als Linearkombinationen von den sechs
reduzierten Termen $1,\,a,\,b,\,b^2,\,a\,b,\,a\,b^2$ darstellen lassen. Mehr
als sechs solcher Ausdrücke sind dann sicher linear abhängig. Wir bestimmen für
unser Beispiel eine solche Abhängigkeitsrelation, indem wir eine
Linearkombination von sieben verschiedenen Potenzen $(a+b)^i$ mit unbestimmten
Koeffizienten $r_i$ aufstellen und die $r_i$ dann so bestimmen, dass die sechs
Koeffizienten vor $1,\,a,\,b,\,b^2,\,a\,b,\,a\,b^2$ alle verschwinden. 
\begin{code}
 p:sum(concat(r,i)*x{\pw}i,i,0,6);\\
 p1:ratsimp(subst(x=a+b,f)),algebraic;
\end{code}
\begin{align*}
  &\br{(30\,a+60)\,b^2+(24\,a+150)\,b+200\,a+33}\,{r_6}\\
  &+\br{(20\,a+5)\,b^2+(25\,a+20 )\,b+4\,a+100}\,{r_5}
  +\br{12\,b^2+(8\,a+5 )\,b+20\,a+4}\,{r_4}\\
  &+\br{3\,a\,b^2+6\,b+2\,a+5}\,{r_3} +\br{b^2+2\,a\,b+2}\,{r_2}
  +(b+a)\,{r_1}+{r_0}
\end{align*}
Dieses Polynom muss zunächst in die distributive Normalform gebracht werden, um
es im zweiten Schritt rekursiv nach Potenzen von $a$ und $b$ zu sortieren und
dann die Koeffizienten vor den reduzierten Termen zu extrahieren. Leider bietet
\cas{Maxima} keine bequeme Möglichkeit, die Koeffizienten eines Polynoms zu
extrahieren, deshalb die umständliche Berechnung von \texttt{coeffs} und
\texttt{sys}.
\begin{code}
  p2:expand(p1);\\
  coeffs:map(lambda([u],makelist(coeff(u,b,j),j,0,2)),\\\>\>
  makelist(coeff(p2,a,i),i,0,1));\\ 
  sys:flatten(coeffs);
\end{code}
\begin{align*}
\big[\quad & r_0 + 2\,r_2 + 5\,r_3 + 4\,r_4 + 100\,r_5 + 33\,r_6,\
 r_1 + 6\,r_3 + 5\,r_4 + 20\,r_5 + 150\,r_6,\\ 
& r_2 + 12\,r_4 + 5\,r_5 + 60\,r_6,\ 
 r_1 + 2\,r_3 + 20\,r_4 + 4\,r_5 + 200\,r_6,\\ 
& 2\,r_2 + 8\,r_4 + 25\,r_5 + 24\,r_6,\
 3\,r_3 + 20\,r_5 + 30\,r_6\quad\big]
\end{align*}
Das ist ein homogenes lineares Gleichungssystem, dessen Lösung von einem
Parameter abhängt.
\begin{code}
 vars:makelist(concat(r,i),i,0,6);\\
 sol:solve(sys,vars);
\end{code}
\begin{gather*}
  \big[\, \sbr{ {r_0}=17\,{\%r_4} , {r_1}=-60\, {\%r_4} , {r_2}=12\,{\%r_4} ,
      {r_3}=-10\,{\%r_4} , {r_4}=-6\,{\%r_4} , {r_5}=0 , {r_6}={\%r_4}}\,\big]
\end{gather*}
Ein Polynom mit der Nullstelle $c=a+b$ lautet also
\begin{code}
 p0:subst(\%r4=1,subst(sol[1],p));
\end{code}
\begin{gather*}
  p_0=x^6\,-\,6\,x^4\,-\,10\,x^3\,+\,12\,x^2\,-\,60\,x\,+\,17
\end{gather*}

Testen wir schließlich noch, ob dieses Polynom irreduzibel ist:
\begin{code}
factor(p0);
\end{code}
Damit wissen wir, dass es sich bei diesem Polynom sogar um das Minimalpolynom
von $c=a+b$ handelt.

\section{Quotienten algebraischer Zahlen}

Von einfachen algebraischen Zahlen wie etwa $1+\sqrt{2}$ oder
$\sqrt{2}+\sqrt{3}$ wissen wir, dass man die jeweilige Inverse dazu recht
einfach darstellen kann, wenn man mit einer auf geeignete Weise definierten
\emph{konjugierten} Zahl erweitert. So gilt etwa
\begin{gather*}
  \frac{1}{1+\sqrt{2}} = \frac{1-\sqrt{2}}{1-2} = \sqrt{2} -1
  \quad\text{und}\quad \frac{1}{\sqrt{2}+\sqrt{3}} =
  \frac{\sqrt{3}-\sqrt{2}}{3-2} =\sqrt{3}-\sqrt{2}\,.
\end{gather*}
Damit kann man in diesen Fällen auch die Inverse einer algebraischen Zahl (und
damit beliebige Quotienten) als lineare Kombination der reduzierten Terme
darstellen.  Es stellt sich die Frage, ob dies auch für kompliziertere
rationale Ausdrücke mit algebraischen Zahlen möglich ist.  Wie sieht es
z.\,B.\ mit
\[\frac{1}{\sqrt{2}+\sqrt{3}+\sqrt{5}}\]
aus?

\cas{Axiom} liefert als Ergebnis sofort
\[\frac{1}{6}\sqrt{3}+\frac{1}{4}\sqrt{2}-\frac{1}{12}\sqrt{30}\]
Für die anderen Systeme sind dazu spezielle Funktionen, Schalter und/oder
Pakete notwendig, so in \cas{Maple} die Funktion \texttt{rationalize} aus der
gleichnamigen Bibliothek und in \cas{MuPAD} die Funktion \texttt{radsimp}. In
\cas{Reduce} muss der Schalter \texttt{rationalize} eingeschaltet sein. In
\cas{Maxima} ist dazu die rationale Normalform im Kontext \texttt{algebraic} zu
berechnen:
\begin{code}
a:sqrt(2)+sqrt(3)+sqrt(5);\\
ratsimp(1/a), algebraic;
\end{code}
\[-{\frac{\sqrt{2}\,\sqrt{3}\,\sqrt{5}-2\,\sqrt{3}-3\,\sqrt{2}}{12}}\]
\medskip

Untersuchen wir, auf welchem Wege sich eine solche Darstellung finden ließe.
Wie oben findet man heraus, dass $a=\sqrt{2}+\sqrt{3}+\sqrt{5}$ das
charakteristische Polynom 
\begin{gather*}
  p(x)=x^8-40\,x^6+352\,x^4-960\,x^2+576
\end{gather*}
hat, d.\,h.\ es gilt $a^8-40\,a^6+352\,a^4-960\,a^2+576=0$ oder
\[a^{-1}=\frac{-a^7+40\,a^5-352\,a^3+960\,a}{576}\,.\]
Ist das charakteristische Polynom bekannt, so ist es also nicht schwer,
$a^{-1}$ zu berechnen. Es werden einzig noch Ringoperationen benötigt, um den
Ausdruck zu vereinfachen:
\begin{gather*}
  a^{-1}=\texttt{subst}\left(a=\sqrt{2}+\sqrt{3}+\sqrt{5},
  \frac{-a^7+40\,a^5-352\,a^3+960\,a}{576}\right)\,.
\end{gather*}

\section{Die Aufgabe $\varphi\ 28$}

Mit dem im letzten Abschnitt beschriebenen Verfahren kann man zwar nun auch
Quotienten in Linearkombinationen reduzierter Terme überführen, allerdings muss
für jeden Divisor erst umständlich das Minimalpolynom bestimmt werden. Wir
zeigen am Beispiel der Aufgabe $\varphi\ 28$, dass ein Ansatz mit unbestimmten
Koeffizienten, wie wir ihn schon oben bei der Bestimmung des Minimalpolynoms
von $c=a+b$ erfolgreich angewendet haben, auch hier weiterhilft.

Gegeben sind in unserem Fall zwei algebraische Zahlen 
\begin{gather*}
  a=\sqrt[4]{2}\ \text{und}\ b=\sqrt{78+52\,a+51\,a^2+34\,a^3}
\end{gather*}
und der zu vereinfachende Ausdruck $A=\frac{14}{b}$.  Die Ersetzungregeln der
beiden algebraischen Zahlen lauten $a^4\to 2$ und $b^2\to
78+52\,a+51\,a^2+34\,a^3$, als reduzierte Terme ergeben sich
$T_{\red}=\cbr{1,a,a^2,a^3,b,a\,b,a^2\,b,a^3\,b}$.  Wir suchen eine
Linearkombination
\begin{gather*}
  p=r_0+r_1\,a+r_2\,a^2+r_3\,a^3+r_4\,b+r_5\,a\,b+r_6\,a^2\,b+r_7\,a^3\,b
\end{gather*}
mit $p\cdot b-14=0$ und verfahren dazu wie oben (alle Rechnungen wiederum mit
\cas{Maxima}):
\begin{code}
  tellrat(b\pw2=78+52*a+51*a\pw2+34*a\pw3,a\pw4=2);\\
  p:r0+r1*a+r2*a\pw2+r3*a\pw3+r4*b+r5*a*b+r6*a\pw2*b+r7*a\pw3*b;\\
  u1:ratsimp(p*b-14),algebraic;
\end{code}
\begin{align*}
  u_1= & (78\,a^3+68\,a^2+102\,a+104)\,r_7
  +(52\,a^3+78\,a^2+68\,a+102)\,r_6\\ &\quad +(51\,a^3+52\,a^2+78\,a+68)\,r_5
  +(34\,a^3+51\,a^2+52\,a+78)\,r_4\\ &\quad +a^3\,b\,r_3
  +a^2\,b\,r_2+a\,b\,r_1+b\,r_0-14\,.
\end{align*}
Dieser Ausdruck ist nach Termen aus $T_{\red}$ zu sortieren. Die Koeffizenten
vor den reduzierten Termen ergeben ein lineares Gleichungssystem zur Bestimmung
der $r_i$.
\begin{code}
  u2:expand(u1);\\
  coeffs:map(lambda([u],makelist(coeff(u,a,j),j,0,3)),\\\>\>
  makelist(coeff(u2,b,i),i,0,1));\\ 
  sys:flatten(coeffs);
\end{code}
\begin{align*}
  \big[\ &
    104\,r_7+102\,r_6+68\,r_5+78\,r_4-14,\ 102\,r_7+68\,r_6+78\,r_5+52\,r_4,\\ &\quad
    68\,r_7+78\,r_6+52\,r_5+51\,r_4,\ 78\,r_7+52\,r_6+51\,r_5+34\,r_4,\ r_0,\ r_1,\ r_2,\ r_3\ \big]
\end{align*}
Wir sehen bereits an dieser Stelle, dass $r_0=r_1=r_2=r_3=0$ sein muss, der
gesuchte Ausdruck also -- anders als in Möwalds Lösung [2] -- ein Vielfaches
von $b$ ist. Wir lösen dieses (inhomogene lineare) Gleichungssystem, setzen die
in diesem Fall eindeutig bestimmte Lösung
\begin{code}
vars:[r0,r1,r2,r3,r4,r5,r6,r7];\\
sol:solve(sys,vars);  
\end{code}
\begin{gather*}
  \sbr{\sbr{r_0 = 0,\ r_1 = 0,\ r_2 = 0,\ r_3 = 0,\ r_4 = \frac67,\ r_5 =
      -\frac47,\ r_6 = -\frac37,\ r_7 = \frac27}}
\end{gather*}
in $p$ ein und erhalten 
\begin{code}
  p0:subst(sol[1],p);
\end{code}
\begin{gather*}
  p_0=\frac{2\,a^3\,b-3\,a^2\,b-4\,a\,b+6\,b}{7}=\frac{b}{7}\,\br{2\,a^3-3\,a^2-4\,a+6}\,.
\end{gather*}
Die gestellte Aufgabe ist gelöst, allein die Differenz zu Möwalds Lösung
$m=a^2-2\,a^2+2$ in [2] bleibt zu diskutieren.  Möwalds größter Trick ist
gleich in der ersten Umformung
\begin{gather*}
  78+52\,a+51\,a^2+34\,a^3=\br{3\,a^3+a^2+5\,a+4}^2
\end{gather*}
enthalten und zeigt, dass $b^2-(78+52\,a+51\,a^2+34\,a^3)$ über $\Q[a]$ nicht
irreduzibel ist. Faktorisieren über algebraischen Erweiterungskörpern ist im
Allgemeinen eine komplizierte Angelegenheit und Lösungsversuche allein mit
Bleistift und Papier sind schnell am Ende. In diesem speziellen Fall findet
aber auch \cas{Maxima} die Faktorisierung:
\begin{code}
  factor(b\pw2-(78+52*a+51*a\pw2+34*a\pw3),a\pw4-2);
\end{code}
\begin{gather*}
  \br{b-3\,a^3-a^2-5\,a-4}\,\br{b+3\,a^3+a^2+5\,a+4}
\end{gather*}
Nehmen wir dagegen Möwalds Lösung als Orakel („Da kam ein Wanderer des Wegs und
sagte: $m$ ist eine Lösung“), so liefert $p_0=m$ eine algebraische
Abhängigkeitsrelation zwischen den reduzierten Termen, aus der wir eine
Darstellung für $b$ extrahieren können:
\begin{code}
  m:a\pw3-2*a\pw2+2;\\
  solve(pp=m,b);
\end{code}
\begin{gather*}
  b = \frac{7\,a^3-14\,a^2+14}{2\,a^3-3\,a^2-4\,a+6},
\end{gather*}
die wir wieder in eine Linearkombination der Terme $a^0,a^1,a^2,a^3$ umrechnen
können
\begin{code}
  ratsimp(solve(pp=m,b)),algebraic;
\end{code}
\begin{gather*}
  b = 3\,a^3+a^2+5\,a+4
\end{gather*}
Wir haben damit also Möwalds Trick „reverse engineered“.

\begin{enumerate}
\item{[1]} Aufgabe $\varphi\ 28$ von Friedhelm Götze. Wurzel {\bf 48}, Heft 6
  (2014), S.\ 143.
\item{[2]} Lösung der $\varphi\ 28$ von Reiner Möwald. Wurzel {\bf 49}, Heft 1
\item{[3]} Maxima 5.32.1, \url{http://maxima.sourceforge.net}
\item{[4]} Hans-Gert Gräbe: Skript zum Kurs „Einführung in das symbolische
  Rechnen“.  Sommersemester 2012.
  \url{http://www.informatik.uni-leipzig.de/~graebe/skripte/esr12.pdf}
\end{enumerate}


\begin{attribution}
graebe (2010-12-16)
\end{attribution}

\end{document}

tellrat(b^2=78+52*a+51*a^2+34*a^3,a^4=2);
p:r0+r1*a+r2*a^2+r3*a^3+r4*b+r5*a*b+r6*a^2*b+r7*a^3*b;
u1:ratsimp(p*b-14),algebraic;
u2:expand(u1);
coeffs:map(lambda([u],makelist(coeff(u,a,j),j,0,3)),makelist(coeff(u2,b,i),i,0,1));
sys:flatten(coeffs);
vars:[r0,r1,r2,r3,r4,r5,r6,r7];
sol:solve(sys,vars);
p0:subst(sol[1],p);

factor(b^2-(78+52*a+51*a^2+34*a^3),a^4-2);

m:a^3-2*a^2+2;
ratsimp(solve(pp=m,b)),algebraic;
ratsimp(7*a^3-14*a^2+14)/(2*a^3-3*a^2-4*a+6)
