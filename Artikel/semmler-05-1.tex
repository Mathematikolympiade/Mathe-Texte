% Version: $Id: semmler-05-1.tex,v 1.1 2008/09/05 09:52:58 graebe Exp $
\documentclass[11pt]{article}  
\usepackage{kosemnet,ko-math,ngerman,url}  

\author{Gunter Semmler, Freiberg\\\url{semmler@math.tu-freiberg.de}}
\title{Funktionalgleichungen\kosemnetlicensemark}
\date{}
\begin{document}
\maketitle         

Funktionalgleichungen sind Gleichungen zur Bestimmung oder Charakterisierung
von unbekannten Funktionen. Dabei kommt die gesuchte Funktion nicht nur in der
Form $f(x)$ vor, sondern auch mit einer Transformation ihres Argumentes $x$,
zum Beispiel in der Form $f(x+2)$, $f(3x)$, $f(x+y)$ oder $f(x-y)$. Es gibt
keine allgemeing\"{u}ltige Technik zur L\"{o}sung jeder beliebig vorgelegten
Funktionalgleichung, aber eine Reihe von erfolgversprechenden Methoden, die
wir an Hand einiger Beispiele kennenlernen wollen.

\section{Die Cauchysche Funktionalgleichung}

Im Jahre 1821 besch\"{a}ftigte sich A.L. Cauchy (1789-1857) mit der Bestimmung
aller auf $\R$ definierten Funktionen, die der Gleichung
\begin{equation} \label{e.cau}
f(x+y) = f(x)+ f(y)\;\;\;\forall x,y\in\R
\end{equation}
gen\"{u}gen. Nehmen wir nun an, dass $f$ eine L"osung ist und leiten
Eigenschaften von $f$ ab. Eine wichtige Methode ist das {\bf Einsetzen
spezieller Werte}:
\begin{align*}
&x=y=0\ \yields f(0)=f(0)+f(0)\ \yields f(0)=0.\\
&x=-y\ \yields f(0)=f(x)+f(-x)\ \yields f(-x)=-f(x).
\intertext{Jede L"osung ist also eine ungerade Funktion.}
&y=x\ \yields f(2x)=f(x)+f(x)=2\,f(x).\\
&y=2x\ \yields f(3x)=f(x)+f(2x)=f(x)+2\,f(x)=3\,f(x).
\end{align*}
Induktiv zeigt man f"ur $n\in \N$
\begin{equation} \label{e.nx}
f(nx)=n\,f(x).
\end{equation}
Wegen $f(x)=-f(x)$ gilt die Formel {(\ref{e.nx})} sogar f"ur alle $n\in\Z$.
Ist $x=m/n$ ($m,n\in\N$) nun eine rationale Zahl, folgt
\begin{align*}
& f(m)=f\br{n\cdot\frac{m}{n}}=nf\br{\frac{m}{n}},
\intertext{woraus man}
& f\br{\frac{m}{n}} = \frac{1}{n}f(m) = \frac{m}{n}\cdot f(1)
\end{align*}
erh"alt. Damit ist $f$ f"ur alle rationalen Argumente bestimmt, wenn $f(1)=c$
vorgegeben wird. 

{\bf Fazit}: Wenn $f$ eine L"osung der Funktionalgleichung ist, so gilt f"ur
rationale $x$
\begin{equation}\label{e.loe}
f(x) = c\cdot x,
\end{equation}
wobei $c = f(1)$ ist. Setzt man nun $f(x)=c\cdot x$ f"ur alle reellen $x$, so
zeigt eine {\bf Probe}, da{ss} diese Funktionen L"osungen der Aufgabe sind.
Allerdings entsteht die Frage, ob noch weitere L\"{o}sungen existieren. Unter
speziellen Voraussetzungen kann das ausgeschlossen werden, zum Beispiel, wenn
man zus\"{a}tzlich fordert, da{ss} $f$ stetig ist. F\"{u}r jedes $x\in\R$
existiert n\"{a}mlich ein Folge rationaler Zahlen $x_n$, die gegen $x$
konvergieren und man kann schlu{ss}folgern, da{ss}
\[f(x)=\lim_{n\to\infty} f(x_n)=\lim_{n\to\infty}cx_n=cx.\]
F\"{u}r schw\"{a}chere Voraussetzungen, unter denen keine weiteren
L\"{o}sungen als die Funktionen $f(x) = c\,x$ existieren, siehe die Aufgaben
\ref{a.1} und \ref{a.3}. 

Es ist \"{u}berraschend, da{ss} ohne Zusatzvoraussetzungen tats\"{a}chlich noch
weitere L\"{o}sungen der Cauchyschen Funktionalgleichung existieren, wie
G. Hamel (1877-1954) erst 1905 gefunden hat. Deren Konstruktion erfordert aber
Kenntnisse, die wesentlich \"{u}ber die Schulmathematik hinausgehen und soll
daher nur angedeutet werden.  Dazu betrachten wir die reellen Zahlen als
Vektorraum \"{u}ber den rationalen Zahlen, d.h. die reellen Zahlen sind die
{\glqq}Vektoren{\grqq}, die rationalen Zahlen die {\glqq}Skalare{\grqq}. Sei
$B=\{b_i:i\in I\}$ eine Basis dieses Vektorraumes, d.h. jedes $x\in\R$ kann
eindeutig dargestellt werden als Linearkombination $x=r_1b_1+\ldots+r_nb_n$
mit gewissen $r_1,\ldots,r_n\in\Q$ und $b_1,\ldots,b_n\in B$. Dann l\"{a}{ss}t
sich zeigen, da{ss} man alle L\"{o}sungen von (\ref{e.cau}) erh\"{a}lt, wenn
man $f(b_i),i\in I,$ beliebig definiert und $f$ dann auf $\R$ fortsetzt mittels
\[f(r_1b_1+\ldots+r_nb_n):=r_1f(b_1)+\ldots+r_nf(b_n).\]
Der Existenznachweis einer Basis $B$ erfordert tiefliegende Hilfsmittel der
Mengenlehre. Die unstetigen L\"{o}sungen von (\ref{e.cau}) sind allerdings
extrem pathologische Funktionen, siehe Aufgabe \ref{a.2}.

\section{Substitutionen}

Eine weitere wichtige Methode ist die {\bf Substitution von Argumenten}.  Wir
demonstrieren dies am Beispiel folgender Verallgemeinerung der Cauchyschen
Funktionalgleichung
\begin{equation}\label{e.cal}
f(ax+by+c)=pf(x)+qf(y)+r\;\;\;\forall x,y\in \R,
\end{equation}
wobei $a,b,c,p,q\in\R$ gegebene Konstanten mit $abpq\neq 0$ sind. Wir
f\"{u}hren nun folgende Substitutionen aus:
\begin{eqnarray}
\label{e.4}x=-\frac{c}{a},y=0\;&\yields
&\;\;f(0)=pf\left(-\frac{c}{a}\right)+q f(0)+r\\
\label{e.5}x=\frac{u-c}{a},y=0\;&\yields
&\;\;f(u)=pf\left(\frac{u-c}{a}\right)+q f(0)+r\\
\label{e.6}x=-\frac{c}{a},y=\frac{v}{b}\;&\yields &\;\;
f(v)=pf\left(-\frac{c}{a}\right)+qf\left(\frac{v}{b}\right)+r\\
\label{e.7}x=\frac{u-c}{a},y=\frac{v}{b}\;&\yields &
\;\;f(u+v)=pf\left(\frac{u-c}{a}\right)+qf\left(\frac{v}{b}\right)+r
\end{eqnarray}
Addition von (\ref{e.4}),(\ref{e.7}) und Subtraktion von (\ref{e.5}),
(\ref{e.6}) f\"{u}hrt auf 
\begin{equation} 
f(u+v)+f(0)-f(u)-f(v)=0
\end{equation}
Den st\"{o}renden Term $f(0)$ zu beseitigen gelingt uns durch {\bf
Einf\"{u}hren einer Hilfsfunktion} mittels $g(u):=f(u)-f(0)$. $g$ erf\"{u}llt
dann die Cauchysche Funktionalgleichung $g(x+y)=g(x)+g(y)$. Als stetige
L\"{o}sungen der Gleichung (\ref{e.cal}) kommen somit nur die linearen
Funktionen $f(x)=\alpha x+\beta$ in Frage. Einsetzen ergibt, da{ss} $\alpha $
und $\beta$ die Gleichungen
\begin{eqnarray*}
\alpha a&=&\alpha p\\
\alpha b&=&\alpha q\\
\beta(1-p-q)&=&r-\alpha c
\end{eqnarray*}
erf\"{u}llen m\"{u}ssen, insbesondere existieren nur f\"{u}r $a=p$ und $b=q$
nichtkonstante L\"{o}sungen.

Eine weitere Gleichung, die auf die Cauchysche Funktionalgleichung
zur\"{u}ckgef\"{u}hrt werden kann, befindet sich in Aufgabe \ref{a.4}.

\section{Anwendung der Differentialrechung}

Auch wenn in der Aufgabenstellung nicht vorausgesetzt wurde, dass die gesuchte
Funktion differenzierbar ist, kann man Differentialrechnung als heuristisches
Mittel einsetzen und die gegebene Funktionalgleichung unter Anwendung der
\"{u}blichen Rechenregeln nach einem oder mehreren Argumenten differenzieren.
Auf diese Weise erh"alt man mitunter sehr schnell eine Vermutung "uber die
L"osung. Die Probe muss zeigen, ob tats"achlich eine L"osung vorliegt. Wenn es
dann noch gelingt nachzuweisen, dass au"ser der (oder den) gefundenen
(differenzierbaren) L"osung keine anderen L"osungen existieren
(Einzigkeitsnachweis), ist der L"osungsweg komplett. \vspace*{1ex}

\noindent Beispiel: (OJM 241233A) Man bestimme alle auf der Menge der
rationalen Zahlen definierten Funktionen $f$ mit $f(1)=1$, die der
Funktionalgleichung
\[f(x+y)=f(x)+f(y)+xy\,(x+y)\]
gen"ugen. \vspace*{2ex}

Unter der (vorl"aufigen) Annahme, dass $f$ sogar f"ur alle reellen Zahlen
definiert und differenzierbar ist, ergibt Differentiation nach $x$
\begin{align*}
f'(x+y) &= f'(x) + 2xy + y^2\\
f''(x+y) &= f''(x) + 2y\\
f'''(x+y) &= f'''(x).
\intertext{Setzt man $x=0$, folgt $f'''(y)={\rm  const}$. Also muss $f$ ein 
Polynom dritten Grades sein, das nun leicht bestimmt werden kann:}
f(x) &= \frac{1}{3} x^3 + \frac{2}{3} x.
\end{align*}
Die Probe best"atigt, dass diese Funktion L"osung ist. Um zu zeigen, dass es
keine weiteren L"osungen gibt, bezeichnen wir die gefundene L"osung mit
$f_0(x) = \frac{1}{3} x^3 + \frac{2}{3} x$ und suchen weitere L"osungen in der
Form $f(x) = f_0(x)+g(x)$, mit einer neuen unbekannten Funktion $g$ (die nun
nicht mehr als differenzierbar vorausgesetzt wird). Einsetzen in die gegebene
Funktionalgleichung f"uhrt auf
\[g(x+y) = g(x) + g(y)\]
Ferner gilt $g(1)=0$. Da $g$ nur f"ur rationale Argumente zu bestimmen ist,
folgt aus den Resultaten "uber die Cauchysche Funktionalgleichung $g(x)\equiv
0$.

Die Technik, sich erst eine spezielle L\"{o}sung zu verschaffen und dann die
allgemeine L\"{o}sung durch einen Ansatz zu bestimmen ist auch bei Aufgabe
\ref{a.13} erfolgversprechend.

\section{Periodische Funktionen}

Eine Funktionen $f:\R\to\R$ hei{\ss}t {\it periodisch}, wenn eine reelle Zahl
$p\neq 0$ existiert, so da{ss}
\begin{equation} \label{e.per}
f(x+p)=f(x)
\end{equation}
f\"{u}r alle $x\in\R$ gilt. Existiert eine kleinste positive Zahl $p$ mit
(\ref{e.per}), so hei{\ss}t $p$ die {\it primitive Periode} der Funktion $f$.
Nicht alle periodischen Funktionen m\"{u}ssen eine primitive Periode haben, so
ist beispielsweise jede rationale Zahl Periode der Dirichletschen Funktion
\[f(x)=
  \begin{cases}
    0 &  \text{falls $x$ rational} \\
    1 & \text{falls $x$ irrational.}
  \end{cases}\]
Als Beispiel zu periodischen Funktionen betrachten wir die folgendes Problem:

\noindent Aufgabe: (OJM 141236) Es sei $a\neq 0$ reell und $f:\R\to\R$ eine
Funktion mit der Eigenschaft da{ss} f\"{u}r alle $x\in\R$ gilt
\[f(x+a)=\frac{f(x)}{3f(x)-1}.\]
Man zeige  da{ss} $f$ periodisch ist.

L\"{o}sung: Wir zeigen da{ss} $2a$ eine Periode von $f$ ist:
\[f(x+2a)=f((x+a)+a)=\frac{f(x+a)}{3f(x+a)-1}
=\frac{\frac{f(x)}{3f(x)-1}}{3\frac{f(x)}{3f(x)-1}-1}
=\frac{f(x)}{3f(x)-(3f(x)-1)}=f(x).\] Ganz \"{a}hnlich l\"{a}{ss}t sich auch
Aufgabe \ref{a.10} l\"{o}sen.

\section{Aufgaben}
\begin{enumerate}
\item \label{a.1} (J.G. Darboux, 1842-1917) Alle L\"{o}sungen $f$ der
Cauchyschen Funktionalgleichung (\ref{e.cau}), f\"{u}r die ein $\varepsilon
>0$ existiert, so da{ss} $f(x)$ nichtnegativ (oder nicht-positiv) ist f\"{u}r
$0<x<\varepsilon$ sind von der Form (\ref{e.loe}).
\item (OJM 191233A) \label{a.3} Man ermittle alle reellen Funktionen mit
(\ref{e.cau}) unter den Zusatzbedingungen $f(1)=1$ und
\[f\br{\frac{1}{x}} =\frac{1}{x^2}f(x)\;\;\;\forall x\in\R\setminus \{0\}.\]
\item \label{a.2} Ist die L\"{o}sung $f$ der Cauchyschen Funktionalgleichung
nicht von der Form {\rm (\ref{e.loe})}, so ist der Graph
\[\{(x,f(x))\colon\ x\in \R \}\]
eine in der $x$-$y$-Ebene dichte Menge, d.h. zu jedem $(x_0,y_0)\in\R^2$
existiert eine Folge $(x_n)$ mit $x_n\to x_0$ und $f(x_n)\to y_0$.
\item \label{a.4} Die Funktion $g:\R\to\R$ erf\"{u}llt genau dann die
Funktionalgleichung
\[g(x+y)=g(x)g(y),\]
wenn sie entweder identisch verschwindet oder von der Form $g(x)=e^{f(x)}$
ist, wobei $f$ eine L\"{o}sung von (\ref{e.cau}) ist.
\item \label{a.6} (OJM 51233) Seien $a,b,c$ reelle Zahlen. Man bestimme in
Abh\"{a}ngigkeit von $a,b,c$ s\"{a}mtliche Funktionen $f:\R\to\R$, die der
Funktionalgleichung
\[af(x-1)+bf(1-x)=cx\]
gen\"{u}gen.
\item Man ermittle alle diejenigen Funktionen $f$, die f\"{u}r alle reellen
Zahlen $x$ definiert sind und den folgenden Bedingungen gen\"{u}gen:
\begin{enumerate}
  \item[(1)] F\"{u}r alle $x,y\in\R$ gilt $f(xy)=f(x)f(y)$.

  \item[(2)] F\"{u}r alle reellen Zahlen  $x\neq 0$  gilt
  \[f(1+x)=f(1)+f(x)\left(1+\frac{2}{x}\right).\]

\end{enumerate}
\item (OJM 171236A) \label{a.5} Gegeben seien eine nat\"{u}rliche Zahl $n>1$
und die Funktionalgleichung
\begin{equation}\label{e.9}
  f(x^n)=f(x).
\end{equation}
\begin{enumerate}
  \item Man ermittle alle stetigen Funktion $f:\R\to\R$ mit (\ref{e.9}).
  \item Man gebe eine unstetige L\"{o}sung von (\ref{e.9}) an.
\end{enumerate}
\item (Bulgarische MO 1982) Man bestimme alle reellwertigen, an der Stelle
$x=0$ stetigen Funktionen $f$ mit dem Definitionsbereich $\R$, welche die
Funktionalgleichung
\[f(x)=\frac{1}{2}f\left(\frac{x}{2}\right)+x\]
erf\"{u}llen.
\item (Russische MO 2000) Gib alle Funktionen $f:\R\to\ R$ an, die f\"{u}r
alle $x,y,z\in\R$ der Ungleichung
\[f(x+y)+f(y+z)+f(z+x)\geq 3f(x+2y+3z)\]
gen\"{u}gen.
\item (IMO 1992) Bestimme alle Funktionen $f:\R\to\R$, die der
Funktionalgleichung
\[f(x^2+f(y))=y+f(x)^2\]
gen\"{u}gen.
\item (OJM 141236) \label{a.10} Es sei $a\neq 0$ reell und $f:\R\to\R$ eine
Funktion mit der Eigenschaft da{ss} f\"{u}r alle $x\in\R$ gilt
\[f(x+a)=\frac{1+f(x)}{1-f(x)}.\]
Man zeige, da{ss} $f$ periodisch ist, und gebe ein Beispiel f\"{u}r eine
solche Funktion an.
\item \label{a.12} Bestimme alle Funktionen $f:\R\setminus \{0,1\}\to\R$ mit
\[f(x)+f\left(\frac{1}{1-x}\right)=x\]
f\"{u}r alle $x\neq 0,1$.
\item \label{a.13} Man l\"{o}se in der Klasse der stetigen Funktionen die
Funktionalgleichung
\[f(x+y)=f(x)+f(y)+f(x)f(y).\]
\item (Neuseel\"{a}ndische MO 1998) Bestimme alle Funktionen $f:\R\to\R$,
welche der Funktionalgleichung
\[f(xf(x)+f(y))=y+f(x)^2\]
gen\"{u}gen.

\end{enumerate}

\begin{thebibliography}{99}
\bibitem{Ac} Acz\'{e}l, J.: On Applications and Theory of Functional
Equations. Birkh\"{a}user Verlag 1969.
\bibitem{Bu} Burosch, G. u.a.: IMO-\"{U}bungsaufgaben, Heft 20.  Herausgegeben
vom Zentralen Komitee f\"{u}r die Olympiaden Junger Mathematiker beim
Ministerium f\"{u}r Volksbildung der DDR.
\bibitem{En} Engel, A.: Problem Solving Strategies. Springer Verlag 1998.
\bibitem{ne} \url{http://www.matholymp.com}
\bibitem{ru} \url{http://www.mccme.ru/olympiads}
\bibitem{Le} Lehmann, J.: Mathematischer Lesebogen {\glqq}Junge
Mathematiker{\grqq} Heft 80, herausgegeben vom Bezirkskabinett f\"{u}r
au{\ss}erunterrichtliche T\"{a}tigkeit, Rat des Bezirkes Leipzig 1987.
\bibitem{SW} Sprengel, H.-J., Wilhelm, O.: Funktionen und
Funktionalgleichungen. Mathematische Sch\"{u}lerb\"{u}cherei 114, VEB
Deutscher Verlag der Wissenschaften 1984.

\end{thebibliography}
\begin{attribution}
semmler (Dec 2004): Contributed to KoSemNet

graebe (2005-01-05): Prepared along the KoSemNet rules
\end{attribution}
\end{document}
