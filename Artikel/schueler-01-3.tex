\documentclass[11pt]{article}
\usepackage{ngerman,schueler,url}
\usepackage{kosemnet,ko-math}

\title{Der Gro"se Fermatsche Satz\kosemnetlicensemark}
\author{Axel Sch"uler}
\date{31.3.2001}

\begin{document}

\maketitle
\subsection*{Gliederung}
\begin{enumerate}
\item Einf"uhrung
\item Pierre de Fermat
\item Die Geschichte einer Gleichung
\item Cambridge, 23. Juni 1993
\item Ein Problem tut sich auf \dots
\item Literatur
\end{enumerate}

\subsection*{Einf"uhrung}
Die Geschichte des gro"sen Fermatschen Satzes (auch \glqq letzter\grqq\ Fermatscher
Satz genannt) ber"uhrt alle gro"sen Themen der Zahlentheorie und ist daher untrennbar  verwoben mit der Geschichte der Mathematik.
Sie gew"ahrt ungew"ohnliche Einsichten in die treibenden Kr"afte der Mathematik und, vielleicht noch wichtiger,
in die Beweggr"unde  und Ziele der Mathematiker selbst. Die Fermatsche Vermutung
bildet das Herzst"uck einer fesselnden Saga, die von K"uhnheit, Geflunker, Scharfsinn und tragischem Leid handelt und in der alle Helden der 
Mathematik auftreten.

Die Faszinatin des Problems liegt auch in seiner Einfachheit. Selbst ein
Grundsch"uler kann die Fragestellung verstehen.
\begin{quote}
Die Gleichung
\begin{align}\label{f}
x^n+y^n=z^n
\end{align}
besitzt f"ur $n\ge 3$ keine L"osungen in nat"urlichen Zahlen
$x,y,z\ge 1$.
\end{quote}
Im 17.\,Jahrhundert lie"s Pierre de Fermat dieses Problem ohne Absicht
zu einer Herausforderung f"ur alle Nachfolger werden. Ein begnadeter Mathematiker
nach dem andern musste vor Fermats Hinterlassenschaft dem"utig kapitulieren,
und drei Jahrzehnte lang gelang es keinem einzigen, das Problem zu l"osen.
Zwar gibt es in der Mathematik auch andere ungel"oste Probleme,
doch das Besondere an Fermats Problem ist seine tr"ugerische Schlichtheit.
Diese kommt sicherlich auch von seiner "Ahnlichkeit mit dem bekannten 
Satz des Pythagoras. Wohl jeder  Zehntkl"assler findet auf Anhieb
eine ganzzahlige L"osung der Gleichung
\begin{align}\label{p}
x^2+y^2=z^2
\end{align}
mit $x,y,z\ge 1$, etwa $(3,4,5)$, $(5,12,13)$ oder $(15,8,17)$.
 Ohne gro"se Probleme best"atigt man durch einfaches Nachrechnen, 
dass man f"ur alle Paare $m>n$ von teilerfremden nat"urlichen Zahlen unterschiedlicher Parit"at  mit
\begin{align*}
x=2mn,\quad y=m^2-n^2,\quad z=m^2+n^2
\end{align*}
eine L"osung von \rf[p] hat. Auch nicht viel schwerer ist es zu zeigen,
dass dies {\em alle} teilerfremden L"osungen von \rf[p] sind.
Dies wurde mit Sch"ulern der Klassen 9 bis 12 aus Sachsen in der Mathematischen
Winterschule 2001 der Leipziger Sch"ulergesellschaft f"ur Mathematik (LSGM) 
in Colditz durchgef"uhrt.

\subsection*{Pierre de Fermat}
Pierre de Fermat wurde am 20. August 1601 in S"udfrankreich als Sohn eines 
wohlhabenden Lederh"andlers geboren. Er genoss eine hervorragende Schulbildung in einem Franziskanerkloster. Auf Dr"angen seiner Eltern
schlug er eine juristische Laufbahn ein und wurde ein gewissenhafter und
flei"siger Staatsdiener. In Europa ging in dieser Zeit die Pest um. Auch
Fermat erkrankte und wurde schon von Kollegen tot geglaubt. Doch Fermat
"uberlebte nicht nur die gesundheitlichen Gefahren, er "uberstand auch
die politischen. Er hatte keinen politischen Ehrgeiz und 
widmete alle seine freien Kr"afte der Mathematik. Manche bezeichnen ihn als
\glqq F"urst der Amateure\grqq\ andere z"ahlen ihn zu den Professionellen.
Eine wichtige integrierende Rolle spielte in dieser Zeit der M"onch
und Mathematiker Mersenne. Er war entschlossen, den damals bestehenden
Ethos der Verschwiegenheit zu bek"ampfen, und er forderte die Mathematiker
auf, ihre Gedanken offenzulegen und sich ihre Arbeiten gegenseitig zunutze zu
machen. Mersenne war offensichtlich der einzige Mathematikerkollege, mit
dem Fermat regelm"a"sig zusammentraf. Mersenne ermunterte Fermat, seine Beweise
zu ver"offentlichen, doch dieser weigerte sich beharrlich. Das scheue
und zur"uckgezogene Genie Fermat besa"s freilich auch einen schelmischen
Zug. Er schrieb Briefe an andere Mathematiker, in denen er seine neuesten
S"atze verk"undete, ohne deren Beweis mitzuliefern. Dann forderte er seine 
Zeitgenossen auf, diesen zu suchen. 
\\
Mersennes Einfluss auf Fermat wird wohl nur noch von der {\em Arithmetica}
des Diophantos von Alexandria "ubertroffen. Dies ist ein Lehrbuch der
Zahlentheorie, vergleichbar mit den {\em Elementen} des Euklid. Fermat
machte eine Vielzahl von Randnotizen in diesem Werk, unter anderem:
\begin{quote}
Es ist nicht m"oglich einen Kubus in zwei Kuben, ein Biquadrat in zwei
Biquadrate und allgemein eine Potenz, h"oher als die zweite, in zwei
Potenzen mit dem selben Exponenten zu zerlegen.
\end{quote}
Und etwas weiter findet man:
\begin{quote}
Ich habe hierf"ur einen wahrhaft wunderbaren Beweis, doch ist dieser 
Rand zu schmal, ihn zu fassen.
\end{quote}
Fermat starb am 12.\,Januar 1665. 
In der Ver"offentlichung seines Nachlasses 1670 durch Cl{\'e}ment-Samuel findet man 48 solcher
Bemerkungen, eine wahre Schatztruhe von Entdeckungen.

\subsection*{Die Geschichte einer Gleichung}
\begin{tabular}{rl}
1635& Fermat formuliert das Problem als Randnotiz. Er l"ost den Fall $n=4$ und 
\\
& f"uhrt dabei die \glqq Methode des unendlichen Abstieges\grqq\ ein. Sie 
besagt,
\\
&dass jede Teilmenge der nat"urlichen Zahlen ein kleinstes Element besitzt.
\\
4.\,8.\,1753& Euler zeigt den Fall $n=3$ mit der Abstiegsmethode unter Benutzung
\\
& der komplexen Zahlen.
\\
1825& Sophie Germain, Dirichlet und Legendre l"osen den Fall $p=5$.
\\
1839& Lam{\'e} erledigt  $p=7$.
\\
1.\,3.\,1847&Cauchy und Lam{\'e} k"undigen gleichzeitig und unabh"angig
voneinander 
\\&vor der Akademie ihre Beweise an.
\\
24.\,5.\,1847&Liouville verliest vor der Akademie 
einen Brief von Ernst Kummer 
und 
\\&
l"ost einen Schock aus:
Die Beweise von Cauchy und Lam{\'e} beruhen
\\&
 auf einem Irrtum.  Die eindeutige 
Primfaktorzerlegung gilt nicht
\\& in allen Ringen $\Z[\zeta]$ mit $\zeta^p=1$,
etwa nicht f"ur $p=37,\,59$ und $67$.
\\
1875& Der 1853 ausgeschriebene Preis der franz"osischen Akademie geht an Kummer.
\\
27.\,6.\,1908& Der verstorbene Darmst"adter Industrielle Paul Wofskehl setzte
\\& ein Preisgeld von $1.000.000$ Mark f"ur denjenigen aus,
\\& der den Gro"sen Satz von Fermat beweist.
\\
1983& Gerd Faltings zeigt, dass die Gleichung \rf[f] f"ur jede nat"urliche
\\& Zahl $n$ h"ochstens endlich viele L"osungen besitzt.
\\
1988& Naom Elkies zeigt, dass die Euler-Vermutung falsch ist: Es gibt eine
\\& L"osung der Gleichung $x^4+y^4+z^4=w^4$ in nat"urlichen Zahlen, n"amlich
\\&$(2682440,15365639,18796760,20615673)$.
\\
1993& Der gro"se Fermatsche Satz gilt f"ur alle Primzahlen $p<4000000$.
\\
23.\,6.\,1993& Andrew Wiles tr"agt den Beweis des Fermats am
Newton-Institut
\\& in Cambridge vor. 
\\
23.\,8.\,1993& Nick Katz entdeckt eine L"ucke im Beweis.
\\
25.\,10.\,1994& Richard Taylor und Andrew Wiles finden einen neuen Teilbeweis
\\&
f"ur die L"ucke. Der gro"se Fermat ist bewiesen.
\end{tabular}

\medskip

Die Frist f"ur die Verleihung des Wolfskehlpreises endet am 13.\,9.\,2007.
Umgerechnet auf heutige Verh"altnisse h"atte das Preisgeld eine 
H"ohe von 2,5 Millionen DM gehabt.
Durch den Wertverlust bei der Inflation sank das Preisgeld allerdings und war
bei seiner "Ubergabe an Andrew Wiles etwa 70000 DM wert.
Im ersten Jahr sind 621 \glqq L"osungen\grqq\  bei der G"ottinger K"oniglichen
Gesellschaft der Wissenschaften eingegangen. Der Sekret"ar der Akademie teilt
die eingehenden Manuskripte auf in
 v"olligen Unsinn, der sofort zur"uckgeschickt wird und
Material, das wie Mathematik aussieht.
\subsubsection*{Cambridge, 23.\,Juni 1993}
Es war die wichtigste Mathematikvorlesung des Jahrhunderts. Zweihundert 
Mathematiker lauschten wie gebannt. Nur ein Viertel von ihnen verstand das 
dichte Gemenge aus griechischen Symbolen und algebraischen Formeln an der 
Tafel. Die "ubrigen waren einfach in der Hoffnung gekommen, Zeugen eines 
historischen Ereignisses zu werden. 
\\
Tags zuvor waren Ger"uchte aufgekommen. In der elektronischen Post des Internet 
wurde gemunkelt, die Vorlesung werde mit der L"osung eines weltber"uhmten 
mathematischen Problems enden, mit dem Beweis von Fermats letztem Satz.
\\
Die drei Tafeln waren nun vollgeschrieben mit Rechnungen, und der Vortragende 
hielt inne, um die erste Tafel zu wischen. Dann setzte er seine algebraischen 
Er"orterungen fort. Jede Zeile schien ihn der L"osung einen kleinen Schritt 
n"aher zu bringen, doch auch eine Dreiviertelstunde sp"ater hatte er den Beweis 
noch nicht verk"undet. Die Professoren, dicht gedr"angt in den vordersten 
Stuhlreihen warteten ungeduldig auf die L"osung. Hinten im Raum standen viele 
Studenten, die sich fragend nach den "alteren Semestern umsahen, um vielleicht 
einen Fingerzeig auf die L"osung zu erhalten. Waren sie Zeugen eines 
vollst"andigen Beweises von Fermats letztem Satz, oder wurde dort vorne blo"s 
ein unvollst"andiger Gedankengang vorgetragen, dem die Pointe fehlte?

Am 23.\,Juni begann Andrew Wiles seinen dritten und letzten Vortrag, erinnerte 
sich John Coates. Erstaunlicherweise waren praktisch alle, die Ideen zu dem 
Beweis beigesteuert hatten, im Raum versammelt, Mazur, Ribet, Kolywagin und 
viele, viele andere. Inzwischen hatten sich die Ger"uchte so sehr verdichtet, 
dass die gesamte Mathematikergemeinde von Cambridge zum letzten Vortrag 
erschien. Wer Gl"uck hatte, konnte sich noch ins  Auditorium zw"angen, die 
anderen mussten drau"sen im Gang bleiben, wo sie auf Zehenspitzen stehend 
durchs Fenster sp"ahten. Ken Ribet hatte sich vorgenommen, die wichtigsten 
mathematischen Darlegung des Jahrhunderts nicht zu vers"aumen:\glqq Ich kam 
ziemlich fr"uh und setzte mich mit Barry Mazur in die erste Reihe. Um das 
Ereignis festzuhalten, hatte ich meine Kamera dabei.  Die Athmosph"are war sehr 
geladen, die Leute waren aufgeregt. Nat"urlich hatten wir das Gef"uhl, an einem 
historischen Moment teilzuhaben. Vor und w"ahrend des Vortrages hatten die 
Leute verschmitzte Gesichter. Die Spannung hatte sich "uber mehrere Tage hin 
aufgebaut. Dann kam dieser herrliche Augenblick, als wir uns dem Beweis von 
Fermats letztem Satz n"aherten. Ich habe nie einen so gro"sartigen Vortrag 
erlebt, mit so vielen gl"anzenden Ideen, voll dramatischer Spannung, und so gut 
aufgebaut. Es gab nur eine m"ogliche Pointe.\grqq\ Nach sieben Jahren 
energischer Arbeit war Wiles nun bereit, der Welt seinen Beweis zu verk"unden. 
An die letzten Augenblicke des Vortrags kann er sich merkw"urdigerweise nicht 
besonders gut erinnern, an die Atmosph"are schon:\glqq Obwohl die Presse schon 
Wind von dem Vortrag hatte, war sie gl"ucklicherweise nicht dabei. Doch im 
Publikum sa"sen eine Menge Leute, die gegen Ende Fotos machten, und der 
Institutsdirektor war gut vorbereitet mit einer Flacshe Champagner gekommmen. 
W"ahrend ich den Beweis vortrug, herrschte das typische w"urdevolle Schweigen, 
und dann schrieb ich einfach Fermats letzten Satz an die Tafel. \glq Ich denke, 
das gen"ugt\grq, meinte ich dann, und es gab langen Beifall\grqq.

Wiles reichte sein Manuskript bei der Zeitschrift {\em Inventiones 
Mathematicae} ein, und nun war es an deren Herausgeber Barry Mazur, die 
Gutachter auszuw"ahlen. Wiles hatte eine derartige Vielfalt von modernen und 
klassischen Verfahren f"ur die Arbeit herangezogen, dass Mazur die 
au"sergew"ohnliche Entscheidung traf, nicht nur, wie "ublich, zwei oder drei 
Gutachter zu benennen, sondern sechs. J"ahrlich erscheinen rund um den Globus 
30000 Artikel, doch die schiere Gr"o"se und Bedeutung von 
Wiles' Manuskript verlangte eine besonders kritische Pr"ufung. Um die Sache zu 
vereinfachen, unterteilte man das zweihundertseitige Manuskript in sechs 
Abschnitte, f"ur die jeweils einer der Gutachter die Verantwortung "ubernahm.

\subsection*{Ein Problem tut sich auf \dots}
Wiles:\glqq Ich konnte diese bestimmte, sehr unscheinbar aussehende Frage nicht 
sofort kl"aren. Eine Weile lang schien sie mir vom selben Schlag zu sein, wie 
die anderen Probleme, doch dann, irgendwann im September, wurde mir klar, dass 
dies eben keine belanglose kleine Schwierigkeit war, sondern ein elementarer 
Fehler. Im entscheidenden Teil der Argumentation mit der 
Kolywagin-Flach-Methode steckte ein Irrtum, aber ein derart unterschwelliger, 
dass ich ihn bis dahin v"ollig "ubersehen hatte \dots\grqq. 
\\
Je tiefer es in den Winter hinein ging, desto  mehr schwanden die Hoffnungen auf 
einen Durchbruch, und eine immer gr"o"sere Zahl von Mathematikern "au"serte die 
Meinung, es sei Wiles' Pflicht, das Manuskript freizugeben. Die Ger"uchte 
wollten nicht verstummen, und in einer Zeitung hie"s es, Wiles habe aufgegeben 
und der Beweis sei unwiderruflich gescheitert.

{\em Die Befreiung}
\\
25.\,10.\,1994. Die Iwasawa-Theorie alleine war unzul"anglich. Die Kolywagin-Flach-Methode f"ur 
sich genommen ebenfalls. Zusammen erg"anzten sie sich aufs beste. Diesen Moment 
der Inspiration wird Wiles nie vergessen. Die Erinnerung daran war so 
eindringlich, dass er zu Tr"anen bewegt war:\glqq Es war so unbeschreiblich 
sch"on; so einfach und elegant. Ich konnte nicht begreifen, wie mir das hatte 
entgehen k"onnen, und zwanzig Minuten lang starrte ich nur ungl"aubig auf die 
L"osung. Dann ging ich den Tag "uber im Fachbereich umher und kam immer wieder 
zum Schreibtisch zur"uck, um zu sehen, ob sie noch da war. Ich war ganz aus dem 
H"auschen vor Aufregung. Das war der wichtigste Moment meines Arbeitslebens. 
Nichts, was ich jemals tun werde, wird so viel bedeuten.\grqq

{\em Die Struktur des Beweises}
\\
(1) Wenn Fermats letzter Satz falsch ist, dann existiert Freys elliptische 
Gleichung.
\\
(2) Freys elliptische Gleichung ist so abwegig, dass sie nicht modular sein 
kann.
\\
(3) Die Taniyama-Shimura-Vermutung von 1955 besagt, dass jede elliptische Kurve 
modular ist.
\\
(4) Dann ist aber die Taniyama-Shimura-Vermutung falsch.


\begin{thebibliography}{1}

\bibitem{b-Bell}
E.~T. Bell, \emph{The last problem. Rev. and updated and with an introduction
  and notes by Underwood Dudley}, MAA Spectrum, Mathematical Association of
  America, Washington, 1990.

\bibitem{p-Cornell}
G.~Cornell and J.~H. Silverman (eds.), \emph{Modular forms and Fermat's last
  theorem. Papers from a conference}, New York, Springer-Verlag, 1995.

\bibitem{b-Scharlau}
W.~Scharlau and H.~Opolka, \emph{Von {F}ermat bis {M}inkowski. Eine {V}orlesung
  {\"u}ber {Z}ahlentheorie und ihre {E}ntwicklung}, Springer-Verlag,
  Berlin-Heidelberg-New York, 1980.

\bibitem{b-Singh}
S.~Singh, \emph{Fermats letzter Satz. Die abenteuerliche Geschichte eines
  mathematischen R{\" a}tsels}, Deutscher Taschenbuch Verlag, M{\" u}nchen,
  2000.

\end{thebibliography}
\begin{attribution}
schueler (2004-09-09): Contributed to KoSemNet

graebe (2004-09-09): Prepared along the KoSemNet rules
\end{attribution}
\end{document}
