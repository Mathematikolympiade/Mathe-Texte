\documentclass[11pt,a4paper]{article}
\usepackage{kosemnet,ko-math,ngerman,url}
\usepackage[utf8]{inputenc}

\title{Sätzesammlung Elementargeometrie.\\ Ein Kompendium für Klasse
  10\kosemnetlicensemark} 
\author{Lisa Sauermann}
\date{März 2013}

\begin{document}
\maketitle

Grundlage zum Lösen elementargeometrischer Aufgaben ist die Kenntnis
geometrischer Sachverhalte und Sätze, von denen einige hier überblicksartig
aufgeführt sind. Die Beweise lassen sich bei Bedarf in zahlreichen
Geometriebüchern oder im Internet nachlesen. Besonders empfohlen sei das
Geometriekapitel von Christian Reiher in „Ein-Blick in die Mathematik“ (von
Richard Bamler, Christian Reiher et al., Aulis Verlag Deubner).

\subsection*{Dreiecksgeometrie}

\begin{itemize}
\item Die Winkelhalbierende und die Mittelsenkrechte der gegenüberliegenden
  Seite eines Dreiecks schneiden sich auf dem Umkreis.
\item Der Bildpunkt des Höhenschnittpunkts eines Dreiecks bei Spiegelung an
  einer Dreiecksseite liegt auf dem Umkreis. Der Bildpunkt des
  Höhenschnittpunkts bei Spiegelung an einem Seitenmittelpunkt liegt ebenfalls
  auf dem Umkreis und liegt dem der Dreiecksseite gegenüberliegenden Eckpunkt
  auf dem Umkreis diametral gegenüber.
\item Der Inkreismittelpunkt eines Dreiecks ist der Höhenschnittpunkt des
  zugehörigen Dreiecks aus den drei Ankreismittelpunkten.
\item Der Höhenschnittpunkt eines Dreiecks ist der Inkreismittelpunkt des
  zugehörigen Dreiecks aus den drei Höhenfußpunkten.
\item Der Umkreismittelpunkt eines Dreiecks ist der Höhenschnittpunkt des
  zugehörigen Dreiecks aus den drei Seitenmittelpunkten.
\item \textbf{Euler-Gerade:} Höhenschnittpunkt, Schwerpunkt,
  Feuerbachkreismittelpunkt und Umkreismittelpunkt eines Dreiecks liegen auf
  einer Gerade, der Euler-Gerade. Dabei wird die Strecke vom
  Umkreismittelpunkt zum Höhenschnittpunkt vom Schwerpunkt im Verhältnis $1:2$
  geteilt und der Feuerbachkreismittelpunkt ist ihr Mittelpunkt.
\item \textbf{Feuerbachkreis:} Die drei Seitenmittelpunkte, die drei
  Höhenfußpunkte und die drei Mittelpunkte der oberen Höhenabschnitte
  (d.\,h.\ der Strecken zwischen Eckpunkt und Höhenschnittpunkt) liegen auf
  einem Kreis, welcher Feuerbachkreis oder Neun-Punkte-Kreis genannt wird.
\item \textbf{Simson-Gerade:} Für ein Dreieck $\ktriangle*{ABC}$ und einen
  Punkt $P$ liegen die Lotfußpunkte von $P$ auf die Geraden $\kline{AB}$,
  $\kline{BC}$ und $\kline{CA}$ durch die Seiten des Dreiecks genau dann auf
  einer Geraden, wenn $P$ auf dem Umkreis des Dreiecks $\ktriangle*{ABC}$
  liegt.
\item \textbf{Satz von Ceva:} Für ein Dreieck $\ktriangle*{ABC}$ und von den
  Ecken verschiedene Punkte $X$, $Y$ und $Z$ auf den Geraden $\kline{BC}$,
  $\kline{CA}$ bzw.\ $\kline{AB}$ schneiden sich die Geraden $\kline{AX}$,
  $\kline{BY}$ bzw.\ $\kline{CZ}$ genau dann in einem Punkt, wenn 
  \begin{gather*}
    \frac{\msegment{BX}}{\msegment{XC}}\cdot
    \frac{\msegment{CY}}{\msegment{YA}}\cdot
    \frac{\msegment{AZ}}{\msegment{ZB}}=1
  \end{gather*}
  gilt.  Die Streckenlängen sind dabei als orientierte Streckenlängen zu
  verstehen. 
\item \textbf{Trigonometrischer Ceva:} Für ein Dreieck $\ktriangle*{ABC}$ und
  von den Ecken verschiedene Punkte $X$, $Y$ und $Z$ auf den Geraden
  $\kline{BC}$, $\kline{CA}$ bzw.\ $\kline{AB}$ schneiden sich die Geraden
  $\kline{AX}$, $\kline{BY}$ bzw.\ $\kline{CZ}$ genau dann in einem Punkt,
  wenn 
  \begin{gather*}
    \frac{\sin\br{\kangle{BAX}}}{\sin\br{\kangle{XAC}}}\cdot
    \frac{\sin\br{\kangle{CBY}}}{\sin\br{\kangle{YBA}}}\cdot
    \frac{\sin\br{\kangle{ACZ}}}{\sin\br{\kangle{ZCB}}}=1
  \end{gather*}
  gilt. Die Winkelgrößen sind dabei als orientierte Winkelgrößen zu verstehen.
\item \textbf{Satz von Menelaos:} Für ein Dreieck $\ktriangle*{ABC}$ liegen
  von den Ecken verschiedene Punkte $X$, $Y$ und $Z$ auf den Geraden
  $\kline{BC}$, $\kline{CA}$ bzw.\ $\kline{AB}$ genau dann auf einer Gerade,
  wenn 
  \begin{gather*}
    \frac{\msegment{BX}}{\msegment{XC}}\cdot
    \frac{\msegment{CY}}{\msegment{YA}}\cdot
    \frac{\msegment{AZ}}{\msegment{ZB}}=-1
  \end{gather*}
  gilt.  Die Streckenlängen sind dabei als orientierte Streckenlängen zu
  verstehen. 
\item \textbf{Satz von Carnot:} Für ein Dreieck $\ktriangle*{ABC}$ und Punkte
  $X$, $Y$ und $Z$ schneiden sich die drei Senkrechten zu den Geraden
  $\kline{BC}$, $\kline{CA}$ und $\kline{AB}$ durch $X$, $Y$ bzw.\ $Z$ genau
  dann in einem Punkt, wenn 
  \begin{gather*}
    \msegment{BX}^{2}-\msegment{XC}^{2} +\msegment{CY}^{2}-\msegment{YA}^{2}
    +\msegment{AZ}^{2}-\msegment{ZB}^{2}=0
  \end{gather*}
  gilt.
\item \textbf{Satz von Stewart:} Für ein Dreieck $\ktriangle*{ABC}$ und einen
  Punkt $P$ auf der Geraden $\kline{BC}$ gilt 
  \begin{gather*}
    \msegment{AC}^{2}\cdot \msegment{BP} +\msegment{AB}^{2}\cdot \msegment{CP}
    =\msegment{BC}\cdot\br{\msegment{BP}\cdot
      \msegment{CP}+\msegment{AP}^{2}}\,.
  \end{gather*}
\item \textbf{Dreiecksfläche:} Die Fläche eines Dreiecks mit Seitenlängen $a$,
  $b$, $c$, Umkreisradius $R$, Inkreisradius $r$, und Halbumfang
  $s=\frac{a+b+c}{2}$ beträgt 
  \begin{gather*}
    \sqrt{s\,(s-a)\,(s-b)\,(s-c)}=r\cdot s=\frac{a\,b\,c}{4\,R}\,.
  \end{gather*}
\end{itemize}
\subsection*{Vierecke}
\begin{itemize}
\item \textbf{Satz von Ptolemäus:} Für ein Sehnenviereck $ABCD$ gilt
  \begin{gather*}
    \msegment{AB}\cdot \msegment{CD} +\msegment{BC}\cdot \msegment{DA}
    =\msegment{AC}\cdot \msegment{BD}\,.
  \end{gather*}
\item \textbf{Ungleichung von Ptolemäus:} Für ein beliebiges Viereck $ABCD$
  gilt 
  \begin{gather*}
    \msegment{AB}\cdot \msegment{CD}+\msegment{BC}\cdot \msegment{DA}
    \geq\msegment{AC}\cdot \msegment{BD}
  \end{gather*}
  mit Gleichheit genau dann, wenn $A$, $B$, $C$, $D$ in dieser Reihenfolge auf
  einem Kreis oder auf einer Geraden (als entartetem Kreis) liegen.
\item \textbf{Satz von Newton:} Für ein Viereck $ABCD$, dessen
  gegenüberliegende Seiten sich in $E$ bzw.\ $F$ schneiden, liegen die
  Mittelpunkte der Strecken $\msegment{AC}$, $\msegment{BD}$ und
  $\msegment{EF}$ auf einer Geraden.
\item \textbf{Parallelogrammungleichung:} Für ein beliebiges Viereck $ABCD$
  gilt
  \begin{gather*}
    \msegment{AB}^{2}+\msegment{BC}^{2}+\msegment{CD}^{2}+\msegment{DA}^{2}
    \geq \msegment{AC}^{2}+\msegment{BD}^{2}
  \end{gather*}
  mit Gleichheit genau dann wenn $ABCD$ ein Parallelogramm ist.
\end{itemize}

\subsection*{Wichtige Sätze der projektiven Geometrie}

\begin{itemize}
\item \textbf{Satz von Pappos:} Gegeben sind zwei Geraden und $A$, $B$, $C$
  bzw.\ $D$, $E$, $F$ drei Punkte auf jeweils einer der Geraden. Dann liegen
  die drei Schnittpunkte von $\kline{AE}$ mit $\kline{BD}$, $\kline{AF}$ mit
  $\kline{CD}$ bzw.\ $\kline{BF}$ mit $\kline{CE}$ auf einer Geraden.
\item \textbf{Satz von Pascal:} Seien $A$, $B$, $C$, $D$, $E$ und $F$ sechs
  Punkte auf einem Kreis. Dann liegen die drei Schnittpunkte von $\kline{AE}$
  mit $\kline{BD}$, $\kline{AF}$ mit $\kline{CD}$ bzw.\ $\kline{BF}$ mit
  $\kline{CE}$ auf einer Geraden. Das heißt, die drei Schnittpunkte der drei
  Paare gegenüberliegender Seiten eines Sehnensechsecks (hier $AECDBF$) liegen
  auf einer Geraden.
\item \textbf{Satz von Briançon:} Sei $ABCDEF$ ein Tangentensechseck. Dann
  schneiden sich die drei Geraden $\kline{AD}$, $\kline{BE}$ und $\kline{CF}$
  in einem Punkt.
\item In den Sätzen von Pascal und Briançon dürfen auch Punkte des Kreises
  bzw.\ Berührpunkte am Kreis doppelt benutzt werden.
\end{itemize}

\subsection*{weitere Hilfsmittel}
\begin{itemize}
\item zentrische Streckungen
\item ähnliche Dreiecke
\item Sehnenvierecke suchen
\item eine gute Skizze (!!)
\item scharfes Hinschauen und ein geübter Blick
\end{itemize}

\begin{attribution}
sauermann (März 2013): Für KoSemNet freigegeben.

graebe (2014-01-01): Nach den KoSemNet Regeln aufbereitet.
\end{attribution}
\end{document}
