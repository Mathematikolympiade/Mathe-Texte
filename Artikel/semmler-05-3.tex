% Version: $Id: semmler-05-3.tex,v 1.1 2008/09/05 09:52:58 graebe Exp $
\documentclass[11pt]{article}  
\usepackage{kosemnet,ko-math,ngerman,url}  

\author{Gunter Semmler, Freiberg\\\url{semmler@math.tu-freiberg.de}}
\title{Funktionen in der Zahlentheorie\kosemnetlicensemark}
\date{}

\begin{document}
\maketitle         

\section{Die Funktion $[\,]$}
Die {\it Gau{\ss}klammer} oder {\it Integerfunktion} ordnet jeder reellen Zahl
$x$ die gr\"{o}{\ss}te ganze Zahl zu, die h\"{o}chstens so gro{\ss} ist wie
$x$. Schreibt man daf\"{u}r $[x]$, so gilt demzufolge
\[ [x]\leq x<[x]+1.\]
\begin{theorem}
\label{T1}
Sei $n$ ganz und positiv. Der Exponent einer Primzahl $p$ in der
Primfaktorzerlegung von $n!$ ist
\[\left[\frac{n}{p}\right]+\left[\frac{n}{p^2}\right]
+\left[\frac{n}{p^3}\right]+\ldots\] 
\end{theorem}
\begin{beweis}
  \begin{enumerate}
  \item[] $\left[\frac{n}{p}\right]$ der Zahlen $1,\ldots,n$ sind durch $p$
    teilbar.
  \item[] $\left[\frac{n}{p^2}\right]$ der Zahlen $1,\ldots,n$ sind durch $p^2$
    teilbar.
  \end{enumerate}
  \hspace*{5cm}$\vdots$
\end{beweis}

\section{Multiplikative Funktionen}
Eine f\"{u}r alle positiven ganzen Zahlen definierte Funktion $\vartheta$
hei{\ss}t {\it multiplikativ}, wenn f\"{u}r teilerfremde $a_1,a_2$ stets
\begin{equation}\label{1}
\vartheta(a_1 a_2)=\vartheta(a_1)\vartheta(a_2)\end{equation}
gilt.

\noindent Beispiel: $\vartheta(a)=a^s, s\in\C$

\noindent Eigenschaften multiplikativer Funktionen:
\begin{itemize}
\item $\vartheta(1)=1$
\item $\vartheta(a_1\cdots a_n)=\vartheta(a_1)\cdots\vartheta(a_n)$,
falls $a_1,\cdots,a_n$ paarweise teilerfremd sind.
\item Ist $a=p_1^{\alpha_1} p_2^{\alpha_2}\cdots p_k^{\alpha_k}$ kanonische
Zerlegung\footnote{d.h. die $p_i$ sind paarweise verschiedene Primzahlen} von
$a$, so gilt
\begin{equation}\label{2}
\vartheta(p_1^{\alpha_1}\cdots p_k^{\alpha_k}) =\vartheta(p_k^{\alpha_1})
\cdots \vartheta(p_k^{\alpha_k})
\end{equation},
\item Man erh\"{a}lt immer eine multiplikative Funktion, wenn man $\vartheta
(1)=1$ und $\vartheta(p^\alpha) \;(\alpha>0)$ beliebig setzt und $\vartheta$
f\"{u}r alle anderen positiven ganzen Zahlen durch (\ref{2}) erkl\"{a}rt.
\item Das Produkt multiplikativer Funktionen ist wieder multiplikativ.
\end{itemize}
\begin{theorem} Ist $a=p_1^{\alpha_1} p_2^{\alpha_2}\cdots p_k^{\alpha_k}$
kanonische Zerlegung von $a$ so gilt
\begin{equation}\label{3}
\sum_{d|a}\vartheta(d)=(1+\vartheta(p_1)+\vartheta(p_1^2)+\ldots
+\vartheta(p_1^{\alpha_1}))\cdots (1+\vartheta(p_k)+\vartheta(p_k^2)+
\ldots+\vartheta(p_k^{\alpha_k})) 
\end{equation}
\end{theorem}
\begin{beweis}
  L\"{o}st man auf der rechten Seite die Klammern auf, so erh\"{a}lt man
  \[\sum_{0\leq\beta_i\leq\alpha_i}\vartheta(p_1^{\beta_1})\cdots\vartheta(p_k^{\beta_k})=\sum_{0\leq\beta_i\leq{\alpha_i}} 
  \vartheta(p_1^{\beta_1}\cdots p_k^{\beta_k})\] Die Produkte
  $p_1^{\beta_1}\cdots p_k^{\beta_k}$ mit
  $0\leq\beta_1\leq\alpha_1,\ldots,0\leq\beta_k\leq\alpha_k$
  durchlaufen aber gerade die Teiler von $a$.
\end{beweis}

\noindent Bemerkung: F\"{u}r $a=1$ ist die rechte Seite von (\ref{3}) gleich
$1$ zu setzen.

\section{Teilerzahl und -summe}

Setzt man in (\ref{3}) $\vartheta\equiv 1$, so erh\"{a}lt man f\"{u}r die
Anzahl $\tau(a)$ der Teiler von $a$
\[\tau(a)=\sum_{d|a}1=(\alpha_1+1)\cdots(\alpha_k+1)\]
Folgerungen:
\begin{itemize}
\item $\tau$ ist eine multiplikative Funktion.
\item F\"{u}r jede Primzahl $p$ gilt speziell $\tau(p^\alpha)=\alpha+1$.
\end{itemize}

\noindent Setzt man in (\ref{3}) $\vartheta(a)=a$, so erh\"{a}lt man f\"{u}r
die Summe $S(a)$ der Teiler von $a$
\begin{eqnarray*}S(a)&=&\sum_{d|a}d=(1+p_1+\ldots+p_1^{\alpha_1})\cdots(1+p_k+\ldots+p_k^{\alpha_k})\\
S(a)&=&\frac{p_1^{\alpha_1+1}-1}{p_1-1}\cdots\frac{p_k^{\alpha_k+1}-1}{p_k-1}
\end{eqnarray*}

\noindent Folgerungen:
\begin{itemize}
\item $S$ ist eine multiplikative Funktion.
\item F\"{u}r jede Primzahl $p$ gilt speziell
$S(p^\alpha)=\frac{p^{\alpha+1}-1}{p-1}$.
\end{itemize}

\section{Die M\"{o}bius-Funktion}

Die {\it M\"{o}biusfunktion} $\mu$ ist diejenige multiplikative Funktion, die
f\"{u}r alle Primzahlen $p$
\[\mu (p)=-1\;\;\;\mu (p^\alpha)=0 \;(\alpha>1)\]
erf\"{u}llt. Es gilt:
\begin{itemize}
\item Ist $a$ durch eine Quadratzahl $>1$ teilbar (d.h. in der kanonischen
Zerlegung $a=p_1^{\alpha_1} p_2^{\alpha_2}\cdots p_k^{\alpha_k}$ ist
mindestens einer der Exponenten $\alpha_1,\ldots,\alpha_k$ gr\"{o}{\ss}er als
$1$), so ist $\mu(a)=0$.
\item Ist $a=p_1\cdots p_k$ das Produkt $k$ verschiedener Primzahlen, so gilt
$\mu(a)=(-1)^k$.
\end{itemize}
\begin{theorem}
F\"{u}r jede multiplikative Funktion $\vartheta$ gilt
\[\sum_{d|a}\mu(d)\vartheta(d)=(1-\vartheta(p_1))\cdots(1-\vartheta(p_k))\]
\end{theorem}
Beweis. $\vartheta_1(a)=\mu(a)\theta(a)$ ist als Produkt
multiplikativer Funktionen wieder multiplikativ und (\ref{3})
liefert
\begin{eqnarray*}
\sum_{d|a}\mu(d)\vartheta(d)&=&\sum_{d|a}\vartheta_1(d)\\
&=&(1+\vartheta_1(p_1)+\vartheta_1(p_1^2)+\ldots
+\vartheta_1(p_1^{\alpha_1}))\\ 
&& \ \ \ \cdots
(1+\vartheta_1(p_k)+\vartheta_1(p_k^2)+\ldots+\vartheta_1(p_k^{\alpha_k}))\\
&=&(1+\mu(p_1)\vartheta(p_1)+\mu(p_1^2)\vartheta(p_1^2)+\ldots
+\mu(p_1^{\alpha_1})\vartheta(p_1^{\alpha_1}))\\
&& \ \ \ \cdots
(1+\mu(p_k)\vartheta(p_k)+\mu(p_k^2)\vartheta(p_k^2)+\ldots
+\mu(p_k^{\alpha_k})\vartheta(p_k^{\alpha_k}))\\ 
&=&(1-\vartheta(p_1))\cdots(1-\vartheta(p_k))
\end{eqnarray*}

\noindent Anwendungen:
\begin{itemize}
\item F\"{u}r $\vartheta\equiv 1$ folgt
\[\sum_{d|a}\mu(d)=\begin{cases}0&a>1\\1&a=1\end{cases}\]
\item F\"{u}r $\vartheta(a)=\frac{1}{a}$ folgt
\[\sum_{d|a}\frac{\mu(d)}{d}=
\begin{cases}
(1-\frac{1}{p_1})\cdots(1-\frac{1}{p_k})&a>1\\1&a=1
\end{cases}\]
\end{itemize}

\section{Die Eulersche Funktion}
F\"{u}r positives ganzes $a$ bedeute $\varphi(a)$ die Anzahl der Zahlen aus
$1,2,\ldots,a$, die zu $a$ teilerfremd sind. $\varphi$ hei{\ss}t {\it
Eulersche Funktion}.
\begin{theorem}
Es gilt f\"{u}r $a=p_1^{\alpha_1} p_2^{\alpha_2}\cdots p_k^{\alpha_k}$ die
Darstellung
\begin{eqnarray*}
\varphi(a)&=&a\left(1-\frac{1}{p_1}\right)\cdots\left(1-\frac{1}{p_k}\right)\\
&=&(p_1^{\alpha_1}-p_1^{\alpha_1-1})\cdots(p_k^{\alpha_k}-p_k^{\alpha_k-1})
\end{eqnarray*}
\end{theorem}
\begin{beweis}
  Setzt man $d_1=(1,a),d_2=(2,a),\ldots,d_{a}=(a,a)$, so gilt
\[\varphi(a)=\sum_{d|d_1}\mu(d)+\ldots+\sum_{d|d_{a}}\mu(d)
=\sum_{d|a}S_d\,\mu(d),
\] 
denn $d$ durchl\"{a}uft genau die Teiler von $a$: Ist n\"{a}mlich $d$ Teiler
eines $d_j$ so ist es wegen $d_j|a$ auch Teiler von a.  Umgekehrt taucht jeder
Teiler $d$ von $a$ unter den Teilern der $d_j$ auf, denn es ist
$d_d=(d,a)=d$. $S_d$ bezeichnet f\"{u}r jeden Teiler $d$ von $a$ die Anzahl
derjenigen $d_j$, die durch $d$ teilbar sind. Da dies genau die Zahlen
$d_d,d_{2d},\ldots,d_a$ sind, gilt $S_d=\frac{a}{d}$ und wir erhalten weiter
\[\varphi(a)=\sum_{d|a}\mu(d)\frac{a}{d}=a\sum_{d|a}\frac{\mu(d)}{d}
=a(1-\frac{1}{p_1})\cdots(1-\frac{1}{p_k})
\] 
\end{beweis}

\noindent Folgerungen:
\begin{itemize}
\item $\varphi$ ist eine multiplikative Funktion.
\item F\"{u}r jede Primzahl $p$ und $\alpha>0$ gilt speziell
$\varphi(p^\alpha)=p^\alpha-p^{\alpha-1}$.
\end{itemize}
\begin{theorem}
Es gilt \[\sum_{d|a}\varphi(d)=a\]
\end{theorem}
\begin{beweis}
  Nach (\ref{3}) gilt
\begin{eqnarray*}
  \sum_{d|a}\varphi(d)&=&(1+\varphi(p_1)+\varphi(p_1^2)+\ldots
    +\varphi(p_1^{\alpha_1}))\cdots 
    (1+\varphi(p_k)+\varphi(p_k^2)+\ldots+\varphi(p_k^{\alpha_k}))\\
    &=&(1+(p_1-1)+(p_1^2-p_1)+\ldots+(p_1^{\alpha_1}-p_1^{\alpha_1-1}))
    \\&& \cdots 
    (1+(p_k-1)+(p_k^2-p_k)+\ldots+(p_k^{\alpha_k}-p_k^{\alpha_k-1}))\\
    &=&p_1^{\alpha_1}\cdots p_k^{\alpha_k}=a 
  \end{eqnarray*}
\end{beweis}
\end{document}
