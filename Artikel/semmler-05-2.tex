% Version: $Id: semmler-05-2.tex,v 1.1 2008/09/05 09:52:58 graebe Exp $
\documentclass[11pt]{article}  
\usepackage{kosemnet,ko-math,ngerman,url}  

\author{Gunter Semmler, Freiberg\\\url{semmler@math.tu-freiberg.de}}
\title{Kombinatorik\kosemnetlicensemark}
\date{}

\renewcommand{\theenumi}{\alph{enumi}}
\begin{document}
\maketitle         

\section{Kombinatorische Grundformeln}
Kombinatorik besch\"{a}ftigt sich mit der Anzahlbestimmung gewisser
Anordnungen von Objekten. In Olympiaden ist dieses Thema deshalb so beliebt,
weil man ohne gro{\ss}e Vorkenntnisse Aufgaben bearbeiten kann. Die
eingerahmten Formeln dieses Artikels sind so n\"{u}tzlich, da{ss} sie jeder
Olympiadeteilnehmer auswendig wissen sollte.

\begin{beispiel}
Wieviele M\"{o}glichkeiten gibt es, die Buchstaben des Wortes ''CANTOR''
anzuordnen?
\end{beispiel}
 
Es gibt $6$ M\"{o}glichkeiten, den ersten Buchstaben auszusuchen, f\"{u}r jede
Wahl eines ersten Buchstaben $5$ M\"{o}glichkeiten, den zweiten Buchstaben zu
w\"{a}hlen, usw. Insgesamt gibt es also $6\cdot 5\cdot 4\cdots 1=6!$
Anordnungsm\"{o}glichkeiten f\"{u}r die sechs Buchstaben von ''CANTOR''.  Eine
jede Umordnung der Buchstaben hei{\ss}t auch {\em Permutation}. Allgemein kann
man sich genauso \"{u}berlegen, da{ss} die
\begin{center}
\fbox{Zahl der Permutationen von $n$ paarweise verschiedenen Objekten gleich
$n!$}
\end{center}
ist.
\begin{beispiel}
Auf wieviele Arten lassen sich die Buchstaben des Wortes ''EULER'' anordnen?
\end{beispiel} 
Obwohl das Wort ''EULER'' $5$ Buchstaben hat, lautet die Antwort nicht $5!$,
denn der Buchstabe E kommt doppelt vor. Bei $5!$ M\"{o}glichkeiten hat man
jede Anordnungsm\"{o}glichkeit doppelt gez\"{a}hlt, denn die beiden E's lassen
sich vertauschen, ohne das Wort zu \"{a}ndern. Die richtige Antwort lautet
also $5!/2$. Auch diese Beipiel ist ein Spezialfall einer allgemeineren
Formel: Gegeben seien $n$ Objekte, die aus $k$ Gruppen von
$n_1,n_2,\ldots,n_k$ jeweils gleichen Objekten bestehen. Es gilt also
$n=n_1+\ldots+n_k$ und die Zahl der Anordnungsm\"{o}glichkeiten ist die
\begin{center}
\fbox{Zahl der Permutationen mit Wiederholungen gleich $\dfrac{n!}{n_1!\cdots
    n_k!}. $} 
\end{center}
\begin{aufgabe} Auf wieviele Arten lassen sich die Buchstaben  des Wortes
  ''RAMANUJAN'' anordnen? 
\end{aufgabe}
\begin{beispiel}
Eine Pizza ist mit 8 verschiedenen Zutaten verf\"{u}gbar: K\"{a}se, Knoblauch,
Ananas, Wurst, Pepperoni, Pilze, Oliven, gr\"{u}ner Pfeffer. Wieviele
M\"{o}glichkeiten gibt es, eine Pizza mit genau drei Beilagen zu w\"{a}hlen?
\end{beispiel}
Um die Aufgabe zu l\"{o}sen, wenden wir ein ebenso einfaches wie
n\"{u}tzliches Prinzip an:
\begin{quote}
\fbox{\parbox{13cm}{Besteht zwischen zwei Mengen $A$ und $B$ eine bijektive
(d.h. eineindeutige) Abbildung, so gen\"{u}gt es, eine der beiden Mengen
abzuz\"{a}hlen, die andere enth\"{a}lt genau so viele Elemente.}}
\end{quote}
Wir betrachten alle verschiedenen Anordnungen der Ziffern aus $11100000$. Eine
Bijektion auf wie Menge der Zutatenkombinationen wird durch folgende
Vorschrift gegeben: Steht in einer Permutation von $11100000$ an $n$-ter,
$m$-ter und $k$-ter Stelle eine Eins, so nehme man die $n$-te, $m$-te und
$k$-te Zutat entsprechend der obigen Reihenfolge. Z.B. bedeutet $01001100$
eine Pizza mit Knoblauch, Pepperoni und Pilzen (geschmackliche Aspekte spielen
bei dieser Aufgabe selbstverst\"{a}ndlich keine Rolle). Die
Anordnungsm\"{o}glichkeiten der Ziffern sind aber Permutationen mit
Wiederholung, also $\dfrac{8!}{5!3!}=\binom{8}{3}$. Diese \"{U}berlegung
f\"{u}hrt uns allgmein darauf, da{ss} die
\begin{center}
\fbox{Zahl der Kombinationen von $k$ aus $n$ Elementen ohne Wiederholungen
  gleich $\binom{n}{k} $}
\end{center}
ist, d.h. es wird die Anzahl der M\"{o}glichkeiten gez\"{a}hlt, aus $n$
verschiedenen Objekten genau $k$ auszuw\"{a}hlen, ohne da{ss} ein Objekt
mehrfach gew\"{a}hlt werden kann. Dabei ist die Reihenfolge egal. Wird sie
zus\"{a}tzlich ber\"{u}cksichtigt, so ist das Ergebnis noch mit der Anzahl
$k!$ von Permutationen von $k$ Objekten zu multiplizieren, d.h.  man findet
da{ss} die
\begin{center}
\fbox{ Zahl der Variationen von $k$ aus $n$ Elementen ohne Wiederholung gleich
$\dfrac{n!}{(n-k)!}$}
\end{center} 
ist.

Wenn man 
\[(x+y)^n=(x+y)(x+y)\cdots (x+y)\]
ausmultipliziert, so w\"{a}hlt man aus jedem Faktor entweder $x$ oder $y$ aus,
multipliziert diese Auswahl von Variablen miteinander und addiert
anschlie{\ss}end alle dabei entstehenden Produkte.  Das Produkt $x^{k}y^{n-k}$
entsteht dabei, wenn man genau $k$-mal $x$ w\"{a}hlt (und notwendigerweise
$(n-k)$ mal $y$). Dazu gibt es nach obigem $\binom{n}{k}$ M\"{o}glichkeiten.
Folglich tritt das Produkt $x^ky^{n-k}$ mit dem Koeffizient $\binom{n}{k}$
auf. F\"{u}r $k$ sind alle Werte von 0 bis $n$ m\"{o}glich. Damit ist ein
kombinatorischer Beweis des wichtigen binomischen Satzes
\begin{center}
  \fbox{$(x+y)^n=\sum_{k=0}^n\binom{n}{k} x^k y^{n-k}$}
\end{center}
geliefert. Auch viele andere Formeln lassen sich statt durch algebraische
Rechnungen durch kombinatorische Interpretation begr\"{u}nden. Dabei geht man
von dem allgemeinen Grundsatz
\begin{center}
  \fbox{Z\"{a}hle dieselben Objekte auf zweierlei Weise ab!}
\end{center}
aus.
\begin{beispiel} Man finde einen kombinatorischen Beweis f\"{u}r die  Formel
\[\binom{2n}{2}=2 \binom{n}{2}+n^2.\]
\end{beispiel}
In einem Raum mit $n$ Frauen und $n$ M\"{a}nnern sollen zwei Personen
ausgew\"{a}hlt werden. Da insgesamt $2n$ Personen anwesend sind, gibt es dazu
$\binom{2n}{2}$ M\"{o}glichkeiten. Die Zahl der M\"{o}glichkeiten, zwei Frauen
auszuw\"{a}hlen, ist gleich $\binom{n}{2}$, die Zahl der M\"{o}glichkeiten,
zwei M\"{a}nner auszuw\"{a}hlen, ist ebenfalls $\binom{n}{2}$. Schlie{\ss}lich
gibt es $n^2$ M\"{o}glichkeiten, ein Paar aus einer Frau und einem Mann
auszusuchen. Andere L\"{o}sungsm\"{o}glichkeit: Man betrachte die Anzahl der
m\"{o}glichen Verbindungslinien zwischen den Eckpunkten eines $2n$-Ecks, wobei
$n$ Punkte rot und $n$ Punkte gr\"{u}n gef\"{a}rbt sind...
\begin{aufgabe}
Man leite folgende Formeln durch kombinatorische Interpretation her:
\begin{enumerate}
\item \[\binom{2n+2}{n+1}=\binom{2n}{n+1}+2\binom{2n}{n}+\binom{2n}{n-1}\]
\item \[\binom{n}{s}=\frac{n}{s}\binom{n-1}{s-1} \]
\item \[\binom{n}{r}\binom{r}{k}=\binom{n}{k}\binom{n-k}{r-k}\]
\item \[ \sum_{i=0}^n{\binom{n}{i}\binom{n}{n-i}} =\binom{2n}{n}
  =\sum_{i=0}^n{\binom{n}{i}^2}\] 
\item \[\binom{n}{0}+\binom{n}{1}+\binom{n}{2}+\ldots+\binom{n}{n}=2^n\]
\end{enumerate}
\end{aufgabe}
\begin{aufgabe}
Von $3n+1$ Objekten seien $n$ ununterscheidbar, die restlichen alle von diesen
$n$ und auch untereinander verschieden. Man zeige, da{ss} es genau $2^{2n}$
M\"{o}glichkeiten gibt, $n$ Objekte auszuw\"{a}hlen.
\end{aufgabe}
\begin{aufgabe}
Wieviele M\"{o}glichkeiten gibt es, zwei disjunkte nichtleere Teilmengen aus
der Menge $\{1,\ldots,n\}$ auszuw\"{a}hlen?
\end{aufgabe}
\begin{aufgabe} 
Wieviele Diagonalen hat ein konvexes $n$-Eck? Und wieviele
Diagonalenschnittpunkte, vorausgesetzt, keine drei Diagonalen gehen durch
einen Punkt?
\end{aufgabe}
\begin{aufgabe}
In der Ebene sei ein kartesisches Koordinatensystem festgelegt. Es werden Wege
entlang der Gitterpunkte (das sind Punkte mit ganzzahligen Koordinaten)
betrachtet, bei denen vom Punkt $(i,j)$ ein Schritt nur nach $(i+1,j)$ oder
$(i,j+1)$ m\"{o}glich ist. Wieviele solcher Wege gibt es vom Punkt $(0,0)$ zum
Punkt $(m,n)$, die v\"{o}llig unterhalb der Geraden $y=x+1$ liegen?
\end{aufgabe}
%\section{Aufgaben}
%\begin{enumerate}
%\item
%\end{enumerate}
\section{Die Formel von Vereinigung und Durchschnitt}

Mit $\card{A}$ bezeichnen wir die Anzahl der Elemente einer endlichen Menge
$A$. F\"{u}r zwei Mengen $A$ und $B$ findet man leicht
\[\card{A\cup B}=\card{A}+\card{B}-\card{A\cap B}.\]

F\"{u}r drei Mengen $A,B$ und $C$ illustriert das Mengendiagramm (zu erg�nzen)
die Formel
\[\card{A\cup B\cup C}=\card{A}+\card{B}+\card{C}-\card{A\cap B}-\card{B\cap
  C}-\card{A\cap C}+\card{A\cap B\cap C}\] 

Die Verallgemeinerung auf $n$ Mengen sieht folgenderma{\ss}en aus:

\begin{center}
\framebox{
\begin{minipage}{12cm}
\begin{align*}
 \card{A_1\cup\ldots \cup A_n} &= \sum_{i=1}^n{\card{A_i}}-\sum_{1\leq
 i<j\leq n}{\card{A_i\cap A_j}}+\ldots \\ &+(-1)^{k-1}\sum_{1\leq i_1<i_2<
 \ldots i_k\leq n}{\card{A_{i_1}\cap A_{i_2}\cap \ldots \cap A_{i_k}}}\\ 
 & +\ldots +(-1)^{n-1}\card{A_1\cap\ldots\cap A_n}
\end{align*}
\end{minipage}}
\end{center}

\bigskip 
Um diese Formel zu beweisen betrachten wir ein beliebiges Element $x\in
A_1\cup \ldots \cup A_n$. Auf der linken Seite wird dieses Element genau
einmal gez\"{a}hlt. Angenommen, $x$ ist in genau $r$ Mengen aus $A_1,\ldots,
A_n$ enthalten. Das bedeutet, da{ss} $x$ in der Summe $\sum_{i=1}^n\card{A_i}$
genau $r$-fach mitgez\"{a}hlt wurde. In der zweiten Summe $\sum_{1\leq i<j\leq
n} \card{A_i\cap A_j}$ wurde $x$ so oft mitgez\"{a}hlt, wie es Kombinationen
von zwei Mengen gibt, die beide $x$ enthalten, also $\binom{r}{2}$ mal. In der
dritten Summe wurde $x$ entsprechend $\binom{r}{3}$ mal mitgez\"{a}hlt,
usw. Es gen\"{u}gt also die Identit\"{a}t
\[1=r-\binom{r}{2}+\binom{r}{3}-\ldots+(-1)^{r-1}\binom{r}{r}\]
zu beweisen. Wegen $\binom{r}{0}=1$ und $\binom{r}{1}=r$ folgt diese aber
unmittelbar aus dem binomischen Satz:
\[\binom{r}{0}-\binom{r}{1}+\binom{r}{2}-\binom{r}{3} +\ldots +(-1)^r
\binom{r}{r}=(1-1)^r=0.
\] 
Bei Anwendungen dieser Formel kommt es darauf an, die Mengen $A_i$
m\"{o}glichst geschickt zu w\"{a}hlen. Wie \"{u}brigens bei allen
kombinatorischen Aufgaben halte man sich immer vor Augen:
\begin{center}
  \fbox{Manchmal ist es einfacher, das Komplement einer Menge abzuz\"{a}hlen.}
\end{center}

\begin{beispiel}
Wieviele M\"{o}glichkeiten gibt es, aus einem Kartenspiel mit 52 Karten
f\"{u}nf auszuw\"{a}hlen, so da{ss} alle vier Farben Karo, Herz, Pique, Kreuz
vertreten sind?
\end{beispiel}
Insgesamt gibt es $\binom{52}{5}$ M\"{o}glichkeiten, f\"{u}nf Karten
auszuw\"{a}hlen. Bedeute nun $A_1,A_2,A_3 $ bzw. $A_4$ die Menge derjenigen
F\"{u}nfkartenauswahlen, die kein Karo, Herz, Pique bzw.  Kreuz enthalten.
$A_1$ umfa{ss}t beispielsweise alle Auswahlen von 5 Karten aus den 3 Farben
Herz, Pique, Kreuz, also aus 39 Karten:
\[\card{A_1}=\card{A_2}=\card{A_3}=\card{A_4}=\binom{39}{5}\;\;\Rightarrow
\;\;\sum_{i=1}^4 \card{A_i}=4\cdot\binom{39}{5}.
\] 
$A_i\cap A_j\,(i\neq j)$ stellt die Menge derjenigen F\"{u}nferauswahlen dar,
die zwei Farben nicht enthalten, es wird also aus 26 Karten ausgew\"{a}hlt:
\[\card{A_i\cap A_j}=\binom{26}{5}\,(i\neq j)\;\;\Rightarrow\;\;\sum_{1\leq
  i<j\leq 4}\card{A_i\cap A_j}=\binom{4}{2}\binom{26}{5}.
\] 
Genauso folgt
\[\sum_{1\leq i<j<k\leq 4}\card{A_i\cap A_j\cap
  A_k}=\binom{4}{3}\binom{13}{5},\] 
w\"{a}hrend $A_1\cap A_2\cap A_3\cap A_4$ leer ist, denn es k\"{o}nnen nicht
alle Farben gleichzeitig fehlen. Die Formel von Vereinigung und Durchschnitt
liefert aber nun
\[\card{A_1\cup A_2\cup A_3\cup A_4}= \binom{4}{1}\binom{39}{5}
-\binom{4}{2}\binom{26}{5} +\binom{4}{3}\binom{13}{5}-0
\] 
$A_1\cup A_2\cup A_3\cup A_4$ stellt die Menge derjenigen F\"{u}nferauswahlen
dar, in der mindestens eine Farbe fehlt, also gerade das Komplement der
gesuchten Menge. Somit ist die Anzahl der F\"{u}nferauswahlen, in denen alle
Farben vertreten sind, gleich
\[\binom{52}{5} -\binom{4}{1}\binom{39}{5} +\binom{4}{2}\binom{26}{5}
-\binom{4}{3}\binom{13}{5}.\] 

\begin{aufgabe}
Wieviele ganze Zahlen zwischen 1 und 1000 (beide inbegriffen) sind weder durch
2,3, noch 5 teilbar?
\end{aufgabe}
\begin{aufgabe}
Jedes von $n$ Kindern soll ein Eis einer von $k$ verf\"{u}gbaren Sorten
bekommen. Wieviele M\"{o}glichkeiten gibt es, die Eissorten zu verteilen, wenn
jede Sorte mindestens einmal ausgegeben werden soll?
\end{aufgabe}
\begin{aufgabe}
Man berechne die Anzahl der M\"{o}glichkeiten, wie 4 Ehepaare auf einer Reihe
von 8 St\"{u}hlen Platz nehmen k\"{o}nnen, ohne da{ss} jemand nebem seinem
eigenen Ehepartner sitzt!
\end{aufgabe}
\begin{aufgabe}
Wieviele M\"{o}glichkeiten gibt es, da{ss} $n$ Personen miteinander
telefonieren, so da{ss} jeder mit genau einem anderen spricht und keiner sich
selbst anruft?
\end{aufgabe}
\section{Rekursionen und erzeugende Funktionen}
Bei schwierigen Aufgaben ist es oft einfacher, f\"{u}r gesuchte Anzahlen
zun\"{a}chst rekursive Beziehungen aufzustellen. Um daraus eine explizite
Formel zu machen, lernen wir die Methode der erzeugenden Funktionen kennen.
\begin{beispiel}
Auf wieviele Weisen kann ein konvexes $n$-Eck trianguliert werden, d.h. durch
Diagonalen in Dreiecke zerlegt werden?
\end{beispiel} 
Die gesuchte Anzahl sei $t_n$. Offenbar gilt $t_3=1, t_4=2, t_5=5$, siehe
folgendes Bild: ({fuenf.bmp} fehlt) 

Fixieren wir im Sechseck eine Seite und betrachten alle m\"{o}glichen Dreicke
mit dieser Seite als Grundseite, so finden wir
\[ t_6=t_5 +t_4+t_4+t_5=14\] 
Bild {sechs.bmp} fehlt.

Im Siebeneck gilt  die Formel
\[t_7=t_2t_6+t_3t_5+t_4t_4+t_5t_3+t_6t_2,\]
wobei $t_2=1$ gesetzt worden ist. Bild {sieben.bmp} fehlt.

Allgemein  folgt
\[t_n=t_2t_{n-1}+t_3t_{n-2}+\ldots+t_{n-1}t_2=\sum_{k=2}^{n-1}t_kt_{n+1-k}.\]
Damit ist eine rekursive Darstellung gewonnen. Die Zahlen $C_n=t_{n+2}$ (die
Indexverschiebung hat historische Gr\"{u}nde) sind als {\em Catalansche
Zahlen} bekannt. Um aus der Rekursionsformel
\[C_{n+1}=\sum_{k=0}^{n} C_kC_{n-k}\]
eine explizite zu gewinnen, ordnen wir dieser Folge eine Potenzreihe mit den
Catalanschen Zahlen als Koeffizienten zu, n\"{a}mlich
\[f(x)=C_0+C_1x+C_2x^2+\ldots.\]
Solange man keine konkreten Werte f\"{u}r $x$ einsetzt, braucht man sich um
Konvergenzfragen nicht k\"{u}mmern, sondern kann mit diesen Potenzreihen
formal wie mit endlichen Polynomen rechen. (Es gibt eine mathematische
Theorie, die das streng begr\"{u}ndet.) Wir quadrieren:
\[f(x)^2=C_0^2+(C_1C_0+C_0C_1)x+(C_2C_0+C_1C_1+C_0C_2)x^2+\ldots
=C_1+C_2x+C_3x^2+\ldots\]
Also gilt
\[xf(x)^2=f(x)-C_0=f(x)-1\]
Das ist eine quadratische Gleichung f\"{u}r $f(x)$ mit der L\"{o}sung
\[f(x)=\frac{1\pm \sqrt{1-4x}}{2x}\]
Wegen $f(0)=C_0$ ist das negative Vorzeichen zu w\"{a}hlen. Wir m\"{u}ssen
also
\[f(x)=\frac{1- \sqrt{1-4x}}{2x}\]
in eine Potenzreihe entwickeln und k\"{o}nnen dann an den Koeffizienten eine
explizite Darstellung der Catalanschen Zahlen ablesen. Wir erinneren uns an
die Binominalreihe
\begin{equation}\label{1}
\fbox{$(1+x)^\alpha=\sum_{n=0}^\infty \binom{\alpha}{n} x^n$.}
\end{equation}
Dabei ist der Binominalkoeffizient definiert durch
\[\binom{\alpha}{n}= \frac{\alpha(\alpha-1)\cdots (\alpha-n+1)}{n!}.\]
F\"{u}r nat\"{u}rliches $\alpha$ verschwindet $\binom{\alpha}{n}$ f\"{u}r
$n>\alpha$, so da{ss} (\ref{1}) in eine endliche Summe \"{u}bergeht. Die
Binominalreihe ist also eine Verallgemeinerung des binomischen Satzes auf
reelle Exponenten. Man rechnet
\begin{eqnarray*}
\sqrt{1-4x}&=&1+\sum_{n=1}^\infty
\frac{\left(\frac{1}{2}\right)\left(\frac{1}{2}-1\right)\cdots
\left(\frac{1}{2}-n+1\right)}{n!}(-4x)^n\\ &=&1+\sum_{n=1}^\infty
(-1)^{n-1}\frac{1\cdot 1\cdot 3\cdot 5\cdots (2n-3)}{2^n n!}(-4)^nx^n\\
&=&1-2x\sum_{n=1}^\infty \frac{1\cdot 3\cdot 5\cdots
(2n-3)2^{n-1}(n-1)!}{n\cdot (n-1)!(n-1)!}x^{n-1}\\ &=&1-2x\sum_{n=1}^\infty
\frac{(2n-2)!}{n\cdot (n-1)!(n-1)!} x^{n-1}
\end{eqnarray*}
Damit folgt 
\[f(x)=\frac{1- \sqrt{1-4x}}{2x}=\sum_{n=1}^\infty\frac{1}{n}\binom{2n-2}{n-1}
x^{n-1}.\] 
Der Koeffizient  vor $x^n$ ist
\[C_n=\frac{1}{n+1}\binom{2(n+1)-2}{(n+1)-1}=\frac{1}{n+1}\binom{2n}{n}.\]
\begin{aufgabe} 
Unter einem Dominostein verstehen wir ein $1\times 2 $-Rechteck. Zeige da{ss}
die Anzahl $f_n$ der m\"{o}glichen \"{U}berdeckungen eines $1\times
n$-Rechtecks durch Dominosteine der Rekursion
\[f_{n+2}=f_{n+1}+f_n\]
gen\"{u}gt. Die Folge $f_n$ hei{\ss}t auch Fibonacci-Folge. Mit Hilfe von
erzeugenden Funktionen gewinne man die Formel
\[f_n=\frac{1}{\sqrt{5}}\left[\left(\frac{1+\sqrt{5}}{2}\right)^n-
  \left(\frac{1-\sqrt{5}}{2}\right)^n\right].\] 
\end{aufgabe}
\begin{aufgabe} 
Wieviele $n$-Tupel aus den Zahlen $1,2,3,4$ gibt es, in denen $1$ und $2$
nirgends benachbart sind?
\end{aufgabe} 
 
\begin{thebibliography}{99}
\bibitem{En} Engel: IMO-\"{U}bungsaufgaben, Heft 22. Herausgegeben vom
Zentralen Komitee f\"{u}r die Olympiaden Junger Mathematiker beim Ministerium
f\"{u}r Volksbildung der DDR.
\bibitem{E2} Engel, A.: Problem Solving Strategies. Springer Verlag 1998.
\bibitem{Ze} Zeitz, P.: The art and craft of problem solving. John Wiley \&
Sons 1999.
\end{thebibliography}

\begin{comment}
graebe (2005-01-05): Es fehlen noch einige Bilder.
\end{comment}

\end{document}
